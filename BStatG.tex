\chapter[BStatG]{Bundesstatistikgesetz}
\qrcode{https://www.gesetze-im-internet.de/bstatg_1987/BStatG.pdf}
\newline
\url{https://www.gesetze-im-internet.de/bstatg_1987/BStatG.pdf}
    \section{\S 1: Statistik für Bundeszwecke} 
    Die Statistik für Bundeszwecke (Bundesstatistik) hat im föderativ gegliederten Gesamtsystem der amtlichen Statistik die Aufgabe, laufend Daten über Massenerscheinungen zu erheben, zu sammeln, aufzubereiten, darzustellen und zu analysieren. Für sie gelten die Grundsätze der Neutralität, Objektivität und fachlichen Unabhängigkeit. Sie gewinnt die Daten unter Verwendung wissenschaftlicher Erkenntnisse und unter Einsatz der jeweils sachgerechten Methoden und Informationstechniken. Durch die Ergebnisse der Bundesstatistik werden gesellschaftliche, wirtschaftliche und ökologische Zusammenhänge für Bund, Länder einschliesslich Gemeinden und Gemeindeverbände, Gesellschaft, Wirtschaft, Wissenschaft und Forschung aufgeschlüsselt. Die Bundesstatistik ist Voraussetzung für eine am Sozialstaatsprinzip ausgerichtete Politik. Die für die Bundesstatistik erhobenen Einzelangaben dienen ausschliesslich den durch dieses Gesetz oder eine andere eine Bundesstatistik anordnende Rechtsvorschrift festgelegten Zwecken.
    \section{\S 10: Erhebungs- und Hilfsmerkmale}
        \begin{enumerate}[label=(\arabic*)]
            \item Bundesstatistiken werden auf der Grundlage von Erhebungs- und Hilfsmerkmalen erstellt. Erhebungsmerkmale umfassen Angaben über persönliche und sachliche Verhältnisse, die zur statistischen Verwendung bestimmt sind. Hilfsmerkmale sind Angaben, die der technischen Durchführung von Bundesstatistiken dienen. Für andere Zwecke dürfen sie nur verwendet werden, soweit Absatz 2 oder ein sonstiges Gesetz es zulassen.
            \item Der Name der Gemeinde, die Blockseite und die geografische Gitterzelle dürfen für die regionale Zuordnung der Erhebungsmerkmale genutzt werden. Die übrigen Teile der Anschrift dürfen für die Zuordnung zu Blockseiten und geografischen Gitterzellen für einen Zeitraum von bis zu vier Jahren nach Abschluss der jeweiligen Erhebung genutzt werden. Besondere Regelungen in einer eine Bundesstatistik anordnenden Rechtsvorschrift bleiben unberührt.
            \item Blockseite ist innerhalb eines Gemeindegebiets die Seite mit gleicher Strassenbezeichnung von der durch Strasseneinmündungen oder vergleichbare Begrenzungen umschlossenen Fläche. Eine geografische Gitterzelle ist eine Gebietseinheit, die bezogen auf eine vorgegebene Kartenprojektion quadratisch ist und mindestens 1 Hektar gross ist.
        \end{enumerate}
    \section{\S 12 Trennung und Löschung der Hilfsmerkmale}
        \begin{enumerate}[label=(\arabic*)]
            \item Hilfsmerkmale sind, soweit Absatz 2, § 10 Absatz 2, § 13 oder eine sonstige Rechtsvorschrift nichts anderes bestimmen, zu löschen, sobald bei den statistischen Ämtern die Überprüfung der Erhebungs- und Hilfsmerkmale auf ihre Schlüssigkeit und Vollständigkeit abgeschlossen ist. Sie sind von den Erhebungsmerkmalen zum frühestmöglichen Zeitpunkt zu trennen und gesondert aufzubewahren oder gesondert zu speichern.
            \item Bei periodischen Erhebungen für Zwecke der Bundesstatistik dürfen die zur Bestimmung des Kreises der zu Befragenden erforderlichen Hilfsmerkmale, soweit sie für nachfolgende Erhebungen benötigt werden, gesondert aufbewahrt oder gesondert gespeichert werden. Nach Beendigung des Zeitraumes der wiederkehrenden Erhebungen sind sie zu löschen.
        \end{enumerate}
    \section{\S 16 Geheimhaltung}
        \begin{enumerate}[label=(\arabic*)]
            \item Einzelangaben über persönliche und sachliche Verhältnisse, die für eine Bundesstatistik gemacht werden, sind von den Amtsträgern und Amtsträgerinnen und für den öffentlichen Dienst besonders Verpflichteten, die mit der Durchführung von Bundesstatistiken betraut sind, geheim zu halten, soweit durch besondere Rechtsvorschrift nichts anderes bestimmt ist. Die Geheimhaltungspflicht besteht auch nach Beendigung ihrer Tätigkeit fort. Die Geheimhaltungspflicht gilt nicht für
                \begin{enumerate}[label=\arabic*.]
                    \item Einzelangaben, in deren Übermittlung oder Veröffentlichung die Betroffenen schriftlich eingewilligt haben, soweit nicht wegen besonderer Umstände eine andere Form der Einwilligung angemessen ist,
                    \item Einzelangaben aus allgemein zugänglichen Quellen, wenn sie sich auf die in § 15 Absatz 1 genannten öffentlichen Stellen beziehen, auch soweit eine Auskunftspflicht aufgrund einer eine Bundesstatistik anordnenden Rechtsvorschrift besteht,
                    \item Einzelangaben, die vom Statistischen Bundesamt oder den statistischen Äm\-tern der Länder mit den Einzelangaben anderer Befragter zusammengefasst und in statistischen Ergebnissen dargestellt sind, 
                    \item Einzelangaben, wenn sie den Befragten oder Betroffenen nicht zuzuordnen sind. Die §§ 93, 97, 105 Absatz 1, § 111 Absatz 5 in Verbindung mit § 105 Absatz 1 sowie § 116 Absatz 1 der Abgabenordnung vom 16. März 1976 (BGBl. I S. 613; 1977 I S. 269), zuletzt geändert durch Artikel 1 des Gesetzes vom 19. Dezember 1985 (BGBl. I S. 2436), gelten nicht für Personen und Stellen, soweit sie mit der Durchführung von Bundes- , Landes- oder Kommunalstatistiken betraut sind.
                \end{enumerate}
            \item Die Übermittlung von Einzelangaben zwischen den mit der Durchführung einer Bundesstatistik betrauten Personen und Stellen ist zulässig, soweit dies zur Erstellung der Bundesstatistik erforderlich ist. Darüber hinaus ist die Übermittlung von Einzelangaben zwischen den an einer Zusammenarbeit nach § 3a beteiligten statistischen Ämtern und die zentrale Verarbeitung und Nutzung dieser Einzelangaben in einem oder mehreren statistischen Ämtern zulässig.
            \item Das Statistische Bundesamt darf an die statistischen Ämter der Länder die ihren jeweiligen Erhebungsbereich betreffenden Einzelangaben für Sonderaufbereitungen auf regionaler Ebene übermitteln. Für die Erstellung der Volkswirtschaftlichen Gesamtrechnungen und sonstiger Gesamtsysteme des Bundes und der Länder dürfen sich das Statistische Bundesamt und die statistischen Ämter der Länder untereinander Einzelangaben aus Bundesstatistiken übermitteln.
            \item Für die Verwendung gegenüber den gesetzgebenden Körperschaften und für Zwecke der Planung, jedoch nicht für die Regelung von Einzelfällen, dürfen den obersten Bundes- oder Landesbehörden vom Statistischen Bundesamt und den statistischen Ämtern der Länder Tabellen mit statistischen Ergebnissen übermittelt werden, auch soweit Tabellenfelder nur einen einzigen Fall ausweisen. Die Übermittlung nach Satz 1 ist nur zulässig, soweit in den eine Bundesstatistik anordnenden Rechtsvorschriften die Übermittlung von Einzelangaben an oberste Bundes- oder Landesbehörden zugelassen ist.
            \item Für ausschließlich statistische Zwecke dürfen vom Statistischen Bundesamt und den statistischen Ämtern der Länder Einzelangaben an die zur Durchführung statistischer Aufgaben zuständigen Stellen der Gemeinden und Gemeindeverbände übermittelt werden, wenn die Übermittlung in einem eine Bundesstatistik anordnenden Gesetz vorgesehen ist sowie Art und Umfang der zu übermittelnden Einzelangaben bestimmt sind. Die Übermittlung ist nur zulässig, wenn durch Landesgesetz eine Trennung dieser Stellen von anderen kommunalen Verwaltungsstellen sichergestellt und das Statistikgeheimnis durch Organisation und Verfahren gewährleistet ist. 
            \item Für die Durchführung wissenschaftlicher Vorhaben dürfen das Statistische Bundesamt und die statistischen Ämter der Länder Hochschulen oder sonstigen Einrichtungen mit der Aufgabe unabhängiger wissenschaftlicher Forschung
                \begin{enumerate}[label=\arabic*.]
                    \item Einzelangaben übermitteln, wenn die Einzelangaben nur mit einem un\-ver\-hält\-nis\-mäßig großen Aufwand an Zeit, Kosten und Arbeitskraft zugeordnet werden können (faktisch anonymisierte Einzelangaben),
                    \item innerhalb speziell abgesicherter Bereiche des Statistischen Bundesamtes und der statistischen Ämter der Länder Zugang zu formal anonymisierten Einzelangaben gewähren, wenn wirksame Vorkehrungen zur Wahrung der Geheimhaltung getroffen werden. Berechtigte können nur Amtsträger oder Amtsträgerinnen, für den öffentlichen Dienst besonders Verpflichtete oder Verpflichtete nach Absatz 7 sein.
                \end{enumerate}
            \item Personen, die Einzelangaben nach Absatz 6 erhalten sollen, sind vor der Über\-mitt\-lung zur Geheimhaltung zu verpflichten, soweit sie nicht Amtsträger oder Amtsträgerinnen oder für den öffentlichen Dienst besonders Verpflichtete sind. § 1 Absatz 2, 3 und 4 Nummer 2 des Verpflichtungsgesetzes vom 2. März 1974 (BGBl. I S. 469, Artikel 42), das durch Gesetz vom 15. August 1974 (BGBl. I S. 1942) geändert worden ist, gilt entsprechend. 
            \item Die aufgrund einer besonderen Rechtsvorschrift oder der Absätze 4, 5 oder 6 übermittelten Einzelangaben dürfen nur für die Zwecke verwendet werden, für die sie übermittelt wurden. In den Fällen des Absatzes 6 Satz 1 Nummer 1 sind sie zu löschen, sobald das wissenschaftliche Vorhaben durchgeführt ist. Bei den Stellen, denen Einzelangaben übermittelt werden, muss durch organisatorische und technische Maßnahmen sichergestellt sein, dass nur Amtsträger, für den öffentlichen Dienst besonders Verpflichtete oder Verpflichtete nach Absatz 7 Satz 1 Empfänger von Einzelangaben sind.
            \item Die Übermittlung aufgrund einer besonderen Rechtsvorschrift oder nach den Ab\-sätzen 4, 5 oder 6 ist nach Inhalt, Stelle, der übermittelt wird, Datum und Zweck der Weitergabe von den statistischen Ämtern aufzuzeichnen. Die Aufzeichnungen sind mindestens fünf Jahre aufzubewahren.
            \item Die Pflicht zur Geheimhaltung nach Absatz 1 besteht auch für die Personen, die Empfänger von Einzelangaben aufgrund einer besonderen Rechtsvorschrift, nach den Absätzen 5, 6 oder von Tabellen nach Absatz 4 sind. Dies gilt nicht für offenkundige Tatsachen bei einer Übermittlung nach Absatz 4.
        \end{enumerate}