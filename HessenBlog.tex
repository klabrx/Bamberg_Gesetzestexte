\chapter{Wer wohnt in Deutschland? – Die Haushaltsstichprobe im Zensus 2021}
\chapterauthor{Von Wanda Otto, Hessisches Statistisches Landesamt}
\minitoc
\begin{myblock}{Kurzfassung}
Mit Stichtag 16. Mai 2021 soll offiziell der Zensus 2021 in Deutschland starten. Er wird größtenteils analog zu dem für den Zensus 2011 neu entwickelten registergestützten Verfahren durchgeführt. Bei dem Verfahren handelt es sich um eine Kombination aus Registerauswertung und der Befragung einer Stichprobe von Haushalten. Ziel ist es zum einen die amtliche Einwohnerzahl zu ermitteln und zum anderen wichtige Strukturmerkmale beispielsweise zur Bildung und Erwerbstätigkeit der Bevölkerung zu erhalten.
\end{myblock}


Die Einwohnerzahlen basieren auf den in Deutschland dezentral vorliegenden Melderegisterdaten. Da es aber bspw. aufgrund von Wanderung zu Zu- und Fortzügen kommt und nicht alle Veränderungen auch gemeldet werden, ist eine regelmäßige bundesweite Erhebung zu einem Stichtag notwendig, um die korrekte Einwohnerzahl ermitteln zu können. Angaben zu Bildung und Erwerbstätigkeit wiederum liegen nicht in Registern vor und müssen daher ebenfalls primärstatistisch erhoben werden.\par

Eine präzise Ermittlung der Einwohnerzahl ist wichtig, weil für die Vergabe bzw. Verteilung finanzieller Mittel seitens der EU und der Bundes- oder Landesregierungen an Gemeinden sowie für den Kommunalen Finanzausgleich die amtliche Einwohnerzahl ausschlaggebend ist. Sie wird mit dem regelmäßigen Zensus ermittelt und dann für die nächsten 10 Jahre im Rahmen der Bevölkerungsfortschreibung fortgeschrieben. Um diese und weitere Angaben über die Bevölkerung zu erhalten, die für Politik und Verwaltung sowohl auf nationaler als auch auf europäischer Ebene zentral sind, um aber andererseits die Bürger nicht über Gebühr zu belasten, werden nicht alle Haushalte sondern nur annähernd 10\% – die sogenannte Haushaltsstichprobe – befragt. Im Rahmen der jährlichen Mikrozensus-Befragung liegen zwar detaillierte Angaben auch zur Bildungs- und Erwerbsbiografie der Befragten vor, allerdings handelt es sich hierbei lediglich um eine 1\%-Stichprobe. Zeitgemäße Erhebungswege, wie die Möglichkeit, Online Angaben zu machen, und weitere Maßnahmen sollen die auskunftspflichtigen Bürgerinnen und Bürger der Haushaltsstichprobe entlasten.\par

\section[Wer ist auskunftspflichtig?]{Wer ist für die Haushaltebefragung im Zensus 2021 auskunftspflichtig?}
Die Stichprobe für die Haushaltebefragung wird auf Basis der im Steuerungsregister konsolidierten Daten gezogen. Das Steuerregister ist das zentrale Anschriften- und Personenregister des Zensus. In Hessen werden insgesamt etwa 750.000 Bürgerinnen und Bürger innerhalb von 12 Wochen befragt.\par Auskunftspflichtig sind alle Personen und Haushalte einer Anschrift, die in die Stichprobe gezogen wurde (siehe Infokasten 1). Es wird das sogenannte ``Wohnhaushaltsprinzip'' zugrunde gelegt, dementsprechend gehören diejenigen Personen zu einem Haushalt, die gemeinsam dort wohnhaft sind, unabhängig davon, ob sie auch gemeinsam wirtschaften. Wer alleine wohnt, bildet einen eigenen Haushalt, auch wenn die Person noch minderjährig ist. Jede an einer Anschrift wohnhafte Person wird erfasst, unabhängig davon, ob es sich bei der Anschrift um den Haupt- oder Nebenwohnsitz der Person handelt.\par
Um die Erhebung so belastungsarm wie möglich zu gestalten, können einzelne Personen auch Auskunft für andere im selben Haushalt lebende Personen machen, bspw. Partner für ihre Partnerinnen und Eltern für ihre Kinder. Diese sogenannten Proxy-Interviews dienen dazu, die Angaben schnell und unkompliziert für alle auskunftspflichtigen Personen eines Haushalts erheben zu können.\par

\begin{myblock}{Infokasten 1: Auskunftspflicht}
Um die Vollzähligkeit und Vollständigkeit der Erhebung zu gewährleisten, sieht der Bundesgesetzgeber eine Auskunftspflicht aller Personen bzw. Haushalte vor, die Teil der Stichprobe sind. Die Auskunftspflicht für den Zensus wird im Zensusgesetz 2021 in § 3 geregelt. Auskunftspflichtig sind demnach alle Personen und Haushalte, die zur Bevölkerung Deutschlands zählen. Dies sind Einwohner der Gemeinden, unabhängig davon, ob sie dort mit Haupt- oder Nebenwohnsitz gemeldet sind, sowie Personen, die im Rahmen ihrer beruflichen Tätigkeit für den deutschen Staat im Ausland tätig sind, bspw. Angehörige der Bundeswehr, der Polizeibehörden des Bundes und der Länder und des Auswärtigen Dienstes sowie ihre dort ansässigen Familien. Für die Ermittlung der Einwohnerzahl der Gemeinden werden dann in einem weiteren Schritt nur diejenigen Personen berücksichtigt, die dort ihren Haupt- bzw. alleinigen Wohnsitz haben. Die Auskunft kann elektronisch, schriftlich, mündlich oder telefonisch erfolgen, soweit diese Möglichkeit zur Antworterteilung von der Erhebungsstelle angeboten wird.
\end{myblock}
\section[Wer erhebt?]{Wie ist die Durchführung des Zensus organisiert?}
Die Erhebungen der Haushaltsstichprobe und auch die Primärerhebungen in den Sonderbereichen (Wohnheime und Gemeinschaftsunterkünfte) sollen, wie schon 2011, von kommunalen Erhebungsstellen organisiert und durchgeführt werden. Daher kommt den Kommunen eine wichtige Rolle in der Realisierung dieses Großprojekts der amtlichen Statistik zu. Die Kommunen in Hessen, genauer gesagt die Kreisverwaltungen, die kreisfreien Städte sowie die Sonderstatusstädte (Städte mit über 50.000 Einwohnern) richten nach jetziger Planung daher ab Sommer 2020 Erhebungsstellen ein. Diese Erhebungsstellen sind insbesondere auf der Grundlage der datenschutzrechtlichen Vorgaben (§ 16 Bundesstatistikgesetz zur Geheimhaltung und europäische Datenschutz-Grundverordnung, DSGVO) von der übrigen Verwaltung personell, organisatorisch und räumlich getrennt und gewährleisten bei der Erfüllung ihrer Aufgaben die statistische Geheimhaltung. Die statistische Geheimhaltung ist ein Grundprinzip der amtlichen Statistik und stellt sicher, dass Erkenntnisse aus der Erhebungstätigkeit nicht für andere Zwecke (bspw. andere Verwaltungsaufgaben) verwendet werden. Ebenso werden alle von den Erhebungsstellen zur Wahrnehmung ihrer Aufgaben eingesetzten Interviewerinnen und Interviewer (Erhebungsbeauftragte) zur Geheimhaltung verpflichtet. Allein in Hessen werden voraussichtlich ca. 7.500 Interviewende angeworben, verpflichtet und von den Erhebungsstellen geschult. Alle Erhebungsunterlagen werden bis zur Beendigung der Feldphase in den Erhebungsstellen auf Vollzähligkeit und auf Vollständigkeit hin überprüft und dann zur weiteren Bearbeitung an die Beleglesung zur Digitalisierung der Erhebungsunterlagen bzw. an die Statistischen Ämter weitergeleitet.

\section[Wie wird erhoben?]{Wie laufen die Haushaltebefragungen ab?}
Um die Befragten zu entlasten wird die Haushaltebefragung im Zensus 2021 drei-stufig erfolgen. Zunächst einmal werden die Stichprobenanschriften durch die Interviewerinnen und die Interviewer begangen, um zu ermitteln, wer an der Anschrift wohnt und um erste Informationen zum Zensus sowie eine Terminankündigung für ein persönliches Interview zu hinterlassen bzw. in den Briefkasten einzulegen. Im anschließenden persönlichen Kontakt werden dann die Merkmale zu den im Haushalt lebenden Personen wie bspw. Vor- und Nachname aufgenommen und entsprechend der Anzahl der Haushaltsmitglieder weitere Informationen und eine individuelle Zugangskennung für die Online-Befragung ausgegeben. So haben die auskunftspflichtigen Bürgerinnen und Bürger im dritten Schritt die Gelegenheit, ihre Angaben bspw. zu Bildungsabschlüssen und ihrer Erwerbstätigkeit zeit- und ortsunabhängig, einfach, schnell und unkompliziert online einzugeben.\par

Wenn eine (weitere) Kontaktaufnahme mit den Auskunftspflichtigen durch die Erhebungsbeauftragten nicht zustande kommt, nimmt die zuständige kommunale Erhebungsstelle schriftlich mit den entsprechenden auskunftspflichtigen Haushalten Kontakt auf, sodass diese ihre Angaben online machen. Ist dies nicht möglich oder nicht gewünscht, gibt es für die Auskunftspflichtigen die Möglichkeit ihre Angaben schriftlich auf einem Papier-Fragebogen oder ggfs. auch telefonisch zu machen.\par

Zwar gab es bereits im Zensus 2011 die Möglichkeit für die Auskunftspflichtigen Ihre Angaben online zu machen, allerdings wurde diese mit einem Anteil von 7 Prozent nur relativ selten genutzt. Dafür sind überwiegend zwei Gründe zu nennen: Zum einen wurden primär Erhebungsbeauftragte eingesetzt, die alle Angaben direkt vor Ort erhoben haben. Zum anderen waren die technischen Möglichkeiten eingeschränkter. Neu ist dieses Mal daher die sogenannte Online-First-Strategie (siehe Infokasten 2).\par

\begin{myblock}{Infokasten 2: Online-First-Strategie}
Die Online-First-Strategie des Zensus 2021 baut auf die bei der Bevölkerung vorhandene Infrastruktur zur Internetnutzung und verspricht dabei den Befragten dieselben Vorteile, die zu Anschaffung und Unterhalt von Smartphone, Tablet, Laptop oder Desktop-Computer geführt haben: Das zeitsouveräne, leichtere, schnellere und kostengünstigere Erledigen von Aufgaben des täglichen Lebens. Damit einher geht ein Zensus, der genauere Ergebnisse in kürzerer Zeit liefert und dabei Ressourcen schont. Alle Erhebungsprozesse werden zuerst auf dieses Ziel hin konzipiert. Dies schließt die Bereitstellung von Papierfragebogen bei Bedarf sowie, wo erforderlich, Interviewer-gestützte Haushaltebefragungen mit ein. Darüber hinaus ist geplant, dass die Angaben auch telefonisch übermittelt werden können.
\end{myblock}

Der öffentliche Dienst und vor allem auch die amtliche Statistik nutzt für den Zensus 2021 einen modernen und zeitgemäßen Erhebungsweg wie die Online-Selbstauskunft. Die Online-Befragung kann mit verschiedensten internetfähigen Geräten neben Desktop-PC, auch mit einem Tablet oder Smartphone durchgeführt werden. Das Layout der Fragebogen passt sich dabei entsprechend des verwendeten Endgerätes der Bildschirmgröße selbstständig an. Zusätzlich wird darauf geachtet, die Befragung nutzerfreundlich zu gestalten. Das heißt, dass sowohl der Login-Bereich zur Online-Befragung leicht erreichbar ist z. B. mittels eines QR-Codes, als auch übersichtlich gestaltet ist, um sich rasch einloggen zu können.\par

Die Online-First-Strategie bietet sowohl für die Befragten als auch für die statistischen Ämter Vorteile. Da bei Online-Meldungen die Filterführung so programmiert ist, dass die Befragten lediglich die für sie relevanten Fragen angezeigt bekommen und zudem Fehler unmittelbar während der Erfassung angezeigt werden können, ist die Beantwortung einfacher und zugleich die Datenqualität höher. Mittels elektronischem Eingang der Meldungen können die Angaben unmittelbar auf Plausibilität geprüft werden und die Daten liegen direkt elektronisch vor. Folglich müssen nicht erst noch die auf Papier ausgefüllten Bögen zu einem Beleglesezentrum transportiert, dort verarbeitet und anschließend plausibilisiert werden. Dies spart auch Zeit – den Befragten und den Statistischen Ämtern. Somit können die Ergebnisse des Zensus 2021 bereits 18 Monate nach Stichtag veröffentlicht werden.\par

\section[Was wird gefragt?]{Was wird gefragt? Welche Merkmale werden erfasst?}
Wie bereits im vorherigen Abschnitt beschrieben erfolgt die Befragung mehrstufig. Die Erhebungsbeauftragten suchen zunächst den persönlichen Kontakt zu den Haushalten, um erste für die Existenzfeststellung notwendige Merkmale zu erheben. Um die Bürgerinnen und Bürger der Haushaltsstichprobe soweit möglich zu entlasten, wird sich dabei auf ein Minimum an Fragen fokussiert (primär auf die Merkmale 1 bis 9). Die Beantwortung aller weiteren Merkmale (ab Merkmal 10) können die Auskunftspflichtigen dann im Anschluss an den Besuch der Interviewerin oder des Interviewers selbstständig vornehmen.\par

Folgende Erhebungsmerkmale werden laut § 13 Zensusgesetz 2021-Entwurf erfragt:
\setlength{\parskip}{1pt}
\begin{enumerate}[label=\arabic*.]
    \item Wohnungsstatus,
    \item Wohnungsstatus,
    \item Geschlecht,
    \item Staatsangehörigkeiten,
    \item Monat und Jahr der Geburt,
    \item Familienstand,
    \item nichteheliche Lebensgemeinschaften,
    \item Jahr der Ankunft in Deutschland (für Personen, die nach dem 31. Dezember 1955 nach Deutschland zugezogen sind),
    \item Anzahl der Personen im Haushalt,
    \item Geburtsstaat,
    \item Erwerbsstatus in der Woche des Zensusstichtags,
    \item Hauptstatus in der Woche des Zensusstichtags,
    \item Stellung im Beruf,
    \item ausgeübter Beruf,
    \item Wirtschaftszweig des Betriebs,
    \item Anschrift des Betriebs (nur Postleitzahl und Gemeinde),
    \item höchster allgemeiner Schulabschluss,
    \item höchster beruflicher Bildungsabschluss,
    \item aktueller Schulbesuch.
\end{enumerate}
\setlength{\parskip}{\baselineskip}
Darüber hinaus werden noch folgende Hilfsmerkmale benötigt:
\setlength{\parskip}{1pt}
\begin{enumerate}[label=\arabic*.]
    \item Familienname und Vornamen,
    \item Anschrift der Wohnung und Lage der Wohnung im Gebäude,
    \item Tag der Geburt ohne Monats- und Jahresangabe,
    \item Kontaktdaten der Auskunftspflichtigen oder einer anderen für Rückfragen zur Verfügung stehenden Person.
\end{enumerate}
\setlength{\parskip}{\baselineskip}
Die Datenschutzbestimmungen werden sowohl im Zensusgesetz 2021 (ZensG § 27 ff) als auch nach der europäischen Datenschutz-Grundverordnung (DSGVO) geregelt. Während im ZensG überwiegend die Verantwortlichkeit und die Verarbeitungsrechte der zu erhebenden Daten geregelt werden, befasst sich die DSGVO verstärkt mit den Rechten der betroffenen Personen, bspw. Artikel 13 ``Informationspflicht bei Erhebung von personenbezogenen Daten''.\par

\section[Wie wird berechnet?]{Wie fließen die Befragungsergebnisse in die Ermittlung der amtlichen Einwohnerzahl ein?}
Die Melderegisterdaten spielen eine große Rolle bei der Ermittlung der amtlichen Einwohnerzahl. Sie dienen auch als Ausgangsbasis für die Stichprobenziehung zur Haushaltebefragung. Zur Ermittlung der Einwohnerzahl werden vom Melderegisterbestand zum Zensusstichtag zunächst die Nebenwohnsitze und die freiwilligen Meldungen abgezogen und Zuzüge sowie Geburten hinzugezählt. Anschließend werden für eine erste Korrektur im Rahmen der sogenannten Mehrfachfallprüfung Meldungen von Personen, die bspw. an zwei Adressen einen Hauptwohnsitz angemeldet haben, subtrahiert und Personen, die bspw. bislang ausschließlich mit einem Nebenwohnsitz gemeldet waren, entsprechend addiert.

\setlength{\parskip}{1pt}
\begin{description}
    \item[]Melderegisterbestand (zum Zensusstichtag)
    \item[--]Nebenwohnsitze
    \item[--]Freiwillige Meldungen (Angehörige ausländischer Streitkräfte, Diplomaten)
    \item[+]Zuzüge und Geburten
    \item[=]\textbf{Konsolidierter Melderegisterbestand}
    \item[--]Abgänge durch Mehrfachfallprüfung
    \item[+]Zugänge durch Mehrfachfallprüfung
    \item[=]\textbf{1. korrigierter Melderegisterbestand}
    \item[--]Abgänge Sonderbereichserhebung
    \item[+]Zugänge Sonderbereichserhebung
    \item[=]\textbf{2. korrigierter Melderegisterbestand}
    \item[--]Karteileichen (hochgerechnet) aus Haushaltsstichprobe
    \item[+]Fehlbestände (hochgerechnet) aus Haushaltsstichprobe
    \item[=]\textbf{Einwohnerzahl}
\end{description}
\setlength{\parskip}{\baselineskip}

Im zweiten Korrekturschritt wird der Melderegisterbestand dann um Meldungen von Personen aktualisiert, die in Sonderbereichen wie Pflegeheimen gemeldet sind. Mittels der Erhebungen im Rahmen der Haushaltsstichprobe können die an einer Anschrift tatsächlich wohnhaften Personen – durch die kommunalen Erhebungsstellen bestätigt sowie an der Anschrift nicht (mehr) wohnhafte Personen (``Karteileichen'') und dort noch nicht gemeldete Personen (``Fehlbestände'') ermittelt werden. Die so festgestellten Karteileichen und Fehlbestände auf Basis der Haushaltsstichprobe werden entsprechend auf die Gesamtzahl der Anschriften der Gemeinde hochgerechnet und dem 2. korrigierten Melderegisterbestand entsprechend ab- bzw. aufgeschlagen. Der Prozess vom Stichtag bis zur Ermittlung der amtlichen Einwohnerzahl soll innerhalb von 18 Monaten mit der Veröffentlichung der Ergebnisse abgeschlossen sein.

\section{Fazit}
Die mithilfe des Zensus, also der Volkszählung auf Basis einer rund 10\%-Stichprobe ermittelten amtlichen Einwohnerzahl dient unter anderem der statistischen Korrektur der Bevölkerungsfortschreibung bis zum nächsten Zensus, der die Fortschreibung dann wieder bereinigt. Darüber hinaus dient die amtliche Einwohnerzahl als Bemessungsgrundlage für die Finanzausgleiche auf Ebene der Länder und der Kommunen. Des Weiteren wird sie als Richtgröße für die Einteilung der Bundestagswahlkreise, für die Berechnung der Zahl der Stimmen der Länder im Bundesrat und für die Berechnung der Zahl der Sitze in den Gemeinderäten genutzt.\par

Auch für die Statistik selbst kommt den Zensus-Daten eine zentrale Stellung zu, da sie als Auswahlgrundlage und Hochrechnungsrahmen für weitere amtliche und nichtamtliche Stichprobenerhebungen dienen. Dazu zählen bspw. der Mikrozensus, die größte amtliche Haushaltsbefragung in Deutschland, oder die Allgemeine Bevölkerungsumfrage (ALLBUS) sowie zahlreiche weitere Erhebungen in Wissenschaft und Wirtschaft. Die ebenfalls im Rahmen des Zensus 2021 zu erhebenden strukturellen Zusatzmerkmale sollen laut Bundesinnenministerium für politische Planungen und Entscheidungen genutzt werden. Hier geht es beispielsweise etwa um die Frage, wo Schulen, Studienplätze und Altersheime benötigt werden.
