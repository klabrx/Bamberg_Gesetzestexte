\documentclass[A4, 12pt]{scrbook}

\usepackage[utf8]{inputenc}
\usepackage{hyperref}
\usepackage[ngerman]{babel}
\usepackage{qrcode}

\usepackage{enumitem}
\setcounter{secnumdepth}{0}

\usepackage{fancyhdr}

\pagestyle{fancy}
\renewcommand{\chaptermark}[1]{\markboth{#1}{}}
% \renewcommand{\sectionmark}[1]{\markboth{#1}{}}

\def\@chapter[#1]#2{\ifnum \c@secnumdepth >\m@ne
                        \if@mainmatter
                          \refstepcounter{chapter}%
                          \typeout{\@chapapp\space\thechapter.}%
                          \addcontentsline{toc}{chapter}%
                                    {\protect\numberline{\thechapter}#2}%
                        \else
                          \addcontentsline{toc}{chapter}{#2}%
                        \fi
                     \else
                       \addcontentsline{toc}{chapter}{#2}%
                     \fi
                     \chaptermark{#1}%
                     \addtocontents{lof}{\protect\addvspace{10\p@}}%
                     \addtocontents{lot}{\protect\addvspace{10\p@}}%
                     \if@twocolumn
                       \@topnewpage[\@makechapterhead{#2}]%
                     \else
                       \@makechapterhead{#2}%
                       \@afterheading
                     \fi}



\title{Einführung in die Kommunalstatistik}
\subtitle{Arbeitsunterlagen und Gesetzestexte}
\author{Zusammenstellung: Klaus Brückner, Statistikstelle der Stadt Passau}
\date{Herbst 2020}

\begin{document}
\maketitle
\tableofcontents

\chapter{EU-DSGVO}
\minitoc
    \section{Art. 1: Gegenstand und Ziele}
        \begin{enumerate}[label=(\arabic*)]
            \item Diese Verordnung enthält Vorschriften zum Schutz natürlicher Personen bei der Verarbeitung personenbezogener Daten und zum freien Verkehr solcher Daten.
            \item Diese Verordnung schützt die Grundrechte und Grundfreiheiten natürlicher Personen und insbesondere deren Recht auf Schutz personenbezogener Daten.
            \item Der freie Verkehr personenbezogener Daten in der Union darf aus Gründen des Schutzes natürlicher Personen bei der Verarbeitung personenbezogener Daten weder eingeschränkt noch verboten werden. 
        \end{enumerate}


    \section{Art. 4: Begriffsbestimmungen}
    Im Sinne dieser Verordnung bezeichnet der Ausdruck:
        \begin{enumerate}[label=\arabic*.]
            \item ``personenbezogene Daten'' alle Informationen, die sich auf eine identifizierte oder identifizierbare natürliche Person (im Folgenden ``betroffene Person'') beziehen; als identifizierbar wird eine natürliche Person angesehen, die direkt oder indirekt, insbesondere mittels Zuordnung zu einer Kennung wie einem Namen, zu einer Kennnummer, zu Standortdaten, zu einer Online-Kennung oder zu einem oder mehreren besonderen Merkmalen, die Ausdruck der physischen, physiologischen, genetischen, psychischen, wirtschaftlichen, kulturellen oder sozialen Identität dieser natürlichen Person sind, identifiziert werden kann; 
            \item ``Verarbeitung'' jeden mit oder ohne Hilfe automatisierter Verfahren ausgeführten Vorgang oder jede solche Vorgangsreihe im Zusammenhang mit personenbezogenen Daten wie das Erheben, das Erfassen, die Organisation, das Ordnen, die Speicherung, die Anpassung oder Veränderung, das Auslesen, das Abfragen, die Verwendung, die Offenlegung durch Übermittlung, Verbreitung oder eine andere Form der Bereitstellung, den Abgleich oder die Verknüpfung, die Einschränkung, das Löschen oder die Vernichtung;
            \item ``Einschränkung der Verarbeitung'' die Markierung gespeicherter personenbezogener Daten mit dem Ziel, ihre künftige Verarbeitung einzuschränken;
            \item ``Profiling'' jede Art der automatisierten Verarbeitung personenbezogener Daten, die darin besteht, dass diese personenbezogenen Daten verwendet werden, um bestimmte persönliche Aspekte, die sich auf eine natürliche Person beziehen, zu bewerten, insbesondere um Aspekte bezüglich Arbeitsleistung, wirtschaftliche Lage, Gesundheit, persönliche Vorlieben, Interessen, Zuverlässigkeit, Verhalten, Aufenthaltsort oder Ortswechsel dieser natürlichen Person zu analysieren oder vorherzusagen;
            \item ``Pseudonymisierung'' die Verarbeitung personenbezogener Daten in einer Weise, dass die personenbezogenen Daten ohne Hinzuziehung zusätzlicher Informationen nicht mehr einer spezifischen betroffenen Person zugeordnet werden können, sofern diese zusätzlichen Informationen gesondert aufbewahrt werden und technischen und organisatorischen Maßnahmen unterliegen, die gewährleisten, dass die personenbezogenen Daten nicht einer identifizierten oder identifizierbaren natürlichen Person zugewiesen werden; 
            \item ``Dateisystem'' jede strukturierte Sammlung personenbezogener Daten, die nach bestimmten Kriterien zugänglich sind, unabhängig davon, ob diese Sammlung zentral, dezentral oder nach funktionalen oder geografischen Gesichtspunkten geordnet geführt wird;
            \item ``Verantwortlicher'' die natürliche oder juristische Person, Behörde, Einrichtung oder andere Stelle, die allein oder gemeinsam mit anderen über die Zwecke und Mittel der Verarbeitung von personenbezogenen Daten entscheidet; sind die Zwecke und Mittel dieser Verarbeitung durch das Unionsrecht oder das Recht der Mitgliedstaaten vorgegeben, so kann der Verantwortliche beziehungsweise können die bestimmten Kriterien seiner Benennung nach dem Unionsrecht oder dem Recht der Mitgliedstaaten vorgesehen werden;
            \item ``Auftragsverarbeiter'' eine natürliche oder juristische Person, Behörde, Einrichtung oder andere Stelle, die personenbezogene Daten im Auftrag des Verantwortlichen verarbeitet;
            \item ``Empfänger'' eine natürliche oder juristische Person, Behörde, Einrichtung oder andere Stelle, der personenbezogene Daten offengelegt werden, unabhängig davon, ob es sich bei ihr um einen Dritten handelt oder nicht. Behörden, die im Rahmen eines bestimmten Untersuchungsauftrags nach dem Unionsrecht oder dem Recht der Mitgliedstaaten möglicherweise personenbezogene Daten erhalten, gelten jedoch nicht als Empfänger; die Verarbeitung dieser Daten durch die genannten Behörden erfolgt im Einklang mit den geltenden Datenschutzvorschriften gemäß den Zwecken der Verarbeitung;
            \item  ``Dritter'' eine natürliche oder juristische Person, Behörde, Einrichtung oder andere Stelle, außer der betroffenen Person, dem Verantwortlichen, dem Auftragsverarbeiter und den Personen, die unter der unmittelbaren Verantwortung des Verantwortlichen oder des Auftragsverarbeiters befugt sind, die personenbezogenen Daten zu verarbeiten;
            \item ``Einwilligung'' der betroffenen Person jede freiwillig für den bestimmten Fall, in informierter Weise und unmissverständlich abgegebene Willensbekundung in Form einer Erklärung oder einer sonstigen eindeutigen bestätigenden Handlung, mit der die betroffene Person zu verstehen gibt, dass sie mit der Verarbeitung der sie betreffenden personenbezogenen Daten einverstanden ist;
            \item ``Verletzung des Schutzes personenbezogener Daten'' eine Verletzung der Sicherheit, die, ob unbeabsichtigt oder unrechtmäßig, zur Vernichtung, zum Verlust, zur Veränderung, oder zur unbefugten Offenlegung von beziehungsweise zum unbefugten Zugang zu personenbezogenen Daten führt, die übermittelt, gespeichert oder auf sonstige Weise verarbeitet wurden;
            \item ``genetische Daten'' personenbezogene Daten zu den ererbten oder erworbenen genetischen Eigenschaften einer natürlichen Person, die eindeutige Informationen über die Physiologie oder die Gesundheit dieser natürlichen Person liefern und insbesondere aus der Analyse einer biologischen Probe der betreffenden natürlichen Person gewonnen wurden;
            \item ``biometrische Daten'' mit speziellen technischen Verfahren gewonnene personenbezogene Daten zu den physischen, physiologischen oder verhaltenstypischen Merkmalen einer natürlichen Person, die die eindeutige Identifizierung dieser natürlichen Person ermöglichen oder bestätigen, wie Gesichtsbilder oder daktyloskopische Daten;
            \item ``Gesundheitsdaten'' personenbezogene Daten, die sich auf die körperliche oder geistige Gesundheit einer natürlichen Person, einschließlich der Erbringung von Gesundheitsdienstleistungen, beziehen und aus denen Informationen über deren Gesundheitszustand hervorgehen;
            \item ``Hauptniederlassung''
                \begin{enumerate}[label=\alph*)]
                    \item im Falle eines Verantwortlichen mit Niederlassungen in mehr als einem Mitgliedstaat den Ort seiner Hauptverwaltung in der Union, es sei denn, die Entscheidungen hinsichtlich der Zwecke und Mittel der Verarbeitung personenbezogener Daten werden in einer anderen Niederlassung des Verantwortlichen in der Union getroffen und diese Niederlassung ist befugt, diese Entscheidungen umsetzen zu lassen; in diesem Fall gilt die Niederlassung, die derartige Entscheidungen trifft, als Hauptniederlassung;
                    \item im Falle eines Auftragsverarbeiters mit Niederlassungen in mehr als einem Mitgliedstaat den Ort seiner Hauptverwaltung in der Union oder, sofern der Auftragsverarbeiter keine Hauptverwaltung in der Union hat, die Niederlassung des Auftragsverarbeiters in der Union, in der die Ver\-ar\-bei\-tungs\-tä\-tig\-kei\-ten im Rahmen der Tätigkeiten einer Niederlassung eines Auftragsverarbeiters hauptsächlich stattfinden, soweit der Auftragsverarbeiter spezifischen Pflichten aus dieser Verordnung unterliegt;
                \end{enumerate} 
            \item ``Vertreter'' eine in der Union niedergelassene natürliche oder juristische Person, die von dem Verantwortlichen oder Auftragsverarbeiter schriftlich gemäß Artikel 27 bestellt wurde und den Verantwortlichen oder Auftragsverarbeiter in Bezug auf die ihnen jeweils nach dieser Verordnung obliegenden Pflichten vertritt;
            \item ``Unternehmen'' eine natürliche und juristische Person, die eine wirtschaftliche Tätigkeit ausübt, unabhängig von ihrer Rechtsform, einschließlich Personengesellschaften oder Vereinigungen, die regelmäßig einer wirtschaftlichen Tätigkeit nachgehen;
            \item ``Unternehmensgruppe'' eine Gruppe, die aus einem herrschenden Unternehmen und den von diesem abhängigen Unternehmen besteht; 
            \item ``verbindliche interne Datenschutzvorschriften'' Maßnahmen zum Schutz personenbezogener Daten, zu deren Einhaltung sich ein im Hoheitsgebiet eines Mitgliedstaats niedergelassener Verantwortlicher oder Auftragsverarbeiter verpflichtet im Hinblick auf Datenübermittlungen oder eine Kategorie von Datenübermittlungen personenbezogener Daten an einen Verantwortlichen oder Auftragsverarbeiter derselben Unternehmensgruppe oder derselben Gruppe von Unternehmen, die eine gemeinsame Wirtschaftstätigkeit ausüben, in einem oder mehreren Drittländern;
            \item ``Aufsichtsbehörde'' eine von einem Mitgliedstaat gemäß Artikel 51 eingerichtete unabhängige staatliche Stelle; 
            \item ``betroffene Aufsichtsbehörde'' eine Aufsichtsbehörde, die von der Verarbeitung personenbezogener Daten betroffen ist, weil
                \begin{enumerate}[label=\alph*)]
                    \item der Verantwortliche oder der Auftragsverarbeiter im Hoheitsgebiet des Mitgliedstaats dieser Aufsichtsbehörde niedergelassen ist,
                    \item diese Verarbeitung erhebliche Auswirkungen auf betroffene Personen mit Wohnsitz im Mitgliedstaat dieser Aufsichtsbehörde hat oder haben kann oder
                    \item eine Beschwerde bei dieser Aufsichtsbehörde eingereicht wurde;
                \end{enumerate}
            \item ``grenzüberschreitende Verarbeitung'' entweder
                \begin{enumerate}[label=\alph*)]
                    \item eine Verarbeitung personenbezogener Daten, die im Rahmen der Tätigkeiten von Niederlassungen eines Verantwortlichen oder eines Auftragsverarbeiters in der Union in mehr als einem Mitgliedstaat erfolgt, wenn der Verantwortliche oder Auftragsverarbeiter in mehr als einem Mitgliedstaat niedergelassen ist, oder
                    \item eine Verarbeitung personenbezogener Daten, die im Rahmen der Tätigkeiten einer einzelnen Niederlassung eines Verantwortlichen oder eines Auftragsverarbeiters in der Union erfolgt, die jedoch erhebliche Auswirkungen auf betroffene Personen in mehr als einem Mitgliedstaat hat oder haben kann;
                \end{enumerate}              
            \item ``maßgeblicher und begründeter Einspruch'' einen Einspruch gegen einen Beschlussentwurf im Hinblick darauf, ob ein Verstoß gegen diese Verordnung vorliegt oder ob beabsichtigte Maßnahmen gegen den Verantwortlichen oder den Auftragsverarbeiter im Einklang mit dieser Verordnung steht, wobei aus diesem Einspruch die Tragweite der Risiken klar hervorgeht, die von dem Beschlussentwurf in Bezug auf die Grundrechte und Grundfreiheiten der betroffenen Personen und gegebenenfalls den freien Verkehr personenbezogener Daten in der Union ausgehen;
            \item ``Dienst der Informationsgesellschaft'' eine Dienstleistung im Sinne des Artikels 1 Nummer 1 Buchstabe b der Richtlinie (EU) 2015/1535 des Europäischen Parlaments und des Rates 
            \item ``internationale Organisation'' eine völkerrechtliche Organisation und ihre nachgeordneten Stellen oder jede sonstige Einrichtung, die durch eine zwischen zwei oder mehr Ländern geschlossene Übereinkunft oder auf der Grundlage einer solchen Übereinkunft geschaffen wurde. 
        \end{enumerate}
    \section{Art. 5: Grundsätze für die Verarbeitung personenbezogener Daten}
        \begin{enumerate}[label=(\arabic*)]
            \item Personenbezogene Daten müssen
                \begin{enumerate}[label=\alph*)]
                    \item auf rechtmäßige Weise, nach Treu und Glauben und in einer für die betroffene Person nachvollziehbaren Weise verarbeitet werden (``Rechtmäßigkeit, Verarbeitung nach Treu und Glauben, Transparenz'');
                    \item für festgelegte, eindeutige und legitime Zwecke erhoben werden und dürfen nicht in einer mit diesen Zwecken nicht zu vereinbarenden Weise weiterverarbeitet werden; eine Weiterverarbeitung für im öffentlichen Interesse liegende Archivzwecke, für wissenschaftliche oder historische Forschungszwecke oder für statistische Zwecke gilt gemäß Artikel 89 Absatz 1 nicht als unvereinbar mit den ursprünglichen Zwecken (``Zweckbindung'');
                    \item dem Zweck angemessen und erheblich sowie auf das für die Zwecke der Verarbeitung notwendige Maß beschränkt sein (``Datenminimierung''); 
                    \item sachlich richtig und erforderlichenfalls auf dem neuesten Stand sein; es sind alle angemessenen Maßnahmen zu treffen, damit personenbezogene Daten, die im Hinblick auf die Zwecke ihrer Verarbeitung unrichtig sind, unverzüglich gelöscht oder berichtigt werden (``Richtigkeit'');
                    \item in einer Form gespeichert werden, die die Identifizierung der betroffenen Personen nur so lange ermöglicht, wie es für die Zwecke, für die sie verarbeitet werden, erforderlich ist; personenbezogene Daten dürfen länger gespeichert werden, soweit die personenbezogenen Daten vorbehaltlich der Durchführung geeigneter technischer und organisatorischer Maßnahmen, die von dieser Verordnung zum Schutz der Rechte und Freiheiten der betroffenen Person gefordert werden, ausschließlich für im öffentlichen Interesse liegende Archivzwecke oder für wissenschaftliche und historische Forschungszwecke oder für statistische Zwecke gemäß Artikel 89 Absatz 1 verarbeitet werden (``Speicherbegrenzung'');
                    \item in einer Weise verarbeitet werden, die eine angemessene Sicherheit der personenbezogenen Daten gewährleistet, einschließlich Schutz vor unbefugter oder unrechtmäßiger Verarbeitung und vor unbeabsichtigtem Verlust, unbeabsichtigter Zerstörung oder unbeabsichtigter Schädigung durch geeignete technische und organisatorische Maßnahmen (``Integrität und Vertraulichkeit'');
                \end{enumerate}
            \item Der Verantwortliche ist für die Einhaltung des Absatzes 1 verantwortlich und muss dessen Einhaltung nachweisen können (``Rechenschaftspflicht'')
        \end{enumerate}
    \section[Art 9: Verarbeitung besonderer Daten]{Art. 9: Verarbeitung besonderer Kategorien personenbezogener Daten}
        \begin{enumerate}
            \item Die Verarbeitung personenbezogener Daten, aus denen die rassische und ethnische Herkunft, politische Meinungen, religiöse oder weltanschauliche Überzeugungen oder die Gewerkschaftszugehörigkeit hervorgehen, sowie die Verarbeitung von genetischen Daten, biometrischen Daten zur eindeutigen Identifizierung einer na\-tür\-lichen Person, Gesundheitsdaten oder Daten zum Sexualleben oder der sexuellen Orientierung einer natürlichen Person ist untersagt.
            \item Absatz 1 gilt nicht in folgenden Fällen:
                \begin{enumerate}[label=\alph*)]
                    \item Die betroffene Person hat in die Verarbeitung der genannten personenbezogenen Daten für einen oder mehrere festgelegte Zwecke ausdrücklich eingewilligt, es sei denn, nach Unionsrecht oder dem Recht der Mitgliedstaaten kann das Verbot nach Absatz 1 durch die Einwilligung der betroffenen Person nicht aufgehoben werden,
                    \item die Verarbeitung ist erforderlich, damit der Verantwortliche oder die betroffene Person die ihm bzw. ihr aus dem Arbeitsrecht und dem Recht der sozialen Sicherheit und des Sozialschutzes erwachsenden Rechte ausüben und seinen bzw. ihren diesbezüglichen Pflichten nachkommen kann, soweit dies nach Unionsrecht oder dem Recht der Mitgliedstaaten oder einer Kollektivvereinbarung nach dem Recht der Mitgliedstaaten, das geeignete Garantien für die Grundrechte und die Interessen der betroffenen Person vorsieht, zulässig ist,
                    \item die Verarbeitung ist zum Schutz lebenswichtiger Interessen der betroffenen Person oder einer anderen natürlichen Person erforderlich und die betroffene Person ist aus körperlichen oder rechtlichen Gründen außerstande, ihre Einwilligung zu geben, 
                    \item die Verarbeitung erfolgt auf der Grundlage geeigneter Garantien durch eine politisch, weltanschaulich, religiös oder gewerkschaftlich ausgerichtete Stiftung, Vereinigung oder sonstige Organisation ohne Gewinnerzielungsabsicht im Rahmen ihrer rechtmäßigen Tätigkeiten und unter der Voraussetzung, dass sich die Verarbeitung ausschließlich auf die Mitglieder oder ehemalige Mitglieder der Organisation oder auf Personen, die im Zusammenhang mit deren Tätigkeitszweck regelmäßige Kontakte mit ihr unterhalten, bezieht und die personenbezogenen Daten nicht ohne Einwilligung der betroffenen Personen nach außen offengelegt werden,
                    \item die Verarbeitung bezieht sich auf personenbezogene Daten, die die betroffene Person offensichtlich öffentlich gemacht hat,
                    \item die Verarbeitung ist zur Geltendmachung, Ausübung oder Verteidigung von Rechtsansprüchen oder bei Handlungen der Gerichte im Rahmen ihrer justiziellen Tätigkeit erforderlich, 
                    \item die Verarbeitung ist auf der Grundlage des Unionsrechts oder des Rechts eines Mitgliedstaats, das in angemessenem Verhältnis zu dem verfolgten Ziel steht, den Wesensgehalt des Rechts auf Datenschutz wahrt und angemessene und spezifische Maßnahmen zur Wahrung der Grundrechte und Interessen der betroffenen Person vorsieht, aus Gründen eines erheblichen öffentlichen Interesses erforderlich,
                    \item die Verarbeitung ist für Zwecke der Gesundheitsvorsorge oder der Arbeitsmedizin, für die Beurteilung der Arbeitsfähigkeit des Beschäftigten, für die medizinische Diagnostik, die Versorgung oder Behandlung im Gesundheitsoder Sozialbereich oder für die Verwaltung von Systemen und Diensten im Gesundheits- oder Sozialbereich auf der Grundlage des Unionsrechts oder des Rechts eines Mitgliedstaats oder aufgrund eines Vertrags mit einem Angehörigen eines Gesundheitsberufs und vorbehaltlich der in Absatz 3 genannten Bedingungen und Garantien erforderlich,
                    \item die Verarbeitung ist aus Gründen des öffentlichen Interesses im Bereich der öffentlichen Gesundheit, wie dem Schutz vor schwerwiegenden grenz\-über\-schrei\-ten\-den Gesundheitsgefahren oder zur Gewährleistung hoher Qualitäts- und Sicherheitsstandards bei der Gesundheitsversorgung und bei Arzneimitteln und Medizinprodukten, auf der Grundlage des Unionsrechts oder des Rechts eines Mitgliedstaats, das angemessene und spezifische Maßnahmen zur Wahrung der Rechte und Freiheiten der betroffenen Person, insbesondere des Berufsgeheimnisses, vorsieht, erforderlich, oder
                    \item die Verarbeitung ist auf der Grundlage des Unionsrechts oder des Rechts eines Mitgliedstaats, das in angemessenem Verhältnis zu dem verfolgten Ziel steht, den Wesensgehalt des Rechts auf Datenschutz wahrt und angemessene und spezifische Maßnahmen zur Wahrung der Grundrechte und Interessen der betroffenen Person vorsieht, für im öffentlichen Interesse liegende Archivzwecke, für wissenschaftliche oder historische Forschungszwecke oder für statistische Zwecke gemäß Artikel 89 Absatz 1 erforderlich.
                \end{enumerate}
            \item Die in Absatz 1 genannten personenbezogenen Daten dürfen zu den in Absatz 2 Buchstabe h genannten Zwecken verarbeitet werden, wenn diese Daten von Fachpersonal oder unter dessen Verantwortung verarbeitet werden und dieses Fachpersonal nach dem Unionsrecht oder dem Recht eines Mitgliedstaats oder den Vorschriften nationaler zuständiger Stellen dem Berufsgeheimnis unterliegt, oder wenn die Verarbeitung durch eine andere Person erfolgt, die ebenfalls nach dem Unionsrecht oder dem Recht eines Mitgliedstaats oder den Vorschriften nationaler zuständiger Stellen einer Geheimhaltungspflicht unterliegt.
            \item Die Mitgliedstaaten können zusätzliche Bedingungen, einschließlich Be\-schrän\-kun\-gen, einführen oder aufrechterhalten, soweit die Verarbeitung von genetischen, biometrischen oder Gesundheitsdaten betroffen ist. 
        \end{enumerate}
    \section[Art. 14: Informationspflicht]{Art. 14: Informationspflicht, wenn die personenbezogenen Daten nicht bei der betroffenen Person
erhoben wurden}
    \begin{enumerate}[label=(\arabic*)]
        \item Werden personenbezogene Daten nicht bei der betroffenen Person erhoben, so teilt der Verantwortliche der betroffenen Person Folgendes mit:
            \begin{enumerate}[label=\alph*)]
                \item den Namen und die Kontaktdaten des Verantwortlichen sowie gegebenenfalls seines Vertreters;
                \item zusätzlich die Kontaktdaten des Datenschutzbeauftragten;
                \item die Zwecke, für die die personenbezogenen Daten verarbeitet werden sollen, sowie die Rechtsgrundlage für die
Verarbeitung;
                \item die Kategorien personenbezogener Daten, die verarbeitet werden;
                \item gegebenenfalls die Empfänger oder Kategorien von Empfängern der personenbezogenen Daten;
                \item gegebenenfalls die Absicht des Verantwortlichen, die personenbezogenen Daten an einen Empfänger in einem Drittland oder einer internationalen Organisation zu übermitteln, sowie das Vorhandensein oder das Fehlen eines Angemessenheitsbeschlusses der Kommission oder im Falle von Übermittlungen gemäß Artikel 46 oder Artikel 47 oder Artikel 49 Absatz 1 Unterabsatz 2 einen Verweis auf die geeigneten oder angemessenen Garantien und die Möglichkeit, eine Kopie von ihnen zu erhalten, oder wo sie verfügbar sind. 
            \end{enumerate}
        \item Zusätzlich zu den Informationen gemäß Absatz 1 stellt der Verantwortliche der betroffenen Person die folgenden Informationen zur Verfügung, die erforderlich sind, um der betroffenen Person gegenüber eine faire und transparente Verarbeitung zu gewährleisten:
            \begin{enumerate}[label=\alph*)]
                \item die Dauer, für die die personenbezogenen Daten gespeichert werden oder, falls dies nicht möglich ist, die Kriterien für die Festlegung dieser Dauer;
                \item wenn die Verarbeitung auf Artikel 6 Absatz 1 Buchstabe f beruht, die berechtigten Interessen, die von dem Verantwortlichen oder einem Dritten verfolgt werden;
                \item das Bestehen eines Rechts auf Auskunft seitens des Verantwortlichen über die betreffenden personenbezogenen Daten sowie auf Berichtigung oder Löschung oder auf Einschränkung der Verarbeitung und eines Widerspruchsrechts gegen die Verarbeitung sowie des Rechts auf Datenübertragbarkeit; 
                \item wenn die Verarbeitung auf Artikel 6 Absatz 1 Buchstabe a oder Artikel 9 Absatz 2 Buchstabe a beruht, das Bestehen eines Rechts, die Einwilligung jederzeit zu widerrufen, ohne dass die Rechtmäßigkeit der aufgrund der Einwilligung bis zum Widerruf erfolgten Verarbeitung berührt wird;
                \item das Bestehen eines Beschwerderechts bei einer Aufsichtsbehörde;
                \item aus welcher Quelle die personenbezogenen Daten stammen und gegebenenfalls ob sie aus öffentlich zugänglichen Quellen stammen;
                \item das Bestehen einer automatisierten Entscheidungsfindung einschließlich Profiling gemäß Artikel 22 Absätze 1 und 4 und — zumindest in diesen Fällen — aussagekräftige Informationen über die involvierte Logik sowie die Tragweite und die angestrebten Auswirkungen einer derartigen Verarbeitung für die betroffene Person.
            \end{enumerate}
        \item Der Verantwortliche erteilt die Informationen gemäß den Absätzen 1 und 2
            \begin{enumerate}
                \item unter Berücksichtigung der spezifischen Umstände der Verarbeitung der personenbezogenen Daten innerhalb einer angemessenen Frist nach Erlangung der personenbezogenen Daten, längstens jedoch innerhalb eines Monats, 
                \item falls die personenbezogenen Daten zur Kommunikation mit der betroffenen Person verwendet werden sollen, spätestens zum Zeitpunkt der ersten Mitteilung an sie, oder, 
                \item falls die Offenlegung an einen anderen Empfänger beabsichtigt ist, spätestens zum Zeitpunkt der ersten Offenlegung.
            \end{enumerate}
        \item Beabsichtigt der Verantwortliche, die personenbezogenen Daten für einen anderen Zweck weiterzuverarbeiten als den, für den die personenbezogenen Daten erlangt wurden, so stellt er der betroffenen Person vor dieser Weiterverarbeitung Informationen über diesen anderen Zweck und alle anderen maßgeblichen Informationen gemäß Absatz 2 zur Verfügung.
        \item Die Absätze 1 bis 4 finden keine Anwendung, wenn und soweit
        \begin{enumerate}[label=\alph*)]
            \item die betroffene Person bereits über die Informationen verfügt,
            \item die Erteilung dieser Informationen sich als unmöglich erweist oder einen unverhältnismäßigen Aufwand erfordern würde; dies gilt insbesondere für die Verarbeitung für im öffentlichen Interesse liegende Archivzwecke, für wissenschaftliche oder historische Forschungszwecke oder für statistische Zwecke vorbehaltlich der in Artikel 89 Absatz 1 genannten Bedingungen und Garantien oder soweit die in Absatz 1 des vorliegenden Artikels genannte Pflicht voraussichtlich die Verwirklichung der Ziele dieser Verarbeitung unmöglich macht oder ernsthaft beeinträchtigt In diesen Fällen ergreift der Verantwortliche geeignete Maßnahmen zum Schutz der Rechte und Freiheiten sowie der berechtigten Interessen der betroffenen Person, einschließlich der Bereitstellung dieser Informationen für die Öffentlichkeit,
            \item die Erlangung oder Offenlegung durch Rechtsvorschriften der Union oder der Mitgliedstaaten, denen der Verantwortliche unterliegt und die geeignete Maßnahmen zum Schutz der berechtigten Interessen der betroffenen Person vorsehen, ausdrücklich geregelt ist oder
            \item die personenbezogenen Daten gemäß dem Unionsrecht oder dem Recht der Mitgliedstaaten dem Berufsgeheimnis, einschließlich einer satzungsmäßigen Geheimhaltungspflicht, unterliegen und daher vertraulich behandelt werden müssen. 
        \end{enumerate}
    \end{enumerate}

    \section[Art. 15: Auskunftsrecht]{Art. 15: Auskunftsrecht der betroffenen Person}
        \begin{enumerate}[label=(\arabic*)]
            \item Die betroffene Person hat das Recht, von dem Verantwortlichen eine Bestätigung darüber zu verlangen, ob sie betreffende personenbezogene Daten verarbeitet werden; ist dies der Fall, so hat sie ein Recht auf Auskunft über diese personenbezogenen Daten und auf folgende Informationen:
                \begin{enumerate}[label=\alph*)]
                    \item die Verarbeitungszwecke;
                    \item die Kategorien personenbezogener Daten, die verarbeitet werden;
                    \item die Empfänger oder Kategorien von Empfängern, gegenüber denen die personenbezogenen Daten offengelegt worden sind oder noch offengelegt werden, insbesondere bei Empfängern in Drittländern oder bei internationalen Organisationen;
                    \item falls möglich die geplante Dauer, für die die personenbezogenen Daten gespeichert werden, oder, falls dies nicht möglich ist, die Kriterien für die Festlegung dieser Dauer;
                    \item das Bestehen eines Rechts auf Berichtigung oder Löschung der sie betreffenden personenbezogenen Daten oder auf Einschränkung der Verarbeitung durch den Verantwortlichen oder eines Widerspruchsrechts gegen diese Verarbeitung;
                    \item das Bestehen eines Beschwerderechts bei einer Aufsichtsbehörde;
                    \item wenn die personenbezogenen Daten nicht bei der betroffenen Person erhoben werden, alle verfügbaren Informationen über die Herkunft der Daten;
                    \item das Bestehen einer automatisierten Entscheidungsfindung einschließlich Profiling gemäß Artikel 22 Absätze 1 und 4 und — zumindest in diesen Fällen — aussagekräftige Informationen über die involvierte Logik sowie die Tragweite und die angestrebten Auswirkungen einer derartigen Verarbeitung für die betroffene Person.
                \end{enumerate}
            \item Werden personenbezogene Daten an ein Drittland oder an eine internationale Organisation übermittelt, so hat die betroffene Person das Recht, über die geeigneten Garantien gemäß Artikel 46 im Zusammenhang mit der Übermittlung unterrichtet zu werden.
            \item Der Verantwortliche stellt eine Kopie der personenbezogenen Daten, die Gegenstand der Verarbeitung sind, zur Verfügung. Für alle weiteren Kopien, die die betroffene Person beantragt, kann der Verantwortliche ein angemessenes Entgelt auf der Grundlage der Verwaltungskosten verlangen. Stellt die betroffene Person den Antrag elektronisch, so sind die Informationen in einem gängigen elektronischen Format zur Verfügung zu stellen, sofern sie nichts anderes angibt.
            \item Das Recht auf Erhalt einer Kopie gemäß Absatz 1b darf die Rechte und Freiheiten anderer Personen nicht beeinträchtigen. 
        \end{enumerate}

    \section{Art. 16: Recht auf Berichtigung}
        Die betroffene Person hat das Recht, von dem Verantwortlichen unverzüglich die Berichtigung sie betreffender unrichtiger personenbezogener Daten zu verlangen. Unter Berücksichtigung der Zwecke der Verarbeitung hat die betroffene Person das Recht, die Vervollständigung unvollständiger personenbezogener Daten — auch mittels einer ergänzenden Erklärung — zu verlangen.



    \section[Art. 17 : Recht auf Löschung]{Art. 17: Recht auf Löschung (``Recht auf Vergessenwerden'')}
        \begin{enumerate}[label=(\arabic*)]
            \item Die betroffene Person hat das Recht, von dem Verantwortlichen zu verlangen, dass sie betreffende personenbezogene Daten unverzüglich gelöscht werden, und der Verantwortliche ist verpflichtet, personenbezogene Daten unverzüglich zu löschen, sofern einer der folgenden Gründe zutrifft:
                \begin{enumerate}[label=\alph*)]
                    \item Die personenbezogenen Daten sind für die Zwecke, für die sie erhoben oder auf sonstige Weise verarbeitet wurden, nicht mehr notwendig.
                    \item Die betroffene Person widerruft ihre Einwilligung, auf die sich die Verarbeitung gemäß Artikel 6 Absatz 1 Buchstabe a oder Artikel 9 Absatz 2 Buchstabe a stützte, und es fehlt an einer anderweitigen Rechtsgrundlage für die Verarbeitung.
                    \item Die betroffene Person legt gemäß Artikel 21 Absatz 1 Widerspruch gegen die Verarbeitung ein und es liegen keine vorrangigen berechtigten Gründe für die Verarbeitung vor, oder die betroffene Person legt gemäß Artikel 21 Absatz 2 Widerspruch gegen die Verarbeitung ein.
                    \item Die personenbezogenen Daten wurden unrechtmäßig verarbeitet. 
                    \item Die Löschung der personenbezogenen Daten ist zur Erfüllung einer rechtlichen Verpflichtung nach dem Unionsrecht oder dem Recht der Mitgliedstaaten erforderlich, dem der Verantwortliche unterliegt. 
                    \item Die personenbezogenen Daten wurden in Bezug auf angebotene Dienste der Informationsgesellschaft gemäß Artikel 8 Absatz 1 erhoben.
                \end{enumerate}
            \item Hat der Verantwortliche die personenbezogenen Daten öffentlich gemacht und ist er gemäß Absatz 1 zu deren Löschung verpflichtet, so trifft er unter Berücksichtigung der verfügbaren Technologie und der Implementierungskosten angemessene Maßnahmen, auch technischer Art, um für die Datenverarbeitung Verantwortliche, die die personenbezogenen Daten verarbeiten, darüber zu informieren, dass eine betroffene Person von ihnen die Löschung aller Links zu diesen personenbezogenen Daten oder von Kopien oder Replikationen dieser personenbezogenen Daten verlangt hat.
            \item Die Absätze 1 und 2 gelten nicht, soweit die Verarbeitung erforderlich ist
                \begin{enumerate}[label=\alph*)]
                    \item zur Ausübung des Rechts auf freie Meinungsäußerung und Information;
                    \item zur Erfüllung einer rechtlichen Verpflichtung, die die Verarbeitung nach dem Recht der Union oder der Mitgliedstaaten, dem der Verantwortliche unterliegt, erfordert, oder zur Wahrnehmung einer Aufgabe, die im öffentlichen Interesse liegt oder in Ausübung öffentlicher Gewalt erfolgt, die dem Verantwortlichen übertragen wurde;
                    \item aus Gründen des öffentlichen Interesses im Bereich der öffentlichen Gesundheit gemäß Artikel 9 Absatz 2 Buchstaben h und i sowie Artikel 9 Absatz 3; 
                    \item für im öffentlichen Interesse liegende Archivzwecke, wissenschaftliche oder historische Forschungszwecke oder für statistische Zwecke gemäß Artikel 89 Absatz 1, soweit das in Absatz 1 genannte Recht voraussichtlich die Verwirklichung der Ziele dieser Verarbeitung unmöglich macht oder ernsthaft beeinträchtigt, oder 
                    \item zur Geltendmachung, Ausübung oder Verteidigung von Rechtsansprüchen. 
                \end{enumerate}
        \end{enumerate}

    \section{Art. 18: Recht auf Einschränkung der Verarbeitung}
        \begin{enumerate}[label=(\arabic*)]
            \item Die betroffene Person hat das Recht, von dem Verantwortlichen die Einschränkung der Verarbeitung zu verlangen, wenn eine der folgenden Voraussetzungen gegeben ist:
                \begin{enumerate}[label=\alph*)]
                    \item die Richtigkeit der personenbezogenen Daten von der betroffenen Person bestritten wird, und zwar für eine Dauer, die es dem Verantwortlichen ermöglicht, die Richtigkeit der personenbezogenen Daten zu überprüfen,
                    \item die Verarbeitung unrechtmäßig ist und die betroffene Person die Löschung der personenbezogenen Daten ablehnt und stattdessen die Einschränkung der Nutzung der personenbezogenen Daten verlangt; 
                    \item der Verantwortliche die personenbezogenen Daten für die Zwecke der Verarbeitung nicht länger benötigt, die betroffene Person sie jedoch zur Geltendmachung, Ausübung oder Verteidigung von Rechtsansprüchen benötigt, oder
                    \item die betroffene Person Widerspruch gegen die Verarbeitung gemäß Artikel 21 Absatz 1 eingelegt hat, solange noch nicht feststeht, ob die berechtigten Gründe des Verantwortlichen gegenüber denen der betroffenen Person überwiegen.
                \end{enumerate}
            \item Wurde die Verarbeitung gemäß Absatz 1 eingeschränkt, so dürfen diese personenbezogenen Daten — von ihrer Speicherung abgesehen — nur mit Einwilligung der betroffenen Person oder zur Geltendmachung, Ausübung oder Verteidigung von Rechtsansprüchen oder zum Schutz der Rechte einer anderen natürlichen oder juristischen Person oder aus Gründen eines wichtigen öffentlichen Interesses der Union oder eines Mitgliedstaats verarbeitet werden.
            \item Eine betroffene Person, die eine Einschränkung der Verarbeitung gemäß Absatz 1 erwirkt hat, wird von dem Verantwortlichen unterrichtet, bevor die Einschränkung aufgehoben wird. 
        \end{enumerate}

    \section{Art. 21: Widerspruchsrecht}
        \begin{enumerate}[label=(\arabic*]
            \item Die betroffene Person hat das Recht, aus Gründen, die sich aus ihrer besonderen Situation ergeben, jederzeit gegen die Verarbeitung sie betreffender personenbezogener Daten, die aufgrund von Artikel 6 Absatz 1 Buchstaben e oder f erfolgt, Widerspruch einzulegen; dies gilt auch für ein auf diese Bestimmungen gestütztes Profiling. Der Verantwortliche verarbeitet die personenbezogenen Daten nicht mehr, es sei denn, er kann zwingende schutzwürdige Gründe für die Verarbeitung nachweisen, die die Interessen, Rechte und Freiheiten der betroffenen Person überwiegen, oder die Verarbeitung dient der Geltendmachung, Ausübung oder Verteidigung von Rechtsansprüchen.
            \item Werden personenbezogene Daten verarbeitet, um Direktwerbung zu betreiben, so hat die betroffene Person das Recht, jederzeit Widerspruch gegen die Verarbeitung sie betreffender personenbezogener Daten zum Zwecke derartiger Werbung einzulegen; dies gilt auch für das Profiling, soweit es mit solcher Direktwerbung in Verbindung steht. 
            \item Widerspricht die betroffene Person der Verarbeitung für Zwecke der Direktwerbung, so werden die personenbezogenen Daten nicht mehr für diese Zwecke verarbeitet.
            \item Die betroffene Person muss spätestens zum Zeitpunkt der ersten Kommunikation mit ihr ausdrücklich auf das in den Absätzen 1 und 2 genannte Recht hingewiesen werden; dieser Hinweis hat in einer verständlichen und von anderen Informationen getrennten Form zu erfolgen.
            \item Im Zusammenhang mit der Nutzung von Diensten der Informationsgesellschaft kann die betroffene Person ungeachtet der Richtlinie 2002/58/EG ihr Widerspruchsrecht mittels automatisierter Verfahren ausüben, bei denen technische Spezifikationen verwendet werden. 
            \item Die betroffene Person hat das Recht, aus Gründen, die sich aus ihrer besonderen Situation ergeben, gegen die sie betreffende Verarbeitung sie betreffender personenbezogener Daten, die zu wissenschaftlichen oder historischen Forschungszwecken oder zu statistischen Zwecken gemäß Artikel 89 Absatz 1 erfolgt, Widerspruch einzulegen, es sei denn, die Verarbeitung ist zur Erfüllung einer im öffentlichen Interesse liegenden Aufgabe erforderlich. 
        \end{enumerate}


    \section[Art. 89: Garantien]{Art. 89: Garantien und Ausnahmen in Bezug auf die Verarbeitung zu im öffentlichen Interesse liegenden Archivzwecken, zu wissenschaftlichen oder historischen Forschungszwecken und zu statistischen Zwecken}
        \begin{enumerate}[label=(\arabic*)]
            \item Die Verarbeitung zu im öffentlichen Interesse liegenden Archivzwecken, zu wissenschaftlichen oder historischen Forschungszwecken oder zu statistischen Zwecken unterliegt geeigneten Garantien für die Rechte und Freiheiten der betroffenen Person gemäß dieser Verordnung. Mit diesen Garantien wird sichergestellt, dass technische und organisatorische Maßnahmen bestehen, mit denen insbesondere die Achtung des Grundsatzes der Datenminimierung gewährleistet wird. Zu diesen Maßnahmen kann die Pseudonymisierung gehören, sofern es möglich ist, diese Zwecke auf diese Weise zu erfüllen. In allen Fällen, in denen diese Zwecke durch die Weiterverarbeitung, bei der die Identifizierung von betroffenen Personen nicht oder nicht mehr möglich ist, erfüllt werden können, werden diese Zwecke auf diese Weise erfüllt.
            \item Werden personenbezogene Daten zu wissenschaftlichen oder historischen Forschungszwecken oder zu statistischen Zwecken verarbeitet, können vorbehaltlich der Bedingungen und Garantien gemäß Absatz 1 des vorliegenden Artikels im Unionsrecht oder im Recht der Mitgliedstaaten insoweit Ausnahmen von den Rechten gemäß der Artikel 15, 16, 18 und 21 vorgesehen werden, als diese Rechte voraussichtlich die Verwirklichung der spezifischen Zwecke unmöglich machen oder ernsthaft beeinträchtigen und solche Ausnahmen für die Erfüllung dieser Zwecke notwendig sind.
            \item Werden personenbezogene Daten für im öffentlichen Interesse liegende Archivzwecke verarbeitet, können vorbehaltlich der Bedingungen und Garantien gemäß Absatz 1 des vorliegenden Artikels im Unionsrecht oder im Recht der Mitgliedstaaten insoweit Ausnahmen von den Rechten gemäß der Artikel 15, 16, 18, 19, 20 und 21 vorgesehen werden, als diese Rechte voraussichtlich die Verwirklichung der spezifischen Zwecke unmöglich machen oder ernsthaft beeinträchtigen und solche Ausnahmen für die Erfüllung dieser Zwecke notwendig sind. 
            \item Dient die in den Absätzen 2 und 3 genannte Verarbeitung gleichzeitig einem anderen Zweck, gelten die Ausnahmen nur für die Verarbeitung zu den in diesen Absätzen genannten Zwecken. 
        \end{enumerate}

\chapter[BStatG]{Bundesstatistikgesetz}
\qrcode{https://www.gesetze-im-internet.de/bstatg_1987/BStatG.pdf}
\newline
\url{https://www.gesetze-im-internet.de/bstatg_1987/BStatG.pdf}
    \section{\S 1: Statistik für Bundeszwecke} 
    Die Statistik für Bundeszwecke (Bundesstatistik) hat im föderativ gegliederten Gesamtsystem der amtlichen Statistik die Aufgabe, laufend Daten über Massenerscheinungen zu erheben, zu sammeln, aufzubereiten, darzustellen und zu analysieren. Für sie gelten die Grundsätze der Neutralität, Objektivität und fachlichen Unabhängigkeit. Sie gewinnt die Daten unter Verwendung wissenschaftlicher Erkenntnisse und unter Einsatz der jeweils sachgerechten Methoden und Informationstechniken. Durch die Ergebnisse der Bundesstatistik werden gesellschaftliche, wirtschaftliche und ökologische Zusammenhänge für Bund, Länder einschliesslich Gemeinden und Gemeindeverbände, Gesellschaft, Wirtschaft, Wissenschaft und Forschung aufgeschlüsselt. Die Bundesstatistik ist Voraussetzung für eine am Sozialstaatsprinzip ausgerichtete Politik. Die für die Bundesstatistik erhobenen Einzelangaben dienen ausschliesslich den durch dieses Gesetz oder eine andere eine Bundesstatistik anordnende Rechtsvorschrift festgelegten Zwecken.
    \section{\S 10: Erhebungs- und Hilfsmerkmale}
        \begin{enumerate}[label=(\arabic*)]
            \item Bundesstatistiken werden auf der Grundlage von Erhebungs- und Hilfsmerkmalen erstellt. Erhebungsmerkmale umfassen Angaben über persönliche und sachliche Verhältnisse, die zur statistischen Verwendung bestimmt sind. Hilfsmerkmale sind Angaben, die der technischen Durchführung von Bundesstatistiken dienen. Für andere Zwecke dürfen sie nur verwendet werden, soweit Absatz 2 oder ein sonstiges Gesetz es zulassen.
            \item Der Name der Gemeinde, die Blockseite und die geografische Gitterzelle dürfen für die regionale Zuordnung der Erhebungsmerkmale genutzt werden. Die übrigen Teile der Anschrift dürfen für die Zuordnung zu Blockseiten und geografischen Gitterzellen für einen Zeitraum von bis zu vier Jahren nach Abschluss der jeweiligen Erhebung genutzt werden. Besondere Regelungen in einer eine Bundesstatistik anordnenden Rechtsvorschrift bleiben unberührt.
            \item Blockseite ist innerhalb eines Gemeindegebiets die Seite mit gleicher Strassenbezeichnung von der durch Strasseneinmündungen oder vergleichbare Begrenzungen umschlossenen Fläche. Eine geografische Gitterzelle ist eine Gebietseinheit, die bezogen auf eine vorgegebene Kartenprojektion quadratisch ist und mindestens 1 Hektar gross ist.
        \end{enumerate}
    \section{\S 12 Trennung und Löschung der Hilfsmerkmale}
        \begin{enumerate}[label=(\arabic*)]
            \item Hilfsmerkmale sind, soweit Absatz 2, § 10 Absatz 2, § 13 oder eine sonstige Rechtsvorschrift nichts anderes bestimmen, zu löschen, sobald bei den statistischen Ämtern die Überprüfung der Erhebungs- und Hilfsmerkmale auf ihre Schlüssigkeit und Vollständigkeit abgeschlossen ist. Sie sind von den Erhebungsmerkmalen zum frühestmöglichen Zeitpunkt zu trennen und gesondert aufzubewahren oder gesondert zu speichern.
            \item Bei periodischen Erhebungen für Zwecke der Bundesstatistik dürfen die zur Bestimmung des Kreises der zu Befragenden erforderlichen Hilfsmerkmale, soweit sie für nachfolgende Erhebungen benötigt werden, gesondert aufbewahrt oder gesondert gespeichert werden. Nach Beendigung des Zeitraumes der wiederkehrenden Erhebungen sind sie zu löschen.
        \end{enumerate}
    \section{\S 16 Geheimhaltung}
        \begin{enumerate}[label=(\arabic*)]
            \item Einzelangaben über persönliche und sachliche Verhältnisse, die für eine Bundesstatistik gemacht werden, sind von den Amtsträgern und Amtsträgerinnen und für den öffentlichen Dienst besonders Verpflichteten, die mit der Durchführung von Bundesstatistiken betraut sind, geheim zu halten, soweit durch besondere Rechtsvorschrift nichts anderes bestimmt ist. Die Geheimhaltungspflicht besteht auch nach Beendigung ihrer Tätigkeit fort. Die Geheimhaltungspflicht gilt nicht für
                \begin{enumerate}[label=\arabic*.]
                    \item Einzelangaben, in deren Übermittlung oder Veröffentlichung die Betroffenen schriftlich eingewilligt haben, soweit nicht wegen besonderer Umstände eine andere Form der Einwilligung angemessen ist,
                    \item Einzelangaben aus allgemein zugänglichen Quellen, wenn sie sich auf die in § 15 Absatz 1 genannten öffentlichen Stellen beziehen, auch soweit eine Auskunftspflicht aufgrund einer eine Bundesstatistik anordnenden Rechtsvorschrift besteht,
                    \item Einzelangaben, die vom Statistischen Bundesamt oder den statistischen Ämtern der Länder mit den Einzelangaben anderer Befragter zusammengefasst und in statistischen Ergebnissen dargestellt sind, 
                    \item Einzelangaben, wenn sie den Befragten oder Betroffenen nicht zuzuordnen sind. Die §§ 93, 97, 105 Absatz 1, § 111 Absatz 5 in Verbindung mit § 105 Absatz 1 sowie § 116 Absatz 1 der Abgabenordnung vom 16. März 1976 (BGBl. I S. 613; 1977 I S. 269), zuletzt geändert durch Artikel 1 des Gesetzes vom 19. Dezember 1985 (BGBl. I S. 2436), gelten nicht für Personen und Stellen, soweit sie mit der Durchführung von Bundes- , Landes- oder Kommunalstatistiken betraut sind.
                \end{enumerate}
            \item Die Übermittlung von Einzelangaben zwischen den mit der Durchführung einer Bundesstatistik betrauten Personen und Stellen ist zulässig, soweit dies zur Erstellung der Bundesstatistik erforderlich ist. Darüber hinaus ist die Übermittlung von Einzelangaben zwischen den an einer Zusammenarbeit nach § 3a beteiligten statistischen Ämtern und die zentrale Verarbeitung und Nutzung dieser Einzelangaben in einem oder mehreren statistischen Ämtern zulässig.
            \item Das Statistische Bundesamt darf an die statistischen Ämter der Länder die ihren jeweiligen Erhebungsbereich betreffenden Einzelangaben für Sonderaufbereitungen auf regionaler Ebene übermitteln. Für die Erstellung der Volkswirtschaftlichen Gesamtrechnungen und sonstiger Gesamtsysteme des Bundes und der Länder dürfen sich das Statistische Bundesamt und die statistischen Ämter der Länder untereinander Einzelangaben aus Bundesstatistiken übermitteln.
            \item Für die Verwendung gegenüber den gesetzgebenden Körperschaften und für Zwecke der Planung, jedoch nicht für die Regelung von Einzelfällen, dürfen den obersten Bundes- oder Landesbehörden vom Statistischen Bundesamt und den statistischen Ämtern der Länder Tabellen mit statistischen Ergebnissen übermittelt werden, auch soweit Tabellenfelder nur einen einzigen Fall ausweisen. Die Übermittlung nach Satz 1 ist nur zulässig, soweit in den eine Bundesstatistik anordnenden Rechtsvorschriften die Übermittlung von Einzelangaben an oberste Bundes- oder Landesbehörden zugelassen ist.
            \item Für ausschließlich statistische Zwecke dürfen vom Statistischen Bundesamt und den statistischen Ämtern der Länder Einzelangaben an die zur Durchführung statistischer Aufgaben zuständigen Stellen der Gemeinden und Gemeindeverbände übermittelt werden, wenn die Übermittlung in einem eine Bundesstatistik anordnenden Gesetz vorgesehen ist sowie Art und Umfang der zu übermittelnden Einzelangaben bestimmt sind. Die Übermittlung ist nur zulässig, wenn durch Landesgesetz eine Trennung dieser Stellen von anderen kommunalen Verwaltungsstellen sichergestellt und das Statistikgeheimnis durch Organisation und Verfahren gewährleistet ist. 
            \item Für die Durchführung wissenschaftlicher Vorhaben dürfen das Statistische Bundesamt und die statistischen Ämter der Länder Hochschulen oder sonstigen Einrichtungen mit der Aufgabe unabhängiger wissenschaftlicher Forschung
                \begin{enumerate}[label=\arabic*.]
                    \item Einzelangaben übermitteln, wenn die Einzelangaben nur mit einem unverhältnismäßig großen Aufwand an Zeit, Kosten und Arbeitskraft zugeordnet werden können (faktisch anonymisierte Einzelangaben),
                    \item innerhalb speziell abgesicherter Bereiche des Statistischen Bundesamtes und der statistischen Ämter der Länder Zugang zu formal anonymisierten Einzelangaben gewähren, wenn wirksame Vorkehrungen zur Wahrung der Geheimhaltung getroffen werden. Berechtigte können nur Amtsträger oder Amtsträgerinnen, für den öffentlichen Dienst besonders Verpflichtete oder Verpflichtete nach Absatz 7 sein.
                \end{enumerate}
            \item Personen, die Einzelangaben nach Absatz 6 erhalten sollen, sind vor der Übermittlung zur Geheimhaltung zu verpflichten, soweit sie nicht Amtsträger oder Amtsträgerinnen oder für den öffentlichen Dienst besonders Verpflichtete sind. § 1 Absatz 2, 3 und 4 Nummer 2 des Verpflichtungsgesetzes vom 2. März 1974 (BGBl. I S. 469, Artikel 42), das durch Gesetz vom 15. August 1974 (BGBl. I S. 1942) geändert worden ist, gilt entsprechend. 
            \item Die aufgrund einer besonderen Rechtsvorschrift oder der Absätze 4, 5 oder 6 übermittelten Einzelangaben dürfen nur für die Zwecke verwendet werden, für die sie übermittelt wurden. In den Fällen des Absatzes 6 Satz 1 Nummer 1 sind sie zu löschen, sobald das wissenschaftliche Vorhaben durchgeführt ist. Bei den Stellen, denen Einzelangaben übermittelt werden, muss durch organisatorische und technische Maßnahmen sichergestellt sein, dass nur Amtsträger, für den öffentlichen Dienst besonders Verpflichtete oder Verpflichtete nach Absatz 7 Satz 1 Empfänger von Einzelangaben sind.
            \item Die Übermittlung aufgrund einer besonderen Rechtsvorschrift oder nach den Absätzen 4, 5 oder 6 ist nach Inhalt, Stelle, der übermittelt wird, Datum und Zweck der Weitergabe von den statistischen Ämtern aufzuzeichnen. Die Aufzeichnungen sind mindestens fünf Jahre aufzubewahren.
            \item Die Pflicht zur Geheimhaltung nach Absatz 1 besteht auch für die Personen, die Empfänger von Einzelangaben aufgrund einer besonderen Rechtsvorschrift, nach den Absätzen 5, 6 oder von Tabellen nach Absatz 4 sind. Dies gilt nicht für offenkundige Tatsachen bei einer Übermittlung nach Absatz 4.
        \end{enumerate}
\chapter[BayStatG]{Bayerisches Statistikgesetz}
\qrcode{https://www.gesetze-bayern.de/Content/Pdf/BayStatG?all=True}
\newline
\url{https://www.gesetze-bayern.de/Content/Pdf/BayStatG?all=True}
    \section{Art. 1: Geltungsbereich}
        \begin{enumerate}[label=(\arabic*)]
            \item \textsuperscript{1}Dieses Gesetz gilt für die Durchführung von Statistiken durch öffentliche Stellen. \textsuperscript{2}Führen diese Stellen Bundesstatistiken oder europäische Statistiken durch und haben sie dabei andere Rechtsvorschriften anzuwenden, so finden die Vorschriften dieses Gesetzes nur ergänzend Anwendung.
            \item Für Geschäftsstatistiken gilt dieses Gesetz nur, soweit das ausdrücklich bestimmt ist.
        \end{enumerate}
    \section{Art. 2: Begriffe}
        \begin{enumerate}[label=(\arabic*)]
            \item \textsuperscript{1}Amtliche Statistiken sind Landesstatistiken, Bundesstatistiken und europäische Statistiken. \textsuperscript{2}Landesstatistiken sind Statistiken, die von Organen des Freistaates Bayern angeordnet und von staatlichen Stellen durchgeführt werden.
            \item Kommunale Statistiken sind Statistiken, die von Gemeinden oder Gemeindever\-bän\-den zur Wahrnehmung ihrer Aufgaben durchgeführt werden.
            \item Geschäftsstatistiken sind statistische Aufbereitungen von Daten, die bei öffentli\-chen Stellen im Vollzug ihrer Aufgaben, die nicht die Durchführung von Statistiken betreffen, erhoben werden oder auf sonstige Weise anfallen.
            \item Öffentliche Stellen sind alle Behörden, Gerichte und sonstige öffentliche Stellen des Freistaates Bayern, die Gemeinden und Gemeindeverbände sowie die der Aufsicht des Freistaates Bayern unterstehenden juristischen Personen des öffentlichen Rechts und deren Vereinigungen.
            \item Einzelangaben sind Daten über persönliche oder sachliche Verhältnisse bestimmter oder bestimmbarer natürlicher oder juristischer Personen und deren Vereinigungen, die bei der Durchführung einer Statistik erhoben oder übermittelt werden.
        \end{enumerate}
    \section[Art. 3: EU-DSGVO und BayDatSchG]{Art. 3: Anwendbarkeit der Datenschutz-Grundverordnung und des Bayerischen Datenschutzgesetzes}
        \begin{enumerate}[label=(\arabic*)]
            \item Die Ansprüche nach den Art. 15, 16, 18 und 21 der Verordnung (EU) 2016/679 (Datenschutz- Grundverordnung– DSGVO) bestehen nicht, soweit diese Rechte die Verwirklichung statistischer Zwecke ernsthaft beeinträchtigen würden.
            \item Einzelangaben dürfen an das Landesamt und an Statistikstellen für die Durch\-füh\-rung von Geschäftsstatistiken übermittelt und von dort – auch in aufbereiteter Form – rückübermittelt werden
        \end{enumerate}
    (\dots)
    \section{Art. 9: Anordnung}
        \begin{enumerate}[label=(\arabic*)]
            \item \textsuperscript{1}Statistiken werden durch Gesetz oder Rechtsverordnung angeordnet. 2Die Anordnung bedarf keiner Rechtsvorschrift, wenn
            \begin{enumerate}[label=\arabic*.]
                \item die einer Statistik zugrundeliegenden Daten
                    \begin{enumerate}[label=(\alph*)]
                        \item auf freiwilligen Auskünften oder allgemein zugänglichen Quellen beruhen,
                        \item keine Einzelangaben enthalten oder
                        \item der die Statistik durchführenden Stelle rechtmässig übermittelt werden oder ihrem Zugriff auf Grund einer Rechtsvorschrift zur Verfügung stehen;
                    \end{enumerate}
                \item lediglich Sonderauswertungen vorhandenen statistischen Materials vorgenommen werden, dessen Verwendung eine Zweckbindung nicht entgegensteht, oder
                \item zur Anordnung der Statistik eine Rechtsvorschrift ermächtigt.
            \end{enumerate}
            \item Die eine Landesstatistik anordnende Rechtsvorschrift muss die näheren Bestimmungen treffen über die Art der Erhebung, den Kreis der zu Befragenden, sonstige Auskunftsstellen, die durch Erhebungsmerkmale zu erfassenden Sachverhalte, die Hilfsmerkmale, den Berichtszeitraum, den Berichtszeitpunkt, die Häufigkeit der Erhebung (Periodizität) sowie über Art und Umfang einer Auskunftspflicht.
        \end{enumerate}
    \section{Art. 14: Erhebungsbeauftragte}
        \begin{enumerate}[label=(\arabic*)]
            \item Als Erhebungsbeauftragte dürfen nur Personen eingesetzt werden, die Gewähr für Zuverlässigkeit und Verschwiegenheit bieten und bei denen nicht auf Grund ihrer beruflichen Tätigkeit oder aus anderen Gründen Anlass zur Besorgnis besteht, dass Erkenntnisse aus der Tätigkeit als Erhebungsbeauftragte zu Lasten der Auskunftspflichtigen genutzt werden.
            \item \textsuperscript{1}Erhebungsbeauftragte sind verpflichtet, die Anweisungen der Erhebungsstellen zu befolgen. \textsuperscript{2}Bei der Ausübung ihrer Tätigkeit haben sie sich auszuweisen. \textsuperscript{3}Sie dürfen statistische Einzelangaben und gelegentlich ihrer Tätigkeit gewonnene Erkenntnisse auch nach Beendigung ihrer Tätigkeit nicht für andere Verfahren oder andere Zwecke verarbeiten.
            \item \textsuperscript{1}Erhebungsbeauftragte sind über ihre Rechte und Pflichten sowie über die Rechte und Pflichten der zu Befragenden zu belehren. \textsuperscript{2}Vor ihrem Einsatz sind sie auf die Wahrung des Statistikgeheimnisses und zur Geheimhaltung der Erkenntnisse, die sie bei ihrer Tätigkeit gewonnen haben, schriftlich zu verpflichten.
        \end{enumerate}
    \section{Art. 15: Erhebungs- und Hilfsmerkmale}
        \begin{enumerate}[label=(\arabic*)]
            \item \textsuperscript{1}Erhebungsmerkmale sind zur Erstellung einer Statistik bestimmte Angaben über persönliche oder sachliche Verhältnisse. \textsuperscript{2}Hilfsmerkmale sind Angaben, die der technischen Durchführung von Statistiken dienen.
            \item \textsuperscript{1}Hilfsmerkmale sind von den Erhebungsmerkmalen zum frühestmöglichen Zeitpunkt zu trennen und gesondert aufzubewahren. \textsuperscript{2}Laufende Nummern und Ordnungsnummern können auf den Erhebungsunterlagen verbleiben. \textsuperscript{3}Sie dürfen auf die für die maschinelle Weiterverarbeitung bestimmten Datenträger übernommen werden.
            \item \textsuperscript{1}Die Hilfsmerkmale sind zu löschen, sobald die Überprüfung der Erhebungs- und Hilfsmerkmale auf Schlüssigkeit und Vollständigkeit abgeschlossen ist. \textsuperscript{2}Bei wiederkehrenden Erhebungen kann die Löschung der Hilfsmerkmale unterbleiben, soweit sie noch künftig zur Bestimmung des Kreises der zu Befragenden benötigt werden. \textsuperscript{3}Die Hilfsmerkmale sind gesondert aufzubewahren und nach Beendigung der wiederkehrenden Erhebungen zu löschen. \textsuperscript{4}Diese Vorschriften gelten entsprechend für die Vernichtung von Erhebungsunterlagen, die Hilfsmerkmale enthalten.
            \item \textsuperscript{1}Die Namen von Gemeinden und von Gemeindeteilen sowie Blockseiten dürfen für die regionale Zuordnung von Erhebungsmerkmalen genutzt werden. \textsuperscript{2}Blockseite ist innerhalb eines Gemeindegebiets die Seite mit gleicher Strassenbezeichnung von der durch Strasseneinmündungen oder vergleichbare Begrenzungen umschlossenen Fläche. \textsuperscript{3}Die übrigen Teile der Anschrift dürfen für die Zuordnung zu Blockseiten für einen Zeitraum bis zu vier Jahren nach Abschluss der jeweiligen Erhebung genutzt werden. 4Besondere Regelungen in einer eine amtliche Statistik anordnenden Rechtsvorschrift bleiben unberührt.
            \item Absatz 2 Satz 1 und Absatz 3 Satz 1 und 3 gelten nicht für Daten, die ausschliesslich einer öffentlichen Stelle zugeordnet werden können.
        \end{enumerate}   
    
    \section{Art. 17: Geheimhaltung}
        \begin{enumerate}[label=(\arabic*)]
            \item \textsuperscript{1}Einzelangaben sind von den mit der Durchführung der Statistik betrauten Stellen und Personen geheimzuhalten. \textsuperscript{2}Dies gilt nicht für
                \begin{enumerate}[label=\arabic*.]
                    \item Einzelangaben, in deren Übermittlung oder Veröffentlichung die Auskunftgebenden oder die betroffenen Personen schriftlich eingewilligt haben;
                    \item Einzelangaben, soweit deren Übermittlung oder Veröffentlichung durch Art. 18 oder durch besondere Rechtsvorschrift zugelassen ist;
                    \item Einzelangaben aus allgemein zugänglichen Quellen;
                    \item Einzelangaben, die ausschliesslich einer öffentlichen Stelle, die nicht am wirtschaftlichen Wettbewerb teilnimmt, zugeordnet werden können;
                    \item Einzelangaben, die keiner befragten oder betroffenen Person zuzuordnen sind, insbesondere, wenn sie mit den Einzelangaben anderer zusammengefasst und in statistischen Ergebnissen dargestellt sind.
                \end{enumerate}
            \textsuperscript{3}Die Pflicht zur Geheimhaltung besteht auch für Personen, die Empfänger von Einzelangaben nach Art. 18 oder auf Grund einer besonderen Rechtsvorschrift sind.
            \item Sonstige Vorschriften über die Geheimhaltung und Verschwiegenheit bleiben unberührt.
        \end{enumerate}
    \section{Art. 18: Zweckbindung und Übermittlung von Einzelangaben}
        \begin{enumerate}[label=(\arabic*)]
            \item Einzelangaben dürfen ausschliesslich für statistische Zwecke verarbeitet werden, es sei denn sie beruhen auf allgemein zugänglichen Quellen oder eine Rechtsvorschrift lässt eine andere Verwendung zu.
            \item \textsuperscript{1}Das Landesamt darf Einzelangaben, wenn eine ausdrückliche Zweckbindung nicht entgegensteht, an Statistikstellen anderer öffentlicher Stellen für deren Zuständig\-keitsbereich zu ausschliesslich statistischen Zwecken übermitteln. \textsuperscript{2}Soweit durch Rechtsvorschrift nichts anderes bestimmt ist, dürfen Hilfsmerkmale nicht über\-mit\-telt werden.
            \item Zur Erstellung koordinierter Länderstatistiken darf das Landesamt Einzel\-an\-ga\-ben an das Statistische Bundesamt und die Statistischen Ämter der Länder über\-mitteln.
            \item \textsuperscript{1}Für Gesetzesvorhaben und für Zwecke der Planung, nicht jedoch für die Regelung von Einzelfällen, darf das Landesamt den Staatsministerien Tabellen mit statistischen Ergebnissen übermitteln, auch soweit Tabellenfelder nur einen einzigen Fall ausweisen. \textsuperscript{2}Durch organisatorische und technische Massnahmen muss sichergestellt sein, dass nur Amtsträger und für den öffentlichen Dienst besonders Verpflichtete Kenntnis von Einzelangaben erhalten.
            \item \textsuperscript{1}Für die Durchführung wissenschaftlicher Vorhaben darf das Landesamt Einzelangaben an Hochschulen oder sonstige Einrichtungen mit der Aufgabe unabhängiger wissenschaftlicher Forschung übermitteln, wenn die Einzelangaben nur mit einem unverhältnismässig grossen Aufwand an Zeit, Kosten und Arbeitskraft zugeordnet werden können. \textsuperscript{2}Sofern es sich bei den Empfängern nicht um Amtsträger oder für den öffentlichen Dienst besonders Verpflichtete handelt, sind sie vor der Übermittlung vom Landesamt besonders zur Geheimhaltung zu verpflichten. \textsuperscript{3}§ 1 Abs. 2, 3 und 4 Nr. 2 des Verpflichtungsgesetzes sind entsprechend anwendbar. \textsuperscript{4}Personen, die nach Satz 2 besonders verpflichtet worden sind, stehen für die Anwendung der Vorschriften des Strafgesetzbuches über die Verletzung von Privatgeheimnissen (§ 203 Abs. 2, 4, 5, § 204, 205) den für den öffentlichen Dienst besonders Verpflichteten gleich. 5Empfänger haben durch technische und organisatorische Massnahmen sicherzustellen, dass sonstige Personen keine Kenntnis von Einzelangaben erhalten. 6Die Einzelangaben sind zu löschen oder zu vernichten, sobald das wissenschaftliche Vorhaben abgeschlossen ist, zu dessen Durchführung sie übermittelt wurden.
            \item \textsuperscript{1}Einzelangaben, die auf Grund der Absätze 2 bis 5 oder auf Grund einer besonderen Rechtsvorschrift übermittelt werden, dürfen nur für den Zweck verwendet werden, für den sie über\-mittelt worden sind. \textsuperscript{2}Die Übermittlung ist vom Landesamt unter Angabe von Inhalt, empfangender Stelle, Datum und Zweck aufzuzeichnen. \textsuperscript{3}Die Aufzeichnungen sind mindestens fünf Jahre aufzubewahren.
            \item Einzelangaben dürfen vom Landesamt wieder an die auskunftsgebende Stelle über\-mittelt werden. 
            \item Die Absätze 2 bis 7 gelten entsprechend, wenn Statistikstellen anderer staatlicher Stellen für die Durchführung von Landesstatistiken zuständig sind. 
        \end{enumerate}
        
    \section{Art. 20: Statistikstellen}
        \begin{enumerate}[label=(\arabic*)]
            \item \textsuperscript{1}Werden Statistiken ausserhalb des Landesamts durchgeführt, so sind besondere Statistikstellen einzurichten. \textsuperscript{2}Nichtstatistische Aufgaben des Verwaltungsvollzugs dürfen ihnen nicht übertragen werden. \textsuperscript{3}Statistikstellen veröffentlichen die Ergebnisse ihrer Statistiken oder stellen sie in sonstiger Weise bereit.
            \item \textsuperscript{1}Für jede Statistikstelle ist jemand zu bestimmen, der diese leitet. \textsuperscript{2}Statistikstellen sind räumlich und organisatorisch von anderen Verwaltungsstellen zu trennen, gegen den Zutritt unbefugter Personen hinreichend zu sichern und mit Personal auszustatten, das die Gewähr für Zuverlässigkeit und Verschwiegenheit bietet.
            \item \textsuperscript{1}Die in Statistikstellen tätigen Personen dürfen statistische Einzelangaben und gelegentlich ihrer Tätigkeit gewonnene Erkenntnisse auch nach Beendigung ihrer Tätigkeit nicht in anderen Verfahren oder für andere Zwecke verarbeiten, soweit nicht durch Rechtsvorschrift etwas anderes zugelassen ist. \textsuperscript{2}Sie sind vor ihrem Einsatz auf die Wahrung des Statistikgeheimnisses und über die Folgen seiner Verletzung zu belehren und schriftlich zu verpflichten. \textsuperscript{3}Soweit und solange sie Einzelangaben bearbeiten, dürfen sie nicht andere Aufgaben des Verwaltungsvollzugs wahrnehmen. \textsuperscript{4}Im Anschluss an eine Tätigkeit in der Statistikstelle sollen sie nicht für Aufgaben eingesetzt werden, bei denen eine Nutzung der in den Statistikstellen gewonnenen Erkenntnisse möglich ist, soweit das die organisatorischen und personellen Verhältnisse zulassen.
            \item Statistikstellen können mit der Durchführung von Geschäftsstatistiken beauftragt werden-
        \end{enumerate}



    \section{Art. 21: Erhebungsstellen, Verordnungsermächtigung}
        \begin{enumerate}[label=(\arabic*)]
            \item Das Landesamt ist bei Statistiken, die es als allgemeine Aufgabe durchführt, Erhebungsstelle.
            \item \textsuperscript{1}Die Staatsregierung wird ermächtigt, durch Rechtsverordnung zu bestimmen, dass andere staatliche Stellen sowie Gemeinden Erhebungsstellen einzurichten oder in sonstiger Weise an der Durchführung amtlicher Statistiken mitzuwirken haben, wenn das wegen der Art der Erhebung, der Zahl oder der räumlichen Verteilung der zu Befragenden oder zur Sicherung der Qualität der Erhebung zweckmässig ist. \textsuperscript{2}Eine aufsichtliche Zuständigkeit des Landesamts wird durch eine solche Bestimmung nicht begründet. \textsuperscript{3}Landratsämter erfüllen die Aufgaben der Erhebungsstellen als Staatsaufgaben; für Gemeinden handelt es sich um Aufgaben des übertragenen Wirkungskreises, die sie auch nach den Vorschriften des Gesetzes über die kommunale Zusammenarbeit erfüllen können.
            \item \textsuperscript{1}Die Erhebungsstellen nach Absatz 2 Satz 1 führen in ihrem jeweiligen Zu\-stän\-dig\-keits\-be\-reich die statistischen Erhebungen durch. 2Art. 20 Abs. 2 und 3 gelten entsprechend mit der Massgabe, dass die räumliche und organisatorische Trennung von anderen Verwaltungsstellen ab dem Eingang der Erhebungsunterlagen bis zu ihrer Ablieferung sicherzustellen ist. 3Durch Rechtsverordnung nach Absatz 2 Satz 1 können Abweichungen von den Anforderungen des Art. 20 Abs. 2 und 3 bestimmt werden, wenn das ein erweiterter Schutz von Einzelangaben erforderlich macht oder wenn eine andere staatliche Stelle oder eine Gemeinde an der Erhebung lediglich mitwirkt. 4Soweit nichts anderes bestimmt ist, haben diese Erhebungsstellen
                \begin{enumerate}[label=\arabic*.]
                    \item bei Bedarf Erhebungsbezirke festzulegen;
                    \item die Erhebungsbeauftragten auszuwählen, zu bestellen, zu unterrichten, zu verpflichten und zu beaufsichtigen;
                    \item die zu Befragenden gemäss Art. 19 zu unterrichten, zur Auskunft heranzuziehen, die Erhebungsvordrucke auszuteilen und einzusammeln;
                    \item Personen, die noch keine Auskünfte gegeben haben, zur Auskunftserteilung anzuhalten;
                    \item die Vollzähligkeit der ausgefüllten Erhebungsvordrucke sowie deren Voll\-stän\-dig\-keit und die formale Richtigkeit der Angaben zu überprüfen;
                    \item unvollständige oder offensichtlich fehlerhaft ausgefüllte Erhebungsvordrucke durch Nachfrage bei den Befragten zu ergänzen oder zu berichtigen.
                \end{enumerate}
            \item Stellen nach Absatz 2 Satz 1 sind nicht berechtigt, erhobenes Material für eigene Auswertungen zu nutzen.
        \end{enumerate}

    \section{Art. 22: Zulässigkeit}
        Gemeinden und Gemeindeverbände sowie andere nichtstaatliche juristische Personen des öffentlichen Rechts können für die Wahrnehmung ihrer Aufgaben im Rahmen ihrer Zuständigkeiten und Befugnisse Statistiken durchführen, wenn Einzelangaben oder Ergebnisse vom Landesamt oder von anderen öffentlichen Stellen weder zur Verfügung gestellt noch anderweitig ermittelt werden können und eigene Statistikstellen eingerichtet werden.

    \section{Art. 23: Anordnung}
        \begin{enumerate}[label=(\arabic*)]
            \item \textsuperscript{1}Statistiken für die Wahrnehmung von Selbstverwaltungsaufgaben (eigener Wirkungskreis) sind durch Satzung anzuordnen; in ihr sind zugleich die erforderlichen Bestimmungen nach Art. 9 Abs. 2 zu treffen. \textsuperscript{2}Die Anordnung bedarf keiner Satzung, wenn
                \begin{enumerate}[label=\arabic*.]
                    \item die einer Statistik zugrundeliegenden Daten auf allgemein zugänglichen Quellen beruhen, keine Einzelangaben enthalten, der Statistikstelle rechtmässig übermittelt werden oder ihrem Zugriff auf Grund einer Rechtsvorschrift zur Verfügung stehen oder
                    \item lediglich Sonderauswertungen vorhandenen statistischen Materials vorgenommen werden, dessen Verwendung eine Zweckbindung nicht entgegensteht.
                \end{enumerate}
                \textsuperscript{3}Bei juristischen Personen, denen kein Satzungsrecht zusteht, werden Statistiken durch die zuständigen Organe angeordnet. \textsuperscript{4}Durch Satzungen können Gemeinden, Landkreise und Bezirke auch eine Auskunftspflicht begründen, wenn es der Zweck der Erhebung erfordert, und zulassen, dass Statistikstellen Adressdateien in entsprechender Anwendung der für amtliche Statistiken geltenden Vorschriften führen und nutzen.
            \item \textsuperscript{1}Statistiken für die Wahrnehmung von übertragenen Aufgaben (übertragener Wirkungskreis) bedürfen einer Anordnung durch Gesetz oder staatliche Rechtsverordnung, es sei denn, es liegen die Voraussetzungen vor, unter denen auch für eine amtliche Statistik keine Anordnung durch Rechtsvorschrift erforderlich ist (Art. 9 Abs. 1 Satz 2). \textsuperscript{2}In den Fällen des Art. 9 Abs. 1 Satz 2 Nr. 1 bedarf die Statistik einer Genehmigung durch den Statistischen Genehmigungsausschuss (Art. 10).
        \end{enumerate}

    \section{Art. 24: Statistikstellen}
        \begin{enumerate}[label=(\arabic*)]
            \item \textsuperscript{1}Statistikstellen führen die angeordneten Statistiken durch. \textsuperscript{2}In die Wahrnehmung nichtstatistischer Aufgaben des Verwaltungsvollzugs dürfen Statistikstellen nicht eingeschaltet werden. \textsuperscript{3}Statistikstellen veröffentlichen die Ergebnisse der von ihnen erstellten Statistiken oder stellen sie in sonstiger Weise bereit, wenn ein öffentliches Bedürfnis besteht.
            \item \textsuperscript{1}Statistikstellen sind durch Satzung einzurichten, die auch die wesentlichen organisatorischen Bestimmungen, vornehmlich zur Wahrung des Statistikgeheimnisses zu treffen hat. \textsuperscript{2}Art. 20 Abs. 2 und 3 finden entsprechende Anwendung. \textsuperscript{3}Kommunale Statistikstellen können auch nach Massgabe des Gesetzes über die kommunale Zusammenarbeit eingerichtet werden. \textsuperscript{4}Bei juristischen Personen, denen kein Satzungsrecht zukommt, werden Statistikstellen durch die zuständigen Organe nach Massgabe der Sätze 1 und 2 eingerichtet.
            \item  \textsuperscript{2}Geschäftsstatistiken führen die Statistikstellen durch, wenn sie damit beauftragt werden. \textsuperscript{1}Kommunale Statistikstellen können die Ergebnisse von Europa-, Bundes-, Landes- und Kommunalwahlen aufbereiten. \textsuperscript{3}Sie nehmen die Aufgaben einer Erhebungsstelle im Sinn des Art. 21 Abs. 2 Satz 1 wahr.
        \end{enumerate}
        
    \section{Art.34: Reidentifizierungsverbot}
        Die Zusammenführung
        \begin{enumerate}[label=\arabic*.]
            \item von Einzelangaben aus Statistiken öffentlicher Stellen oder
            \item von Einzelangaben aus Statistiken öffentlicher Stellen mit anderen Angaben
        \end{enumerate}
         zum Zweck der Herstellung eines Personen-, Unternehmens-, Betriebs- oder Ar\-beits\-stät\-ten\-be\-zugs ist untersagt, es sei denn, die Aufgabenstellung dieses Gesetzes oder einer anderen Rechtsvorschrift oder ein sonstiger eine Statistik einer öffentlichen Stelle anordnender Rechtsakt lassen das zu. 
     
    \section{Art. 35: Strafvorschrift}
        Wer entgegen Art. 34 Einzelangaben aus Statistiken öffentlicher Stellen oder solche Einzelangaben mit anderen Angaben zusammenführt, wird mit Freiheitsstrafe bis zu einem Jahr oder mit Geldstrafe bestraft.

\chapter{LStatG NRW}{Statistikgesetz Nordrhein-Westfalen}
\qrcode{https://recht.nrw.de/lmi/owa/br_text_anzeigen?v_id=92220190717090732842}
\newline
\url{https://recht.nrw.de/lmi/owa/br_text_anzeigen?v_id=92220190717090732842}




\chapter{EU-DSGVO}
\qrcode{https://eur-lex.europa.eu/legal-content/DE/TXT/PDF/?uri=CELEX:32016R0679&from=DE}
\newline
\url{https://eur-lex.europa.eu/legal-content/DE/TXT/PDF/?uri=CELEX:32016R0679&from=DE}
    \section{Art. 1: Gegenstand und Ziele}
        \begin{enumerate}[label=(\arabic*)]
            \item Diese Verordnung enthält Vorschriften zum Schutz natürlicher Personen bei der Verarbeitung personenbezogener Daten und zum freien Verkehr solcher Daten.
            \item Diese Verordnung schützt die Grundrechte und Grundfreiheiten natürlicher Personen und insbesondere deren Recht auf Schutz personenbezogener Daten.
            \item Der freie Verkehr personenbezogener Daten in der Union darf aus Gründen des Schutzes natürlicher Personen bei der Verarbeitung personenbezogener Daten weder eingeschränkt noch verboten werden. 
        \end{enumerate}


    \section{Art. 4: Begriffsbestimmungen}
    Im Sinne dieser Verordnung bezeichnet der Ausdruck:
        \begin{enumerate}[label=\arabic*.]
            \item ``personenbezogene Daten'' alle Informationen, die sich auf eine identifizierte oder identifizierbare natürliche Person (im Folgenden ``betroffene Person'') beziehen; als identifizierbar wird eine natürliche Person angesehen, die direkt oder indirekt, insbesondere mittels Zuordnung zu einer Kennung wie einem Namen, zu einer Kennnummer, zu Standortdaten, zu einer Online-Kennung oder zu einem oder mehreren besonderen Merkmalen, die Ausdruck der physischen, physiologischen, genetischen, psychischen, wirtschaftlichen, kulturellen oder sozialen Identität dieser natürlichen Person sind, identifiziert werden kann; 
            \item ``Verarbeitung'' jeden mit oder ohne Hilfe automatisierter Verfahren ausgeführten Vorgang oder jede solche Vorgangsreihe im Zusammenhang mit personenbezogenen Daten wie das Erheben, das Erfassen, die Organisation, das Ordnen, die Speicherung, die Anpassung oder Veränderung, das Auslesen, das Abfragen, die Verwendung, die Offenlegung durch Übermittlung, Verbreitung oder eine andere Form der Bereitstellung, den Abgleich oder die Verknüpfung, die Einschränkung, das Löschen oder die Vernichtung;
            \item ``Einschränkung der Verarbeitung'' die Markierung gespeicherter personenbezogener Daten mit dem Ziel, ihre künftige Verarbeitung einzuschränken;
            \item ``Profiling'' jede Art der automatisierten Verarbeitung personenbezogener Daten, die darin besteht, dass diese personenbezogenen Daten verwendet werden, um bestimmte persönliche Aspekte, die sich auf eine natürliche Person beziehen, zu bewerten, insbesondere um Aspekte bezüglich Arbeitsleistung, wirtschaftliche Lage, Gesundheit, persönliche Vorlieben, Interessen, Zuverlässigkeit, Verhalten, Aufenthaltsort oder Ortswechsel dieser natürlichen Person zu analysieren oder vorherzusagen;
            \item ``Pseudonymisierung'' die Verarbeitung personenbezogener Daten in einer Weise, dass die personenbezogenen Daten ohne Hinzuziehung zusätzlicher Informationen nicht mehr einer spezifischen betroffenen Person zugeordnet werden können, sofern diese zusätzlichen Informationen gesondert aufbewahrt werden und technischen und organisatorischen Maßnahmen unterliegen, die gewährleisten, dass die personenbezogenen Daten nicht einer identifizierten oder identifizierbaren natürlichen Person zugewiesen werden; 
            \item ``Dateisystem'' jede strukturierte Sammlung personenbezogener Daten, die nach bestimmten Kriterien zugänglich sind, unabhängig davon, ob diese Sammlung zentral, dezentral oder nach funktionalen oder geografischen Gesichtspunkten geordnet geführt wird;
            \item ``Verantwortlicher'' die natürliche oder juristische Person, Behörde, Einrichtung oder andere Stelle, die allein oder gemeinsam mit anderen über die Zwecke und Mittel der Verarbeitung von personenbezogenen Daten entscheidet; sind die Zwecke und Mittel dieser Verarbeitung durch das Unionsrecht oder das Recht der Mitgliedstaaten vorgegeben, so kann der Verantwortliche beziehungsweise können die bestimmten Kriterien seiner Benennung nach dem Unionsrecht oder dem Recht der Mitgliedstaaten vorgesehen werden;
            \item ``Auftragsverarbeiter'' eine natürliche oder juristische Person, Behörde, Einrichtung oder andere Stelle, die personenbezogene Daten im Auftrag des Verantwortlichen verarbeitet;
            \item ``Empfänger'' eine natürliche oder juristische Person, Behörde, Einrichtung oder andere Stelle, der personenbezogene Daten offengelegt werden, unabhängig davon, ob es sich bei ihr um einen Dritten handelt oder nicht. Behörden, die im Rahmen eines bestimmten Untersuchungsauftrags nach dem Unionsrecht oder dem Recht der Mitgliedstaaten möglicherweise personenbezogene Daten erhalten, gelten jedoch nicht als Empfänger; die Verarbeitung dieser Daten durch die genannten Behörden erfolgt im Einklang mit den geltenden Datenschutzvorschriften gemäß den Zwecken der Verarbeitung;
            \item  ``Dritter'' eine natürliche oder juristische Person, Behörde, Einrichtung oder andere Stelle, außer der betroffenen Person, dem Verantwortlichen, dem Auftragsverarbeiter und den Personen, die unter der unmittelbaren Verantwortung des Verantwortlichen oder des Auftragsverarbeiters befugt sind, die personenbezogenen Daten zu verarbeiten;
            \item ``Einwilligung'' der betroffenen Person jede freiwillig für den bestimmten Fall, in informierter Weise und unmissverständlich abgegebene Willensbekundung in Form einer Erklärung oder einer sonstigen eindeutigen bestätigenden Handlung, mit der die betroffene Person zu verstehen gibt, dass sie mit der Verarbeitung der sie betreffenden personenbezogenen Daten einverstanden ist;
            \item ``Verletzung des Schutzes personenbezogener Daten'' eine Verletzung der Sicherheit, die, ob unbeabsichtigt oder unrechtmäßig, zur Vernichtung, zum Verlust, zur Veränderung, oder zur unbefugten Offenlegung von beziehungsweise zum unbefugten Zugang zu personenbezogenen Daten führt, die übermittelt, gespeichert oder auf sonstige Weise verarbeitet wurden;
            \item ``genetische Daten'' personenbezogene Daten zu den ererbten oder erworbenen genetischen Eigenschaften einer natürlichen Person, die eindeutige Informationen über die Physiologie oder die Gesundheit dieser natürlichen Person liefern und insbesondere aus der Analyse einer biologischen Probe der betreffenden natürlichen Person gewonnen wurden;
            \item ``biometrische Daten'' mit speziellen technischen Verfahren gewonnene personenbezogene Daten zu den physischen, physiologischen oder verhaltenstypischen Merkmalen einer natürlichen Person, die die eindeutige Identifizierung dieser natürlichen Person ermöglichen oder bestätigen, wie Gesichtsbilder oder daktyloskopische Daten;
            \item ``Gesundheitsdaten'' personenbezogene Daten, die sich auf die körperliche oder geistige Gesundheit einer natürlichen Person, einschließlich der Erbringung von Gesundheitsdienstleistungen, beziehen und aus denen Informationen über deren Gesundheitszustand hervorgehen;
            \item ``Hauptniederlassung''
                \begin{enumerate}[label=\alph*)]
                    \item im Falle eines Verantwortlichen mit Niederlassungen in mehr als einem Mitgliedstaat den Ort seiner Hauptverwaltung in der Union, es sei denn, die Entscheidungen hinsichtlich der Zwecke und Mittel der Verarbeitung personenbezogener Daten werden in einer anderen Niederlassung des Verantwortlichen in der Union getroffen und diese Niederlassung ist befugt, diese Entscheidungen umsetzen zu lassen; in diesem Fall gilt die Niederlassung, die derartige Entscheidungen trifft, als Hauptniederlassung;
                    \item im Falle eines Auftragsverarbeiters mit Niederlassungen in mehr als einem Mitgliedstaat den Ort seiner Hauptverwaltung in der Union oder, sofern der Auftragsverarbeiter keine Hauptverwaltung in der Union hat, die Niederlassung des Auftragsverarbeiters in der Union, in der die Verarbeitungstätigkeiten im Rahmen der Tätigkeiten einer Niederlassung eines Auftragsverarbeiters hauptsächlich stattfinden, soweit der Auftragsverarbeiter spezifischen Pflichten aus dieser Verordnung unterliegt;
                \end{enumerate} 
            \item ``Vertreter'' eine in der Union niedergelassene natürliche oder juristische Person, die von dem Verantwortlichen oder Auftragsverarbeiter schriftlich gemäß Artikel 27 bestellt wurde und den Verantwortlichen oder Auftragsverarbeiter in Bezug auf die ihnen jeweils nach dieser Verordnung obliegenden Pflichten vertritt;
            \item ``Unternehmen'' eine natürliche und juristische Person, die eine wirtschaftliche Tätigkeit ausübt, unabhängig von ihrer Rechtsform, einschließlich Personengesellschaften oder Vereinigungen, die regelmäßig einer wirtschaftlichen Tätigkeit nachgehen;
            \item ``Unternehmensgruppe'' eine Gruppe, die aus einem herrschenden Unternehmen und den von diesem abhängigen Unternehmen besteht; 
            \item ``verbindliche interne Datenschutzvorschriften'' Maßnahmen zum Schutz personenbezogener Daten, zu deren Einhaltung sich ein im Hoheitsgebiet eines Mitgliedstaats niedergelassener Verantwortlicher oder Auftragsverarbeiter verpflichtet im Hinblick auf Datenübermittlungen oder eine Kategorie von Datenübermittlungen personenbezogener Daten an einen Verantwortlichen oder Auftragsverarbeiter derselben Unternehmensgruppe oder derselben Gruppe von Unternehmen, die eine gemeinsame Wirtschaftstätigkeit ausüben, in einem oder mehreren Drittländern;
            \item ``Aufsichtsbehörde'' eine von einem Mitgliedstaat gemäß Artikel 51 eingerichtete unabhängige staatliche Stelle; 
            \item ``betroffene Aufsichtsbehörde'' eine Aufsichtsbehörde, die von der Verarbeitung personenbezogener Daten betroffen ist, weil
                \begin{enumerate}[label=\alph*)]
                    \item der Verantwortliche oder der Auftragsverarbeiter im Hoheitsgebiet des Mitgliedstaats dieser Aufsichtsbehörde niedergelassen ist,
                    \item diese Verarbeitung erhebliche Auswirkungen auf betroffene Personen mit Wohnsitz im Mitgliedstaat dieser Aufsichtsbehörde hat oder haben kann oder
                    \item eine Beschwerde bei dieser Aufsichtsbehörde eingereicht wurde;
                \end{enumerate}
            \item ``grenzüberschreitende Verarbeitung'' entweder
                \begin{enumerate}[label=\alph*)]
                    \item eine Verarbeitung personenbezogener Daten, die im Rahmen der Tätigkeiten von Niederlassungen eines Verantwortlichen oder eines Auftragsverarbeiters in der Union in mehr als einem Mitgliedstaat erfolgt, wenn der Verantwortliche oder Auftragsverarbeiter in mehr als einem Mitgliedstaat niedergelassen ist, oder
                    \item eine Verarbeitung personenbezogener Daten, die im Rahmen der Tätigkeiten einer einzelnen Niederlassung eines Verantwortlichen oder eines Auftragsverarbeiters in der Union erfolgt, die jedoch erhebliche Auswirkungen auf betroffene Personen in mehr als einem Mitgliedstaat hat oder haben kann;
                \end{enumerate}              
            \item ``maßgeblicher und begründeter Einspruch'' einen Einspruch gegen einen Beschlussentwurf im Hinblick darauf, ob ein Verstoß gegen diese Verordnung vorliegt oder ob beabsichtigte Maßnahmen gegen den Verantwortlichen oder den Auftragsverarbeiter im Einklang mit dieser Verordnung steht, wobei aus diesem Einspruch die Tragweite der Risiken klar hervorgeht, die von dem Beschlussentwurf in Bezug auf die Grundrechte und Grundfreiheiten der betroffenen Personen und gegebenenfalls den freien Verkehr personenbezogener Daten in der Union ausgehen;
            \item ``Dienst der Informationsgesellschaft'' eine Dienstleistung im Sinne des Artikels 1 Nummer 1 Buchstabe b der Richtlinie (EU) 2015/1535 des Europäischen Parlaments und des Rates 
            \item ``internationale Organisation'' eine völkerrechtliche Organisation und ihre nachgeordneten Stellen oder jede sonstige Einrichtung, die durch eine zwischen zwei oder mehr Ländern geschlossene Übereinkunft oder auf der Grundlage einer solchen Übereinkunft geschaffen wurde. 
        \end{enumerate}
    \section{Art. 5: Grundsätze für die Verarbeitung personenbezogener Daten}
        \begin{enumerate}[label=(\arabic*)]
            \item Personenbezogene Daten müssen
                \begin{enumerate}[label=\alph*)]
                    \item auf rechtmäßige Weise, nach Treu und Glauben und in einer für die betroffene Person nachvollziehbaren Weise verarbeitet werden (``Rechtmäßigkeit, Verarbeitung nach Treu und Glauben, Transparenz'');
                    \item für festgelegte, eindeutige und legitime Zwecke erhoben werden und dürfen nicht in einer mit diesen Zwecken nicht zu vereinbarenden Weise weiterverarbeitet werden; eine Weiterverarbeitung für im öffentlichen Interesse liegende Archivzwecke, für wissenschaftliche oder historische Forschungszwecke oder für statistische Zwecke gilt gemäß Artikel 89 Absatz 1 nicht als unvereinbar mit den ursprünglichen Zwecken (``Zweckbindung'');
                    \item dem Zweck angemessen und erheblich sowie auf das für die Zwecke der Verarbeitung notwendige Maß beschränkt sein (``Datenminimierung''); 
                    \item sachlich richtig und erforderlichenfalls auf dem neuesten Stand sein; es sind alle angemessenen Maßnahmen zu treffen, damit personenbezogene Daten, die im Hinblick auf die Zwecke ihrer Verarbeitung unrichtig sind, unverzüglich gelöscht oder berichtigt werden (``Richtigkeit'');
                    \item in einer Form gespeichert werden, die die Identifizierung der betroffenen Personen nur so lange ermöglicht, wie es für die Zwecke, für die sie verarbeitet werden, erforderlich ist; personenbezogene Daten dürfen länger gespeichert werden, soweit die personenbezogenen Daten vorbehaltlich der Durchführung geeigneter technischer und organisatorischer Maßnahmen, die von dieser Verordnung zum Schutz der Rechte und Freiheiten der betroffenen Person gefordert werden, ausschließlich für im öffentlichen Interesse liegende Archivzwecke oder für wissenschaftliche und historische Forschungszwecke oder für statistische Zwecke gemäß Artikel 89 Absatz 1 verarbeitet werden (``Speicherbegrenzung'');
                    \item in einer Weise verarbeitet werden, die eine angemessene Sicherheit der personenbezogenen Daten gewährleistet, einschließlich Schutz vor unbefugter oder unrechtmäßiger Verarbeitung und vor unbeabsichtigtem Verlust, unbeabsichtigter Zerstörung oder unbeabsichtigter Schädigung durch geeignete technische und organisatorische Maßnahmen (``Integrität und Vertraulichkeit'');
                \end{enumerate}
            \item Der Verantwortliche ist für die Einhaltung des Absatzes 1 verantwortlich und muss dessen Einhaltung nachweisen können (``Rechenschaftspflicht'')
        \end{enumerate}
    \section[Art 9: Verarbeitung besonderer Daten]{Art. 9: Verarbeitung besonderer Kategorien personenbezogener Daten}
        \begin{enumerate}
            \item Die Verarbeitung personenbezogener Daten, aus denen die rassische und ethnische Herkunft, politische Meinungen, religiöse oder weltanschauliche Überzeugungen oder die Gewerkschaftszugehörigkeit hervorgehen, sowie die Verarbeitung von genetischen Daten, biometrischen Daten zur eindeutigen Identifizierung einer natürlichen Person, Gesundheitsdaten oder Daten zum Sexualleben oder der sexuellen Orientierung einer natürlichen Person ist untersagt.
            \item Absatz 1 gilt nicht in folgenden Fällen:
                \begin{enumerate}[label=\alph*)]
                    \item Die betroffene Person hat in die Verarbeitung der genannten personenbezogenen Daten für einen oder mehrere festgelegte Zwecke ausdrücklich eingewilligt, es sei denn, nach Unionsrecht oder dem Recht der Mitgliedstaaten kann das Verbot nach Absatz 1 durch die Einwilligung der betroffenen Person nicht aufgehoben werden,
                    \item die Verarbeitung ist erforderlich, damit der Verantwortliche oder die betroffene Person die ihm bzw. ihr aus dem Arbeitsrecht und dem Recht der sozialen Sicherheit und des Sozialschutzes erwachsenden Rechte ausüben und seinen bzw. ihren diesbezüglichen Pflichten nachkommen kann, soweit dies nach Unionsrecht oder dem Recht der Mitgliedstaaten oder einer Kollektivvereinbarung nach dem Recht der Mitgliedstaaten, das geeignete Garantien für die Grundrechte und die Interessen der betroffenen Person vorsieht, zulässig ist,
                    \item die Verarbeitung ist zum Schutz lebenswichtiger Interessen der betroffenen Person oder einer anderen natürlichen Person erforderlich und die betroffene Person ist aus körperlichen oder rechtlichen Gründen außerstande, ihre Einwilligung zu geben, 
                    \item die Verarbeitung erfolgt auf der Grundlage geeigneter Garantien durch eine politisch, weltanschaulich, religiös oder gewerkschaftlich ausgerichtete Stiftung, Vereinigung oder sonstige Organisation ohne Gewinnerzielungsabsicht im Rahmen ihrer rechtmäßigen Tätigkeiten und unter der Voraussetzung, dass sich die Verarbeitung ausschließlich auf die Mitglieder oder ehemalige Mitglieder der Organisation oder auf Personen, die im Zusammenhang mit deren Tätigkeitszweck regelmäßige Kontakte mit ihr unterhalten, bezieht und die personenbezogenen Daten nicht ohne Einwilligung der betroffenen Personen nach außen offengelegt werden,
                    \item die Verarbeitung bezieht sich auf personenbezogene Daten, die die betroffene Person offensichtlich öffentlich gemacht hat,
                    \item die Verarbeitung ist zur Geltendmachung, Ausübung oder Verteidigung von Rechtsansprüchen oder bei Handlungen der Gerichte im Rahmen ihrer justiziellen Tätigkeit erforderlich, 
                    \item die Verarbeitung ist auf der Grundlage des Unionsrechts oder des Rechts eines Mitgliedstaats, das in angemessenem Verhältnis zu dem verfolgten Ziel steht, den Wesensgehalt des Rechts auf Datenschutz wahrt und angemessene und spezifische Maßnahmen zur Wahrung der Grundrechte und Interessen der betroffenen Person vorsieht, aus Gründen eines erheblichen öffentlichen Interesses erforderlich,
                    \item die Verarbeitung ist für Zwecke der Gesundheitsvorsorge oder der Arbeitsmedizin, für die Beurteilung der Arbeitsfähigkeit des Beschäftigten, für die medizinische Diagnostik, die Versorgung oder Behandlung im Gesundheitsoder Sozialbereich oder für die Verwaltung von Systemen und Diensten im Gesundheits- oder Sozialbereich auf der Grundlage des Unionsrechts oder des Rechts eines Mitgliedstaats oder aufgrund eines Vertrags mit einem Angehörigen eines Gesundheitsberufs und vorbehaltlich der in Absatz 3 genannten Bedingungen und Garantien erforderlich,
                    \item die Verarbeitung ist aus Gründen des öffentlichen Interesses im Bereich der öffentlichen Gesundheit, wie dem Schutz vor schwerwiegenden grenzüberschreitenden Gesundheitsgefahren oder zur Gewährleistung hoher Qualitäts- und Sicherheitsstandards bei der Gesundheitsversorgung und bei Arzneimitteln und Medizinprodukten, auf der Grundlage des Unionsrechts oder des Rechts eines Mitgliedstaats, das angemessene und spezifische Maßnahmen zur Wahrung der Rechte und Freiheiten der betroffenen Person, insbesondere des Berufsgeheimnisses, vorsieht, erforderlich, oder
                    \item die Verarbeitung ist auf der Grundlage des Unionsrechts oder des Rechts eines Mitgliedstaats, das in angemessenem Verhältnis zu dem verfolgten Ziel steht, den Wesensgehalt des Rechts auf Datenschutz wahrt und angemessene und spezifische Maßnahmen zur Wahrung der Grundrechte und Interessen der betroffenen Person vorsieht, für im öffentlichen Interesse liegende Archivzwecke, für wissenschaftliche oder historische Forschungszwecke oder für statistische Zwecke gemäß Artikel 89 Absatz 1 erforderlich.
                \end{enumerate}
            \item Die in Absatz 1 genannten personenbezogenen Daten dürfen zu den in Absatz 2 Buchstabe h genannten Zwecken verarbeitet werden, wenn diese Daten von Fachpersonal oder unter dessen Verantwortung verarbeitet werden und dieses Fachpersonal nach dem Unionsrecht oder dem Recht eines Mitgliedstaats oder den Vorschriften nationaler zuständiger Stellen dem Berufsgeheimnis unterliegt, oder wenn die Verarbeitung durch eine andere Person erfolgt, die ebenfalls nach dem Unionsrecht oder dem Recht eines Mitgliedstaats oder den Vorschriften nationaler zuständiger Stellen einer Geheimhaltungspflicht unterliegt.
            \item Die Mitgliedstaaten können zusätzliche Bedingungen, einschließlich Beschränkungen, einführen oder aufrechterhalten, soweit die Verarbeitung von genetischen, biometrischen oder Gesundheitsdaten betroffen ist. 
        \end{enumerate}
    \section[Art. 14: Informationspflicht]{Art. 14: Informationspflicht, wenn die personenbezogenen Daten nicht bei der betroffenen Person
erhoben wurden}
    \begin{enumerate}[label=(\arabic*)]
        \item Werden personenbezogene Daten nicht bei der betroffenen Person erhoben, so teilt der Verantwortliche der betroffenen Person Folgendes mit:
            \begin{enumerate}[label=\alph*)]
                \item den Namen und die Kontaktdaten des Verantwortlichen sowie gegebenenfalls seines Vertreters;
                \item zusätzlich die Kontaktdaten des Datenschutzbeauftragten;
                \item die Zwecke, für die die personenbezogenen Daten verarbeitet werden sollen, sowie die Rechtsgrundlage für die
Verarbeitung;
                \item die Kategorien personenbezogener Daten, die verarbeitet werden;
                \item gegebenenfalls die Empfänger oder Kategorien von Empfängern der personenbezogenen Daten;
                \item gegebenenfalls die Absicht des Verantwortlichen, die personenbezogenen Daten an einen Empfänger in einem Drittland oder einer internationalen Organisation zu übermitteln, sowie das Vorhandensein oder das Fehlen eines Angemessenheitsbeschlusses der Kommission oder im Falle von Übermittlungen gemäß Artikel 46 oder Artikel 47 oder Artikel 49 Absatz 1 Unterabsatz 2 einen Verweis auf die geeigneten oder angemessenen Garantien und die Möglichkeit, eine Kopie von ihnen zu erhalten, oder wo sie verfügbar sind. 
            \end{enumerate}
        \item Zusätzlich zu den Informationen gemäß Absatz 1 stellt der Verantwortliche der betroffenen Person die folgenden Informationen zur Verfügung, die erforderlich sind, um der betroffenen Person gegenüber eine faire und transparente Verarbeitung zu gewährleisten:
            \begin{enumerate}[label=\alph*)]
                \item die Dauer, für die die personenbezogenen Daten gespeichert werden oder, falls dies nicht möglich ist, die Kriterien für die Festlegung dieser Dauer;
                \item wenn die Verarbeitung auf Artikel 6 Absatz 1 Buchstabe f beruht, die berechtigten Interessen, die von dem Verantwortlichen oder einem Dritten verfolgt werden;
                \item das Bestehen eines Rechts auf Auskunft seitens des Verantwortlichen über die betreffenden personenbezogenen Daten sowie auf Berichtigung oder Löschung oder auf Einschränkung der Verarbeitung und eines Widerspruchsrechts gegen die Verarbeitung sowie des Rechts auf Datenübertragbarkeit; 
                \item wenn die Verarbeitung auf Artikel 6 Absatz 1 Buchstabe a oder Artikel 9 Absatz 2 Buchstabe a beruht, das Bestehen eines Rechts, die Einwilligung jederzeit zu widerrufen, ohne dass die Rechtmäßigkeit der aufgrund der Einwilligung bis zum Widerruf erfolgten Verarbeitung berührt wird;
                \item das Bestehen eines Beschwerderechts bei einer Aufsichtsbehörde;
                \item aus welcher Quelle die personenbezogenen Daten stammen und gegebenenfalls ob sie aus öffentlich zugänglichen Quellen stammen;
                \item das Bestehen einer automatisierten Entscheidungsfindung einschließlich Profiling gemäß Artikel 22 Absätze 1 und 4 und — zumindest in diesen Fällen — aussagekräftige Informationen über die involvierte Logik sowie die Tragweite und die angestrebten Auswirkungen einer derartigen Verarbeitung für die betroffene Person.
            \end{enumerate}
        \item Der Verantwortliche erteilt die Informationen gemäß den Absätzen 1 und 2
            \begin{enumerate}
                \item unter Berücksichtigung der spezifischen Umstände der Verarbeitung der personenbezogenen Daten innerhalb einer angemessenen Frist nach Erlangung der personenbezogenen Daten, längstens jedoch innerhalb eines Monats, 
                \item falls die personenbezogenen Daten zur Kommunikation mit der betroffenen Person verwendet werden sollen, spätestens zum Zeitpunkt der ersten Mitteilung an sie, oder, 
                \item falls die Offenlegung an einen anderen Empfänger beabsichtigt ist, spätestens zum Zeitpunkt der ersten Offenlegung.
            \end{enumerate}
        \item Beabsichtigt der Verantwortliche, die personenbezogenen Daten für einen anderen Zweck weiterzuverarbeiten als den, für den die personenbezogenen Daten erlangt wurden, so stellt er der betroffenen Person vor dieser Weiterverarbeitung Informationen über diesen anderen Zweck und alle anderen maßgeblichen Informationen gemäß Absatz 2 zur Verfügung.
        \item Die Absätze 1 bis 4 finden keine Anwendung, wenn und soweit
        \begin{enumerate}[label=\alph*)]
            \item die betroffene Person bereits über die Informationen verfügt,
            \item die Erteilung dieser Informationen sich als unmöglich erweist oder einen unverhältnismäßigen Aufwand erfordern würde; dies gilt insbesondere für die Verarbeitung für im öffentlichen Interesse liegende Archivzwecke, für wissenschaftliche oder historische Forschungszwecke oder für statistische Zwecke vorbehaltlich der in Artikel 89 Absatz 1 genannten Bedingungen und Garantien oder soweit die in Absatz 1 des vorliegenden Artikels genannte Pflicht voraussichtlich die Verwirklichung der Ziele dieser Verarbeitung unmöglich macht oder ernsthaft beeinträchtigt In diesen Fällen ergreift der Verantwortliche geeignete Maßnahmen zum Schutz der Rechte und Freiheiten sowie der berechtigten Interessen der betroffenen Person, einschließlich der Bereitstellung dieser Informationen für die Öffentlichkeit,
            \item die Erlangung oder Offenlegung durch Rechtsvorschriften der Union oder der Mitgliedstaaten, denen der Verantwortliche unterliegt und die geeignete Maßnahmen zum Schutz der berechtigten Interessen der betroffenen Person vorsehen, ausdrücklich geregelt ist oder
            \item die personenbezogenen Daten gemäß dem Unionsrecht oder dem Recht der Mitgliedstaaten dem Berufsgeheimnis, einschließlich einer satzungsmäßigen Geheimhaltungspflicht, unterliegen und daher vertraulich behandelt werden müssen. 
        \end{enumerate}
    \end{enumerate}

    \section[Art. 15: Auskunftsrecht]{Art. 15: Auskunftsrecht der betroffenen Person}
        \begin{enumerate}[label=(\arabic*)]
            \item Die betroffene Person hat das Recht, von dem Verantwortlichen eine Bestätigung darüber zu verlangen, ob sie betreffende personenbezogene Daten verarbeitet werden; ist dies der Fall, so hat sie ein Recht auf Auskunft über diese personenbezogenen Daten und auf folgende Informationen:
                \begin{enumerate}[label=\alph*)]
                    \item die Verarbeitungszwecke;
                    \item die Kategorien personenbezogener Daten, die verarbeitet werden;
                    \item die Empfänger oder Kategorien von Empfängern, gegenüber denen die personenbezogenen Daten offengelegt worden sind oder noch offengelegt werden, insbesondere bei Empfängern in Drittländern oder bei internationalen Organisationen;
                    \item falls möglich die geplante Dauer, für die die personenbezogenen Daten gespeichert werden, oder, falls dies nicht möglich ist, die Kriterien für die Festlegung dieser Dauer;
                    \item das Bestehen eines Rechts auf Berichtigung oder Löschung der sie betreffenden personenbezogenen Daten oder auf Einschränkung der Verarbeitung durch den Verantwortlichen oder eines Widerspruchsrechts gegen diese Verarbeitung;
                    \item das Bestehen eines Beschwerderechts bei einer Aufsichtsbehörde;
                    \item wenn die personenbezogenen Daten nicht bei der betroffenen Person erhoben werden, alle verfügbaren Informationen über die Herkunft der Daten;
                    \item das Bestehen einer automatisierten Entscheidungsfindung einschließlich Profiling gemäß Artikel 22 Absätze 1 und 4 und — zumindest in diesen Fällen — aussagekräftige Informationen über die involvierte Logik sowie die Tragweite und die angestrebten Auswirkungen einer derartigen Verarbeitung für die betroffene Person.
                \end{enumerate}
            \item Werden personenbezogene Daten an ein Drittland oder an eine internationale Organisation übermittelt, so hat die betroffene Person das Recht, über die geeigneten Garantien gemäß Artikel 46 im Zusammenhang mit der Übermittlung unterrichtet zu werden.
            \item Der Verantwortliche stellt eine Kopie der personenbezogenen Daten, die Gegenstand der Verarbeitung sind, zur Verfügung. Für alle weiteren Kopien, die die betroffene Person beantragt, kann der Verantwortliche ein angemessenes Entgelt auf der Grundlage der Verwaltungskosten verlangen. Stellt die betroffene Person den Antrag elektronisch, so sind die Informationen in einem gängigen elektronischen Format zur Verfügung zu stellen, sofern sie nichts anderes angibt.
            \item Das Recht auf Erhalt einer Kopie gemäß Absatz 1b darf die Rechte und Freiheiten anderer Personen nicht beeinträchtigen. 
        \end{enumerate}

    \section{Art. 16: Recht auf Berichtigung}
        Die betroffene Person hat das Recht, von dem Verantwortlichen unverzüglich die Berichtigung sie betreffender unrichtiger personenbezogener Daten zu verlangen. Unter Berücksichtigung der Zwecke der Verarbeitung hat die betroffene Person das Recht, die Vervollständigung unvollständiger personenbezogener Daten — auch mittels einer ergänzenden Erklärung — zu verlangen.



    \section[Art. 17 : Recht auf Löschung]{Art. 17: Recht auf Löschung (``Recht auf Vergessenwerden'')}
        \begin{enumerate}[label=(\arabic*)]
            \item Die betroffene Person hat das Recht, von dem Verantwortlichen zu verlangen, dass sie betreffende personenbezogene Daten unverzüglich gelöscht werden, und der Verantwortliche ist verpflichtet, personenbezogene Daten unverzüglich zu löschen, sofern einer der folgenden Gründe zutrifft:
                \begin{enumerate}[label=\alph*)]
                    \item Die personenbezogenen Daten sind für die Zwecke, für die sie erhoben oder auf sonstige Weise verarbeitet wurden, nicht mehr notwendig.
                    \item Die betroffene Person widerruft ihre Einwilligung, auf die sich die Verarbeitung gemäß Artikel 6 Absatz 1 Buchstabe a oder Artikel 9 Absatz 2 Buchstabe a stützte, und es fehlt an einer anderweitigen Rechtsgrundlage für die Verarbeitung.
                    \item Die betroffene Person legt gemäß Artikel 21 Absatz 1 Widerspruch gegen die Verarbeitung ein und es liegen keine vorrangigen berechtigten Gründe für die Verarbeitung vor, oder die betroffene Person legt gemäß Artikel 21 Absatz 2 Widerspruch gegen die Verarbeitung ein.
                    \item Die personenbezogenen Daten wurden unrechtmäßig verarbeitet. 
                    \item Die Löschung der personenbezogenen Daten ist zur Erfüllung einer rechtlichen Verpflichtung nach dem Unionsrecht oder dem Recht der Mitgliedstaaten erforderlich, dem der Verantwortliche unterliegt. 
                    \item Die personenbezogenen Daten wurden in Bezug auf angebotene Dienste der Informationsgesellschaft gemäß Artikel 8 Absatz 1 erhoben.
                \end{enumerate}
            \item Hat der Verantwortliche die personenbezogenen Daten öffentlich gemacht und ist er gemäß Absatz 1 zu deren Löschung verpflichtet, so trifft er unter Berücksichtigung der verfügbaren Technologie und der Implementierungskosten angemessene Maßnahmen, auch technischer Art, um für die Datenverarbeitung Verantwortliche, die die personenbezogenen Daten verarbeiten, darüber zu informieren, dass eine betroffene Person von ihnen die Löschung aller Links zu diesen personenbezogenen Daten oder von Kopien oder Replikationen dieser personenbezogenen Daten verlangt hat.
            \item Die Absätze 1 und 2 gelten nicht, soweit die Verarbeitung erforderlich ist
                \begin{enumerate}[label=\alph*)]
                    \item zur Ausübung des Rechts auf freie Meinungsäußerung und Information;
                    \item zur Erfüllung einer rechtlichen Verpflichtung, die die Verarbeitung nach dem Recht der Union oder der Mitgliedstaaten, dem der Verantwortliche unterliegt, erfordert, oder zur Wahrnehmung einer Aufgabe, die im öffentlichen Interesse liegt oder in Ausübung öffentlicher Gewalt erfolgt, die dem Verantwortlichen übertragen wurde;
                    \item aus Gründen des öffentlichen Interesses im Bereich der öffentlichen Gesundheit gemäß Artikel 9 Absatz 2 Buchstaben h und i sowie Artikel 9 Absatz 3; 
                    \item für im öffentlichen Interesse liegende Archivzwecke, wissenschaftliche oder historische Forschungszwecke oder für statistische Zwecke gemäß Artikel 89 Absatz 1, soweit das in Absatz 1 genannte Recht voraussichtlich die Verwirklichung der Ziele dieser Verarbeitung unmöglich macht oder ernsthaft beeinträchtigt, oder 
                    \item zur Geltendmachung, Ausübung oder Verteidigung von Rechtsansprüchen. 
                \end{enumerate}
        \end{enumerate}

    \section{Art. 18: Recht auf Einschränkung der Verarbeitung}
        \begin{enumerate}[label=(\arabic*)]
            \item Die betroffene Person hat das Recht, von dem Verantwortlichen die Einschränkung der Verarbeitung zu verlangen, wenn eine der folgenden Voraussetzungen gegeben ist:
                \begin{enumerate}[label=\alph*)]
                    \item die Richtigkeit der personenbezogenen Daten von der betroffenen Person bestritten wird, und zwar für eine Dauer, die es dem Verantwortlichen ermöglicht, die Richtigkeit der personenbezogenen Daten zu überprüfen,
                    \item die Verarbeitung unrechtmäßig ist und die betroffene Person die Löschung der personenbezogenen Daten ablehnt und stattdessen die Einschränkung der Nutzung der personenbezogenen Daten verlangt; 
                    \item der Verantwortliche die personenbezogenen Daten für die Zwecke der Verarbeitung nicht länger benötigt, die betroffene Person sie jedoch zur Geltendmachung, Ausübung oder Verteidigung von Rechtsansprüchen benötigt, oder
                    \item die betroffene Person Widerspruch gegen die Verarbeitung gemäß Artikel 21 Absatz 1 eingelegt hat, solange noch nicht feststeht, ob die berechtigten Gründe des Verantwortlichen gegenüber denen der betroffenen Person überwiegen.
                \end{enumerate}
            \item Wurde die Verarbeitung gemäß Absatz 1 eingeschränkt, so dürfen diese personenbezogenen Daten — von ihrer Speicherung abgesehen — nur mit Einwilligung der betroffenen Person oder zur Geltendmachung, Ausübung oder Verteidigung von Rechtsansprüchen oder zum Schutz der Rechte einer anderen natürlichen oder juristischen Person oder aus Gründen eines wichtigen öffentlichen Interesses der Union oder eines Mitgliedstaats verarbeitet werden.
            \item Eine betroffene Person, die eine Einschränkung der Verarbeitung gemäß Absatz 1 erwirkt hat, wird von dem Verantwortlichen unterrichtet, bevor die Einschränkung aufgehoben wird. 
        \end{enumerate}

    \section{Art. 21: Widerspruchsrecht}
        \begin{enumerate}[label=(\arabic*]
            \item Die betroffene Person hat das Recht, aus Gründen, die sich aus ihrer besonderen Situation ergeben, jederzeit gegen die Verarbeitung sie betreffender personenbezogener Daten, die aufgrund von Artikel 6 Absatz 1 Buchstaben e oder f erfolgt, Widerspruch einzulegen; dies gilt auch für ein auf diese Bestimmungen gestütztes Profiling. Der Verantwortliche verarbeitet die personenbezogenen Daten nicht mehr, es sei denn, er kann zwingende schutzwürdige Gründe für die Verarbeitung nachweisen, die die Interessen, Rechte und Freiheiten der betroffenen Person überwiegen, oder die Verarbeitung dient der Geltendmachung, Ausübung oder Verteidigung von Rechtsansprüchen.
            \item Werden personenbezogene Daten verarbeitet, um Direktwerbung zu betreiben, so hat die betroffene Person das Recht, jederzeit Widerspruch gegen die Verarbeitung sie betreffender personenbezogener Daten zum Zwecke derartiger Werbung einzulegen; dies gilt auch für das Profiling, soweit es mit solcher Direktwerbung in Verbindung steht. 
            \item Widerspricht die betroffene Person der Verarbeitung für Zwecke der Direktwerbung, so werden die personenbezogenen Daten nicht mehr für diese Zwecke verarbeitet.
            \item Die betroffene Person muss spätestens zum Zeitpunkt der ersten Kommunikation mit ihr ausdrücklich auf das in den Absätzen 1 und 2 genannte Recht hingewiesen werden; dieser Hinweis hat in einer verständlichen und von anderen Informationen getrennten Form zu erfolgen.
            \item Im Zusammenhang mit der Nutzung von Diensten der Informationsgesellschaft kann die betroffene Person ungeachtet der Richtlinie 2002/58/EG ihr Widerspruchsrecht mittels automatisierter Verfahren ausüben, bei denen technische Spezifikationen verwendet werden. 
            \item Die betroffene Person hat das Recht, aus Gründen, die sich aus ihrer besonderen Situation ergeben, gegen die sie betreffende Verarbeitung sie betreffender personenbezogener Daten, die zu wissenschaftlichen oder historischen Forschungszwecken oder zu statistischen Zwecken gemäß Artikel 89 Absatz 1 erfolgt, Widerspruch einzulegen, es sei denn, die Verarbeitung ist zur Erfüllung einer im öffentlichen Interesse liegenden Aufgabe erforderlich. 
        \end{enumerate}


    \section[Art. 89: Garantien]{Art. 89: Garantien und Ausnahmen in Bezug auf die Verarbeitung zu im öffentlichen Interesse liegenden Archivzwecken, zu wissenschaftlichen oder historischen Forschungszwecken und zu statistischen Zwecken}
        \begin{enumerate}[label=(\arabic*)]
            \item Die Verarbeitung zu im öffentlichen Interesse liegenden Archivzwecken, zu wissenschaftlichen oder historischen Forschungszwecken oder zu statistischen Zwecken unterliegt geeigneten Garantien für die Rechte und Freiheiten der betroffenen Person gemäß dieser Verordnung. Mit diesen Garantien wird sichergestellt, dass technische und organisatorische Maßnahmen bestehen, mit denen insbesondere die Achtung des Grundsatzes der Datenminimierung gewährleistet wird. Zu diesen Maßnahmen kann die Pseudonymisierung gehören, sofern es möglich ist, diese Zwecke auf diese Weise zu erfüllen. In allen Fällen, in denen diese Zwecke durch die Weiterverarbeitung, bei der die Identifizierung von betroffenen Personen nicht oder nicht mehr möglich ist, erfüllt werden können, werden diese Zwecke auf diese Weise erfüllt.
            \item Werden personenbezogene Daten zu wissenschaftlichen oder historischen Forschungszwecken oder zu statistischen Zwecken verarbeitet, können vorbehaltlich der Bedingungen und Garantien gemäß Absatz 1 des vorliegenden Artikels im Unionsrecht oder im Recht der Mitgliedstaaten insoweit Ausnahmen von den Rechten gemäß der Artikel 15, 16, 18 und 21 vorgesehen werden, als diese Rechte voraussichtlich die Verwirklichung der spezifischen Zwecke unmöglich machen oder ernsthaft beeinträchtigen und solche Ausnahmen für die Erfüllung dieser Zwecke notwendig sind.
            \item Werden personenbezogene Daten für im öffentlichen Interesse liegende Archivzwecke verarbeitet, können vorbehaltlich der Bedingungen und Garantien gemäß Absatz 1 des vorliegenden Artikels im Unionsrecht oder im Recht der Mitgliedstaaten insoweit Ausnahmen von den Rechten gemäß der Artikel 15, 16, 18, 19, 20 und 21 vorgesehen werden, als diese Rechte voraussichtlich die Verwirklichung der spezifischen Zwecke unmöglich machen oder ernsthaft beeinträchtigen und solche Ausnahmen für die Erfüllung dieser Zwecke notwendig sind. 
            \item Dient die in den Absätzen 2 und 3 genannte Verarbeitung gleichzeitig einem anderen Zweck, gelten die Ausnahmen nur für die Verarbeitung zu den in diesen Absätzen genannten Zwecken. 
        \end{enumerate}
\chapter{Bundesdatenschutzgesetz}
\qrcode{https://www.gesetze-im-internet.de/bdsg_2018/BDSG.pdf}
\newline
\url{https://www.gesetze-im-internet.de/bdsg_2018/BDSG.pdf}
    \section[\S 27: Datenverarbeitung zu \dots statistischen Zwecken]{\S 27: Datenverarbeitung zu wissenschaftlichen oder historischen Forschungszwecken und zu statistischen Zwecken}
        \begin{enumerate}[label=(\arabic*)]
            \item Abweichend von Artikel 9 Absatz 1 der Verordnung (EU) 2016/679\footnote{EU-DSGVO} ist die Verarbeitung besonderer Kategorien personenbezogener Daten im Sinne des Artikels 9 Absatz 1 der Verordnung (EU) 2016/679 auch ohne Einwilligung für wissenschaftliche oder historische Forschungszwecke oder für statistische Zwecke zulässig, wenn die Verarbeitung zu diesen Zwecken erforderlich ist und die Interessen des Verantwortlichen an der Verarbeitung die Interessen der betroffenen Person an einem Ausschluss der Verarbeitung erheblich überwiegen. Der Verantwortliche sieht angemessene und spezifische Maßnahmen zur Wahrung der Interessen der betroffenen Person gemäß § 22 Absatz 2 Satz 2 vor.
            \item Die in den Artikeln 15, 16, 18 und 21 der Verordnung (EU) 2016/679 vorgesehenen Rechte der betroffenen Person sind insoweit beschränkt, als diese Rechte voraussichtlich die Verwirklichung der Forschungs- oder Statistikzwecke unmöglich machen oder ernsthaft beinträchtigen und die Beschränkung für die Erfüllung der Forschungs- oder Statistikzwecke notwendig ist. Das Recht auf Auskunft gemäß Artikel 15 der Verordnung (EU) 2016/679 besteht darüber hinaus nicht, wenn die Daten für Zwecke der wissenschaftlichen Forschung erforderlich sind und die Auskunftserteilung einen unverhältnismäßigen Aufwand erfordern würde.
            \item Ergänzend zu den in § 22 Absatz 2 genannten Maßnahmen sind zu wissenschaftlichen oder historischen Forschungszwecken oder zu statistischen Zwecken verarbeitete besondere Kategorien personenbezogener Daten im Sinne des Artikels 9 Absatz 1 der Verordnung (EU) 2016/679 zu anonymisieren, sobald dies nach dem Forschungs- oder Statistikzweck möglich ist, es sei denn, berechtigte Interessen der betroffenen Person stehen dem entgegen. Bis dahin sind die Merkmale gesondert zu speichern, mit denen Einzelangaben über persönliche oder sachliche Verhältnisse einer bestimmten oder bestimmbaren Person zugeordnet werden können. Sie dürfen mit den Einzelangaben nur zusammengeführt werden, soweit der Forschungs- oder Statistikzweck dies erfordert.
            \item Der Verantwortliche darf personenbezogene Daten nur veröffentlichen, wenn die betroffene Person eingewilligt hat oder dies für die Darstellung von Forschungsergebnissen über Ereignisse der Zeitgeschichte unerlässlich ist.
        \end{enumerate}

    \section[\S 50 Verarbeitung zu \dots statistischen Zwecken]{\S 50 Verarbeitung zu archivarischen, wissenschaftlichen  und statistischen Zwecken}
    Personenbezogene Daten dürfen im Rahmen der in § 45 genannten Zwecke in archivarischer, wissenschaftlicher oder statistischer Form verarbeitet werden, wenn hieran ein öffentliches Interesse besteht und geeignete Garantien für die Rechtsgüter der betroffenen Personen vorgesehen werden. Solche Garantien können in einer so zeitnah wie möglich erfolgenden Anonymisierung der personenbezogenen Daten, in Vorkehrungen gegen ihre unbefugte Kenntnisnahme durch Dritte oder in ihrer räumlich und organisatorisch von den sonstigen Fachaufgaben getrennten Verarbeitung bestehen.
    
\chapter[GG]{Grundgesetz für die Bundesrepublik Deutschland}
\qrcode{https://www.gesetze-im-internet.de/gg/GG.pdf}
\newline
\url{https://www.gesetze-im-internet.de/gg/GG.pdf}
    \section{Art. 73}
        \begin{enumerate}[label=(\arabic*)]
            \item Der Bund hat die ausschliessliche Gesetzgebung über:
            \newline
            (\dots)
                \begin{enumerate}[label=\arabic*.,start=11]
                    \item  die Statistik für Bundeszwecke;
                \end{enumerate}

            
        \end{enumerate}
    \section{Art. 85}
        \begin{enumerate}[label=(\arabic*)]
            \item Führen die Länder die Bundesgesetze im Auftrage des Bundes aus, so bleibt die Einrichtung der Behörden Angelegenheit der Länder, soweit nicht Bundesgesetze mit Zustimmung des Bundesrates etwas anderes bestimmen. \textbf{Durch Bundesgesetz dürfen Gemeinden und Gemeindeverbänden Aufgaben nicht übertragen werden.}
            \item \dots
            \item \dots
            \item \dots
        \end{enumerate}

\chapter{Statistiksatzung der Stadt Passau}{Satzung über die Kommunalstatistik der Stadt Passau}
  Die Stadt Passau erlässt aufgrund des Art. 23 der Gemeindeordnung für den Freistaat Bayern (Ge-meindeordnung - GO) in der Fassung der Bekanntmachung vom 22. August 1998 (GVBl S. 796, BayRS 2020-1-1-I), zuletzt geändert durch § 10 des Gesetzes vom 27. Juli 2009 (GVBl S. 400), und der Art. 22, 23 und 24 des Bayerischen Statistikgesetzes (BayStatG) vom 10. August 1990 (GVBl S. 270, BayRS 290-1-I), zuletzt geändert durch § 14 des Gesetzes vom 24.12.2002 (GVBl. S. 962), folgende Satzung:
  \section{\S1 Geltungsbereich}
    \begin{enumerate}[label=(\arabic*)]
      \item Diese Satzung gilt für Kommunalstatistiken der Stadt Passau. Für Auftragsstatistiken gilt sie nur, soweit dies ausdrücklich bestimmt ist. Die statistische Aufbereitung von Daten, die bei städtischen Dienststellen im Vollzug ihrer Aufgaben erhoben werden oder auf sonstige Weise anfallen und nicht die ausschließliche Durchführung von Statistiken betreffen (Geschäftsstatis-tiken), bleibt unberührt. 
      \item Die Verarbeitung von Daten, die nicht dem Datenschutz oder der statistischen Geheimhaltung unterliegen, ist von den Bestimmungen dieser Satzung ebenfalls ausgenommen. 
    \end{enumerate}

  \section{\S2 Kommunalstatistik der Stadt Passau}
    \begin{enumerate}[label=(\arabic*)]
      \item Die Stadt Passau betreibt - soweit Einzelangaben oder Ergebnisse vom Bayerischen Landesamt für Statistik und Datenverarbeitung oder von anderen öffentlichen Stellen weder zur Verfügung gestellt noch anderweitig ermittelt werden können - eine Kommunalstatistik und bestimmt eine gem. Art. 20 Abs. 2 Satz 1 BayStatG für die Leitung verantwortliche Person.
      \item Im Rahmen der Kommunalstatistik nach Maßgabe dieser Satzung dürfen bei der Stadt Passau gesetzlich geschützte Daten aus unterschiedlichen Quellen und für nicht abschließend be-stimmte statistische Auswertungszwecke erhoben und verarbeitet werden.
    \end{enumerate}
 
 \section{\S3 Aufgaben der Kommunalstatistik}
 Die Statistikstelle hat insbesondere folgende Aufgaben:
  \begin{enumerate}[label=\arabic*.]
    \item Vorbereitung und Durchführung statistischer Erhebungen aufgrund Bundes- oder Lan-desgesetze sowie freiwilliger kommunalstatistischer Erhebungen und Umfragen, Gewin-nung statistischer Daten aus Verwaltungstätigkeiten, aus Quellen der Landes- und Bun-desstatistiken und aus sonstigen Quellen, 
    \item Aufbau, Pflege und Betreuung der städtischen Datensammlungen zur statistischen Infor-mation in Form von Einzel- und Aggregatdaten aus unterschiedlichen Quellen und für nicht abschließend bestimmte statistische Auswertungszwecke, 
    \item Aufbau, Pflege und Betreuung der Instrumente zur Gewinnung und Darstellung statisti-scher Informationen,
    \item Aufbau, Pflege und Betreuung eines kleinräumig gegliederten Raumbezugssystems sowie der sich daraus ergebenden Schlüsselsysteme,
    \item Datenaufbereitung, Durchführung statistischer Analysen, Prognosen und Modellrechnun-gen (Stadtforschung), Erstellung statistischer Gutachten, 
    \item Erhebung, Aufbereitung und Analyse der Grundlagen, 
    \item Aufgaben der örtlichen Erhebungs- und Berichtsstelle für Volkszählungen, Bundes- und Landesstatistiken, soweit durch Bundes- und Landesrecht nichts anderes bestimmt ist,
    \item Wahrnehmung der Verbindung zum statistischen Bundesamt sowie zu den statistischen Landesämtern, Mitwirkung in den einschlägigen Facharbeitskreisen und im Verband Deutscher Städtestatistiker (VDSt).
  \end{enumerate} 
  \section{\S4 Geheimhaltung}
    Einzelangaben über persönliche und sachliche Verhältnisse, die für die Kommunalstatistik der Stadt Passau gemacht oder zu diesem Zweck an die Statistikstelle übermittelt werden, sind von den Amtsträgern und für den öffentlichen Dienst besonders Verpflichteten, die mit der Durchführung einer solchen Statistik betraut sind, geheim zu halten, soweit durch besondere Rechtsvorschrift nichts anderes bestimmt ist. Die Regelungen von Art. 17 BayStatG bleiben unberührt.
    
  \section{\S5 Abschottung}
    \begin{enumerate}[label=(\arabic*)]
      \item Die Statistikstelle ist räumlich, organisatorisch und personell von anderen Verwaltungsstellen getrennt zu führen. Die Räume, in denen geschützte Einzeldaten verwahrt oder bearbeitet wer-den, sind gegen Zutritt Unbefugter bestmöglich zu sichern. Die Räume der Statistikstelle dür-fen nur von deren Mitarbeitern und den zuständigen Datenschutzbeauftragten betreten werden. Sollte der Zutritt weiterer Personen notwendig sein (z. B. IT-Firmen-Personal, Reini-gungspersonal u. ä.), so sind diese vor Betreten ausdrücklich auf ihre Geheimhaltungspflichten hinzuweisen. 
      \item Die in der Statistikstelle tätigen Personen dürfen nicht gleichzeitig bei anderen Dienststellen der Stadtverwaltung eingesetzt werden und müssen die Gewähr für Zuverlässigkeit und Ver-schwiegenheit bieten. Sie sind auf die Wahrung des Datengeheimnisses nach Art. 5 des Baye-rischen Datenschutzgesetz - BayDSG und des Statistikgeheimnisses nach § 4 dieser Satzung schriftlich zu verpflichten. Sie sind zur Einhaltung dieser Verpflichtungen auch gegenüber den Dienstvorgesetzten verpflichtet. Die dienst- und arbeitsrechtlichen Befugnisse des Dienstvor-gesetzten bleiben unberührt. 
      \item Zur Erfüllung ihrer Aufgaben bedient sich die Statistikstelle der zentralen Datenverarbeitung. Dabei müssen die Einhaltung der Vorschriften des BayDSG, des Statistikgeheimnisses und der Vorgaben dieser Satzung gewährleistet sein. 
    \end{enumerate}
    
    \section{\S6 Inkrafttreten}
    Diese Satzung tritt eine Woche nach ihrer Bekanntmachung im Amtsblatt der Stadt Passau in Kraft.
    
\end{document}
