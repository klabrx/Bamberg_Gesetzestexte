\documentclass[12pt]{scrbook}

\usepackage[utf8]{inputenc}
\usepackage{hyperref}
\usepackage[ngerman]{babel}
\usepackage{qrcode}

\usepackage{enumitem}
\setcounter{secnumdepth}{0}

\usepackage{fancyhdr}

\pagestyle{fancy}
\renewcommand{\chaptermark}[1]{\markboth{#1}{}}
% \renewcommand{\sectionmark}[1]{\markboth{#1}{}}

\def\@chapter[#1]#2{\ifnum \c@secnumdepth >\m@ne
                        \if@mainmatter
                          \refstepcounter{chapter}%
                          \typeout{\@chapapp\space\thechapter.}%
                          \addcontentsline{toc}{chapter}%
                                    {\protect\numberline{\thechapter}#2}%
                        \else
                          \addcontentsline{toc}{chapter}{#2}%
                        \fi
                     \else
                       \addcontentsline{toc}{chapter}{#2}%
                     \fi
                     \chaptermark{#1}%
                     \addtocontents{lof}{\protect\addvspace{10\p@}}%
                     \addtocontents{lot}{\protect\addvspace{10\p@}}%
                     \if@twocolumn
                       \@topnewpage[\@makechapterhead{#2}]%
                     \else
                       \@makechapterhead{#2}%
                       \@afterheading
                     \fi}



\title{Einführung in die Kommunalstatistik}
\subtitle{Arbeitsunterlagen und Gesetzestexte}
\author{Zusammenstellung: Klaus Brückner}

\begin{document}
\maketitle
\tableofcontents

\chapter[EU-StatV]{EU-Statistikverordnung}
\qrcode{https://eur-lex.europa.eu/legal-content/DE/TXT/PDF/?uri=CELEX:32009R0223&from=DE}
Link zum Volltext (pdf)
    \section{Art. 2: Statistische Grundsätze}
        \begin{enumerate}[label=(\arabic*)]
            \item Für die Entwicklung, Erstellung und Verbreitung europäischer Statistiken gelten die folgenden statistischen Grundsätze:
            \begin{enumerate}
                \item ``Fachliche Unabhängigkeit'' bedeutet, dass die Statistiken auf unabhängige Weise entwickelt, erstellt und verbreitet werden müssen, insbesondere was die Wahl der zu verwendenden Verfahren, Definitionen, Methoden und Quellen sowie den Zeitpunkt und den Inhalt aller Verbreitungsformen anbelangt, ohne dass politische Gruppen oder Interessengruppen oder Stellen der Gemeinschaft oder einzelstaatliche Stellen Druck ausüben können; dies gilt unbeschadet institutioneller Rahmenbedingungen wie gemeinschaftlicher oder einzelstaatlicher institutioneller oder haushaltsrechtlicher Bestimmungen oder der Festlegung des statistischen Bedarfs.
                \item ``Unparteilichkeit'' bedeutet, dass die Statistiken auf neutrale Weise entwickelt, erstellt und verbreitet und dass alle Nutzer gleich behandelt werden müssen.
                \item ``Objektivität'' bedeutet, dass die Statistiken in systematischer, zuverlässiger und unvoreingenommener Weise entwickelt, erstellt und verbreitet werden müssen; dabei werden fachliche und ethische Standards angewandt und die angewandten Grundsätze und Verfahren sind für Nutzer und Befragte transparent.
                \item ``Zuverlässigkeit'' bedeutet, dass die Statistiken die Gegebenheiten, die sie abbilden sollen, so getreu, genau und konsistent wie möglich messen müssen, wobei zur Wahl der Quellen, Methoden und Verfahren wissenschaftliche Kriterien herangezogen werden.
                \item ``Statistische Geheimhaltung'' bedeutet, dass direkt für statistische Zwecke oder indirekt aus administrativen oder sonstigen Quellen eingeholte vertrauliche Angaben über einzelne statistische Einheiten geschützt werden müssen, wobei die Verwendung der eingeholten Angaben für nichtstatistische Zwecke und ihre unrechtmässige Offenlegung untersagt sind.
                \item ``Kostenwirksamkeit'' bedeutet, dass die Kosten für die Erstellung der Statistiken in einem angemessenen Verhältnis zur Bedeutung des angestrebten Ergebnisses und Nutzens stehen und die Mittel optimal genutzt werden müssen und dass der Beantwortungsaufwand so gering wie möglich gehalten werden muss. Die verlangten Informationen werden nach Möglichkeit direkt aus vorhandenen Unterlagen oder Quellen entnommen.
            \end{enumerate}
            Die in diesem Absatz dargelegten statistischen Grundsätze werden in dem in Artikel 11 genannten Verhaltenskodex weiter ausgearbeitet.
            \item Bei der Entwicklung, Erstellung und Verbreitung europäischer Statistiken werden internationale Empfehlungen und vorbildliche Verfahren (best practice) berücksichtigt.
        \end{enumerate}
    \section{Art. 3: Definitionen}
        Für die Zwecke dieser Verordnung bezeichnet der Ausdruck:
        \begin{enumerate}
            \item ``Statistiken'' quantitative und qualitative, aggregierte und repräsentative Informationen, die ein Massenphänomen in einer betrachteten Grundgesamtheit beschreiben;
            \item ``Entwicklung'' die Tätigkeiten zur Festlegung, Stärkung und Verbesserung der für die Erstellung und Verbreitung von Statistiken verwendeten statistischen Methoden, Standards und Verfahren sowie zur Konzeption neuer Statistiken und Indikatoren;
            \item ``Erstellung'' alle im Zusammenhang mit der Erhebung, Speicherung, Verarbeitung und Analyse stehenden Tätigkeiten, die zur Erstellung von Statistiken erforderlich sind;
            \item ``Verbreitung'' die Tätigkeit, mit der Statistiken und statistische Analysen den Nutzern zugänglich gemacht werden;
            \item ``Datengewinnung'' Befragungen und alle sonstigen Methoden der Gewinnung von Informationen aus unterschiedlichen Quellen, einschliesslich administrativer Quellen;
            \item ``Statistische Einheit'' die Grundbeobachtungseinheit, das heisst eine natürliche Person, ein Haushalt, ein Wirtschaftsteilnehmer oder eine sonstige Unternehmung, auf die sich die Daten beziehen;
            \item ``Vertrauliche Daten'' Daten, die eine direkte oder indirekte Identifizierung statistischer Einheiten möglich machen und dadurch Einzelinformationen offenlegen. Bei der Entscheidung, ob eine statistische Einheit identifizierbar ist, sind alle Mittel zu berücksichtigen, die nach vernünftigem Ermessen von einem Dritten angewendet werden könnten, um die statistische Einheit zu identifizieren;
            \item ``Verwendung für statistische Zwecke'' die ausschliessliche Verwendung für die Entwicklung und Erstellung statistischer Ergebnisse und Analysen;
            \item ``Direkte Identifizierung'' die Identifizierung einer statistischen Einheit anhand ihres Namens oder ihrer Anschrift oder anhand einer öffentlich zugänglichen Identifikationsnummer;
            \item ``Indirekte Identifizierung'' die Identifizierung einer statistischen Einheit durch andere Mittel als die direkte Identifizierung;
            \item ``Beamte der Kommission (Eurostat)'' Beamte der Gemeinschaften im Sinne von Artikel 1 des Statuts der Beamten der Europäischen Gemeinschaften, die bei der statistischen Stelle der Gemeinschaft tätig sind;
            \item ``Sonstige Mitarbeiter der Kommission (Eurostat)'' Bedienstete der Gemeinschaften im Sinne der Artikel 2 bis 5 der Beschäftigungsbedingungen für die sonstigen Bediensteten der Europäischen Gemeinschaften, die bei der statistischen Stelle der Gemeinschaft tätig sind.
        \end{enumerate}
    \section{Art. 12: Qualität der Statistik}
        \begin{enumerate}[label=(\arabic*)]
            \item Um die Qualität der Ergebnisse zu gewährleisten, werden europäische Statistiken auf der Grundlage einheitlicher Standards und nach harmonisierten Methoden entwickelt, erstellt und verbreitet. Dabei gelten die folgenden Qualitätskriterien:
            \begin{enumerate}
                \item ``Relevanz'': diese bezieht sich auf den Umfang, in dem die Statistiken dem aktuellen und potenziellen Nutzerbedarf entsprechen;
                \item ``Genauigkeit'': diese bezieht sich auf den Grad der Übereinstimmung der Schätzungen mit den unbekannten wahren Werten;
                \item ``Aktualität'': diese bezieht sich auf die Zeitspanne zwischen dem Vorliegen der Information und dem von ihr beschriebenen Ereignis oder Phänomen;
                \item ``Pünktlichkeit'': diese bezieht sich auf die Zeitspanne zwischen dem Zeitpunkt der Veröffentlichung der Daten und dem Zieltermin (Termin, zu dem die Daten geliefert werden sollten);
                \item ``Zugänglichkeit'' und ``Klarheit'': diese beziehen sich auf die Bedingungen und Modalitäten, unter denen die Nutzer Daten erhalten, verwenden und interpretieren können;
                \item ``Vergleichbarkeit'': diese bezieht sich auf die Messung der Auswirkungen von Unterschieden in den verwendeten statistischen Konzepten, Messinstrumenten und -verfahren bei Vergleichen von Statistiken für unterschiedliche geografische Gebiete oder thematische Bereiche oder bei zeitlichen Vergleichen;
                \item ``Kohärenz'': diese bezieht sich auf die Eignung der Daten, auf unterschiedliche Weise und für verschiedene Zwecke zuverlässig kombiniert zu werden.
            \end{enumerate}
            \item Bei der Anwendung der in Absatz 1 festgelegten Qualitätskriterien auf die unter sektorale Rechtsvorschriften in bestimmten Statistikbereichen fallenden Daten werden die Modalitäten, der Aufbau und die Periodizität der in den sektoralen Rechtsvorschriften vorgesehenen Qualitätsberichte von der Kommission nach dem in Artikel 27 Absatz 2 genannten Regelungsverfahren festgelegt.
            Besondere Qualitätsanforderungen wie Zielwerte und Mindeststandards für die Statistikproduktion können in sektoralen Rechtsvorschriften festgelegt sein. Enthalten die sektoralen Rechtsvorschriften keine derartigen Bestimmungen, kann die Kommission entsprechende Massnahmen ergreifen. Diese Massnahmen zur Änderung nicht wesentlicher Bestimmungen dieser Verordnung durch Ergänzung werden nach dem in Artikel 27 Absatz 3 genannten Regelungsverfahren mit Kontrolle erlassen.
            \item Die Mitgliedstaaten legen der Kommission (Eurostat) Berichte über die Qualität der übermittelten Daten vor. Die Kommission (Eurostat) bewertet die Qualität der übermittelten Daten und erstellt und veröffentlicht Berichte über die Qualität der europäischen Statistiken.
        \end{enumerate}
    \section{Art. 20: Schutz vertraulicher Daten}
        \begin{enumerate}[label=(\arabic*)]
            \item Die folgenden Regeln und Massnahmen gelten, um sicherzustellen, dass vertrauliche Daten ausschliesslich für statistische Zwecke verwendet werden und ihre rechtswidrige Offenlegung verhindert wird.
            \item Vertrauliche Daten, die ausschliesslich für die Erstellung europäischer Statistiken erhoben wurden, werden von den NSÄ und anderen einzelstaatlichen Stellen und von der Kommission (Eurostat) ausschliesslich für statistische Zwecke verwendet, es sei denn, die statistische Einheit hat unmissverständlich ihre Zustimmung zur Verwendung der Daten zu anderen Zwecken erteilt.
            \item Statistische Ergebnisse, die die Identifizierung einer statistischen Einheit ermöglichen könnten, dürfen in folgenden Ausnahmefällen von den NSÄ und anderen einzelstaatlichen Stellen und der Kommission (Eurostat) verbreitet werden:
            \begin{enumerate}
                \item wenn in einem Rechtsakt des Europäischen Parlaments und des Rates gemäss Artikel 251 des Vertrags besondere Bedingungen und Modalitäten festgelegt sind und die statistischen Ergebnisse auf Ersuchen der statistischen Einheit so verändert werden, dass ihre Verbreitung die statistische Geheimhaltung nicht gefährdet; oder
                \item wenn die statistische Einheit der Offenlegung der Daten unmissverständlich zugestimmt hat.
            \end{enumerate}
            \item Die NSÄ und andere einzelstaatliche Stellen und die Kommission (Eurostat) ergreifen innerhalb ihrer jeweiligen Zuständigkeitsbereiche alle erforderlichen rechtlichen, administrativen, technischen und organisatorischen Massnahmen, um den physischen und logischen Schutz vertraulicher Daten zu gewährleisten (statistische Offenlegungskontrolle).
            \item Die NSÄ und andere einzelstaatliche Stellen und die Kommission (Eurostat) ergreifen alle erforderlichen Massnahmen, um die Harmonisierung der Grundsätze und Leitlinien für den physischen und logischen Schutz vertraulicher Daten zu gewährleisten. Diese Massnahmen werden von der Kommission nach dem in Artikel 27 Absatz 2 genannten Regelungsverfahren erlassen.
            \item Beamte und sonstige Mitarbeiter der NSÄ und anderer einzelstaatlicher Stellen, die Zugang zu vertraulichen Daten haben, unterliegen auch nach ihrem Ausscheiden aus dem Dienst der statistischen Geheimhaltungspflicht.
        \end{enumerate}
\chapter[BStatG]{Bundesstatistikgesetz}
\qrcode{https://www.gesetze-im-internet.de/bstatg_1987/BStatG.pdf}
Link zum Volltext (pdf)
    \section{\S 1: Statistik für Bundeszwecke} 
    Die Statistik für Bundeszwecke (Bundesstatistik) hat im föderativ gegliederten Gesamtsystem der amtlichen Statistik die Aufgabe, laufend Daten über Massenerscheinungen zu erheben, zu sammeln, aufzubereiten, darzustellen und zu analysieren. Für sie gelten die Grundsätze der Neutralität, Objektivität und fachlichen Unabhängigkeit. Sie gewinnt die Daten unter Verwendung wissenschaftlicher Erkenntnisse und unter Einsatz der jeweils sachgerechten Methoden und Informationstechniken. Durch die Ergebnisse der Bundesstatistik werden gesellschaftliche, wirtschaftliche und ökologische Zusammenhänge für Bund, Länder einschliesslich Gemeinden und Gemeindeverbände, Gesellschaft, Wirtschaft, Wissenschaft und Forschung aufgeschlüsselt. Die Bundesstatistik ist Voraussetzung für eine am Sozialstaatsprinzip ausgerichtete Politik. Die für die Bundesstatistik erhobenen Einzelangaben dienen ausschliesslich den durch dieses Gesetz oder eine andere eine Bundesstatistik anordnende Rechtsvorschrift festgelegten Zwecken.
    \section{\S 10: Erhebungs- und Hilfsmerkmale}
        \begin{enumerate}[label=(\arabic*)]
            \item Bundesstatistiken werden auf der Grundlage von Erhebungs- und Hilfsmerkmalen erstellt. Erhebungsmerkmale umfassen Angaben über persönliche und sachliche Verhältnisse, die zur statistischen Verwendung bestimmt sind. Hilfsmerkmale sind Angaben, die der technischen Durchführung von Bundesstatistiken dienen. Für andere Zwecke dürfen sie nur verwendet werden, soweit Absatz 2 oder ein sonstiges Gesetz es zulassen.
            \item Der Name der Gemeinde, die Blockseite und die geografische Gitterzelle dürfen für die regionale Zuordnung der Erhebungsmerkmale genutzt werden. Die übrigen Teile der Anschrift dürfen für die Zuordnung zu Blockseiten und geografischen Gitterzellen für einen Zeitraum von bis zu vier Jahren nach Abschluss der jeweiligen Erhebung genutzt werden. Besondere Regelungen in einer eine Bundesstatistik anordnenden Rechtsvorschrift bleiben unberührt.
            \item Blockseite ist innerhalb eines Gemeindegebiets die Seite mit gleicher Strassenbezeichnung von der durch Strasseneinmündungen oder vergleichbare Begrenzungen umschlossenen Fläche. Eine geografische Gitterzelle ist eine Gebietseinheit, die bezogen auf eine vorgegebene Kartenprojektion quadratisch ist und mindestens 1 Hektar gross ist.
        \end{enumerate}
    \section{\S 12 Trennung und Löschung der Hilfsmerkmale}
        \begin{enumerate}[label=(\arabic*)]
            \item Hilfsmerkmale sind, soweit Absatz 2, § 10 Absatz 2, § 13 oder eine sonstige Rechtsvorschrift nichts anderes bestimmen, zu löschen, sobald bei den statistischen Ämtern die Überprüfung der Erhebungs- und Hilfsmerkmale auf ihre Schlüssigkeit und Vollständigkeit abgeschlossen ist. Sie sind von den Erhebungsmerkmalen zum frühestmöglichen Zeitpunkt zu trennen und gesondert aufzubewahren oder gesondert zu speichern.
            \item Bei periodischen Erhebungen für Zwecke der Bundesstatistik dürfen die zur Bestimmung des Kreises der zu Befragenden erforderlichen Hilfsmerkmale, soweit sie für nachfolgende Erhebungen benötigt werden, gesondert aufbewahrt oder gesondert gespeichert werden. Nach Beendigung des Zeitraumes der wiederkehrenden Erhebungen sind sie zu löschen.
        \end{enumerate}
\chapter[BayStatG]{Bayerisches Statistikgesetz}
\qrcode{https://www.gesetze-bayern.de/Content/Pdf/BayStatG?all=True}
Link zum Volltext (pdf)
    \section{Art. 1: Geltungsbereich}
        \begin{enumerate}[label=(\arabic*)]
            \item \textsuperscript{1}Dieses Gesetz gilt für die Durchführung von Statistiken durch öffentliche Stellen. \textsuperscript{2}Führen diese Stellen Bundesstatistiken oder europäische Statistiken durch und haben sie dabei andere Rechtsvorschriften anzuwenden, so finden die Vorschriften dieses Gesetzes nur ergänzend Anwendung.
            \item Für Geschäftsstatistiken gilt dieses Gesetz nur, soweit das ausdrücklich bestimmt ist.
        \end{enumerate}
    \section{Art. 2: Begriffe}
        \begin{enumerate}[label=(\arabic*)]
            \item \textsuperscript{1}Amtliche Statistiken sind Landesstatistiken, Bundesstatistiken und europäische Statistiken. \textsuperscript{2}Landesstatistiken sind Statistiken, die von Organen des Freistaates Bayern angeordnet und von staatlichen Stellen durchgeführt werden.
            \item Kommunale Statistiken sind Statistiken, die von Gemeinden oder Gemeindeverbänden zur Wahrnehmung ihrer Aufgaben durchgeführt werden.
            \item Geschäftsstatistiken sind statistische Aufbereitungen von Daten, die bei öffentlichen Stellen im Vollzug ihrer Aufgaben, die nicht die Durchführung von Statistiken betreffen, erhoben werden oder auf sonstige Weise anfallen.
            \item Öffentliche Stellen sind alle Behörden, Gerichte und sonstige öffentliche Stellen des Freistaates Bayern, die Gemeinden und Gemeindeverbände sowie die der Aufsicht des Freistaates Bayern unterstehenden juristischen Personen des öffentlichen Rechts und deren Vereinigungen.
            \item Einzelangaben sind Daten über persönliche oder sachliche Verhältnisse bestimmter oder bestimmbarer natürlicher oder juristischer Personen und deren Vereinigungen, die bei der Durchführung einer Statistik erhoben oder übermittelt werden.
        \end{enumerate}
    \section[Art. 3: EU-DSGVO und BayDatSchG]{Art. 3: Anwendbarkeit der Datenschutz-Grundverordnung und des Bayerischen Datenschutzgesetzes}
        \begin{enumerate}[label=(\arabic*)]
            \item Die Ansprüche nach den Art. 15, 16, 18 und 21 der Verordnung (EU) 2016/679 (Datenschutz- Grundverordnung– DSGVO) bestehen nicht, soweit diese Rechte die Verwirklichung statistischer Zwecke ernsthaft beeinträchtigen würden.
            \item Einzelangaben dürfen an das Landesamt und an Statistikstellen für die Durchführung von Geschäftsstatistiken übermittelt und von dort – auch in aufbereiteter Form – rückübermittelt werden
        \end{enumerate}
    (\dots)
    \section{Art. 9: Anordnung}
        \begin{enumerate}[label=(\arabic*)]
            \item \textsuperscript{1}Statistiken werden durch Gesetz oder Rechtsverordnung angeordnet. 2Die Anordnung bedarf keiner Rechtsvorschrift, wenn
            \begin{enumerate}[label=\arabic*.]
                \item die einer Statistik zugrundeliegenden Daten
                    \begin{enumerate}[label=(\alph*)]
                        \item auf freiwilligen Auskünften oder allgemein zugänglichen Quellen beruhen,
                        \item keine Einzelangaben enthalten oder
                        \item der die Statistik durchführenden Stelle rechtmässig übermittelt werden oder ihrem Zugriff auf Grund einer Rechtsvorschrift zur Verfügung stehen;
                    \end{enumerate}
                \item lediglich Sonderauswertungen vorhandenen statistischen Materials vorgenommen werden, dessen Verwendung eine Zweckbindung nicht entgegensteht, oder
                \item zur Anordnung der Statistik eine Rechtsvorschrift ermächtigt.
            \end{enumerate}
            \item Die eine Landesstatistik anordnende Rechtsvorschrift muss die näheren Bestimmungen treffen über die Art der Erhebung, den Kreis der zu Befragenden, sonstige Auskunftsstellen, die durch Erhebungsmerkmale zu erfassenden Sachverhalte, die Hilfsmerkmale, den Berichtszeitraum, den Berichtszeitpunkt, die Häufigkeit der Erhebung (Periodizität) sowie über Art und Umfang einer Auskunftspflicht.
        \end{enumerate}
    \section{Art. 14: Erhebungsbeauftragte}
        \begin{enumerate}[label=(\arabic*)]
            \item Als Erhebungsbeauftragte dürfen nur Personen eingesetzt werden, die Gewähr für Zuverlässigkeit und Verschwiegenheit bieten und bei denen nicht auf Grund ihrer beruflichen Tätigkeit oder aus anderen Gründen Anlass zur Besorgnis besteht, dass Erkenntnisse aus der Tätigkeit als Erhebungsbeauftragte zu Lasten der Auskunftspflichtigen genutzt werden.
            \item \textsuperscript{1}Erhebungsbeauftragte sind verpflichtet, die Anweisungen der Erhebungsstellen zu befolgen. \textsuperscript{2}Bei der Ausübung ihrer Tätigkeit haben sie sich auszuweisen. \textsuperscript{3}Sie dürfen statistische Einzelangaben und gelegentlich ihrer Tätigkeit gewonnene Erkenntnisse auch nach Beendigung ihrer Tätigkeit nicht für andere Verfahren oder andere Zwecke verarbeiten.
            \item \textsuperscript{1}Erhebungsbeauftragte sind über ihre Rechte und Pflichten sowie über die Rechte und Pflichten der zu Befragenden zu belehren. \textsuperscript{2}Vor ihrem Einsatz sind sie auf die Wahrung des Statistikgeheimnisses und zur Geheimhaltung der Erkenntnisse, die sie bei ihrer Tätigkeit gewonnen haben, schriftlich zu verpflichten.
        \end{enumerate}
    \section{Art. 15: Erhebungs- und Hilfsmerkmale}
        \begin{enumerate}[label=(\arabic*)]
            \item \textsuperscript{1}Erhebungsmerkmale sind zur Erstellung einer Statistik bestimmte Angaben über persönliche oder sachliche Verhältnisse. \textsuperscript{2}Hilfsmerkmale sind Angaben, die der technischen Durchführung von Statistiken dienen.
            \item \textsuperscript{1}Hilfsmerkmale sind von den Erhebungsmerkmalen zum frühestmöglichen Zeitpunkt zu trennen und gesondert aufzubewahren. \textsuperscript{2}Laufende Nummern und Ordnungsnummern können auf den Erhebungsunterlagen verbleiben. \textsuperscript{3}Sie dürfen auf die für die maschinelle Weiterverarbeitung bestimmten Datenträger übernommen werden.
            \item \textsuperscript{1}Die Hilfsmerkmale sind zu löschen, sobald die Überprüfung der Erhebungs- und Hilfsmerkmale auf Schlüssigkeit und Vollständigkeit abgeschlossen ist. \textsuperscript{2}Bei wiederkehrenden Erhebungen kann die Löschung der Hilfsmerkmale unterbleiben, soweit sie noch künftig zur Bestimmung des Kreises der zu Befragenden benötigt werden. \textsuperscript{3}Die Hilfsmerkmale sind gesondert aufzubewahren und nach Beendigung der wiederkehrenden Erhebungen zu löschen. \textsuperscript{4}Diese Vorschriften gelten entsprechend für die Vernichtung von Erhebungsunterlagen, die Hilfsmerkmale enthalten.
            \item \textsuperscript{1}Die Namen von Gemeinden und von Gemeindeteilen sowie Blockseiten dürfen für die regionale Zuordnung von Erhebungsmerkmalen genutzt werden. \textsuperscript{2}Blockseite ist innerhalb eines Gemeindegebiets die Seite mit gleicher Strassenbezeichnung von der durch Strasseneinmündungen oder vergleichbare Begrenzungen umschlossenen Fläche. \textsuperscript{3}Die übrigen Teile der Anschrift dürfen für die Zuordnung zu Blockseiten für einen Zeitraum bis zu vier Jahren nach Abschluss der jeweiligen Erhebung genutzt werden. 4Besondere Regelungen in einer eine amtliche Statistik anordnenden Rechtsvorschrift bleiben unberührt.
            \item Absatz 2 Satz 1 und Absatz 3 Satz 1 und 3 gelten nicht für Daten, die ausschliesslich einer öffentlichen Stelle zugeordnet werden können.
        \end{enumerate}   
    
    \section{Art. 17: Geheimhaltung}
        \begin{enumerate}[label=(\arabic*)]
            \item \textsuperscript{1}Einzelangaben sind von den mit der Durchführung der Statistik betrauten Stellen und Personen geheimzuhalten. \textsuperscript{2}Dies gilt nicht für
                \begin{enumerate}[label=\arabic*.]
                    \item Einzelangaben, in deren Übermittlung oder Veröffentlichung die Auskunftgebenden oder die betroffenen Personen schriftlich eingewilligt haben;
                    \item Einzelangaben, soweit deren Übermittlung oder Veröffentlichung durch Art. 18 oder durch besondere Rechtsvorschrift zugelassen ist;
                    \item Einzelangaben aus allgemein zugänglichen Quellen;
                    \item Einzelangaben, die ausschliesslich einer öffentlichen Stelle, die nicht am wirtschaftlichen Wettbewerb teilnimmt, zugeordnet werden können;
                    \item Einzelangaben, die keiner befragten oder betroffenen Person zuzuordnen sind, insbesondere, wenn sie mit den Einzelangaben anderer zusammengefasst und in statistischen Ergebnissen dargestellt sind.
                \end{enumerate}
            \textsuperscript{3}Die Pflicht zur Geheimhaltung besteht auch für Personen, die Empfänger von Einzelangaben nach Art. 18 oder auf Grund einer besonderen Rechtsvorschrift sind.
            \item Sonstige Vorschriften über die Geheimhaltung und Verschwiegenheit bleiben unberührt.
        \end{enumerate}
    \section{Art. 18: Zweckbindung und Übermittlung von Einzelangaben}
        \begin{enumerate}[label=(\arabic*)]
            \item Einzelangaben dürfen ausschliesslich für statistische Zwecke verarbeitet werden, es sei denn sie beruhen auf allgemein zugänglichen Quellen oder eine Rechtsvorschrift lässt eine andere Verwendung zu.
            \item \textsuperscript{1}Das Landesamt darf Einzelangaben, wenn eine ausdrückliche Zweckbindung nicht entgegensteht, an Statistikstellen anderer öffentlicher Stellen für deren Zuständig\-keitsbereich zu ausschliesslich statistischen Zwecken übermitteln. \textsuperscript{2}Soweit durch Rechtsvorschrift nichts anderes bestimmt ist, dürfen Hilfsmerkmale nicht über\-mit\-telt werden.
            \item Zur Erstellung koordinierter Länderstatistiken darf das Landesamt Einzelangaben an das Statistische Bundesamt und die Statistischen Ämter der Länder übermitteln.
            \item \textsuperscript{1}Für Gesetzesvorhaben und für Zwecke der Planung, nicht jedoch für die Regelung von Einzelfällen, darf das Landesamt den Staatsministerien Tabellen mit statistischen Ergebnissen übermitteln, auch soweit Tabellenfelder nur einen einzigen Fall ausweisen. \textsuperscript{2}Durch organisatorische und technische Massnahmen muss sichergestellt sein, dass nur Amtsträger und für den öffentlichen Dienst besonders Verpflichtete Kenntnis von Einzelangaben erhalten.
            \item \textsuperscript{1}Für die Durchführung wissenschaftlicher Vorhaben darf das Landesamt Einzelangaben an Hochschulen oder sonstige Einrichtungen mit der Aufgabe unabhängiger wissenschaftlicher Forschung übermitteln, wenn die Einzelangaben nur mit einem unverhältnismässig grossen Aufwand an Zeit, Kosten und Arbeitskraft zugeordnet werden können. \textsuperscript{2}Sofern es sich bei den Empfängern nicht um Amtsträger oder für den öffentlichen Dienst besonders Verpflichtete handelt, sind sie vor der Übermittlung vom Landesamt besonders zur Geheimhaltung zu verpflichten. \textsuperscript{3}§ 1 Abs. 2, 3 und 4 Nr. 2 des Verpflichtungsgesetzes sind entsprechend anwendbar. \textsuperscript{4}Personen, die nach Satz 2 besonders verpflichtet worden sind, stehen für die Anwendung der Vorschriften des Strafgesetzbuches über die Verletzung von Privatgeheimnissen (§ 203 Abs. 2, 4, 5, § 204, 205) den für den öffentlichen Dienst besonders Verpflichteten gleich. 5Empfänger haben durch technische und organisatorische Massnahmen sicherzustellen, dass sonstige Personen keine Kenntnis von Einzelangaben erhalten. 6Die Einzelangaben sind zu löschen oder zu vernichten, sobald das wissenschaftliche Vorhaben abgeschlossen ist, zu dessen Durchführung sie übermittelt wurden.
            \item \textsuperscript{1}Einzelangaben, die auf Grund der Absätze 2 bis 5 oder auf Grund einer besonderen Rechtsvorschrift übermittelt werden, dürfen nur für den Zweck verwendet werden, für den sie übermittelt worden sind. \textsuperscript{2}Die Übermittlung ist vom Landesamt unter Angabe von Inhalt, empfangender Stelle, Datum und Zweck aufzuzeichnen. \textsuperscript{3}Die Aufzeichnungen sind mindestens fünf Jahre aufzubewahren.
            \item Einzelangaben dürfen vom Landesamt wieder an die auskunftsgebende Stelle übermittelt werden. 
            \item Die Absätze 2 bis 7 gelten entsprechend, wenn Statistikstellen anderer staatlicher Stellen für die Durchführung von Landesstatistiken zuständig sind. 
        \end{enumerate}
        
    \section{Art. 20: Statistikstellen}
        \begin{enumerate}[label=(\arabic*)]
            \item \textsuperscript{1}Werden Statistiken ausserhalb des Landesamts durchgeführt, so sind besondere Statistikstellen einzurichten. \textsuperscript{2}Nichtstatistische Aufgaben des Verwaltungsvollzugs dürfen ihnen nicht übertragen werden. \textsuperscript{3}Statistikstellen veröffentlichen die Ergebnisse ihrer Statistiken oder stellen sie in sonstiger Weise bereit.
            \item \textsuperscript{1}Für jede Statistikstelle ist jemand zu bestimmen, der diese leitet. \textsuperscript{2}Statistikstellen sind räumlich und organisatorisch von anderen Verwaltungsstellen zu trennen, gegen den Zutritt unbefugter Personen hinreichend zu sichern und mit Personal auszustatten, das die Gewähr für Zuverlässigkeit und Verschwiegenheit bietet.
            \item \textsuperscript{1}Die in Statistikstellen tätigen Personen dürfen statistische Einzelangaben und gelegentlich ihrer Tätigkeit gewonnene Erkenntnisse auch nach Beendigung ihrer Tätigkeit nicht in anderen Verfahren oder für andere Zwecke verarbeiten, soweit nicht durch Rechtsvorschrift etwas anderes zugelassen ist. \textsuperscript{2}Sie sind vor ihrem Einsatz auf die Wahrung des Statistikgeheimnisses und über die Folgen seiner Verletzung zu belehren und schriftlich zu verpflichten. \textsuperscript{3}Soweit und solange sie Einzelangaben bearbeiten, dürfen sie nicht andere Aufgaben des Verwaltungsvollzugs wahrnehmen. \textsuperscript{4}Im Anschluss an eine Tätigkeit in der Statistikstelle sollen sie nicht für Aufgaben eingesetzt werden, bei denen eine Nutzung der in den Statistikstellen gewonnenen Erkenntnisse möglich ist, soweit das die organisatorischen und personellen Verhältnisse zulassen.
            \item Statistikstellen können mit der Durchführung von Geschäftsstatistiken beauftragt werden-
        \end{enumerate}



    \section{Art. 21: Erhebungsstellen, Verordnungsermächtigung}
        \begin{enumerate}[label=(\arabic*)]
            \item Das Landesamt ist bei Statistiken, die es als allgemeine Aufgabe durchführt, Erhebungsstelle.
            \item \textsuperscript{1}Die Staatsregierung wird ermächtigt, durch Rechtsverordnung zu bestimmen, dass andere staatliche Stellen sowie Gemeinden Erhebungsstellen einzurichten oder in sonstiger Weise an der Durchführung amtlicher Statistiken mitzuwirken haben, wenn das wegen der Art der Erhebung, der Zahl oder der räumlichen Verteilung der zu Befragenden oder zur Sicherung der Qualität der Erhebung zweckmässig ist. \textsuperscript{2}Eine aufsichtliche Zuständigkeit des Landesamts wird durch eine solche Bestimmung nicht begründet. \textsuperscript{3}Landratsämter erfüllen die Aufgaben der Erhebungsstellen als Staatsaufgaben; für Gemeinden handelt es sich um Aufgaben des übertragenen Wirkungskreises, die sie auch nach den Vorschriften des Gesetzes über die kommunale Zusammenarbeit erfüllen können.
            \item \textsuperscript{1}Die Erhebungsstellen nach Absatz 2 Satz 1 führen in ihrem jeweiligen Zuständigkeitsbereich die statistischen Erhebungen durch. 2Art. 20 Abs. 2 und 3 gelten entsprechend mit der Massgabe, dass die räumliche und organisatorische Trennung von anderen Verwaltungsstellen ab dem Eingang der Erhebungsunterlagen bis zu ihrer Ablieferung sicherzustellen ist. 3Durch Rechtsverordnung nach Absatz 2 Satz 1 können Abweichungen von den Anforderungen des Art. 20 Abs. 2 und 3 bestimmt werden, wenn das ein erweiterter Schutz von Einzelangaben erforderlich macht oder wenn eine andere staatliche Stelle oder eine Gemeinde an der Erhebung lediglich mitwirkt. 4Soweit nichts anderes bestimmt ist, haben diese Erhebungsstellen
                \begin{enumerate}[label=\arabic*.]
                    \item bei Bedarf Erhebungsbezirke festzulegen;
                    \item die Erhebungsbeauftragten auszuwählen, zu bestellen, zu unterrichten, zu verpflichten und zu beaufsichtigen;
                    \item die zu Befragenden gemäss Art. 19 zu unterrichten, zur Auskunft heranzuziehen, die Erhebungsvordrucke auszuteilen und einzusammeln;
                    \item Personen, die noch keine Auskünfte gegeben haben, zur Auskunftserteilung anzuhalten;
                    \item die Vollzähligkeit der ausgefüllten Erhebungsvordrucke sowie deren Vollständigkeit und die formale Richtigkeit der Angaben zu überprüfen;
                    \item unvollständige oder offensichtlich fehlerhaft ausgefüllte Erhebungsvordrucke durch Nachfrage bei den Befragten zu ergänzen oder zu berichtigen.
                \end{enumerate}
            \item Stellen nach Absatz 2 Satz 1 sind nicht berechtigt, erhobenes Material für eigene Auswertungen zu nutzen.
        \end{enumerate}

    \section{Art. 22: Zulässigkeit}
        Gemeinden und Gemeindeverbände sowie andere nichtstaatliche juristische Personen des öffentlichen Rechts können für die Wahrnehmung ihrer Aufgaben im Rahmen ihrer Zuständigkeiten und Befugnisse Statistiken durchführen, wenn Einzelangaben oder Ergebnisse vom Landesamt oder von anderen öffentlichen Stellen weder zur Verfügung gestellt noch anderweitig ermittelt werden können und eigene Statistikstellen eingerichtet werden.

    \section{Art. 23: Anordnung}
        \begin{enumerate}[label=(\arabic*)]
            \item \textsuperscript{1}Statistiken für die Wahrnehmung von Selbstverwaltungsaufgaben (eigener Wirkungskreis) sind durch Satzung anzuordnen; in ihr sind zugleich die erforderlichen Bestimmungen nach Art. 9 Abs. 2 zu treffen. \textsuperscript{2}Die Anordnung bedarf keiner Satzung, wenn
                \begin{enumerate}[label=\arabic*.]
                    \item die einer Statistik zugrundeliegenden Daten auf allgemein zugänglichen Quellen beruhen, keine Einzelangaben enthalten, der Statistikstelle rechtmässig übermittelt werden oder ihrem Zugriff auf Grund einer Rechtsvorschrift zur Verfügung stehen oder
                    \item lediglich Sonderauswertungen vorhandenen statistischen Materials vorgenommen werden, dessen Verwendung eine Zweckbindung nicht entgegensteht.
                \end{enumerate}
                \textsuperscript{3}Bei juristischen Personen, denen kein Satzungsrecht zusteht, werden Statistiken durch die zuständigen Organe angeordnet. \textsuperscript{4}Durch Satzungen können Gemeinden, Landkreise und Bezirke auch eine Auskunftspflicht begründen, wenn es der Zweck der Erhebung erfordert, und zulassen, dass Statistikstellen Adressdateien in entsprechender Anwendung der für amtliche Statistiken geltenden Vorschriften führen und nutzen.
            \item \textsuperscript{1}Statistiken für die Wahrnehmung von übertragenen Aufgaben (übertragener Wirkungskreis) bedürfen einer Anordnung durch Gesetz oder staatliche Rechtsverordnung, es sei denn, es liegen die Voraussetzungen vor, unter denen auch für eine amtliche Statistik keine Anordnung durch Rechtsvorschrift erforderlich ist (Art. 9 Abs. 1 Satz 2). \textsuperscript{2}In den Fällen des Art. 9 Abs. 1 Satz 2 Nr. 1 bedarf die Statistik einer Genehmigung durch den Statistischen Genehmigungsausschuss (Art. 10).
        \end{enumerate}

    \section{Art. 24: Statistikstellen}
        \begin{enumerate}[label=(\arabic*)]
            \item \textsuperscript{1}Statistikstellen führen die angeordneten Statistiken durch. \textsuperscript{2}In die Wahrnehmung nichtstatistischer Aufgaben des Verwaltungsvollzugs dürfen Statistikstellen nicht eingeschaltet werden. \textsuperscript{3}Statistikstellen veröffentlichen die Ergebnisse der von ihnen erstellten Statistiken oder stellen sie in sonstiger Weise bereit, wenn ein öffentliches Bedürfnis besteht.
            \item \textsuperscript{1}Statistikstellen sind durch Satzung einzurichten, die auch die wesentlichen organisatorischen Bestimmungen, vornehmlich zur Wahrung des Statistikgeheimnisses zu treffen hat. \textsuperscript{2}Art. 20 Abs. 2 und 3 finden entsprechende Anwendung. \textsuperscript{3}Kommunale Statistikstellen können auch nach Massgabe des Gesetzes über die kommunale Zusammenarbeit eingerichtet werden. \textsuperscript{4}Bei juristischen Personen, denen kein Satzungsrecht zukommt, werden Statistikstellen durch die zuständigen Organe nach Massgabe der Sätze 1 und 2 eingerichtet.
            \item  \textsuperscript{2}Geschäftsstatistiken führen die Statistikstellen durch, wenn sie damit beauftragt werden. \textsuperscript{1}Kommunale Statistikstellen können die Ergebnisse von Europa-, Bundes-, Landes- und Kommunalwahlen aufbereiten. \textsuperscript{3}Sie nehmen die Aufgaben einer Erhebungsstelle im Sinn des Art. 21 Abs. 2 Satz 1 wahr.
        \end{enumerate}
        
    \section{Art.34: Reidentifizierungsverbot}
        Die Zusammenführung
        \begin{enumerate}[label=\arabic*.]
            \item von Einzelangaben aus Statistiken öffentlicher Stellen oder
            \item von Einzelangaben aus Statistiken öffentlicher Stellen mit anderen Angaben
        \end{enumerate}
         zum Zweck der Herstellung eines Personen-, Unternehmens-, Betriebs- oder Arbeitsstättenbezugs ist untersagt, es sei denn, die Aufgabenstellung dieses Gesetzes oder einer anderen Rechtsvorschrift oder ein sonstiger eine Statistik einer öffentlichen Stelle anordnender Rechtsakt lassen das zu. 
     
    \section{Art. 35: Strafvorschrift}
        Wer entgegen Art. 34 Einzelangaben aus Statistiken öffentlicher Stellen oder solche Einzelangaben mit anderen Angaben zusammenführt, wird mit Freiheitsstrafe bis zu einem Jahr oder mit Geldstrafe bestraft.



\chapter[GG]{Grundgesetz für die Bundesrepublik Deutschland}
\qrcode{https://www.gesetze-im-internet.de/gg/GG.pdf}
Link zum Volltext (pdf)
    \section{Art. 73}
        \begin{enumerate}[label=(\arabic*)]
            \item Der Bund hat die ausschliessliche Gesetzgebung über:
            \newline
            (\dots)
                \begin{enumerate}[label=\arabic*.,start=11]
                    \item  die Statistik für Bundeszwecke;
                \end{enumerate}

            
        \end{enumerate}
    \section{Art. 85}
        \begin{enumerate}[label=(\arabic*)]
            \item Führen die Länder die Bundesgesetze im Auftrage des Bundes aus, so bleibt die Einrichtung der Behörden Angelegenheit der Länder, soweit nicht Bundesgesetze mit Zustimmung des Bundesrates etwas anderes bestimmen. \textbf{Durch Bundesgesetz dürfen Gemeinden und Gemeindeverbänden Aufgaben nicht übertragen werden.}
            \item \dots
            \item \dots
            \item \dots
        \end{enumerate}


    
\end{document}
