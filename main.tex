\documentclass[A4, 12t, headings=optiontoheadandtoc]{scrbook}
\usepackage[landscape=true]{geometry}
\usepackage[utf8]{inputenc}
\usepackage{hyperref}
\usepackage[ngerman]{babel}
\usepackage{qrcode}

\usepackage{xcolor}
\usepackage{enumitem}
\setcounter{secnumdepth}{0}

\usepackage{fancyhdr}
\setlength{\headheight}{30pt}

\usepackage{minitoc}
    \mtcselectlanguage{german}

\usepackage{booktabs}

%----
\PassOptionsToPackage{svgnames}{xcolor}
\usepackage{tcolorbox}
\usepackage{lipsum}
\tcbuselibrary{skins,breakable}
\usetikzlibrary{shadings,shadows}





\newenvironment{myblock}[1]{%
    \tcolorbox[beamer,%
    noparskip,breakable,
    title=#1]}%
    {\endtcolorbox}
%---




\makeatletter
\newcommand{\chapterauthor}[1]{%
  {\parindent20pt\vspace*{-25pt}%
  \linespread{1.1}\large\scshape#1%
  \par\nobreak\vspace*{35pt}}
  \@afterheading%
}
\makeatother


\pagestyle{fancy}
\renewcommand{\chaptermark}[1]{\markboth{#1}{}}
% \renewcommand{\sectionmark}[1]{\markboth{#1}{}}

\def\@chapter[#1]#2{\ifnum \c@secnumdepth >\m@ne
                        \if@mainmatter
                          \refstepcounter{chapter}%
                          \typeout{\@chapapp\space\thechapter.}%
                          \addcontentsline{toc}{chapter}%
                                    {\protect\numberline{\thechapter}#2}%
                        \else
                          \addcontentsline{toc}{chapter}{#2}%
                        \fi
                     \else
                       \addcontentsline{toc}{chapter}{#2}%
                     \fi
                     \chaptermark{#1}%
                     \addtocontents{lof}{\protect\addvspace{10\p@}}%
                     \addtocontents{lot}{\protect\addvspace{10\p@}}%
                     \if@twocolumn
                       \@topnewpage[\@makechapterhead{#2}]%
                     \else
                       \@makechapterhead{#2}%
                       \@afterheading
                     \fi}

\setlength{\parskip}{\baselineskip}%
\setlength{\parindent}{0pt}%


\title{Einführung in die Kommunalstatistik}
\subtitle{Arbeitsunterlagen und Gesetzestexte}
\author{Zusammenstellung: Klaus Brückner, Statistikstelle der Stadt Passau}
\date{Herbst 2020}

\begin{document}
\dominitoc
\maketitle
\tableofcontents

\chapter[EU-StatV]{EU-Statistikverordnung}
\qrcode{https://eur-lex.europa.eu/legal-content/DE/TXT/PDF/?uri=CELEX:32009R0223&from=DE}
Link zum Volltext (pdf)
    \section{Art. 2: Statistische Grundsätze}
        \begin{enumerate}
            \item Für die Entwicklung, Erstellung und Verbreitung europäischer Statistiken gelten die folgenden statistischen Grundsätze:
            \begin{enumerate}
                \item ``Fachliche Unabhängigkeit'' bedeutet, dass die Statistiken auf unabhängige Weise entwickelt, erstellt und verbreitet werden müssen, insbesondere was die Wahl der zu verwendenden Verfahren, Definitionen, Methoden und Quellen sowie den Zeitpunkt und den Inhalt aller Verbreitungsformen anbelangt, ohne dass politische Gruppen oder Interessengruppen oder Stellen der Gemeinschaft oder einzelstaatliche Stellen Druck ausüben können; dies gilt unbeschadet institutioneller Rahmenbedingungen wie gemeinschaftlicher oder einzelstaatlicher institutioneller oder haushaltsrechtlicher Bestimmungen oder der Festlegung des statistischen Bedarfs.
                \item ``Unparteilichkeit'' bedeutet, dass die Statistiken auf neutrale Weise entwickelt, erstellt und verbreitet und dass alle Nutzer gleich behandelt werden müssen.
                \item ``Objektivität'' bedeutet, dass die Statistiken in systematischer, zuverlässiger und unvoreingenommener Weise entwickelt, erstellt und verbreitet werden müssen; dabei werden fachliche und ethische Standards angewandt und die angewandten Grundsätze und Verfahren sind für Nutzer und Befragte transparent.
                \item ``Zuverlässigkeit'' bedeutet, dass die Statistiken die Gegebenheiten, die sie abbilden sollen, so getreu, genau und konsistent wie möglich messen müssen, wobei zur Wahl der Quellen, Methoden und Verfahren wissenschaftliche Kriterien herangezogen werden.
                \item ``Statistische Geheimhaltung'' bedeutet, dass direkt für statistische Zwecke oder indirekt aus administrativen oder sonstigen Quellen eingeholte vertrauliche Angaben über einzelne statistische Einheiten geschützt werden müssen, wobei die Verwendung der eingeholten Angaben für nichtstatistische Zwecke und ihre unrechtmäßige Offenlegung untersagt sind.
                \item ``Kostenwirksamkeit'' bedeutet, dass die Kosten für die Erstellung der Statistiken in einem angemessenen Verhältnis zur Bedeutung des angestrebten Ergebnisses und Nutzens stehen und die Mittel optimal genutzt werden müssen und dass der Beantwortungsaufwand so gering wie möglich gehalten werden muss. Die verlangten Informationen werden nach Möglichkeit direkt aus vorhandenen Unterlagen oder Quellen entnommen.
            \end{enumerate}
            Die in diesem Absatz dargelegten statistischen Grundsätze werden in dem in Artikel 11 genannten Verhaltenskodex weiter ausgearbeitet.
            \item Bei der Entwicklung, Erstellung und Verbreitung europäischer Statistiken werden internationale Empfehlungen und vorbildliche Verfahren (best practice) berücksichtigt.
        \end{enumerate}
    \section{Art. 3: Definitionen}
        Für die Zwecke dieser Verordnung bezeichnet der Ausdruck:
        \begin{enumerate}
            \item ``Statistiken'' quantitative und qualitative, aggregierte und repräsentative Informationen, die ein Massenphänomen in einer betrachteten Grundgesamtheit beschreiben;
            \item ``Entwicklung'' die Tätigkeiten zur Festlegung, Stärkung und Verbesserung der für die Erstellung und Verbreitung von Statistiken verwendeten statistischen Methoden, Standards und Verfahren sowie zur Konzeption neuer Statistiken und Indikatoren;
            \item ``Erstellung'' alle im Zusammenhang mit der Erhebung, Speicherung, Verarbeitung und Analyse stehenden Tätigkeiten, die zur Erstellung von Statistiken erforderlich sind;
            \item ``Verbreitung'' die Tätigkeit, mit der Statistiken und statistische Analysen den Nutzern zugänglich gemacht werden;
            \item ``Datengewinnung'' Befragungen und alle sonstigen Methoden der Gewinnung von Informationen aus unterschiedlichen Quellen, einschließlich administrativer Quellen;
            \item ``Statistische Einheit'' die Grundbeobachtungseinheit, das heißt eine natürliche Person, ein Haushalt, ein Wirtschaftsteilnehmer oder eine sonstige Unternehmung, auf die sich die Daten beziehen;
            \item ``Vertrauliche Daten'' Daten, die eine direkte oder indirekte Identifizierung statistischer Einheiten möglich machen und dadurch Einzelinformationen offenlegen. Bei der Entscheidung, ob eine statistische Einheit identifizierbar ist, sind alle Mittel zu berücksichtigen, die nach vernünftigem Ermessen von einem Dritten angewendet werden könnten, um die statistische Einheit zu identifizieren;
            \item ``Verwendung für statistische Zwecke'' die ausschließliche Verwendung für die Entwicklung und Erstellung statistischer Ergebnisse und Analysen;
            \item ``Direkte Identifizierung'' die Identifizierung einer statistischen Einheit anhand ihres Namens oder ihrer Anschrift oder anhand einer öffentlich zugänglichen Identifikationsnummer;
            \item ``Indirekte Identifizierung'' die Identifizierung einer statistischen Einheit durch andere Mittel als die direkte Identifizierung;
            \item ``Beamte der Kommission (Eurostat)'' Beamte der Gemeinschaften im Sinne von Artikel 1 des Statuts der Beamten der Europäischen Gemeinschaften, die bei der statistischen Stelle der Gemeinschaft tätig sind;
            \item ``Sonstige Mitarbeiter der Kommission (Eurostat)'' Bedienstete der Gemeinschaften im Sinne der Artikel 2 bis 5 der Beschäftigungsbedingungen für die sonstigen Bediensteten der Europäischen Gemeinschaften, die bei der statistischen Stelle der Gemeinschaft tätig sind.
        \end{enumerate}
    \section{Art. 12: Qualität der Statistik}
        \begin{enumerate}
            \item Um die Qualität der Ergebnisse zu gewährleisten, werden europäische Statistiken auf der Grundlage einheitlicher Standards und nach harmonisierten Methoden entwickelt, erstellt und verbreitet. Dabei gelten die folgenden Qualitätskriterien:
            \begin{enumerate}
                \item ``Relevanz'': diese bezieht sich auf den Umfang, in dem die Statistiken dem aktuellen und potenziellen Nutzerbedarf entsprechen;
                \item ``Genauigkeit'': diese bezieht sich auf den Grad der Übereinstimmung der Schätzungen mit den unbekannten wahren Werten;
                \item ``Aktualität'': diese bezieht sich auf die Zeitspanne zwischen dem Vorliegen der Information und dem von ihr beschriebenen Ereignis oder Phänomen;
                \item ``Pünktlichkeit'': diese bezieht sich auf die Zeitspanne zwischen dem Zeitpunkt der Veröffentlichung der Daten und dem Zieltermin (Termin, zu dem die Daten geliefert werden sollten);
                \item ``Zugänglichkeit'' und ``Klarheit'': diese beziehen sich auf die Bedingungen und Modalitäten, unter denen die Nutzer Daten erhalten, verwenden und interpretieren können;
                \item ``Vergleichbarkeit'': diese bezieht sich auf die Messung der Auswirkungen von Unterschieden in den verwendeten statistischen Konzepten, Messinstrumenten und -verfahren bei Vergleichen von Statistiken für unterschiedliche geografische Gebiete oder thematische Bereiche oder bei zeitlichen Vergleichen;
                \item ``Kohärenz'': diese bezieht sich auf die Eignung der Daten, auf unterschiedliche Weise und für verschiedene Zwecke zuverlässig kombiniert zu werden.
            \end{enumerate}
            \item Bei der Anwendung der in Absatz 1 festgelegten Qualitätskriterien auf die unter sektorale Rechtsvorschriften in bestimmten Statistikbereichen fallenden Daten werden die Modalitäten, der Aufbau und die Periodizität der in den sektoralen Rechtsvorschriften vorgesehenen Qualitätsberichte von der Kommission nach dem in Artikel 27 Absatz 2 genannten Regelungsverfahren festgelegt.
            Besondere Qualitätsanforderungen wie Zielwerte und Mindeststandards für die Statistikproduktion können in sektoralen Rechtsvorschriften festgelegt sein. Enthalten die sektoralen Rechtsvorschriften keine derartigen Bestimmungen, kann die Kommission entsprechende Maßnahmen ergreifen. Diese Maßnahmen zur Änderung nicht wesentlicher Bestimmungen dieser Verordnung durch Ergänzung werden nach dem in Artikel 27 Absatz 3 genannten Regelungsverfahren mit Kontrolle erlassen.
            \item Die Mitgliedstaaten legen der Kommission (Eurostat) Berichte über die Qualität der übermittelten Daten vor. Die Kommission (Eurostat) bewertet die Qualität der übermittelten Daten und erstellt und veröffentlicht Berichte über die Qualität der europäischen Statistiken.
        \end{enumerate}
    \section{Art. 20: Schutz vertraulicher Daten}
        \begin{enumerate}
            \item Die folgenden Regeln und Maßnahmen gelten, um sicherzustellen, dass vertrauliche Daten ausschließlich für statistische Zwecke verwendet werden und ihre rechtswidrige Offenlegung verhindert wird.
            \item Vertrauliche Daten, die ausschließlich für die Erstellung europäischer Statistiken erhoben wurden, werden von den NSÄ und anderen einzelstaatlichen Stellen und von der Kommission (Eurostat) ausschließlich für statistische Zwecke verwendet, es sei denn, die statistische Einheit hat unmissverständlich ihre Zustimmung zur Verwendung der Daten zu anderen Zwecken erteilt.
            \item Statistische Ergebnisse, die die Identifizierung einer statistischen Einheit ermöglichen könnten, dürfen in folgenden Ausnahmefällen von den NSÄ und anderen einzelstaatlichen Stellen und der Kommission (Eurostat) verbreitet werden:
            \begin{enumerate}
                \item wenn in einem Rechtsakt des Europäischen Parlaments und des Rates gemäß Artikel 251 des Vertrags besondere Bedingungen und Modalitäten festgelegt sind und die statistischen Ergebnisse auf Ersuchen der statistischen Einheit so verändert werden, dass ihre Verbreitung die statistische Geheimhaltung nicht gefährdet; oder
                \item wenn die statistische Einheit der Offenlegung der Daten unmissverständlich zugestimmt hat.
            \end{enumerate}
            \item Die NSÄ und andere einzelstaatliche Stellen und die Kommission (Eurostat) ergreifen innerhalb ihrer jeweiligen Zuständigkeitsbereiche alle erforderlichen rechtlichen, administrativen, technischen und organisatorischen Maßnahmen, um den physischen und logischen Schutz vertraulicher Daten zu gewährleisten (statistische Offenlegungskontrolle).
            \item Die NSÄ und andere einzelstaatliche Stellen und die Kommission (Eurostat) ergreifen alle erforderlichen Maßnahmen, um die Harmonisierung der Grundsätze und Leitlinien für den physischen und logischen Schutz vertraulicher Daten zu gewährleisten. Diese Maßnahmen werden von der Kommission nach dem in Artikel 27 Absatz 2 genannten Regelungsverfahren erlassen.
            \item Beamte und sonstige Mitarbeiter der NSÄ und anderer einzelstaatlicher Stellen, die Zugang zu vertraulichen Daten haben, unterliegen auch nach ihrem Ausscheiden aus dem Dienst der statistischen Geheimhaltungspflicht.
        \end{enumerate}
\chapter[BStatG]{Bundesstatistikgesetz}
\minitoc
    \section{\S 1: Statistik für Bundeszwecke} 
    Die Statistik für Bundeszwecke (Bundesstatistik) hat im föderativ gegliederten Gesamtsystem der amtlichen Statistik die Aufgabe, laufend Daten über Massenerscheinungen zu erheben, zu sammeln, aufzubereiten, darzustellen und zu analysieren. Für sie gelten die Grundsätze der Neutralität, Objektivität und fachlichen Unabhängigkeit. Sie gewinnt die Daten unter Verwendung wissenschaftlicher Erkenntnisse und unter Einsatz der jeweils sachgerechten Methoden und Informationstechniken. Durch die Ergebnisse der Bundesstatistik werden gesellschaftliche, wirtschaftliche und ökologische Zusammenhänge für Bund, Länder einschliesslich Gemeinden und Gemeindeverbände, Gesellschaft, Wirtschaft, Wissenschaft und Forschung aufgeschlüsselt. Die Bundesstatistik ist Voraussetzung für eine am Sozialstaatsprinzip ausgerichtete Politik. Die für die Bundesstatistik erhobenen Einzelangaben dienen ausschliesslich den durch dieses Gesetz oder eine andere eine Bundesstatistik anordnende Rechtsvorschrift festgelegten Zwecken.
    \section{\S 10: Erhebungs- und Hilfsmerkmale}
        \begin{enumerate}[label=(\arabic*)]
            \item Bundesstatistiken werden auf der Grundlage von Erhebungs- und Hilfsmerkmalen erstellt. Erhebungsmerkmale umfassen Angaben über persönliche und sachliche Verhältnisse, die zur statistischen Verwendung bestimmt sind. Hilfsmerkmale sind Angaben, die der technischen Durchführung von Bundesstatistiken dienen. Für andere Zwecke dürfen sie nur verwendet werden, soweit Absatz 2 oder ein sonstiges Gesetz es zulassen.
            \item Der Name der Gemeinde, die Blockseite und die geografische Gitterzelle dürfen für die regionale Zuordnung der Erhebungsmerkmale genutzt werden. Die übrigen Teile der Anschrift dürfen für die Zuordnung zu Blockseiten und geografischen Gitterzellen für einen Zeitraum von bis zu vier Jahren nach Abschluss der jeweiligen Erhebung genutzt werden. Besondere Regelungen in einer eine Bundesstatistik anordnenden Rechtsvorschrift bleiben unberührt.
            \item Blockseite ist innerhalb eines Gemeindegebiets die Seite mit gleicher Strassenbezeichnung von der durch Strasseneinmündungen oder vergleichbare Begrenzungen umschlossenen Fläche. Eine geografische Gitterzelle ist eine Gebietseinheit, die bezogen auf eine vorgegebene Kartenprojektion quadratisch ist und mindestens 1 Hektar gross ist.
        \end{enumerate}
    \section{\S 12 Trennung und Löschung der Hilfsmerkmale}
        \begin{enumerate}[label=(\arabic*)]
            \item Hilfsmerkmale sind, soweit Absatz 2, § 10 Absatz 2, § 13 oder eine sonstige Rechtsvorschrift nichts anderes bestimmen, zu löschen, sobald bei den statistischen Ämtern die Überprüfung der Erhebungs- und Hilfsmerkmale auf ihre Schlüssigkeit und Vollständigkeit abgeschlossen ist. Sie sind von den Erhebungsmerkmalen zum frühestmöglichen Zeitpunkt zu trennen und gesondert aufzubewahren oder gesondert zu speichern.
            \item Bei periodischen Erhebungen für Zwecke der Bundesstatistik dürfen die zur Bestimmung des Kreises der zu Befragenden erforderlichen Hilfsmerkmale, soweit sie für nachfolgende Erhebungen benötigt werden, gesondert aufbewahrt oder gesondert gespeichert werden. Nach Beendigung des Zeitraumes der wiederkehrenden Erhebungen sind sie zu löschen.
        \end{enumerate}
    \section{\S 16 Geheimhaltung}
        \begin{enumerate}[label=(\arabic*)]
            \item Einzelangaben über persönliche und sachliche Verhältnisse, die für eine Bundesstatistik gemacht werden, sind von den Amtsträgern und Amtsträgerinnen und für den öffentlichen Dienst besonders Verpflichteten, die mit der Durchführung von Bundesstatistiken betraut sind, geheim zu halten, soweit durch besondere Rechtsvorschrift nichts anderes bestimmt ist. Die Geheimhaltungspflicht besteht auch nach Beendigung ihrer Tätigkeit fort. Die Geheimhaltungspflicht gilt nicht für
                \begin{enumerate}[label=\arabic*.]
                    \item Einzelangaben, in deren Übermittlung oder Veröffentlichung die Betroffenen schriftlich eingewilligt haben, soweit nicht wegen besonderer Umstände eine andere Form der Einwilligung angemessen ist,
                    \item Einzelangaben aus allgemein zugänglichen Quellen, wenn sie sich auf die in § 15 Absatz 1 genannten öffentlichen Stellen beziehen, auch soweit eine Auskunftspflicht aufgrund einer eine Bundesstatistik anordnenden Rechtsvorschrift besteht,
                    \item Einzelangaben, die vom Statistischen Bundesamt oder den statistischen Äm\-tern der Länder mit den Einzelangaben anderer Befragter zusammengefasst und in statistischen Ergebnissen dargestellt sind, 
                    \item Einzelangaben, wenn sie den Befragten oder Betroffenen nicht zuzuordnen sind. Die §§ 93, 97, 105 Absatz 1, § 111 Absatz 5 in Verbindung mit § 105 Absatz 1 sowie § 116 Absatz 1 der Abgabenordnung vom 16. März 1976 (BGBl. I S. 613; 1977 I S. 269), zuletzt geändert durch Artikel 1 des Gesetzes vom 19. Dezember 1985 (BGBl. I S. 2436), gelten nicht für Personen und Stellen, soweit sie mit der Durchführung von Bundes- , Landes- oder Kommunalstatistiken betraut sind.
                \end{enumerate}
            \item Die Übermittlung von Einzelangaben zwischen den mit der Durchführung einer Bundesstatistik betrauten Personen und Stellen ist zulässig, soweit dies zur Erstellung der Bundesstatistik erforderlich ist. Darüber hinaus ist die Übermittlung von Einzelangaben zwischen den an einer Zusammenarbeit nach § 3a beteiligten statistischen Ämtern und die zentrale Verarbeitung und Nutzung dieser Einzelangaben in einem oder mehreren statistischen Ämtern zulässig.
            \item Das Statistische Bundesamt darf an die statistischen Ämter der Länder die ihren jeweiligen Erhebungsbereich betreffenden Einzelangaben für Sonderaufbereitungen auf regionaler Ebene übermitteln. Für die Erstellung der Volkswirtschaftlichen Gesamtrechnungen und sonstiger Gesamtsysteme des Bundes und der Länder dürfen sich das Statistische Bundesamt und die statistischen Ämter der Länder untereinander Einzelangaben aus Bundesstatistiken übermitteln.
            \item Für die Verwendung gegenüber den gesetzgebenden Körperschaften und für Zwecke der Planung, jedoch nicht für die Regelung von Einzelfällen, dürfen den obersten Bundes- oder Landesbehörden vom Statistischen Bundesamt und den statistischen Ämtern der Länder Tabellen mit statistischen Ergebnissen übermittelt werden, auch soweit Tabellenfelder nur einen einzigen Fall ausweisen. Die Übermittlung nach Satz 1 ist nur zulässig, soweit in den eine Bundesstatistik anordnenden Rechtsvorschriften die Übermittlung von Einzelangaben an oberste Bundes- oder Landesbehörden zugelassen ist.
            \item Für ausschließlich statistische Zwecke dürfen vom Statistischen Bundesamt und den statistischen Ämtern der Länder Einzelangaben an die zur Durchführung statistischer Aufgaben zuständigen Stellen der Gemeinden und Gemeindeverbände übermittelt werden, wenn die Übermittlung in einem eine Bundesstatistik anordnenden Gesetz vorgesehen ist sowie Art und Umfang der zu übermittelnden Einzelangaben bestimmt sind. Die Übermittlung ist nur zulässig, wenn durch Landesgesetz eine Trennung dieser Stellen von anderen kommunalen Verwaltungsstellen sichergestellt und das Statistikgeheimnis durch Organisation und Verfahren gewährleistet ist. 
            \item Für die Durchführung wissenschaftlicher Vorhaben dürfen das Statistische Bundesamt und die statistischen Ämter der Länder Hochschulen oder sonstigen Einrichtungen mit der Aufgabe unabhängiger wissenschaftlicher Forschung
                \begin{enumerate}[label=\arabic*.]
                    \item Einzelangaben übermitteln, wenn die Einzelangaben nur mit einem un\-ver\-hält\-nis\-mäßig großen Aufwand an Zeit, Kosten und Arbeitskraft zugeordnet werden können (faktisch anonymisierte Einzelangaben),
                    \item innerhalb speziell abgesicherter Bereiche des Statistischen Bundesamtes und der statistischen Ämter der Länder Zugang zu formal anonymisierten Einzelangaben gewähren, wenn wirksame Vorkehrungen zur Wahrung der Geheimhaltung getroffen werden. Berechtigte können nur Amtsträger oder Amtsträgerinnen, für den öffentlichen Dienst besonders Verpflichtete oder Verpflichtete nach Absatz 7 sein.
                \end{enumerate}
            \item Personen, die Einzelangaben nach Absatz 6 erhalten sollen, sind vor der Über\-mitt\-lung zur Geheimhaltung zu verpflichten, soweit sie nicht Amtsträger oder Amtsträgerinnen oder für den öffentlichen Dienst besonders Verpflichtete sind. § 1 Absatz 2, 3 und 4 Nummer 2 des Verpflichtungsgesetzes vom 2. März 1974 (BGBl. I S. 469, Artikel 42), das durch Gesetz vom 15. August 1974 (BGBl. I S. 1942) geändert worden ist, gilt entsprechend. 
            \item Die aufgrund einer besonderen Rechtsvorschrift oder der Absätze 4, 5 oder 6 übermittelten Einzelangaben dürfen nur für die Zwecke verwendet werden, für die sie übermittelt wurden. In den Fällen des Absatzes 6 Satz 1 Nummer 1 sind sie zu löschen, sobald das wissenschaftliche Vorhaben durchgeführt ist. Bei den Stellen, denen Einzelangaben übermittelt werden, muss durch organisatorische und technische Maßnahmen sichergestellt sein, dass nur Amtsträger, für den öffentlichen Dienst besonders Verpflichtete oder Verpflichtete nach Absatz 7 Satz 1 Empfänger von Einzelangaben sind.
            \item Die Übermittlung aufgrund einer besonderen Rechtsvorschrift oder nach den Ab\-sätzen 4, 5 oder 6 ist nach Inhalt, Stelle, der übermittelt wird, Datum und Zweck der Weitergabe von den statistischen Ämtern aufzuzeichnen. Die Aufzeichnungen sind mindestens fünf Jahre aufzubewahren.
            \item Die Pflicht zur Geheimhaltung nach Absatz 1 besteht auch für die Personen, die Empfänger von Einzelangaben aufgrund einer besonderen Rechtsvorschrift, nach den Absätzen 5, 6 oder von Tabellen nach Absatz 4 sind. Dies gilt nicht für offenkundige Tatsachen bei einer Übermittlung nach Absatz 4.
        \end{enumerate}
\chapter[BayStatG]{Bayerisches Statistikgesetz}
\qrcode{https://www.gesetze-bayern.de/Content/Pdf/BayStatG?all=True}
\newline
\url{https://www.gesetze-bayern.de/Content/Pdf/BayStatG?all=True}
    \section{Art. 1: Geltungsbereich}
        \begin{enumerate}[label=(\arabic*)]
            \item \textsuperscript{1}Dieses Gesetz gilt für die Durchführung von Statistiken durch öffentliche Stellen. \textsuperscript{2}Führen diese Stellen Bundesstatistiken oder europäische Statistiken durch und haben sie dabei andere Rechtsvorschriften anzuwenden, so finden die Vorschriften dieses Gesetzes nur ergänzend Anwendung.
            \item Für Geschäftsstatistiken gilt dieses Gesetz nur, soweit das ausdrücklich bestimmt ist.
        \end{enumerate}
    \section{Art. 2: Begriffe}
        \begin{enumerate}[label=(\arabic*)]
            \item \textsuperscript{1}Amtliche Statistiken sind Landesstatistiken, Bundesstatistiken und europäische Statistiken. \textsuperscript{2}Landesstatistiken sind Statistiken, die von Organen des Freistaates Bayern angeordnet und von staatlichen Stellen durchgeführt werden.
            \item Kommunale Statistiken sind Statistiken, die von Gemeinden oder Gemeindever\-bän\-den zur Wahrnehmung ihrer Aufgaben durchgeführt werden.
            \item Geschäftsstatistiken sind statistische Aufbereitungen von Daten, die bei öffentli\-chen Stellen im Vollzug ihrer Aufgaben, die nicht die Durchführung von Statistiken betreffen, erhoben werden oder auf sonstige Weise anfallen.
            \item Öffentliche Stellen sind alle Behörden, Gerichte und sonstige öffentliche Stellen des Freistaates Bayern, die Gemeinden und Gemeindeverbände sowie die der Aufsicht des Freistaates Bayern unterstehenden juristischen Personen des öffentlichen Rechts und deren Vereinigungen.
            \item Einzelangaben sind Daten über persönliche oder sachliche Verhältnisse bestimmter oder bestimmbarer natürlicher oder juristischer Personen und deren Vereinigungen, die bei der Durchführung einer Statistik erhoben oder übermittelt werden.
        \end{enumerate}
    \section[Art. 3: EU-DSGVO und BayDatSchG]{Art. 3: Anwendbarkeit der Datenschutz-Grundverordnung und des Bayerischen Datenschutzgesetzes}
        \begin{enumerate}[label=(\arabic*)]
            \item Die Ansprüche nach den Art. 15, 16, 18 und 21 der Verordnung (EU) 2016/679 (Datenschutz- Grundverordnung– DSGVO) bestehen nicht, soweit diese Rechte die Verwirklichung statistischer Zwecke ernsthaft beeinträchtigen würden.
            \item Einzelangaben dürfen an das Landesamt und an Statistikstellen für die Durch\-füh\-rung von Geschäftsstatistiken übermittelt und von dort – auch in aufbereiteter Form – rückübermittelt werden
        \end{enumerate}
    (\dots)
    \section{Art. 9: Anordnung}
        \begin{enumerate}[label=(\arabic*)]
            \item \textsuperscript{1}Statistiken werden durch Gesetz oder Rechtsverordnung angeordnet. 2Die Anordnung bedarf keiner Rechtsvorschrift, wenn
            \begin{enumerate}[label=\arabic*.]
                \item die einer Statistik zugrundeliegenden Daten
                    \begin{enumerate}[label=(\alph*)]
                        \item auf freiwilligen Auskünften oder allgemein zugänglichen Quellen beruhen,
                        \item keine Einzelangaben enthalten oder
                        \item der die Statistik durchführenden Stelle rechtmässig übermittelt werden oder ihrem Zugriff auf Grund einer Rechtsvorschrift zur Verfügung stehen;
                    \end{enumerate}
                \item lediglich Sonderauswertungen vorhandenen statistischen Materials vorgenommen werden, dessen Verwendung eine Zweckbindung nicht entgegensteht, oder
                \item zur Anordnung der Statistik eine Rechtsvorschrift ermächtigt.
            \end{enumerate}
            \item Die eine Landesstatistik anordnende Rechtsvorschrift muss die näheren Bestimmungen treffen über die Art der Erhebung, den Kreis der zu Befragenden, sonstige Auskunftsstellen, die durch Erhebungsmerkmale zu erfassenden Sachverhalte, die Hilfsmerkmale, den Berichtszeitraum, den Berichtszeitpunkt, die Häufigkeit der Erhebung (Periodizität) sowie über Art und Umfang einer Auskunftspflicht.
        \end{enumerate}
    \section{Art. 14: Erhebungsbeauftragte}
        \begin{enumerate}[label=(\arabic*)]
            \item Als Erhebungsbeauftragte dürfen nur Personen eingesetzt werden, die Gewähr für Zuverlässigkeit und Verschwiegenheit bieten und bei denen nicht auf Grund ihrer beruflichen Tätigkeit oder aus anderen Gründen Anlass zur Besorgnis besteht, dass Erkenntnisse aus der Tätigkeit als Erhebungsbeauftragte zu Lasten der Auskunftspflichtigen genutzt werden.
            \item \textsuperscript{1}Erhebungsbeauftragte sind verpflichtet, die Anweisungen der Erhebungsstellen zu befolgen. \textsuperscript{2}Bei der Ausübung ihrer Tätigkeit haben sie sich auszuweisen. \textsuperscript{3}Sie dürfen statistische Einzelangaben und gelegentlich ihrer Tätigkeit gewonnene Erkenntnisse auch nach Beendigung ihrer Tätigkeit nicht für andere Verfahren oder andere Zwecke verarbeiten.
            \item \textsuperscript{1}Erhebungsbeauftragte sind über ihre Rechte und Pflichten sowie über die Rechte und Pflichten der zu Befragenden zu belehren. \textsuperscript{2}Vor ihrem Einsatz sind sie auf die Wahrung des Statistikgeheimnisses und zur Geheimhaltung der Erkenntnisse, die sie bei ihrer Tätigkeit gewonnen haben, schriftlich zu verpflichten.
        \end{enumerate}
    \section{Art. 15: Erhebungs- und Hilfsmerkmale}
        \begin{enumerate}[label=(\arabic*)]
            \item \textsuperscript{1}Erhebungsmerkmale sind zur Erstellung einer Statistik bestimmte Angaben über persönliche oder sachliche Verhältnisse. \textsuperscript{2}Hilfsmerkmale sind Angaben, die der technischen Durchführung von Statistiken dienen.
            \item \textsuperscript{1}Hilfsmerkmale sind von den Erhebungsmerkmalen zum frühestmöglichen Zeitpunkt zu trennen und gesondert aufzubewahren. \textsuperscript{2}Laufende Nummern und Ordnungsnummern können auf den Erhebungsunterlagen verbleiben. \textsuperscript{3}Sie dürfen auf die für die maschinelle Weiterverarbeitung bestimmten Datenträger übernommen werden.
            \item \textsuperscript{1}Die Hilfsmerkmale sind zu löschen, sobald die Überprüfung der Erhebungs- und Hilfsmerkmale auf Schlüssigkeit und Vollständigkeit abgeschlossen ist. \textsuperscript{2}Bei wiederkehrenden Erhebungen kann die Löschung der Hilfsmerkmale unterbleiben, soweit sie noch künftig zur Bestimmung des Kreises der zu Befragenden benötigt werden. \textsuperscript{3}Die Hilfsmerkmale sind gesondert aufzubewahren und nach Beendigung der wiederkehrenden Erhebungen zu löschen. \textsuperscript{4}Diese Vorschriften gelten entsprechend für die Vernichtung von Erhebungsunterlagen, die Hilfsmerkmale enthalten.
            \item \textsuperscript{1}Die Namen von Gemeinden und von Gemeindeteilen sowie Blockseiten dürfen für die regionale Zuordnung von Erhebungsmerkmalen genutzt werden. \textsuperscript{2}Blockseite ist innerhalb eines Gemeindegebiets die Seite mit gleicher Strassenbezeichnung von der durch Strasseneinmündungen oder vergleichbare Begrenzungen umschlossenen Fläche. \textsuperscript{3}Die übrigen Teile der Anschrift dürfen für die Zuordnung zu Blockseiten für einen Zeitraum bis zu vier Jahren nach Abschluss der jeweiligen Erhebung genutzt werden. 4Besondere Regelungen in einer eine amtliche Statistik anordnenden Rechtsvorschrift bleiben unberührt.
            \item Absatz 2 Satz 1 und Absatz 3 Satz 1 und 3 gelten nicht für Daten, die ausschliesslich einer öffentlichen Stelle zugeordnet werden können.
        \end{enumerate}   
    
    \section{Art. 17: Geheimhaltung}
        \begin{enumerate}[label=(\arabic*)]
            \item \textsuperscript{1}Einzelangaben sind von den mit der Durchführung der Statistik betrauten Stellen und Personen geheimzuhalten. \textsuperscript{2}Dies gilt nicht für
                \begin{enumerate}[label=\arabic*.]
                    \item Einzelangaben, in deren Übermittlung oder Veröffentlichung die Auskunftgebenden oder die betroffenen Personen schriftlich eingewilligt haben;
                    \item Einzelangaben, soweit deren Übermittlung oder Veröffentlichung durch Art. 18 oder durch besondere Rechtsvorschrift zugelassen ist;
                    \item Einzelangaben aus allgemein zugänglichen Quellen;
                    \item Einzelangaben, die ausschliesslich einer öffentlichen Stelle, die nicht am wirtschaftlichen Wettbewerb teilnimmt, zugeordnet werden können;
                    \item Einzelangaben, die keiner befragten oder betroffenen Person zuzuordnen sind, insbesondere, wenn sie mit den Einzelangaben anderer zusammengefasst und in statistischen Ergebnissen dargestellt sind.
                \end{enumerate}
            \textsuperscript{3}Die Pflicht zur Geheimhaltung besteht auch für Personen, die Empfänger von Einzelangaben nach Art. 18 oder auf Grund einer besonderen Rechtsvorschrift sind.
            \item Sonstige Vorschriften über die Geheimhaltung und Verschwiegenheit bleiben unberührt.
        \end{enumerate}
    \section{Art. 18: Zweckbindung und Übermittlung von Einzelangaben}
        \begin{enumerate}[label=(\arabic*)]
            \item Einzelangaben dürfen ausschliesslich für statistische Zwecke verarbeitet werden, es sei denn sie beruhen auf allgemein zugänglichen Quellen oder eine Rechtsvorschrift lässt eine andere Verwendung zu.
            \item \textsuperscript{1}Das Landesamt darf Einzelangaben, wenn eine ausdrückliche Zweckbindung nicht entgegensteht, an Statistikstellen anderer öffentlicher Stellen für deren Zuständig\-keitsbereich zu ausschliesslich statistischen Zwecken übermitteln. \textsuperscript{2}Soweit durch Rechtsvorschrift nichts anderes bestimmt ist, dürfen Hilfsmerkmale nicht über\-mit\-telt werden.
            \item Zur Erstellung koordinierter Länderstatistiken darf das Landesamt Einzel\-an\-ga\-ben an das Statistische Bundesamt und die Statistischen Ämter der Länder über\-mitteln.
            \item \textsuperscript{1}Für Gesetzesvorhaben und für Zwecke der Planung, nicht jedoch für die Regelung von Einzelfällen, darf das Landesamt den Staatsministerien Tabellen mit statistischen Ergebnissen übermitteln, auch soweit Tabellenfelder nur einen einzigen Fall ausweisen. \textsuperscript{2}Durch organisatorische und technische Massnahmen muss sichergestellt sein, dass nur Amtsträger und für den öffentlichen Dienst besonders Verpflichtete Kenntnis von Einzelangaben erhalten.
            \item \textsuperscript{1}Für die Durchführung wissenschaftlicher Vorhaben darf das Landesamt Einzelangaben an Hochschulen oder sonstige Einrichtungen mit der Aufgabe unabhängiger wissenschaftlicher Forschung übermitteln, wenn die Einzelangaben nur mit einem unverhältnismässig grossen Aufwand an Zeit, Kosten und Arbeitskraft zugeordnet werden können. \textsuperscript{2}Sofern es sich bei den Empfängern nicht um Amtsträger oder für den öffentlichen Dienst besonders Verpflichtete handelt, sind sie vor der Übermittlung vom Landesamt besonders zur Geheimhaltung zu verpflichten. \textsuperscript{3}\S 1 Abs. 2, 3 und 4 Nr. 2 des Verpflichtungsgesetzes sind entsprechend anwendbar. \textsuperscript{4}Personen, die nach Satz 2 besonders verpflichtet worden sind, stehen für die Anwendung der Vorschriften des Strafgesetzbuches über die Verletzung von Privatgeheimnissen (\S 203 Abs. 2, 4, 5, \S 204, 205) den für den öffentlichen Dienst besonders Verpflichteten gleich. 5Empfänger haben durch technische und organisatorische Massnahmen sicherzustellen, dass sonstige Personen keine Kenntnis von Einzelangaben erhalten. 6Die Einzelangaben sind zu löschen oder zu vernichten, sobald das wissenschaftliche Vorhaben abgeschlossen ist, zu dessen Durchführung sie übermittelt wurden.
            \item \textsuperscript{1}Einzelangaben, die auf Grund der Absätze 2 bis 5 oder auf Grund einer besonderen Rechtsvorschrift übermittelt werden, dürfen nur für den Zweck verwendet werden, für den sie über\-mittelt worden sind. \textsuperscript{2}Die Übermittlung ist vom Landesamt unter Angabe von Inhalt, empfangender Stelle, Datum und Zweck aufzuzeichnen. \textsuperscript{3}Die Aufzeichnungen sind mindestens fünf Jahre aufzubewahren.
            \item Einzelangaben dürfen vom Landesamt wieder an die auskunftsgebende Stelle über\-mittelt werden. 
            \item Die Absätze 2 bis 7 gelten entsprechend, wenn Statistikstellen anderer staatlicher Stellen für die Durchführung von Landesstatistiken zuständig sind. 
        \end{enumerate}
        
    \section{Art. 20: Statistikstellen}
        \begin{enumerate}[label=(\arabic*)]
            \item \textsuperscript{1}Werden Statistiken ausserhalb des Landesamts durchgeführt, so sind besondere Statistikstellen einzurichten. \textsuperscript{2}Nichtstatistische Aufgaben des Verwaltungsvollzugs dürfen ihnen nicht übertragen werden. \textsuperscript{3}Statistikstellen veröffentlichen die Ergebnisse ihrer Statistiken oder stellen sie in sonstiger Weise bereit.
            \item \textsuperscript{1}Für jede Statistikstelle ist jemand zu bestimmen, der diese leitet. \textsuperscript{2}Statistikstellen sind räumlich und organisatorisch von anderen Verwaltungsstellen zu trennen, gegen den Zutritt unbefugter Personen hinreichend zu sichern und mit Personal auszustatten, das die Gewähr für Zuverlässigkeit und Verschwiegenheit bietet.
            \item \textsuperscript{1}Die in Statistikstellen tätigen Personen dürfen statistische Einzelangaben und gelegentlich ihrer Tätigkeit gewonnene Erkenntnisse auch nach Beendigung ihrer Tätigkeit nicht in anderen Verfahren oder für andere Zwecke verarbeiten, soweit nicht durch Rechtsvorschrift etwas anderes zugelassen ist. \textsuperscript{2}Sie sind vor ihrem Einsatz auf die Wahrung des Statistikgeheimnisses und über die Folgen seiner Verletzung zu belehren und schriftlich zu verpflichten. \textsuperscript{3}Soweit und solange sie Einzelangaben bearbeiten, dürfen sie nicht andere Aufgaben des Verwaltungsvollzugs wahrnehmen. \textsuperscript{4}Im Anschluss an eine Tätigkeit in der Statistikstelle sollen sie nicht für Aufgaben eingesetzt werden, bei denen eine Nutzung der in den Statistikstellen gewonnenen Erkenntnisse möglich ist, soweit das die organisatorischen und personellen Verhältnisse zulassen.
            \item Statistikstellen können mit der Durchführung von Geschäftsstatistiken beauftragt werden-
        \end{enumerate}



    \section{Art. 21: Erhebungsstellen, Verordnungsermächtigung}
        \begin{enumerate}[label=(\arabic*)]
            \item Das Landesamt ist bei Statistiken, die es als allgemeine Aufgabe durchführt, Erhebungsstelle.
            \item \textsuperscript{1}Die Staatsregierung wird ermächtigt, durch Rechtsverordnung zu bestimmen, dass andere staatliche Stellen sowie Gemeinden Erhebungsstellen einzurichten oder in sonstiger Weise an der Durchführung amtlicher Statistiken mitzuwirken haben, wenn das wegen der Art der Erhebung, der Zahl oder der räumlichen Verteilung der zu Befragenden oder zur Sicherung der Qualität der Erhebung zweckmässig ist. \textsuperscript{2}Eine aufsichtliche Zuständigkeit des Landesamts wird durch eine solche Bestimmung nicht begründet. \textsuperscript{3}Landratsämter erfüllen die Aufgaben der Erhebungsstellen als Staatsaufgaben; für Gemeinden handelt es sich um Aufgaben des übertragenen Wirkungskreises, die sie auch nach den Vorschriften des Gesetzes über die kommunale Zusammenarbeit erfüllen können.
            \item \textsuperscript{1}Die Erhebungsstellen nach Absatz 2 Satz 1 führen in ihrem jeweiligen Zu\-stän\-dig\-keits\-be\-reich die statistischen Erhebungen durch. 2Art. 20 Abs. 2 und 3 gelten entsprechend mit der Massgabe, dass die räumliche und organisatorische Trennung von anderen Verwaltungsstellen ab dem Eingang der Erhebungsunterlagen bis zu ihrer Ablieferung sicherzustellen ist. 3Durch Rechtsverordnung nach Absatz 2 Satz 1 können Abweichungen von den Anforderungen des Art. 20 Abs. 2 und 3 bestimmt werden, wenn das ein erweiterter Schutz von Einzelangaben erforderlich macht oder wenn eine andere staatliche Stelle oder eine Gemeinde an der Erhebung lediglich mitwirkt. 4Soweit nichts anderes bestimmt ist, haben diese Erhebungsstellen
                \begin{enumerate}[label=\arabic*.]
                    \item bei Bedarf Erhebungsbezirke festzulegen;
                    \item die Erhebungsbeauftragten auszuwählen, zu bestellen, zu unterrichten, zu verpflichten und zu beaufsichtigen;
                    \item die zu Befragenden gemäss Art. 19 zu unterrichten, zur Auskunft heranzuziehen, die Erhebungsvordrucke auszuteilen und einzusammeln;
                    \item Personen, die noch keine Auskünfte gegeben haben, zur Auskunftserteilung anzuhalten;
                    \item die Vollzähligkeit der ausgefüllten Erhebungsvordrucke sowie deren Voll\-stän\-dig\-keit und die formale Richtigkeit der Angaben zu überprüfen;
                    \item unvollständige oder offensichtlich fehlerhaft ausgefüllte Erhebungsvordrucke durch Nachfrage bei den Befragten zu ergänzen oder zu berichtigen.
                \end{enumerate}
            \item Stellen nach Absatz 2 Satz 1 sind nicht berechtigt, erhobenes Material für eigene Auswertungen zu nutzen.
        \end{enumerate}

    \section{Art. 22: Zulässigkeit}
        Gemeinden und Gemeindeverbände sowie andere nichtstaatliche juristische Personen des öffentlichen Rechts können für die Wahrnehmung ihrer Aufgaben im Rahmen ihrer Zuständigkeiten und Befugnisse Statistiken durchführen, wenn Einzelangaben oder Ergebnisse vom Landesamt oder von anderen öffentlichen Stellen weder zur Verfügung gestellt noch anderweitig ermittelt werden können und eigene Statistikstellen eingerichtet werden.

    \section{Art. 23: Anordnung}
        \begin{enumerate}[label=(\arabic*)]
            \item \textsuperscript{1}Statistiken für die Wahrnehmung von Selbstverwaltungsaufgaben (eigener Wirkungskreis) sind durch Satzung anzuordnen; in ihr sind zugleich die erforderlichen Bestimmungen nach Art. 9 Abs. 2 zu treffen. \textsuperscript{2}Die Anordnung bedarf keiner Satzung, wenn
                \begin{enumerate}[label=\arabic*.]
                    \item die einer Statistik zugrundeliegenden Daten auf allgemein zugänglichen Quellen beruhen, keine Einzelangaben enthalten, der Statistikstelle rechtmässig übermittelt werden oder ihrem Zugriff auf Grund einer Rechtsvorschrift zur Verfügung stehen oder
                    \item lediglich Sonderauswertungen vorhandenen statistischen Materials vorgenommen werden, dessen Verwendung eine Zweckbindung nicht entgegensteht.
                \end{enumerate}
                \textsuperscript{3}Bei juristischen Personen, denen kein Satzungsrecht zusteht, werden Statistiken durch die zuständigen Organe angeordnet. \textsuperscript{4}Durch Satzungen können Gemeinden, Landkreise und Bezirke auch eine Auskunftspflicht begründen, wenn es der Zweck der Erhebung erfordert, und zulassen, dass Statistikstellen Adressdateien in entsprechender Anwendung der für amtliche Statistiken geltenden Vorschriften führen und nutzen.
            \item \textsuperscript{1}Statistiken für die Wahrnehmung von übertragenen Aufgaben (übertragener Wirkungskreis) bedürfen einer Anordnung durch Gesetz oder staatliche Rechtsverordnung, es sei denn, es liegen die Voraussetzungen vor, unter denen auch für eine amtliche Statistik keine Anordnung durch Rechtsvorschrift erforderlich ist (Art. 9 Abs. 1 Satz 2). \textsuperscript{2}In den Fällen des Art. 9 Abs. 1 Satz 2 Nr. 1 bedarf die Statistik einer Genehmigung durch den Statistischen Genehmigungsausschuss (Art. 10).
        \end{enumerate}

    \section{Art. 24: Statistikstellen}
        \begin{enumerate}[label=(\arabic*)]
            \item \textsuperscript{1}Statistikstellen führen die angeordneten Statistiken durch. \textsuperscript{2}In die Wahrnehmung nichtstatistischer Aufgaben des Verwaltungsvollzugs dürfen Statistikstellen nicht eingeschaltet werden. \textsuperscript{3}Statistikstellen veröffentlichen die Ergebnisse der von ihnen erstellten Statistiken oder stellen sie in sonstiger Weise bereit, wenn ein öffentliches Bedürfnis besteht.
            \item \textsuperscript{1}Statistikstellen sind durch Satzung einzurichten, die auch die wesentlichen organisatorischen Bestimmungen, vornehmlich zur Wahrung des Statistikgeheimnisses zu treffen hat. \textsuperscript{2}Art. 20 Abs. 2 und 3 finden entsprechende Anwendung. \textsuperscript{3}Kommunale Statistikstellen können auch nach Massgabe des Gesetzes über die kommunale Zusammenarbeit eingerichtet werden. \textsuperscript{4}Bei juristischen Personen, denen kein Satzungsrecht zukommt, werden Statistikstellen durch die zuständigen Organe nach Massgabe der Sätze 1 und 2 eingerichtet.
            \item  \textsuperscript{2}Geschäftsstatistiken führen die Statistikstellen durch, wenn sie damit beauftragt werden. \textsuperscript{1}Kommunale Statistikstellen können die Ergebnisse von Europa-, Bundes-, Landes- und Kommunalwahlen aufbereiten. \textsuperscript{3}Sie nehmen die Aufgaben einer Erhebungsstelle im Sinn des Art. 21 Abs. 2 Satz 1 wahr.
        \end{enumerate}
        
    \section{Art.34: Reidentifizierungsverbot}
        Die Zusammenführung
        \begin{enumerate}[label=\arabic*.]
            \item von Einzelangaben aus Statistiken öffentlicher Stellen oder
            \item von Einzelangaben aus Statistiken öffentlicher Stellen mit anderen Angaben
        \end{enumerate}
         zum Zweck der Herstellung eines Personen-, Unternehmens-, Betriebs- oder Ar\-beits\-stät\-ten\-be\-zugs ist untersagt, es sei denn, die Aufgabenstellung dieses Gesetzes oder einer anderen Rechtsvorschrift oder ein sonstiger eine Statistik einer öffentlichen Stelle anordnender Rechtsakt lassen das zu. 
     
    \section{Art. 35: Strafvorschrift}
        Wer entgegen Art. 34 Einzelangaben aus Statistiken öffentlicher Stellen oder solche Einzelangaben mit anderen Angaben zusammenführt, wird mit Freiheitsstrafe bis zu einem Jahr oder mit Geldstrafe bestraft.
% \chapter{LStatG NRW}{Statistikgesetz Nordrhein-Westfalen}
\qrcode{https://recht.nrw.de/lmi/owa/br_text_anzeigen?v_id=92220190717090732842}
\newline
\url{https://recht.nrw.de/lmi/owa/br_text_anzeigen?v_id=92220190717090732842}

\chapter[LStatGMV]{Landesstatistikgesetz Mecklenburg-Vorpommern}
\minitoc
    \section{\S1 Geltungsbereich}
    Dieses Gesetz gilt
    \begin{enumerate}[label=\arabic*.]
        \item ergänzend zum Bundesstatistikgesetz (BStatG) für die Durchführung von
            \begin{enumerate}[label=\alph*)]
                \item Statistiken aufgrund von unmittelbar geltenden Rechtsakten der Europäischen Union (EU-Statistiken) und
                \item Statistiken aufgrund von Rechtsvorschriften des Bundes (Bundesstatistiken),
            \end{enumerate}
        \item für die Durchführung von
            \begin{enumerate}[label=\alph*)]
                \item Landestatistiken und
                \item Kommunalstatistiken,
            \end{enumerate}
        \item für Statistiken, bei denen Daten verwendet werden, die im Geschäftsgang der Behörden und Gerichte des Landes sowie der der Aufsicht des Landes unterstehenden juristischen Personen des öffentlichen Rechts anfallen, die bei diesen oder den übergeordneten Behörden oder Stellen geführt werden (Geschäftsstatistiken), sowie
        \item für die statistische Aufbereitung und Auswertung von Daten aus dem Verwaltungsvollzug.
    \end{enumerate}


    \section[Grundsätze]{\S2 Grundsätze der Landesstatistik und Kommunalstatistik}
        \begin{enumerate}[label=(\arabic*)]
            \item Die amtliche Statistik des Landes (Landes- und Kommunalstatistik) hat im föderativ gegliederten Gesamtsystem der amtlichen Statistik die Aufgabe, für den Informationsbedarf von Bund, Ländern, Landkreisen, Gemeinden, Gesellschaft, Wirtschaft, Wissenschaft und Forschung Daten über Massenerscheinungen zu erheben, zu sammeln, aufzubereiten, darzustellen und zu analysieren. Sie gewinnt die Daten unter Verwendung wissenschaftlicher Erkenntnisse und unter Einsatz der jeweils sachgerechten Methoden und Informationstechniken. Für die amtliche Statistik gelten die Grundsätze der Neutralität, Objektivität, wissenschaftlichen Unabhängigkeit und der statistischen Geheimhaltung.
            \item Landesstatistiken und Kommunalstatistiken sollen nur angeordnet werden, wenn die benötigten Informationen nicht auf andere Art beschafft werden können. Sie sind auf das notwendige Maß zu beschränken. Erhobene Einzelangaben dienen ausschließlich den durch dieses Gesetz oder durch eine andere eine Landes- oder Kommunalstatistik anordnende Rechtsvorschrift festgelegten Zwecken.
        \end{enumerate}

    \section{\S3 Statistisches Amt}
        \begin{enumerate}[label=(\arabic*)]
            \item Die Aufgaben der amtlichen Statistik in Mecklenburg-Vorpommern werden vom Statistischen Amt wahrgenommen. Es ist als besondere Organisationseinheit in das Landesamt für innere Verwaltung eingegliedert. Das Statistische Amt ist organisatorisch und räumlich von den anderen Verwaltungsstellen des Landesamtes für innere Verwaltung und der sonstigen Landesverwaltung abzugrenzen, gegen den Zutritt unbefugter Personen ausreichend zu sichern und mit gesondertem Personal auszustatten. Das Weisungsrecht gegenüber dem Statistischen Amt erstreckt sich nicht auf die Weitergabe von Einzeldaten, die der statistischen Geheimhaltung unterliegen.
            \item Das Statistische Amt ist zuständige Behörde für die Durchführung von Bundes- und Landesstatistiken sowie für statistische Erhebungen aufgrund unmittelbar geltender Rechtsakte der Europäischen Union. Die Aufgabe des Statistischen Amtes ist es,
            \begin{enumerate}[label=\arabic*.]
                \item EU-, Bundes- und Landesstatistiken zu erheben und aufzubereiten, soweit in diesem Gesetz oder in einer sonstigen Rechtsvorschrift nichts anderes bestimmt ist, und statistische Ergebnisse zusammenzustellen, auszuwerten, darzustellen und zu veröffentlichen,
                \item Landesstatistiken methodisch und technisch vorzubereiten und weiterzuentwickeln sowie bei der Vorbereitung und Weiterentwicklung von EU- und Bundesstatistiken mitzuwirken,
                \item Volkswirtschaftliche und Umweltökonomische Gesamtrechnungen sowie andere Gesamtsysteme statistischer Daten für Bundes- und Landeszwecke darzustellen und zu veröffentlichen,
                \item das statistische Informationssystem des Landes einzurichten, zu betreiben und inhaltlich und technisch weiterzuentwickeln sowie an der Koordinierung von speziellen Informationssystemen anderer Stellen des Landes mitzuwirken,
                \item wissenschaftliche Analysen, Prognosen und Modellrechnungen auf der Grundlage statistischer Daten vorzunehmen,
                \item auf Anforderung insbesondere der Kommission der Europäischen Union, oberster Bundesbehörden oder oberster Landesbehörden Forschungsaufträge auszuführen, Gutachten zu erstellen und sonstige Arbeiten statistischer Art durchzuführen,
                \item die Behörden und Gerichte des Landes, die Landkreise, kreisfreien Städte, Ämter und amtsfreien Gemeinden sowie die sonstigen der Aufsicht des Landes unterstehenden juristischen Personen des öffentlichen Rechts in statistischen Angelegenheiten zu beraten und zu unterstützen,
                \item an der Vorbereitung von Rechts- und Verwaltungsvorschriften mitzuwirken, die die Bundes- und Landesstatistik betreffen,
                \item bei der Durchführung von allgemeinen Wahlen und Volksabstimmungen mitzuwirken,
                \item sonstige durch Rechtsvorschrift oder durch die fachlich zuständige oberste Landesbehörde im Einvernehmen mit dem Innenminister übertragene Aufgaben wahrzunehmen.
            \end{enumerate}
            \item Die im Statistischen Amt tätigen Personen dürfen die aus oder gelegentlich ihrer Tätigkeit gewonnenen Erkenntnisse mit Personenbezug auch nach ihrem Ausscheiden aus dieser Stelle nicht in anderen Verfahren oder für andere Zwecke verwenden. Sie sind vor ihrem Einsatz auf die Wahrung der statistischen Geheimhaltung, zur Beachtung der gesetzlichen Gebote und Verbote zur Sicherung des Datenschutzes schriftlich zu verpflichten und über die Folgen ihrer Verletzung zu belehren. Sie dürfen während der Tätigkeit im Statistischen Amt nicht mit anderen Aufgaben des Verwaltungsvollzuges betraut werden.
            \item Das Ministerium für Inneres und Europa legt die zur Durchführung der Absätze 1 und 3 erforderlichen Maßnahmen in einer schriftlichen Dienstanweisung fest.
        \end{enumerate}

    \section[Zusammenarbeit]{\S4 Zusammenarbeit der statistischen Ämter}
        \begin{enumerate}[label=(\arabic*)]
            \item Das Statistische Amt darf, soweit es für die Durchführung von Landesstatistiken und für sonstige Arbeiten statistischer Art im Rahmen der Landesstatistik zuständig ist, die Ausführung einzelner Arbeiten oder hierzu erforderlicher Hilfsmaßnahmen durch Verwaltungsvereinbarung oder aufgrund einer Verwaltungsvereinbarung auf andere statistische Ämter übertragen. Davon ausgenommen sind die Heranziehung zur Auskunftserteilung und die Durchsetzung der Auskunftspflicht.
            \item Zu den statistischen Arbeiten nach Absatz 1 gehört auch die Bereitstellung von Daten für die Wissenschaft.
        \end{enumerate}
 
    \section{\S5 Landesstatistiken}
        \begin{enumerate}[label=(\arabic*)]
            \item Landesstatistiken werden, soweit in diesem Gesetz oder in einem anderen Landesgesetz nichts anderes bestimmt ist, durch Gesetz angeordnet.
            \item Die Landesregierung wird ermächtigt, Landesstatistiken mit Auskunftspflicht für die Dauer von bis zu drei Jahren durch Rechtsverordnung anzuordnen, wenn die Ergebnisse der Statistik für Zwecke der Planung oder zur Vorbereitung einer Entscheidung erforderlich sind und die Erhebung nur einen begrenzten Befragtenkreis betrifft.
            \item Landesstatistiken, die auf freiwilliger Grundlage durchgeführt werden, bedürfen keiner Anordnung durch Rechtsvorschrift. Das gleiche gilt für Landesstatistiken, bei denen ausschließlich Angaben aus allgemein zugänglichen Quellen oder aus öffentlichen Registern, zu denen dem Statistischen Amt in einer Rechtsvorschrift ein besonderes Zugangsrecht gewährt wird, verwendet werden. Landesstatistiken nach Satz 1 werden durch Verwaltungsvorschriften der Landesregierung oder der fachlich zuständigen obersten Landesbehörde im Einvernehmen mit dem Ministerium für Inneres und Europa angeordnet; die Finanzierung muß gesichert sein.
            \item Die eine Landesstatistik anordnende Rechts- oder Verwaltungsvorschrift muß die Erhebungsmerkmale, die Hilfsmerkmale, die Art und Weise der Erhebung, den Berichtszeitraum, den Berichtszeitpunkt, die Periodizität und den Kreis der zu Befragenden bestimmen. Ferner ist festzulegen, ob und in welchem Umfang die Erhebung mit oder ohne Auskunftspflicht erfolgen soll. Laufende Nummern und Ordnungsnummern sind nur dann anzuordnen und inhaltlich zu bestimmen, wenn sie Angaben über persönliche und sachliche Verhältnisse enthalten, die über die Erhebungs- und Hilfsmerkmale hinausgehen.
            \item Die Landesregierung wird ermächtigt, durch Rechtsverordnung die Durchführung einer durch Rechtsvorschrift angeordneten Landesstatistik oder die Erhebung einzelner Merkmale auszusetzen, die Periodizität zu verlängern, Erhebungstermine zu ändern sowie den Kreis der zu Befragenden einzuschränken, wenn und soweit die Ergebnisse nicht mehr benötigt werden. Die Landesregierung wird außerdem ermächtigt, durch Rechtsverordnung von der in einer Rechtsvorschrift vorgesehenen Befragung mit Auskunftspflicht zu einer Befragung ohne Auskunftspflicht überzugehen, wenn und soweit ausreichende Ergebnisse einer Landesstatistik auch durch Befragung ohne Auskunftspflicht erreicht werden können.
        \end{enumerate}

    \section{\S6 Statistiken aus dem Verwaltungsvollzug}
        \begin{enumerate}[label=(\arabic*)]
            \item Die Landesregierung kann durch Rechtsverordnung bestimmen, daß dem Statistischen Amt für statistische Zwecke solche personenbezogenen Daten aus automatisierten Registern des Verwaltungsvollzugs zur Verfügung gestellt werden, die beim Vollzug eines Landesgesetzes erhoben worden sind, soweit das Gesetz dies vorsieht. Personenbezogene Daten, die freiwillig für Zwecke des Verwaltungsvollzugs gegeben wurden und in automatisierten Registern oder Dateien verarbeitet werden, dürfen mit Einwilligung des Betroffenen dem Statistischen Amt für die Erfüllung seiner Aufgaben zur Verfügung gestellt werden.
            \item Die Rechtsverordnung muß folgende Angaben enthalten:
            \begin{enumerate}[label=\arabic*.]
                \item Bezeichnung des Registers und der Datei,
                \item speichernde Stelle,
                \item die an das Statistische Amt zu übermittelnden Daten,
                \item den statistischen Zweck, für den die Daten verwendet werden sollen,
                \item Zeitpunkt und Periodizität der Übermittlung. 
            \end{enumerate}
            \item Vor Erlaß der Rechtsverordnung ist der Landesbeauftragte für den Datenschutz zu hören.
        \end{enumerate}


 
    \section{\S7 Maßnahmen zur Vorbereitung von Landesstatistiken}
    Das Statistische Amt kann zur Vorbereitung einer eine Landesstatistik anordnenden Rechtsvorschrift
        \begin{enumerate}[label=\arabic*.]
            \item zur Bestimmung des Kreises der zu Befragenden und deren statistischer Zuordnung Angaben erheben sowie
            \item Erhebungsunterlagen und Erhebungsverfahren auf ihre Zweckmäßigkeit erproben.
        \end{enumerate}
    Für die Angaben nach Nummern 1 und 2 besteht keine Auskunftspflicht. Sie sind zum frühestmöglichen Zeitpunkt zu löschen, die Angaben nach Nummer 2 sind spätestens drei Jahre nach Durchführung der Erprobung zu löschen.

    \section{\S8 Geschäftsstatistiken}
        \begin{enumerate}[label=(\arabic*)]
            \item Geschäftsstatistiken bedürfen, auch soweit personenbezogene Daten verwendet werden, keiner Anordnung durch Rechtsvorschrift, wenn sie ausschließlich der Aufgabenbewältigung der Dienststelle, in deren Geschäftsgang die Daten anfallen, oder der Ausübung von Aufgaben oder Befugnissen der jeweils übergeordneten Dienststellen dienen.
            \item Die statistische Aufbereitung von Geschäftsstatistiken der Behörden und Gerichte des Landes sowie der der Aufsicht des Landes unterstehenden juristischen Personen des öffentlichen Rechts kann mit Zustimmung des fachlich zuständigen Ministeriums und des Ministeriums für Inneres und Europa ganz oder teilweise dem Statistischen Amt übertragen werden. Zu diesem Zweck dürfen mit Ausnahme von Name und Anschrift auch Einzelangaben über persönliche und sachliche Verhältnisse übermittelt werden. Gesetzliche Übermittlungs- und Offenbarungsverbote bleiben unberührt. Das Statistische Amt ist mit Einwilligung der auftraggebenden Stelle berechtigt, aus aufbereiteten Daten der Geschäftsstatistiken statistische Ergebnisse für allgemeine Zwecke darzustellen und zu veröffentlichen.
        \end{enumerate}

    \section{\S9 Erhebungsstellen}
        \begin{enumerate}[label=(\arabic*)]
            \item Die Landesregierung wird ermächtigt, durch Rechtsverordnung zu bestimmen, daß andere staatliche Stellen sowie Landkreise, kreisfreie Städte, Ämter und amtsfreie Gemeinden Erhebungsstellen einzurichten oder in sonstiger Weise an der amtlichen Statistik mitzuwirken haben, wenn dies wegen der Art der Erhebung, der Zahl oder der räumlichen Verteilung der zu Befragenden oder zur Sicherung der Qualität der Erhebung zweckmäßig ist. Die Landkreise, kreisfreien Städte, Ämter und amtsfreien Gemeinden nehmen die Aufgaben nach Satz 1 als Aufgaben im übertragenen Wirkungskreis wahr.
            \item Werden zur Erhebung von EU-, Bundes- oder Landesstatistiken örtliche Erhebungsstellen eingerichtet, so haben diese, soweit durch Rechtsvorschrift nichts anderes bestimmt ist, insbesondere
                \begin{enumerate}[label=\arabic*.]
                    \item die Erhebungsbeauftragten auszuwählen, zu bestellen, über ihre Rechte und Pflichten zu belehren, auf die in § 12 Abs. 2 genannten Geheimhaltungspflichten schriftlich zu verpflichten und zu beaufsichtigten,
                    \item bei der Auswahl der Berichtstellen mitzuwirken, die Erhebungsunterlagen auszuteilen und einzusammeln, die zu Befragenden über die Erhebung zu unterrichten und zur Auskunft aufzufordern, soweit Auskunftspflicht besteht,
                    \item unvollständige oder fehlerhaft ausgefüllte Erhebungsunterlagen durch Nachfrage bei den Befragten zu ergänzen oder zu berichtigen und
                    \item die Erhebungsunterlagen nach Prüfung auf Vollzähligkeit dem Statistischen Amt oder der überörtlichen Erhebungsstelle zuzuleiten.
                \end{enumerate}
            \item Werden überörtliche Erhebungsstellen eingerichtet, so haben diese, soweit durch Rechtsvorschrift nichts anderes bestimmt ist, insbesondere
                \begin{enumerate}[label=\arabic*.]
                    \item die Erhebungsunterlagen an die örtlichen Erhebungsstellen zu verteilen und von diesen wieder einzusammeln und
                    \item die empfangenen Erhebungsunterlagen auf Vollzähligkeit zu überprüfen und dem Statistischen Amt zuzuleiten.

                \end{enumerate}
            \item Die Erhebungsstellen sind für die Dauer der Bearbeitung von statistischen Einzelangaben von anderen Verwaltungsstellen zu trennen. § 11 Abs. 1 bis 3 gilt entsprechend.

            \item Sind bei Landkreisen, kreisfreien Städten, Ämtern und amtsfreien Gemeinden kommunale Statistikstellen eingerichtet, so können diese die Aufgaben der Erhebungsstellen wahrnehmen.
            \item Nehmen die Landkreise, kreisfreien Städte, Ämter und amtsfreien Gemeinden die Einrichtung der Erhebungsstellen als Aufgabe im übertragenen Wirkungskreis wahr, so unterliegen sie insoweit vorbehaltlich abweichender Regelungen der Fachaufsicht der nachfolgenden Behörden:
                \begin{enumerate}[label=\arabic*.]
                    \item Fachaufsichtsbehörde ist der Landrat, soweit örtliche Erhebungsstellen bei einer Gemeinde oder einem Amt eingerichtet sind, die der Rechtsaufsicht des Landrates unterstehen, im übrigen das Landesamt für innere Verwaltung,
                    \item obere Fachaufsichtsbehörde ist das Landesamt für innere Verwaltung,
                    \item oberste Fachaufsichtsbehörde ist das für die Erhebung jeweils fachlich zuständige Ministerium.
                \end{enumerate}
                Soweit das Landesamt für innere Verwaltung Fachaufsichtsbehörde oder obere Fachaufsichtsbehörde ist, werden diese Aufgaben vom Statistischen Amt wahrgenommen.
        \end{enumerate}

    \section{\S10 Kommunalstatistiken}
        Die Landkreise, kreisfreien Städte, Ämter und amtsfreien Gemeinden können zur Wahrnehmung ihrer öffentlichen Aufgaben statistische Erhebungen durchführen, soweit weder die benötigten statistischen Einzelangaben noch die erforderlichen Ergebnisse vom Statistischen Amt zur Verfügung gestellt werden können. Kommunalstatistiken mit Auskunftspflicht bedürfen einer Regelung durch Satzung. Kommunalstatistiken ohne Auskunftspflicht können auch durch Anordnung des Landrates, Oberbürgermeisters, Amtsvorstehers oder Bürgermeisters geregelt werden. § 5 Abs. 4 gilt jeweils entsprechend.

    \section{\S11 Kommunale Statistikstellen}
    \begin{enumerate}[label=(\arabic*)]
        \item Die Aufgaben der Kommunalstatistik dürfen nur von einer Dienststelle des Landkreises, der kreisfreien Stadt, des Amtes und der amtsfreien Gemeinde wahrgenommen werden, die organisatorisch und räumlich von den anderen Verwaltungsstellen der Körperschaft getrennt, gegen den Zutritt unbefugter Personen hinreichend gesichert und mit eigenem Personal ausgestattet ist (kommunale Statistikstelle).
        \item § 3 Abs. 3 gilt entsprechend für die in den kommunalen Statistikstellen tätigen Personen.
        \item Der Landrat, Oberbürgermeister, Amtsvorsteher oder Bürgermeister legt die zur Durchführung der Absätze 1 und 2 erforderlichen Maßnahmen in einer schriftlichen Dienstanweisung fest.
        \item Die Einrichtung sowie die Auflösung einer kommunalen Statistikstelle ist ortsüblich bekanntzugeben sowie dem Statistischen Amt, der Rechtsaufsichtsbehörde und dem Landesbeauftragten für den Datenschutz schriftlich anzuzeigen.
        \item Für ausschließlich statistische Zwecke dürfen an die kommunale Statistikstelle Daten, die im Geschäftsgang anderer Verwaltungsstellen der Landkreise, kreisfreien Städte, Ämter und amtsfreien Gemeinden anfallen, weitergegeben werden, soweit die Auswertungen zur Wahrnehmung der Aufgaben erforderlich sind und gesetzliche Weitergabeverbote nicht entgegenstehen. Regelmäßige Weitergaben sind nur aufgrund einer Satzung zulässig. § 6 Abs. 2 gilt dabei entsprechend.

    \end{enumerate}

    \section{\S12 Erhebungsbeauftragte}
        \begin{enumerate}[label=(\arabic*)]
            \item Werden zur Durchführung einer Landes- oder Kommunalstatistik Erhebungsbeauftragte eingesetzt, müssen sie die Gewähr für Zuverlässigkeit und Verschwiegenheit bieten. Erhebungsbeauftragte dürfen nicht eingesetzt werden, wenn aufgrund der beruflichen Tätigkeit oder aus anderen Gründen Anlaß zur Besorgnis besteht, daß Erkenntnisse aus der Tätigkeit als Erhebungsbeauftragte zu Lasten der Auskunftspflichtigen genutzt werden.
            \item Erhebungsbeauftragte dürfen die aus ihrer Tätigkeit gewonnenen Erkenntnisse nicht in anderen Verfahren oder für andere Zwecke verwenden. Sie sind über ihre Rechte und Pflichten zu belehren und auf die Wahrung der statistischen Geheimhaltung schriftlich zu verpflichten. Die Verpflichtung gilt auch nach Beendigung ihrer Tätigkeit fort.
            \item Erhebungsbeauftragte sind verpflichtet, die Anweisungen der mit der Durchführung der Landes- oder Kommunalstatistik betrauten Dienststellen zu befolgen. Bei der Ausübung ihrer Tätigkeit haben sie sich auszuweisen.
            \item Die Landkreise, kreisfreien Städte, Ämter und amtsfreien Gemeinden sind verpflichtet, bei der Bestellung von Erhebungsbeauftragten, insbesondere bei deren Benennung und Auswahl, mitzuwirken.
        \end{enumerate}
 
    \section{\S13 Erhebungs- und Hilfsmerkmale}
        \begin{enumerate}[label=(\arabic*)]
            \item Erhebungsmerkmale umfassen Angaben über persönliche und sachliche Verhältnisse, die zur statistischen Verwendung bestimmt sind. Hilfsmerkmale sind Angaben, die der technischen Durchführung von Landes- oder Kommunalstatistiken dienen.
            \item Für die regionale Zuordnung der Erhebungsmerkmale und für die regionale Darstellung statistischer Ergebnisse darf innerhalb einer Gemeinde als kleinste regionale Einheit die Blockseite genutzt und gespeichert werden. Besondere Regelungen in einer eine Landes- oder Kommunalstatistik anordnenden Rechts- oder Verwaltungsvorschrift bleiben unberührt.
            \item Soweit nicht eine Rechtsvorschrift etwas anderes bestimmt, sind die Hilfsmerkmale zu löschen, sobald die Überprüfung der Erhebungs- und Hilfsmerkmale auf ihre Schlüssigkeit und Vollständigkeit abgeschlossen ist. Sie sind von den Erhebungsmerkmalen zum frühestmöglichen Zeitpunkt zu trennen und gesondert aufzubewahren.
            \item Bei periodischen Erhebungen dürfen die zur Bestimmung des Kreises der zu Befragenden erforderlichen Hilfsmerkmale, soweit sie für nachfolgende Erhebungen benötigt werden, gesondert aufbewahrt werden. Nach Beendigung des Zeitraums der wiederkehrenden Erhebungen sind sie zu löschen.
            \item Absatz 3 gilt nicht für Einzelangaben, die ausschließlich einer öffentlichen Stelle zugeordnet werden können.
        \end{enumerate}

    \section{\S14 Auskunftspflicht}
    \begin{enumerate}[label=(\arabic*)]
        \item Besteht eine Auskunftspflicht, so sind alle in die Erhebung einbezogenen Personen und Stellen zur Beantwortung der gestellten Fragen gegenüber den mit der Durchführung der Statistik betrauten Stellen und Personen verpflichtet. Die Antwort ist für den Empfänger kosten- und portofrei zu erteilen.
        \item Die Antwort ist wahrheitsgemäß, vollständig und innerhalb der durch Rechtsvorschrift oder von der Erhebungsstelle gesetzten Frist zu erteilen. Eine schriftlich oder elektronisch zu übermittelnde Auskunft ist erst erteilt, wenn sie der Erhebungsstelle zugegangen ist. Elektronisch übermittelte Erhebungsvordrucke sind zugegangen, sobald die für den Empfang bestimmte Einrichtung sie in einer für die Erhebungsstelle bearbeitbaren Weise aufgezeichnet hat.
        \item Sind von den Auskunftspflichtigen Erhebungsvordrucke auszufüllen, sind die Antworten in den Vordrucken schriftlich oder elektronisch in der vorgegebenen Form zu erteilen, soweit in einer Rechtsvorschrift nichts anderes bestimmt ist. Die Richtigkeit ist unterschriftlich zu bestätigen, soweit dies in den Erhebungsvordrucken vorgesehen ist. Öffentliche Stellen des Landes haben aus dem Verwaltungsvollzug gewonnene Daten elektronisch zu übermitteln, soweit diese in geeigneter Form vorliegen.
        \item Werden Erhebungsbeauftragte eingesetzt, können die Fragen mündlich, schriftlich oder elektronisch beantwortet werden. Bei schriftlicher oder elektronischer Beantwortung sind die ausgefüllten Erhebungsvordrucke den Erhebungsbeauftragten offen oder in einem verschlossenen Umschlag zu übergeben oder bei der Erhebungsstelle abzugeben oder dorthin zu übersenden oder elektronisch zu übermitteln.
        \item Widerspruch und Anfechtungsklage gegen die Aufforderung zur Auskunftserteilung bei der Durchführung von Landes- oder Kommunalstatistiken haben keine aufschiebende Wirkung.
    \end{enumerate}

    \section{\S15 Informationspflicht}
    Die zu Befragenden sind über die Informationspflichten gemäß Artikel 13 und Artikel 14 der Verordnung (EU) 2016/679 des Europäischen Parlaments und des Rates vom 27. April 2016 zum Schutz natürlicher Personen bei der Verarbeitung personenbezogener Daten zum freien Datenverkehr und zur Aufhebung der Richtlinie 95/46/EG (ABl. L 119 vom 04.05.2016, S. 1; L 314 vom 22.11.2016, S. 72) hinaus schriftlich oder elektronisch zu unterrichten über
        \begin{enumerate}[label=\arabic*.]
            \item Art und Umfang der Erhebung,
            \item die statistische Geheimhaltung,
            \item die Auskunftspflicht oder die Freiwilligkeit der Auskunftserteilung,
            \item die bei der Durchführung der Erhebung verwendeten Hilfsmerkmale,
            \item die Trennung der Erhebungsmerkmale von den Hilfsmerkmalen und die Löschung der Hilfsmerkmale,
            \item die Hilfs- und Erhebungsmerkmale zur Führung von Adreßdateien,
            \item die Rechte und Pflichten der Erhebungsbeauftragten,
            \item die verschiedenen Möglichkeiten, Auskunft zu erteilen,
            \item die Möglichkeit der Übermittlung von Einzelangaben,
            \item die Bedeutung und den Inhalt von laufenden Nummern und Ordnungsnummern,
            \item den Ausschluß der aufschiebenden Wirkung von Widerspruch und Anfechtungsklage gegen die Aufforderung zur Auskunftserteilung.
        \end{enumerate}

    \section{\S15a Beschränkung von Rechten der betroffenen Personen}
    Die in den Artikeln 15, 16, 18 und 21 der Verordnung (EU) 2016/679 vorgesehenen Rechte der betroffenen Person sind insoweit beschränkt, als diese Rechte voraussichtlich die Verwirklichung der statistischen Zwecke unmöglich machen oder ernsthaft beeinträchtigen und solche Ausnahmen für die Erfüllung der Statistikzwecke notwendig sind.

    \section{\S16 Adreßdateien}
    Adreßdateien, die nach den jeweils geltenden bundesrechtlichen Vorschriften geführt werden, führt und nutzt das Statistische Amt in entsprechender Anwendung dieser Bestimmungen.
    
    \section{\S17 Statistische Geheimhaltung}
        \begin{enumerate}[label=(\arabic*)]
            \item Einzelangaben, die für eine Landes- oder Kommunalstatistik gemacht werden und die dem Befragten oder Betroffenen zugeordnet werden können, sind von den mit der Durchführung der Statistik betrauten Personen geheimzuhalten, soweit in diesem Gesetz oder in einer eine Landes- oder Kommunalstatistik anordnenden Rechtsvorschrift nichts anderes bestimmt ist. Die Pflicht zur statistischen Geheimhaltung gilt nicht für
                \begin{enumerate}[label=\arabic*.]
                    \item Einzelangaben, in deren Übermittlung oder Veröffentlichung der Befragte oder Betroffene schriftlich eingewilligt hat,
                    \item Einzelangaben, die aus allgemein zugänglichen Quellen entnommen werden können, auch soweit sie aufgrund einer Auskunftspflicht erlangt wurden,
                    \item Einzelangaben, die ausschließlich einer öffentlichen Stelle, die nicht am wirtschaftlichen Wettbewerb teilnimmt, zugeordnet werden können.
                \end{enumerate}
            \item Die Pflicht zur Geheimhaltung besteht auch für Personen, die Empfänger von Einzelangaben nach § 18 und § 19 Abs. 3 oder einer anderen Rechtsvorschrift sind.
        \end{enumerate}



    \section[Übermittlung von Einzelangaben]{\S18 Übermittlung von Einzelangaben aus Landes- und Kommunalstatistiken}
        \begin{enumerate}
            \item Die Übermittlung von Einzelangaben zwischen den mit der Durchführung der Statistik betrauten Personen und Stellen ist zulässig, soweit dies zur Erstellung der Statistik erforderlich ist. Darüber hinaus ist die Übermittlung von Einzelangaben zwischen den an einer Zusammenarbeit nach § 4 beteiligten statistischen Ämtern und die zentrale Verarbeitung dieser Einzelangaben in einem oder mehreren statistischen Ämtern zulässig.
            \item Für ausschließlich statistische Zwecke darf das Statistische Amt den kommunalen Statistikstellen Einzelangaben für ihren Zuständigkeitsbereich übermitteln, wenn die Voraussetzungen des § 11 Abs. 1 bis 4 erfüllt sind und die Übermittlung in einer eine Landesstatistik anordnenden Rechtsvorschrift vorgesehen ist sowie Art und Umfang der zu übermittelnden Einzelangaben bestimmt sind. Vor der erstmaligen Übermittlung von Einzelangaben ist dem Statistischen Amt die Dienstanweisung nach § 11 Abs. 3 vorzulegen
            \item Das Statistische Amt darf dem Statistischen Bundesamt und den Statistischen Ämtern der anderen Länder zur Erstellung koordinierter Länderstatistiken oder für methodische Untersuchungen Einzelangaben übermitteln.
            \item Für die Verwendung gegenüber dem Landtag und für Zwecke der Planung, jedoch nicht für die Regelung von Einzelfällen, dürfen den obersten Landesbehörden Tabellen mit statistischen Ergebnissen übermittelt werden, auch soweit Tabellenfeldern nur ein einziger Fall zugrunde liegt. Entsprechendes gilt für die Übermittlung von Daten an oberste Bundesbehörden und an oberste Behörden anderer Länder. Die Übermittlung nach Satz 1 und 2 ist nur zulässig, soweit in der eine Landesstatistik anordnenden Rechtsvorschrift die Übermittlung von Einzelangaben an oberste Landesbehörden oder oberste Bundesbehörden zugelassen ist.
            \item Für die Durchführung wissenschaftlicher Vorhaben darf das Statistische Amt Einzelangaben an Hochschulen oder sonstige Einrichtungen mit der Aufgabe unabhängiger wissenschaftlicher Forschung übermitteln, wenn die Einzelangaben nur mit einem unverhältnismäßig großen Aufwand an Zeit, Kosten und Arbeitskraft den Betroffenen zugeordnet werden können. Sofern es sich bei den Empfängern nicht um Amtsträger oder für den öffentlichen Dienst besonders Verpflichtete handelt, sind sie vor der Übermittlung besonders zur Geheimhaltung zu verpflichten. § 1 Abs. 2 und 3 und § 1 Abs. 4 Nr. 2 des Verpflichtungsgesetzes vom 2. März 1974 (BGBl. I S. 469), geändert durch Gesetz vom 15. August 1974 (BGBl. I S. 1942), gilt entsprechend. Personen, die nach Satz 2 besonders verpflichtet worden sind, stehen für die Anwendung der Vorschriften des Strafgesetzbuches über die Verletzung von Privatgeheimnissen ( § 203 Abs. 2, 4, 5 StGB und §§ 204 , 205 StGB ) und des Dienstgeheimnisses ( § 353 b Abs. 1 StGB ) den für den öffentlichen Dienst besonders Verpflichteten gleich. Empfänger haben durch technische und organisatorische Maßnahmen sicherzustellen, daß sonstige Personen keine Kenntnis von Einzelangaben erhalten. Die Einzelangaben sind zu löschen, sobald das wissenschaftliche Vorhaben abgeschlossen ist, zu dessen Durchführung sie übermittelt wurden. Die Löschung ist dem Statistischen Amt anzuzeigen.
            \item Die übermittelten Einzelangaben dürfen nur für Zwecke verwendet werden, für die sie übermittelt wurden. Die Übermittlung ist vom Statistischen Amt unter Angabe von Inhalt, empfangender Stelle, Datum und Zweck aufzuzeichnen. Die Aufzeichnungen sind von der empfangenden Stelle gegenzuzeichnen und beim Statistischen Amt fünf Jahre lang aufzubewahren.
            \item Die Vorschriften der Absätze 5 und 6 gelten für kommunale Statistikstellen entsprechend.
        \end{enumerate}
 
    \section{\S19 Vergabe statistischer Arbeiten}
    Behörden des Landes dürfen privaten oder öffentlichen Stellen Forschungs-, Planungs- und Untersuchungsaufträge, deren Erledigung statistische Erhebungen oder die Auswertung von Angaben aus Statistiken nach § 1 Nr. 1 a und Nr. 4 erfordern, nur im Einvernehmen mit dem Statistischen Amt erteilen. Vor der Auftragserteilung sind Art und Umfang der statistischen Erhebungen oder Auswertungen mit dem Statistischen Amt abzustimmen. Können die benötigten Angaben vom Statistischen Amt zur Verfügung gestellt werden, darf der Auftrag insoweit nicht erteilt werden.
 
    \section{\S20 Verbot der Reidentifizierung}
    Eine Zusammenführung von Einzeldaten aus Landes- oder Kommunalstatistiken oder von Einzelangaben mit anderen Angaben zum Zwecke der Herstellung eines Personen-, Unternehmens-, Betriebs- oder Arbeitsstättenbezuges außerhalb der Aufgabenstellung dieses Gesetzes oder einer eine Statistik anordnenden Rechtsvorschrift ist verboten.

    \section{\S21 Strafvorschrift}
    Wer entgegen § 20 Einzeldaten aus Landesstatistiken oder Kommunalstatistiken oder solche Einzelangaben mit anderen Angaben zusammenführt, wird mit Freiheitsstrafe bis zu einem Jahr oder mit Geldstrafe bestraft.
 
    \section{\S22 Bußgeldvorschrift}
        \begin{enumerate}
            \item Ordnungswidrig handelt, wer vorsätzlich oder fahrlässig entgegen § 14 Abs. 1 Satz 1, Absatz 2 oder 3 eine Auskunft nicht, nicht wahrheitsgemäß, nicht vollständig, nicht rechtzeitig oder nicht auf den Erhebungsvordrucken in der dort vorgegebenen Form erteilt.
            \item Ordnungswidrig handelt auch, wer vorsätzlich oder fahrlässig einer Auskunftspflicht zuwiderhandelt, die in einer nach § 10 Satz 2 erlassenen Satzung festgelegt ist, soweit die Satzung für einen bestimmten Tatbestand auf diese Bußgeldvorschrift verweist.
            \item Die Ordnungswidrigkeit kann mit einer Geldbuße bis zu 5 000 Euro geahndet werden.
            \item Verwaltungsbehörde im Sinne von § 36 Abs. 1 Nr. 1 des Gesetzes über Ordnungswidrigkeiten ist bei EU-, Bundes- und Landesstatistiken das Landesamt für innere Verwaltung, soweit durch Rechtsvorschrift nichts anderes bestimmt wird. Diese Aufgaben werden vom Statistischen Amt wahrgenommen. Verwaltungsbehörde im Sinne von § 36 Abs. 1 Nr. 1 des Gesetzes über Ordnungswidrigkeiten ist bei Statistiken nach § 10 Satz 2 der Landrat, der Oberbürgermeister, der Amtsvorsteher oder der Bürgermeister der amtsfreien Gemeinden.
        \end{enumerate}

    \section{\S23 (aufgehoben)}
    
    \section{\S24 Inkrafttreten}
    Dieses Gesetz tritt am Tage nach seiner Verkündung in Kraft.


\chapter{Statistiksatzung der Stadt Passau}{Satzung über die Kommunalstatistik der Stadt Passau}
\minitoc
  Die Stadt Passau erlässt aufgrund des Art. 23 der Gemeindeordnung für den Freistaat Bayern (Ge-meindeordnung - GO) in der Fassung der Bekanntmachung vom 22. August 1998 (GVBl S. 796, BayRS 2020-1-1-I), zuletzt geändert durch \S 10 des Gesetzes vom 27. Juli 2009 (GVBl S. 400), und der Art. 22, 23 und 24 des Bayerischen Statistikgesetzes (BayStatG) vom 10. August 1990 (GVBl S. 270, BayRS 290-1-I), zuletzt geändert durch \S 14 des Gesetzes vom 24.12.2002 (GVBl. S. 962), folgende Satzung:
  \section{\S1 Geltungsbereich}
    \begin{enumerate}[label=(\arabic*)]
      \item Diese Satzung gilt für Kommunalstatistiken der Stadt Passau. Für Auftragsstatistiken gilt sie nur, soweit dies ausdrücklich bestimmt ist. Die statistische Aufbereitung von Daten, die bei städtischen Dienststellen im Vollzug ihrer Aufgaben erhoben werden oder auf sonstige Weise anfallen und nicht die ausschließliche Durchführung von Statistiken betreffen (Geschäftsstatis-tiken), bleibt unberührt. 
      \item Die Verarbeitung von Daten, die nicht dem Datenschutz oder der statistischen Geheimhaltung unterliegen, ist von den Bestimmungen dieser Satzung ebenfalls ausgenommen. 
    \end{enumerate}

  \section{\S2 Kommunalstatistik der Stadt Passau}
    \begin{enumerate}[label=(\arabic*)]
      \item Die Stadt Passau betreibt - soweit Einzelangaben oder Ergebnisse vom Bayerischen Landesamt für Statistik und Datenverarbeitung oder von anderen öffentlichen Stellen weder zur Verfügung gestellt noch anderweitig ermittelt werden können - eine Kommunalstatistik und bestimmt eine gem. Art. 20 Abs. 2 Satz 1 BayStatG für die Leitung verantwortliche Person.
      \item Im Rahmen der Kommunalstatistik nach Maßgabe dieser Satzung dürfen bei der Stadt Passau gesetzlich geschützte Daten aus unterschiedlichen Quellen und für nicht abschließend be-stimmte statistische Auswertungszwecke erhoben und verarbeitet werden.
    \end{enumerate}
 
 \section{\S3 Aufgaben der Kommunalstatistik}
 Die Statistikstelle hat insbesondere folgende Aufgaben:
  \begin{enumerate}[label=\arabic*.]
    \item Vorbereitung und Durchführung statistischer Erhebungen aufgrund Bundes- oder Lan-desgesetze sowie freiwilliger kommunalstatistischer Erhebungen und Umfragen, Gewin-nung statistischer Daten aus Verwaltungstätigkeiten, aus Quellen der Landes- und Bun-desstatistiken und aus sonstigen Quellen, 
    \item Aufbau, Pflege und Betreuung der städtischen Datensammlungen zur statistischen Infor-mation in Form von Einzel- und Aggregatdaten aus unterschiedlichen Quellen und für nicht abschließend bestimmte statistische Auswertungszwecke, 
    \item Aufbau, Pflege und Betreuung der Instrumente zur Gewinnung und Darstellung statisti-scher Informationen,
    \item Aufbau, Pflege und Betreuung eines kleinräumig gegliederten Raumbezugssystems sowie der sich daraus ergebenden Schlüsselsysteme,
    \item Datenaufbereitung, Durchführung statistischer Analysen, Prognosen und Mo\-dell\-rech\-nun\-gen (Stadtforschung), Erstellung statistischer Gutachten, 
    \item Erhebung, Aufbereitung und Analyse der Grundlagen, 
    \item Aufgaben der örtlichen Erhebungs- und Berichtsstelle für Volkszählungen, Bundes- und Landesstatistiken, soweit durch Bundes- und Landesrecht nichts anderes bestimmt ist,
    \item Wahrnehmung der Verbindung zum statistischen Bundesamt sowie zu den statistischen Landesämtern, Mitwirkung in den einschlägigen Facharbeitskreisen und im Verband Deutscher Städtestatistiker (VDSt).
  \end{enumerate} 
  \section{\S4 Geheimhaltung}
    Einzelangaben über persönliche und sachliche Verhältnisse, die für die Kommunalstatistik der Stadt Passau gemacht oder zu diesem Zweck an die Statistikstelle übermittelt werden, sind von den Amtsträgern und für den öffentlichen Dienst besonders Verpflichteten, die mit der Durchführung einer solchen Statistik betraut sind, geheim zu halten, soweit durch besondere Rechtsvorschrift nichts anderes bestimmt ist. Die Regelungen von Art. 17 BayStatG bleiben unberührt.
    
  \section{\S5 Abschottung}
    \begin{enumerate}[label=(\arabic*)]
      \item Die Statistikstelle ist räumlich, organisatorisch und personell von anderen Verwaltungsstellen getrennt zu führen. Die Räume, in denen geschützte Einzeldaten verwahrt oder bearbeitet wer-den, sind gegen Zutritt Unbefugter bestmöglich zu sichern. Die Räume der Statistikstelle dür-fen nur von deren Mitarbeitern und den zuständigen Datenschutzbeauftragten betreten werden. Sollte der Zutritt weiterer Personen notwendig sein (z. B. IT-Firmen-Personal, Reini-gungspersonal u. ä.), so sind diese vor Betreten ausdrücklich auf ihre Geheimhaltungspflichten hinzuweisen. 
      \item Die in der Statistikstelle tätigen Personen dürfen nicht gleichzeitig bei anderen Dienststellen der Stadtverwaltung eingesetzt werden und müssen die Gewähr für Zuverlässigkeit und Ver-schwiegenheit bieten. Sie sind auf die Wahrung des Datengeheimnisses nach Art. 5 des Baye-rischen Datenschutzgesetz - BayDSG und des Statistikgeheimnisses nach \S 4 dieser Satzung schriftlich zu verpflichten. Sie sind zur Einhaltung dieser Verpflichtungen auch gegenüber den Dienstvorgesetzten verpflichtet. Die dienst- und arbeitsrechtlichen Befugnisse des Dienstvor-gesetzten bleiben unberührt. 
      \item Zur Erfüllung ihrer Aufgaben bedient sich die Statistikstelle der zentralen Datenverarbeitung. Dabei müssen die Einhaltung der Vorschriften des BayDSG, des Statistikgeheimnisses und der Vorgaben dieser Satzung gewährleistet sein. 
    \end{enumerate}
    
    \section{\S6 Inkrafttreten}
    Diese Satzung tritt eine Woche nach ihrer Bekanntmachung im Amtsblatt der Stadt Passau in Kraft.

\chapter{EU-DSGVO}
\minitoc
    \section{Art. 1: Gegenstand und Ziele}
        \begin{enumerate}[label=(\arabic*)]
            \item Diese Verordnung enthält Vorschriften zum Schutz natürlicher Personen bei der Verarbeitung personenbezogener Daten und zum freien Verkehr solcher Daten.
            \item Diese Verordnung schützt die Grundrechte und Grundfreiheiten natürlicher Personen und insbesondere deren Recht auf Schutz personenbezogener Daten.
            \item Der freie Verkehr personenbezogener Daten in der Union darf aus Gründen des Schutzes natürlicher Personen bei der Verarbeitung personenbezogener Daten weder eingeschränkt noch verboten werden. 
        \end{enumerate}


    \section{Art. 4: Begriffsbestimmungen}
    Im Sinne dieser Verordnung bezeichnet der Ausdruck:
        \begin{enumerate}[label=\arabic*.]
            \item ``personenbezogene Daten'' alle Informationen, die sich auf eine identifizierte oder identifizierbare natürliche Person (im Folgenden ``betroffene Person'') beziehen; als identifizierbar wird eine natürliche Person angesehen, die direkt oder indirekt, insbesondere mittels Zuordnung zu einer Kennung wie einem Namen, zu einer Kennnummer, zu Standortdaten, zu einer Online-Kennung oder zu einem oder mehreren besonderen Merkmalen, die Ausdruck der physischen, physiologischen, genetischen, psychischen, wirtschaftlichen, kulturellen oder sozialen Identität dieser natürlichen Person sind, identifiziert werden kann; 
            \item ``Verarbeitung'' jeden mit oder ohne Hilfe automatisierter Verfahren ausgeführten Vorgang oder jede solche Vorgangsreihe im Zusammenhang mit personenbezogenen Daten wie das Erheben, das Erfassen, die Organisation, das Ordnen, die Speicherung, die Anpassung oder Veränderung, das Auslesen, das Abfragen, die Verwendung, die Offenlegung durch Übermittlung, Verbreitung oder eine andere Form der Bereitstellung, den Abgleich oder die Verknüpfung, die Einschränkung, das Löschen oder die Vernichtung;
            \item ``Einschränkung der Verarbeitung'' die Markierung gespeicherter personenbezogener Daten mit dem Ziel, ihre künftige Verarbeitung einzuschränken;
            \item ``Profiling'' jede Art der automatisierten Verarbeitung personenbezogener Daten, die darin besteht, dass diese personenbezogenen Daten verwendet werden, um bestimmte persönliche Aspekte, die sich auf eine natürliche Person beziehen, zu bewerten, insbesondere um Aspekte bezüglich Arbeitsleistung, wirtschaftliche Lage, Gesundheit, persönliche Vorlieben, Interessen, Zuverlässigkeit, Verhalten, Aufenthaltsort oder Ortswechsel dieser natürlichen Person zu analysieren oder vorherzusagen;
            \item ``Pseudonymisierung'' die Verarbeitung personenbezogener Daten in einer Weise, dass die personenbezogenen Daten ohne Hinzuziehung zusätzlicher Informationen nicht mehr einer spezifischen betroffenen Person zugeordnet werden können, sofern diese zusätzlichen Informationen gesondert aufbewahrt werden und technischen und organisatorischen Maßnahmen unterliegen, die gewährleisten, dass die personenbezogenen Daten nicht einer identifizierten oder identifizierbaren natürlichen Person zugewiesen werden; 
            \item ``Dateisystem'' jede strukturierte Sammlung personenbezogener Daten, die nach bestimmten Kriterien zugänglich sind, unabhängig davon, ob diese Sammlung zentral, dezentral oder nach funktionalen oder geografischen Gesichtspunkten geordnet geführt wird;
            \item ``Verantwortlicher'' die natürliche oder juristische Person, Behörde, Einrichtung oder andere Stelle, die allein oder gemeinsam mit anderen über die Zwecke und Mittel der Verarbeitung von personenbezogenen Daten entscheidet; sind die Zwecke und Mittel dieser Verarbeitung durch das Unionsrecht oder das Recht der Mitgliedstaaten vorgegeben, so kann der Verantwortliche beziehungsweise können die bestimmten Kriterien seiner Benennung nach dem Unionsrecht oder dem Recht der Mitgliedstaaten vorgesehen werden;
            \item ``Auftragsverarbeiter'' eine natürliche oder juristische Person, Behörde, Einrichtung oder andere Stelle, die personenbezogene Daten im Auftrag des Verantwortlichen verarbeitet;
            \item ``Empfänger'' eine natürliche oder juristische Person, Behörde, Einrichtung oder andere Stelle, der personenbezogene Daten offengelegt werden, unabhängig davon, ob es sich bei ihr um einen Dritten handelt oder nicht. Behörden, die im Rahmen eines bestimmten Untersuchungsauftrags nach dem Unionsrecht oder dem Recht der Mitgliedstaaten möglicherweise personenbezogene Daten erhalten, gelten jedoch nicht als Empfänger; die Verarbeitung dieser Daten durch die genannten Behörden erfolgt im Einklang mit den geltenden Datenschutzvorschriften gemäß den Zwecken der Verarbeitung;
            \item  ``Dritter'' eine natürliche oder juristische Person, Behörde, Einrichtung oder andere Stelle, außer der betroffenen Person, dem Verantwortlichen, dem Auftragsverarbeiter und den Personen, die unter der unmittelbaren Verantwortung des Verantwortlichen oder des Auftragsverarbeiters befugt sind, die personenbezogenen Daten zu verarbeiten;
            \item ``Einwilligung'' der betroffenen Person jede freiwillig für den bestimmten Fall, in informierter Weise und unmissverständlich abgegebene Willensbekundung in Form einer Erklärung oder einer sonstigen eindeutigen bestätigenden Handlung, mit der die betroffene Person zu verstehen gibt, dass sie mit der Verarbeitung der sie betreffenden personenbezogenen Daten einverstanden ist;
            \item ``Verletzung des Schutzes personenbezogener Daten'' eine Verletzung der Sicherheit, die, ob unbeabsichtigt oder unrechtmäßig, zur Vernichtung, zum Verlust, zur Veränderung, oder zur unbefugten Offenlegung von beziehungsweise zum unbefugten Zugang zu personenbezogenen Daten führt, die übermittelt, gespeichert oder auf sonstige Weise verarbeitet wurden;
            \item ``genetische Daten'' personenbezogene Daten zu den ererbten oder erworbenen genetischen Eigenschaften einer natürlichen Person, die eindeutige Informationen über die Physiologie oder die Gesundheit dieser natürlichen Person liefern und insbesondere aus der Analyse einer biologischen Probe der betreffenden natürlichen Person gewonnen wurden;
            \item ``biometrische Daten'' mit speziellen technischen Verfahren gewonnene personenbezogene Daten zu den physischen, physiologischen oder verhaltenstypischen Merkmalen einer natürlichen Person, die die eindeutige Identifizierung dieser natürlichen Person ermöglichen oder bestätigen, wie Gesichtsbilder oder daktyloskopische Daten;
            \item ``Gesundheitsdaten'' personenbezogene Daten, die sich auf die körperliche oder geistige Gesundheit einer natürlichen Person, einschließlich der Erbringung von Gesundheitsdienstleistungen, beziehen und aus denen Informationen über deren Gesundheitszustand hervorgehen;
            \item ``Hauptniederlassung''
                \begin{enumerate}[label=\alph*)]
                    \item im Falle eines Verantwortlichen mit Niederlassungen in mehr als einem Mitgliedstaat den Ort seiner Hauptverwaltung in der Union, es sei denn, die Entscheidungen hinsichtlich der Zwecke und Mittel der Verarbeitung personenbezogener Daten werden in einer anderen Niederlassung des Verantwortlichen in der Union getroffen und diese Niederlassung ist befugt, diese Entscheidungen umsetzen zu lassen; in diesem Fall gilt die Niederlassung, die derartige Entscheidungen trifft, als Hauptniederlassung;
                    \item im Falle eines Auftragsverarbeiters mit Niederlassungen in mehr als einem Mitgliedstaat den Ort seiner Hauptverwaltung in der Union oder, sofern der Auftragsverarbeiter keine Hauptverwaltung in der Union hat, die Niederlassung des Auftragsverarbeiters in der Union, in der die Ver\-ar\-bei\-tungs\-tä\-tig\-kei\-ten im Rahmen der Tätigkeiten einer Niederlassung eines Auftragsverarbeiters hauptsächlich stattfinden, soweit der Auftragsverarbeiter spezifischen Pflichten aus dieser Verordnung unterliegt;
                \end{enumerate} 
            \item ``Vertreter'' eine in der Union niedergelassene natürliche oder juristische Person, die von dem Verantwortlichen oder Auftragsverarbeiter schriftlich gemäß Artikel 27 bestellt wurde und den Verantwortlichen oder Auftragsverarbeiter in Bezug auf die ihnen jeweils nach dieser Verordnung obliegenden Pflichten vertritt;
            \item ``Unternehmen'' eine natürliche und juristische Person, die eine wirtschaftliche Tätigkeit ausübt, unabhängig von ihrer Rechtsform, einschließlich Personengesellschaften oder Vereinigungen, die regelmäßig einer wirtschaftlichen Tätigkeit nachgehen;
            \item ``Unternehmensgruppe'' eine Gruppe, die aus einem herrschenden Unternehmen und den von diesem abhängigen Unternehmen besteht; 
            \item ``verbindliche interne Datenschutzvorschriften'' Maßnahmen zum Schutz personenbezogener Daten, zu deren Einhaltung sich ein im Hoheitsgebiet eines Mitgliedstaats niedergelassener Verantwortlicher oder Auftragsverarbeiter verpflichtet im Hinblick auf Datenübermittlungen oder eine Kategorie von Datenübermittlungen personenbezogener Daten an einen Verantwortlichen oder Auftragsverarbeiter derselben Unternehmensgruppe oder derselben Gruppe von Unternehmen, die eine gemeinsame Wirtschaftstätigkeit ausüben, in einem oder mehreren Drittländern;
            \item ``Aufsichtsbehörde'' eine von einem Mitgliedstaat gemäß Artikel 51 eingerichtete unabhängige staatliche Stelle; 
            \item ``betroffene Aufsichtsbehörde'' eine Aufsichtsbehörde, die von der Verarbeitung personenbezogener Daten betroffen ist, weil
                \begin{enumerate}[label=\alph*)]
                    \item der Verantwortliche oder der Auftragsverarbeiter im Hoheitsgebiet des Mitgliedstaats dieser Aufsichtsbehörde niedergelassen ist,
                    \item diese Verarbeitung erhebliche Auswirkungen auf betroffene Personen mit Wohnsitz im Mitgliedstaat dieser Aufsichtsbehörde hat oder haben kann oder
                    \item eine Beschwerde bei dieser Aufsichtsbehörde eingereicht wurde;
                \end{enumerate}
            \item ``grenzüberschreitende Verarbeitung'' entweder
                \begin{enumerate}[label=\alph*)]
                    \item eine Verarbeitung personenbezogener Daten, die im Rahmen der Tätigkeiten von Niederlassungen eines Verantwortlichen oder eines Auftragsverarbeiters in der Union in mehr als einem Mitgliedstaat erfolgt, wenn der Verantwortliche oder Auftragsverarbeiter in mehr als einem Mitgliedstaat niedergelassen ist, oder
                    \item eine Verarbeitung personenbezogener Daten, die im Rahmen der Tätigkeiten einer einzelnen Niederlassung eines Verantwortlichen oder eines Auftragsverarbeiters in der Union erfolgt, die jedoch erhebliche Auswirkungen auf betroffene Personen in mehr als einem Mitgliedstaat hat oder haben kann;
                \end{enumerate}              
            \item ``maßgeblicher und begründeter Einspruch'' einen Einspruch gegen einen Beschlussentwurf im Hinblick darauf, ob ein Verstoß gegen diese Verordnung vorliegt oder ob beabsichtigte Maßnahmen gegen den Verantwortlichen oder den Auftragsverarbeiter im Einklang mit dieser Verordnung steht, wobei aus diesem Einspruch die Tragweite der Risiken klar hervorgeht, die von dem Beschlussentwurf in Bezug auf die Grundrechte und Grundfreiheiten der betroffenen Personen und gegebenenfalls den freien Verkehr personenbezogener Daten in der Union ausgehen;
            \item ``Dienst der Informationsgesellschaft'' eine Dienstleistung im Sinne des Artikels 1 Nummer 1 Buchstabe b der Richtlinie (EU) 2015/1535 des Europäischen Parlaments und des Rates 
            \item ``internationale Organisation'' eine völkerrechtliche Organisation und ihre nachgeordneten Stellen oder jede sonstige Einrichtung, die durch eine zwischen zwei oder mehr Ländern geschlossene Übereinkunft oder auf der Grundlage einer solchen Übereinkunft geschaffen wurde. 
        \end{enumerate}
    \section{Art. 5: Grundsätze für die Verarbeitung personenbezogener Daten}
        \begin{enumerate}[label=(\arabic*)]
            \item Personenbezogene Daten müssen
                \begin{enumerate}[label=\alph*)]
                    \item auf rechtmäßige Weise, nach Treu und Glauben und in einer für die betroffene Person nachvollziehbaren Weise verarbeitet werden (``Rechtmäßigkeit, Verarbeitung nach Treu und Glauben, Transparenz'');
                    \item für festgelegte, eindeutige und legitime Zwecke erhoben werden und dürfen nicht in einer mit diesen Zwecken nicht zu vereinbarenden Weise weiterverarbeitet werden; eine Weiterverarbeitung für im öffentlichen Interesse liegende Archivzwecke, für wissenschaftliche oder historische Forschungszwecke oder für statistische Zwecke gilt gemäß Artikel 89 Absatz 1 nicht als unvereinbar mit den ursprünglichen Zwecken (``Zweckbindung'');
                    \item dem Zweck angemessen und erheblich sowie auf das für die Zwecke der Verarbeitung notwendige Maß beschränkt sein (``Datenminimierung''); 
                    \item sachlich richtig und erforderlichenfalls auf dem neuesten Stand sein; es sind alle angemessenen Maßnahmen zu treffen, damit personenbezogene Daten, die im Hinblick auf die Zwecke ihrer Verarbeitung unrichtig sind, unverzüglich gelöscht oder berichtigt werden (``Richtigkeit'');
                    \item in einer Form gespeichert werden, die die Identifizierung der betroffenen Personen nur so lange ermöglicht, wie es für die Zwecke, für die sie verarbeitet werden, erforderlich ist; personenbezogene Daten dürfen länger gespeichert werden, soweit die personenbezogenen Daten vorbehaltlich der Durchführung geeigneter technischer und organisatorischer Maßnahmen, die von dieser Verordnung zum Schutz der Rechte und Freiheiten der betroffenen Person gefordert werden, ausschließlich für im öffentlichen Interesse liegende Archivzwecke oder für wissenschaftliche und historische Forschungszwecke oder für statistische Zwecke gemäß Artikel 89 Absatz 1 verarbeitet werden (``Speicherbegrenzung'');
                    \item in einer Weise verarbeitet werden, die eine angemessene Sicherheit der personenbezogenen Daten gewährleistet, einschließlich Schutz vor unbefugter oder unrechtmäßiger Verarbeitung und vor unbeabsichtigtem Verlust, unbeabsichtigter Zerstörung oder unbeabsichtigter Schädigung durch geeignete technische und organisatorische Maßnahmen (``Integrität und Vertraulichkeit'');
                \end{enumerate}
            \item Der Verantwortliche ist für die Einhaltung des Absatzes 1 verantwortlich und muss dessen Einhaltung nachweisen können (``Rechenschaftspflicht'')
        \end{enumerate}
    \section[Art 9: Verarbeitung besonderer Daten]{Art. 9: Verarbeitung besonderer Kategorien personenbezogener Daten}
        \begin{enumerate}
            \item Die Verarbeitung personenbezogener Daten, aus denen die rassische und ethnische Herkunft, politische Meinungen, religiöse oder weltanschauliche Überzeugungen oder die Gewerkschaftszugehörigkeit hervorgehen, sowie die Verarbeitung von genetischen Daten, biometrischen Daten zur eindeutigen Identifizierung einer na\-tür\-lichen Person, Gesundheitsdaten oder Daten zum Sexualleben oder der sexuellen Orientierung einer natürlichen Person ist untersagt.
            \item Absatz 1 gilt nicht in folgenden Fällen:
                \begin{enumerate}[label=\alph*)]
                    \item Die betroffene Person hat in die Verarbeitung der genannten personenbezogenen Daten für einen oder mehrere festgelegte Zwecke ausdrücklich eingewilligt, es sei denn, nach Unionsrecht oder dem Recht der Mitgliedstaaten kann das Verbot nach Absatz 1 durch die Einwilligung der betroffenen Person nicht aufgehoben werden,
                    \item die Verarbeitung ist erforderlich, damit der Verantwortliche oder die betroffene Person die ihm bzw. ihr aus dem Arbeitsrecht und dem Recht der sozialen Sicherheit und des Sozialschutzes erwachsenden Rechte ausüben und seinen bzw. ihren diesbezüglichen Pflichten nachkommen kann, soweit dies nach Unionsrecht oder dem Recht der Mitgliedstaaten oder einer Kollektivvereinbarung nach dem Recht der Mitgliedstaaten, das geeignete Garantien für die Grundrechte und die Interessen der betroffenen Person vorsieht, zulässig ist,
                    \item die Verarbeitung ist zum Schutz lebenswichtiger Interessen der betroffenen Person oder einer anderen natürlichen Person erforderlich und die betroffene Person ist aus körperlichen oder rechtlichen Gründen außerstande, ihre Einwilligung zu geben, 
                    \item die Verarbeitung erfolgt auf der Grundlage geeigneter Garantien durch eine politisch, weltanschaulich, religiös oder gewerkschaftlich ausgerichtete Stiftung, Vereinigung oder sonstige Organisation ohne Gewinnerzielungsabsicht im Rahmen ihrer rechtmäßigen Tätigkeiten und unter der Voraussetzung, dass sich die Verarbeitung ausschließlich auf die Mitglieder oder ehemalige Mitglieder der Organisation oder auf Personen, die im Zusammenhang mit deren Tätigkeitszweck regelmäßige Kontakte mit ihr unterhalten, bezieht und die personenbezogenen Daten nicht ohne Einwilligung der betroffenen Personen nach außen offengelegt werden,
                    \item die Verarbeitung bezieht sich auf personenbezogene Daten, die die betroffene Person offensichtlich öffentlich gemacht hat,
                    \item die Verarbeitung ist zur Geltendmachung, Ausübung oder Verteidigung von Rechtsansprüchen oder bei Handlungen der Gerichte im Rahmen ihrer justiziellen Tätigkeit erforderlich, 
                    \item die Verarbeitung ist auf der Grundlage des Unionsrechts oder des Rechts eines Mitgliedstaats, das in angemessenem Verhältnis zu dem verfolgten Ziel steht, den Wesensgehalt des Rechts auf Datenschutz wahrt und angemessene und spezifische Maßnahmen zur Wahrung der Grundrechte und Interessen der betroffenen Person vorsieht, aus Gründen eines erheblichen öffentlichen Interesses erforderlich,
                    \item die Verarbeitung ist für Zwecke der Gesundheitsvorsorge oder der Arbeitsmedizin, für die Beurteilung der Arbeitsfähigkeit des Beschäftigten, für die medizinische Diagnostik, die Versorgung oder Behandlung im Gesundheitsoder Sozialbereich oder für die Verwaltung von Systemen und Diensten im Gesundheits- oder Sozialbereich auf der Grundlage des Unionsrechts oder des Rechts eines Mitgliedstaats oder aufgrund eines Vertrags mit einem Angehörigen eines Gesundheitsberufs und vorbehaltlich der in Absatz 3 genannten Bedingungen und Garantien erforderlich,
                    \item die Verarbeitung ist aus Gründen des öffentlichen Interesses im Bereich der öffentlichen Gesundheit, wie dem Schutz vor schwerwiegenden grenz\-über\-schrei\-ten\-den Gesundheitsgefahren oder zur Gewährleistung hoher Qualitäts- und Sicherheitsstandards bei der Gesundheitsversorgung und bei Arzneimitteln und Medizinprodukten, auf der Grundlage des Unionsrechts oder des Rechts eines Mitgliedstaats, das angemessene und spezifische Maßnahmen zur Wahrung der Rechte und Freiheiten der betroffenen Person, insbesondere des Berufsgeheimnisses, vorsieht, erforderlich, oder
                    \item die Verarbeitung ist auf der Grundlage des Unionsrechts oder des Rechts eines Mitgliedstaats, das in angemessenem Verhältnis zu dem verfolgten Ziel steht, den Wesensgehalt des Rechts auf Datenschutz wahrt und angemessene und spezifische Maßnahmen zur Wahrung der Grundrechte und Interessen der betroffenen Person vorsieht, für im öffentlichen Interesse liegende Archivzwecke, für wissenschaftliche oder historische Forschungszwecke oder für statistische Zwecke gemäß Artikel 89 Absatz 1 erforderlich.
                \end{enumerate}
            \item Die in Absatz 1 genannten personenbezogenen Daten dürfen zu den in Absatz 2 Buchstabe h genannten Zwecken verarbeitet werden, wenn diese Daten von Fachpersonal oder unter dessen Verantwortung verarbeitet werden und dieses Fachpersonal nach dem Unionsrecht oder dem Recht eines Mitgliedstaats oder den Vorschriften nationaler zuständiger Stellen dem Berufsgeheimnis unterliegt, oder wenn die Verarbeitung durch eine andere Person erfolgt, die ebenfalls nach dem Unionsrecht oder dem Recht eines Mitgliedstaats oder den Vorschriften nationaler zuständiger Stellen einer Geheimhaltungspflicht unterliegt.
            \item Die Mitgliedstaaten können zusätzliche Bedingungen, einschließlich Be\-schrän\-kun\-gen, einführen oder aufrechterhalten, soweit die Verarbeitung von genetischen, biometrischen oder Gesundheitsdaten betroffen ist. 
        \end{enumerate}
    \section[Art. 14: Informationspflicht]{Art. 14: Informationspflicht, wenn die personenbezogenen Daten nicht bei der betroffenen Person
erhoben wurden}
    \begin{enumerate}[label=(\arabic*)]
        \item Werden personenbezogene Daten nicht bei der betroffenen Person erhoben, so teilt der Verantwortliche der betroffenen Person Folgendes mit:
            \begin{enumerate}[label=\alph*)]
                \item den Namen und die Kontaktdaten des Verantwortlichen sowie gegebenenfalls seines Vertreters;
                \item zusätzlich die Kontaktdaten des Datenschutzbeauftragten;
                \item die Zwecke, für die die personenbezogenen Daten verarbeitet werden sollen, sowie die Rechtsgrundlage für die
Verarbeitung;
                \item die Kategorien personenbezogener Daten, die verarbeitet werden;
                \item gegebenenfalls die Empfänger oder Kategorien von Empfängern der personenbezogenen Daten;
                \item gegebenenfalls die Absicht des Verantwortlichen, die personenbezogenen Daten an einen Empfänger in einem Drittland oder einer internationalen Organisation zu übermitteln, sowie das Vorhandensein oder das Fehlen eines Angemessenheitsbeschlusses der Kommission oder im Falle von Übermittlungen gemäß Artikel 46 oder Artikel 47 oder Artikel 49 Absatz 1 Unterabsatz 2 einen Verweis auf die geeigneten oder angemessenen Garantien und die Möglichkeit, eine Kopie von ihnen zu erhalten, oder wo sie verfügbar sind. 
            \end{enumerate}
        \item Zusätzlich zu den Informationen gemäß Absatz 1 stellt der Verantwortliche der betroffenen Person die folgenden Informationen zur Verfügung, die erforderlich sind, um der betroffenen Person gegenüber eine faire und transparente Verarbeitung zu gewährleisten:
            \begin{enumerate}[label=\alph*)]
                \item die Dauer, für die die personenbezogenen Daten gespeichert werden oder, falls dies nicht möglich ist, die Kriterien für die Festlegung dieser Dauer;
                \item wenn die Verarbeitung auf Artikel 6 Absatz 1 Buchstabe f beruht, die berechtigten Interessen, die von dem Verantwortlichen oder einem Dritten verfolgt werden;
                \item das Bestehen eines Rechts auf Auskunft seitens des Verantwortlichen über die betreffenden personenbezogenen Daten sowie auf Berichtigung oder Löschung oder auf Einschränkung der Verarbeitung und eines Widerspruchsrechts gegen die Verarbeitung sowie des Rechts auf Datenübertragbarkeit; 
                \item wenn die Verarbeitung auf Artikel 6 Absatz 1 Buchstabe a oder Artikel 9 Absatz 2 Buchstabe a beruht, das Bestehen eines Rechts, die Einwilligung jederzeit zu widerrufen, ohne dass die Rechtmäßigkeit der aufgrund der Einwilligung bis zum Widerruf erfolgten Verarbeitung berührt wird;
                \item das Bestehen eines Beschwerderechts bei einer Aufsichtsbehörde;
                \item aus welcher Quelle die personenbezogenen Daten stammen und gegebenenfalls ob sie aus öffentlich zugänglichen Quellen stammen;
                \item das Bestehen einer automatisierten Entscheidungsfindung einschließlich Profiling gemäß Artikel 22 Absätze 1 und 4 und — zumindest in diesen Fällen — aussagekräftige Informationen über die involvierte Logik sowie die Tragweite und die angestrebten Auswirkungen einer derartigen Verarbeitung für die betroffene Person.
            \end{enumerate}
        \item Der Verantwortliche erteilt die Informationen gemäß den Absätzen 1 und 2
            \begin{enumerate}
                \item unter Berücksichtigung der spezifischen Umstände der Verarbeitung der personenbezogenen Daten innerhalb einer angemessenen Frist nach Erlangung der personenbezogenen Daten, längstens jedoch innerhalb eines Monats, 
                \item falls die personenbezogenen Daten zur Kommunikation mit der betroffenen Person verwendet werden sollen, spätestens zum Zeitpunkt der ersten Mitteilung an sie, oder, 
                \item falls die Offenlegung an einen anderen Empfänger beabsichtigt ist, spätestens zum Zeitpunkt der ersten Offenlegung.
            \end{enumerate}
        \item Beabsichtigt der Verantwortliche, die personenbezogenen Daten für einen anderen Zweck weiterzuverarbeiten als den, für den die personenbezogenen Daten erlangt wurden, so stellt er der betroffenen Person vor dieser Weiterverarbeitung Informationen über diesen anderen Zweck und alle anderen maßgeblichen Informationen gemäß Absatz 2 zur Verfügung.
        \item Die Absätze 1 bis 4 finden keine Anwendung, wenn und soweit
        \begin{enumerate}[label=\alph*)]
            \item die betroffene Person bereits über die Informationen verfügt,
            \item die Erteilung dieser Informationen sich als unmöglich erweist oder einen unverhältnismäßigen Aufwand erfordern würde; dies gilt insbesondere für die Verarbeitung für im öffentlichen Interesse liegende Archivzwecke, für wissenschaftliche oder historische Forschungszwecke oder für statistische Zwecke vorbehaltlich der in Artikel 89 Absatz 1 genannten Bedingungen und Garantien oder soweit die in Absatz 1 des vorliegenden Artikels genannte Pflicht voraussichtlich die Verwirklichung der Ziele dieser Verarbeitung unmöglich macht oder ernsthaft beeinträchtigt In diesen Fällen ergreift der Verantwortliche geeignete Maßnahmen zum Schutz der Rechte und Freiheiten sowie der berechtigten Interessen der betroffenen Person, einschließlich der Bereitstellung dieser Informationen für die Öffentlichkeit,
            \item die Erlangung oder Offenlegung durch Rechtsvorschriften der Union oder der Mitgliedstaaten, denen der Verantwortliche unterliegt und die geeignete Maßnahmen zum Schutz der berechtigten Interessen der betroffenen Person vorsehen, ausdrücklich geregelt ist oder
            \item die personenbezogenen Daten gemäß dem Unionsrecht oder dem Recht der Mitgliedstaaten dem Berufsgeheimnis, einschließlich einer satzungsmäßigen Geheimhaltungspflicht, unterliegen und daher vertraulich behandelt werden müssen. 
        \end{enumerate}
    \end{enumerate}

    \section[Art. 15: Auskunftsrecht]{Art. 15: Auskunftsrecht der betroffenen Person}
        \begin{enumerate}[label=(\arabic*)]
            \item Die betroffene Person hat das Recht, von dem Verantwortlichen eine Bestätigung darüber zu verlangen, ob sie betreffende personenbezogene Daten verarbeitet werden; ist dies der Fall, so hat sie ein Recht auf Auskunft über diese personenbezogenen Daten und auf folgende Informationen:
                \begin{enumerate}[label=\alph*)]
                    \item die Verarbeitungszwecke;
                    \item die Kategorien personenbezogener Daten, die verarbeitet werden;
                    \item die Empfänger oder Kategorien von Empfängern, gegenüber denen die personenbezogenen Daten offengelegt worden sind oder noch offengelegt werden, insbesondere bei Empfängern in Drittländern oder bei internationalen Organisationen;
                    \item falls möglich die geplante Dauer, für die die personenbezogenen Daten gespeichert werden, oder, falls dies nicht möglich ist, die Kriterien für die Festlegung dieser Dauer;
                    \item das Bestehen eines Rechts auf Berichtigung oder Löschung der sie betreffenden personenbezogenen Daten oder auf Einschränkung der Verarbeitung durch den Verantwortlichen oder eines Widerspruchsrechts gegen diese Verarbeitung;
                    \item das Bestehen eines Beschwerderechts bei einer Aufsichtsbehörde;
                    \item wenn die personenbezogenen Daten nicht bei der betroffenen Person erhoben werden, alle verfügbaren Informationen über die Herkunft der Daten;
                    \item das Bestehen einer automatisierten Entscheidungsfindung einschließlich Profiling gemäß Artikel 22 Absätze 1 und 4 und — zumindest in diesen Fällen — aussagekräftige Informationen über die involvierte Logik sowie die Tragweite und die angestrebten Auswirkungen einer derartigen Verarbeitung für die betroffene Person.
                \end{enumerate}
            \item Werden personenbezogene Daten an ein Drittland oder an eine internationale Organisation übermittelt, so hat die betroffene Person das Recht, über die geeigneten Garantien gemäß Artikel 46 im Zusammenhang mit der Übermittlung unterrichtet zu werden.
            \item Der Verantwortliche stellt eine Kopie der personenbezogenen Daten, die Gegenstand der Verarbeitung sind, zur Verfügung. Für alle weiteren Kopien, die die betroffene Person beantragt, kann der Verantwortliche ein angemessenes Entgelt auf der Grundlage der Verwaltungskosten verlangen. Stellt die betroffene Person den Antrag elektronisch, so sind die Informationen in einem gängigen elektronischen Format zur Verfügung zu stellen, sofern sie nichts anderes angibt.
            \item Das Recht auf Erhalt einer Kopie gemäß Absatz 1b darf die Rechte und Freiheiten anderer Personen nicht beeinträchtigen. 
        \end{enumerate}

    \section{Art. 16: Recht auf Berichtigung}
        Die betroffene Person hat das Recht, von dem Verantwortlichen unverzüglich die Berichtigung sie betreffender unrichtiger personenbezogener Daten zu verlangen. Unter Berücksichtigung der Zwecke der Verarbeitung hat die betroffene Person das Recht, die Vervollständigung unvollständiger personenbezogener Daten — auch mittels einer ergänzenden Erklärung — zu verlangen.



    \section[Art. 17 : Recht auf Löschung]{Art. 17: Recht auf Löschung (``Recht auf Vergessenwerden'')}
        \begin{enumerate}[label=(\arabic*)]
            \item Die betroffene Person hat das Recht, von dem Verantwortlichen zu verlangen, dass sie betreffende personenbezogene Daten unverzüglich gelöscht werden, und der Verantwortliche ist verpflichtet, personenbezogene Daten unverzüglich zu löschen, sofern einer der folgenden Gründe zutrifft:
                \begin{enumerate}[label=\alph*)]
                    \item Die personenbezogenen Daten sind für die Zwecke, für die sie erhoben oder auf sonstige Weise verarbeitet wurden, nicht mehr notwendig.
                    \item Die betroffene Person widerruft ihre Einwilligung, auf die sich die Verarbeitung gemäß Artikel 6 Absatz 1 Buchstabe a oder Artikel 9 Absatz 2 Buchstabe a stützte, und es fehlt an einer anderweitigen Rechtsgrundlage für die Verarbeitung.
                    \item Die betroffene Person legt gemäß Artikel 21 Absatz 1 Widerspruch gegen die Verarbeitung ein und es liegen keine vorrangigen berechtigten Gründe für die Verarbeitung vor, oder die betroffene Person legt gemäß Artikel 21 Absatz 2 Widerspruch gegen die Verarbeitung ein.
                    \item Die personenbezogenen Daten wurden unrechtmäßig verarbeitet. 
                    \item Die Löschung der personenbezogenen Daten ist zur Erfüllung einer rechtlichen Verpflichtung nach dem Unionsrecht oder dem Recht der Mitgliedstaaten erforderlich, dem der Verantwortliche unterliegt. 
                    \item Die personenbezogenen Daten wurden in Bezug auf angebotene Dienste der Informationsgesellschaft gemäß Artikel 8 Absatz 1 erhoben.
                \end{enumerate}
            \item Hat der Verantwortliche die personenbezogenen Daten öffentlich gemacht und ist er gemäß Absatz 1 zu deren Löschung verpflichtet, so trifft er unter Berücksichtigung der verfügbaren Technologie und der Implementierungskosten angemessene Maßnahmen, auch technischer Art, um für die Datenverarbeitung Verantwortliche, die die personenbezogenen Daten verarbeiten, darüber zu informieren, dass eine betroffene Person von ihnen die Löschung aller Links zu diesen personenbezogenen Daten oder von Kopien oder Replikationen dieser personenbezogenen Daten verlangt hat.
            \item Die Absätze 1 und 2 gelten nicht, soweit die Verarbeitung erforderlich ist
                \begin{enumerate}[label=\alph*)]
                    \item zur Ausübung des Rechts auf freie Meinungsäußerung und Information;
                    \item zur Erfüllung einer rechtlichen Verpflichtung, die die Verarbeitung nach dem Recht der Union oder der Mitgliedstaaten, dem der Verantwortliche unterliegt, erfordert, oder zur Wahrnehmung einer Aufgabe, die im öffentlichen Interesse liegt oder in Ausübung öffentlicher Gewalt erfolgt, die dem Verantwortlichen übertragen wurde;
                    \item aus Gründen des öffentlichen Interesses im Bereich der öffentlichen Gesundheit gemäß Artikel 9 Absatz 2 Buchstaben h und i sowie Artikel 9 Absatz 3; 
                    \item für im öffentlichen Interesse liegende Archivzwecke, wissenschaftliche oder historische Forschungszwecke oder für statistische Zwecke gemäß Artikel 89 Absatz 1, soweit das in Absatz 1 genannte Recht voraussichtlich die Verwirklichung der Ziele dieser Verarbeitung unmöglich macht oder ernsthaft beeinträchtigt, oder 
                    \item zur Geltendmachung, Ausübung oder Verteidigung von Rechtsansprüchen. 
                \end{enumerate}
        \end{enumerate}

    \section{Art. 18: Recht auf Einschränkung der Verarbeitung}
        \begin{enumerate}[label=(\arabic*)]
            \item Die betroffene Person hat das Recht, von dem Verantwortlichen die Einschränkung der Verarbeitung zu verlangen, wenn eine der folgenden Voraussetzungen gegeben ist:
                \begin{enumerate}[label=\alph*)]
                    \item die Richtigkeit der personenbezogenen Daten von der betroffenen Person bestritten wird, und zwar für eine Dauer, die es dem Verantwortlichen ermöglicht, die Richtigkeit der personenbezogenen Daten zu überprüfen,
                    \item die Verarbeitung unrechtmäßig ist und die betroffene Person die Löschung der personenbezogenen Daten ablehnt und stattdessen die Einschränkung der Nutzung der personenbezogenen Daten verlangt; 
                    \item der Verantwortliche die personenbezogenen Daten für die Zwecke der Verarbeitung nicht länger benötigt, die betroffene Person sie jedoch zur Geltendmachung, Ausübung oder Verteidigung von Rechtsansprüchen benötigt, oder
                    \item die betroffene Person Widerspruch gegen die Verarbeitung gemäß Artikel 21 Absatz 1 eingelegt hat, solange noch nicht feststeht, ob die berechtigten Gründe des Verantwortlichen gegenüber denen der betroffenen Person überwiegen.
                \end{enumerate}
            \item Wurde die Verarbeitung gemäß Absatz 1 eingeschränkt, so dürfen diese personenbezogenen Daten — von ihrer Speicherung abgesehen — nur mit Einwilligung der betroffenen Person oder zur Geltendmachung, Ausübung oder Verteidigung von Rechtsansprüchen oder zum Schutz der Rechte einer anderen natürlichen oder juristischen Person oder aus Gründen eines wichtigen öffentlichen Interesses der Union oder eines Mitgliedstaats verarbeitet werden.
            \item Eine betroffene Person, die eine Einschränkung der Verarbeitung gemäß Absatz 1 erwirkt hat, wird von dem Verantwortlichen unterrichtet, bevor die Einschränkung aufgehoben wird. 
        \end{enumerate}

    \section{Art. 21: Widerspruchsrecht}
        \begin{enumerate}[label=(\arabic*]
            \item Die betroffene Person hat das Recht, aus Gründen, die sich aus ihrer besonderen Situation ergeben, jederzeit gegen die Verarbeitung sie betreffender personenbezogener Daten, die aufgrund von Artikel 6 Absatz 1 Buchstaben e oder f erfolgt, Widerspruch einzulegen; dies gilt auch für ein auf diese Bestimmungen gestütztes Profiling. Der Verantwortliche verarbeitet die personenbezogenen Daten nicht mehr, es sei denn, er kann zwingende schutzwürdige Gründe für die Verarbeitung nachweisen, die die Interessen, Rechte und Freiheiten der betroffenen Person überwiegen, oder die Verarbeitung dient der Geltendmachung, Ausübung oder Verteidigung von Rechtsansprüchen.
            \item Werden personenbezogene Daten verarbeitet, um Direktwerbung zu betreiben, so hat die betroffene Person das Recht, jederzeit Widerspruch gegen die Verarbeitung sie betreffender personenbezogener Daten zum Zwecke derartiger Werbung einzulegen; dies gilt auch für das Profiling, soweit es mit solcher Direktwerbung in Verbindung steht. 
            \item Widerspricht die betroffene Person der Verarbeitung für Zwecke der Direktwerbung, so werden die personenbezogenen Daten nicht mehr für diese Zwecke verarbeitet.
            \item Die betroffene Person muss spätestens zum Zeitpunkt der ersten Kommunikation mit ihr ausdrücklich auf das in den Absätzen 1 und 2 genannte Recht hingewiesen werden; dieser Hinweis hat in einer verständlichen und von anderen Informationen getrennten Form zu erfolgen.
            \item Im Zusammenhang mit der Nutzung von Diensten der Informationsgesellschaft kann die betroffene Person ungeachtet der Richtlinie 2002/58/EG ihr Widerspruchsrecht mittels automatisierter Verfahren ausüben, bei denen technische Spezifikationen verwendet werden. 
            \item Die betroffene Person hat das Recht, aus Gründen, die sich aus ihrer besonderen Situation ergeben, gegen die sie betreffende Verarbeitung sie betreffender personenbezogener Daten, die zu wissenschaftlichen oder historischen Forschungszwecken oder zu statistischen Zwecken gemäß Artikel 89 Absatz 1 erfolgt, Widerspruch einzulegen, es sei denn, die Verarbeitung ist zur Erfüllung einer im öffentlichen Interesse liegenden Aufgabe erforderlich. 
        \end{enumerate}


    \section[Art. 89: Garantien]{Art. 89: Garantien und Ausnahmen in Bezug auf die Verarbeitung zu im öffentlichen Interesse liegenden Archivzwecken, zu wissenschaftlichen oder historischen Forschungszwecken und zu statistischen Zwecken}
        \begin{enumerate}[label=(\arabic*)]
            \item Die Verarbeitung zu im öffentlichen Interesse liegenden Archivzwecken, zu wissenschaftlichen oder historischen Forschungszwecken oder zu statistischen Zwecken unterliegt geeigneten Garantien für die Rechte und Freiheiten der betroffenen Person gemäß dieser Verordnung. Mit diesen Garantien wird sichergestellt, dass technische und organisatorische Maßnahmen bestehen, mit denen insbesondere die Achtung des Grundsatzes der Datenminimierung gewährleistet wird. Zu diesen Maßnahmen kann die Pseudonymisierung gehören, sofern es möglich ist, diese Zwecke auf diese Weise zu erfüllen. In allen Fällen, in denen diese Zwecke durch die Weiterverarbeitung, bei der die Identifizierung von betroffenen Personen nicht oder nicht mehr möglich ist, erfüllt werden können, werden diese Zwecke auf diese Weise erfüllt.
            \item Werden personenbezogene Daten zu wissenschaftlichen oder historischen Forschungszwecken oder zu statistischen Zwecken verarbeitet, können vorbehaltlich der Bedingungen und Garantien gemäß Absatz 1 des vorliegenden Artikels im Unionsrecht oder im Recht der Mitgliedstaaten insoweit Ausnahmen von den Rechten gemäß der Artikel 15, 16, 18 und 21 vorgesehen werden, als diese Rechte voraussichtlich die Verwirklichung der spezifischen Zwecke unmöglich machen oder ernsthaft beeinträchtigen und solche Ausnahmen für die Erfüllung dieser Zwecke notwendig sind.
            \item Werden personenbezogene Daten für im öffentlichen Interesse liegende Archivzwecke verarbeitet, können vorbehaltlich der Bedingungen und Garantien gemäß Absatz 1 des vorliegenden Artikels im Unionsrecht oder im Recht der Mitgliedstaaten insoweit Ausnahmen von den Rechten gemäß der Artikel 15, 16, 18, 19, 20 und 21 vorgesehen werden, als diese Rechte voraussichtlich die Verwirklichung der spezifischen Zwecke unmöglich machen oder ernsthaft beeinträchtigen und solche Ausnahmen für die Erfüllung dieser Zwecke notwendig sind. 
            \item Dient die in den Absätzen 2 und 3 genannte Verarbeitung gleichzeitig einem anderen Zweck, gelten die Ausnahmen nur für die Verarbeitung zu den in diesen Absätzen genannten Zwecken. 
        \end{enumerate}
\chapter{Bundesdatenschutzgesetz}
\minitoc
    \section[\S 27: Datenverarbeitung zu \dots statistischen Zwecken]{\S 27: Datenverarbeitung zu wissenschaftlichen oder historischen Forschungszwecken und zu statistischen Zwecken}
        \begin{enumerate}[label=(\arabic*)]
            \item Abweichend von Artikel 9 Absatz 1 der Verordnung (EU) 2016/679\footnote{EU-DSGVO} ist die Verarbeitung besonderer Kategorien personenbezogener Daten im Sinne des Artikels 9 Absatz 1 der Verordnung (EU) 2016/679 auch ohne Einwilligung für wissenschaftliche oder historische Forschungszwecke oder für statistische Zwecke zulässig, wenn die Verarbeitung zu diesen Zwecken erforderlich ist und die Interessen des Verantwortlichen an der Verarbeitung die Interessen der betroffenen Person an einem Ausschluss der Verarbeitung erheblich überwiegen. Der Verantwortliche sieht angemessene und spezifische Maßnahmen zur Wahrung der Interessen der betroffenen Person gemäß \S 22 Absatz 2 Satz 2 vor.
            \item Die in den Artikeln 15, 16, 18 und 21 der Verordnung (EU) 2016/679 vorgesehenen Rechte der betroffenen Person sind insoweit beschränkt, als diese Rechte voraussichtlich die Verwirklichung der Forschungs- oder Statistikzwecke unmöglich machen oder ernsthaft beinträchtigen und die Beschränkung für die Erfüllung der Forschungs- oder Statistikzwecke notwendig ist. Das Recht auf Auskunft gemäß Artikel 15 der Verordnung (EU) 2016/679 besteht darüber hinaus nicht, wenn die Daten für Zwecke der wissenschaftlichen Forschung erforderlich sind und die Auskunftserteilung einen unverhältnismäßigen Aufwand erfordern würde.
            \item Ergänzend zu den in \S 22 Absatz 2 genannten Maßnahmen sind zu wissenschaftlichen oder historischen Forschungszwecken oder zu statistischen Zwecken verarbeitete besondere Kategorien personenbezogener Daten im Sinne des Artikels 9 Absatz 1 der Verordnung (EU) 2016/679 zu anonymisieren, sobald dies nach dem Forschungs- oder Statistikzweck möglich ist, es sei denn, berechtigte Interessen der betroffenen Person stehen dem entgegen. Bis dahin sind die Merkmale gesondert zu speichern, mit denen Einzelangaben über persönliche oder sachliche Verhältnisse einer bestimmten oder bestimmbaren Person zugeordnet werden können. Sie dürfen mit den Einzelangaben nur zusammengeführt werden, soweit der Forschungs- oder Statistikzweck dies erfordert.
            \item Der Verantwortliche darf personenbezogene Daten nur veröffentlichen, wenn die betroffene Person eingewilligt hat oder dies für die Darstellung von Forschungsergebnissen über Ereignisse der Zeitgeschichte unerlässlich ist.
        \end{enumerate}

    \section[\S 50 Verarbeitung zu \dots statistischen Zwecken]{\S 50 Verarbeitung zu archivarischen, wissenschaftlichen  und statistischen Zwecken}
    Personenbezogene Daten dürfen im Rahmen der in \S 45 genannten Zwecke in archivarischer, wissenschaftlicher oder statistischer Form verarbeitet werden, wenn hieran ein öffentliches Interesse besteht und geeignete Garantien für die Rechtsgüter der betroffenen Personen vorgesehen werden. Solche Garantien können in einer so zeitnah wie möglich erfolgenden Anonymisierung der personenbezogenen Daten, in Vorkehrungen gegen ihre unbefugte Kenntnisnahme durch Dritte oder in ihrer räumlich und organisatorisch von den sonstigen Fachaufgaben getrennten Verarbeitung bestehen.


\chapter[GG]{Grundgesetz für die Bundesrepublik Deutschland}
    \section{Art. 73}
        \begin{enumerate}[label=(\arabic*)]
            \item Der Bund hat die ausschliessliche Gesetzgebung über:
            \newline
            (\dots)
                \begin{enumerate}[label=\arabic*.,start=11]
                    \item  die Statistik für Bundeszwecke;
                \end{enumerate}

            
        \end{enumerate}
    \section{Art. 85}
        \begin{enumerate}[label=(\arabic*)]
            \item Führen die Länder die Bundesgesetze im Auftrage des Bundes aus, so bleibt die Einrichtung der Behörden Angelegenheit der Länder, soweit nicht Bundesgesetze mit Zustimmung des Bundesrates etwas anderes bestimmen. \textbf{Durch Bundesgesetz dürfen Gemeinden und Gemeindeverbänden Aufgaben nicht übertragen werden.}
            \item \dots
            \item \dots
            \item \dots
        \end{enumerate}

\include{Volkszählungsurteil}


\chapter{Leitfaden zur Einrichtung abgeschotteter kommunaler Statistikstellen in Bayern}
Jede Kommune hat die Möglichkeit, eine abgeschottete Statistikstelle entsprechend Art. 20 BayStatG einzurichten, um in bestimmten, gesetzlich geregelten Fällen Einzeldaten übermittelt zu bekommen (vgl. \S 16 Abs. 5 BStatG), die Auswertungen in tiefer regionaler Gliederung erlauben (z.B. Zensus, Unternehmensregister). Ferner können die abgeschotteten Statistikstellen bei künftigen Großzählungen die Aufgaben einer örtlichen Erhebungsstelle wahrnehmen (vgl. Art. 27 BayStatG; die relevanten Regelungen im BayStatG sind im Anhang aufgeführt). Durch die Abschottung der kommunalen Statistikstellen soll eine eindeutige und für jedermann nachvollziehbare Trennung der Statistik von der übrigen Verwaltung sichergestellt werden. Abschottung bedeutet dabei organisatorische, personelle und räumliche Trennung der Statistikstelle von anderen Verwaltungsstellen (Art. 21 Abs. 3 i.V.m. Art. 20 Abs. 2 und 3 BayStatG). Insbesondere ist auch die IT-Infrastruktur der abgeschotteten Statistikstelle hinreichend vom übrigen Verwaltungsnetz zu trennen. Das strenge Abschottungsgebot resultiert aus dem besonderen Schutzbedarf der statistischen Einzeldaten. Die Geheimhaltung der statistischen Einzelangaben (\S 16 BStatG, Art. 17 BayStatG) als spezielle Form des Datenschutzes ist seit jeher das Fundament der Statistik. Ihre Gewährleistung dient folgenden Zielen:
\begin{itemize}
    \item Schutz des Einzelnen vor der Offenlegung seiner persönlichen und sachlichen Verhältnisse,
    \item Erhaltung des Vertrauensverhältnisses zwischen den Befragten und den statistischen Ämtern,
    \item Gewährleistung der Zuverlässigkeit der Angaben und der Berichtswilligkeit der Befragten.
\end{itemize}
Das Bundesverfassungsgericht hat im Volkszählungsurteil (BVerfGE 65,1) die herausragende Bedeutung des Statistikgeheimnisses hervorgehoben. Es betrachtet den Grundsatz, die zu statistischen Zwecken erhobenen Einzelangaben strikt geheim zu halten, nicht nur als konstitutiv für die Funktionsfähigkeit der Statistik, sondern auch im Hinblick auf den Schutz des Rechts auf informationelle Selbstbestimmung als unverzichtbar.
Da es in Bayern für die zu treffenden Abschottungsmaßnahmen  -- insbesondere im IT-Bereich -- keine weiterführende rechtliche Regelung gibt, wird hiermit ein mit dem Bayerischen Landesbeauftragten für den Datenschutz abgestimmter Leitfaden zur Verfügung gestellt. Eine individuelle, davon abweichende kommunale Regelung sollte trotzdem immer noch mit dem behördlichen Datenschutzbeauftragten abgestimmt werden. 
Der Leitfaden richtet sich insbesondere an Kommunen, die erstmalig eine abgeschottete Statistikstelle einrichten möchten; es ist nicht erforderlich, dass eine Kommune mit bereits bestehender Statistikstelle Änderungen an ihrer bewährten individuellen Lösung vornimmt, sofern diese mit dem behördlichen Datenschutzbeauftragten abgestimmt ist.
Die im Folgenden aufgeführten Maßnahmen sind zu unterscheiden nach Maßnahmen, die konkret (z.B. im BayStatG) gefordert sind und Maßnahmen, die mangels konkreter gesetzlicher Vorgaben aus dem allgemeinen Abschottungsgebot abgeleitet wurden. Erstere sind direkt durch den Verweis auf die entsprechende Rechtsgrundlage gekennzeichnet, letztere (vorrangig die Maßnahmen zur IT-Abschottung) sind grundsätzlich nicht als Vorschrift, sondern als Empfehlung zu verstehen. D.h., eine Kommune, die sich bei der Einrichtung einer abgeschotteten Statistikstelle an die aufgeführten Maßnahmen hält, kann mit einer problemlosen Abstimmung mit dem behördlichen Datenschutzbeauftragten rechnen; weicht die Kommune von den Empfehlungen ab, entsteht evtl. weiterer Abstimmungsbedarf. Für die statistischen Einzeldaten wird das Schutzbedarfsniveau "normal" vorausgesetzt (Schutzstufe C gemäß Datensicherungskatalog des Koordinierungsausschusses Datenverarbeitung; Nä-heres hierzu in Abschnitt 4).

\chapter{Wer wohnt in Deutschland? – Die Haushaltsstichprobe im Zensus 2021}
\chapterauthor{Von Wanda Otto, Hessisches Statistisches Landesamt}
\minitoc
\begin{myblock}{Kurzfassung}
Mit Stichtag 16. Mai 2021 soll offiziell der Zensus 2021 in Deutschland starten. Er wird größtenteils analog zu dem für den Zensus 2011 neu entwickelten registergestützten Verfahren durchgeführt. Bei dem Verfahren handelt es sich um eine Kombination aus Registerauswertung und der Befragung einer Stichprobe von Haushalten. Ziel ist es zum einen die amtliche Einwohnerzahl zu ermitteln und zum anderen wichtige Strukturmerkmale beispielsweise zur Bildung und Erwerbstätigkeit der Bevölkerung zu erhalten.
\end{myblock}


Die Einwohnerzahlen basieren auf den in Deutschland dezentral vorliegenden Melderegisterdaten. Da es aber bspw. aufgrund von Wanderung zu Zu- und Fortzügen kommt und nicht alle Veränderungen auch gemeldet werden, ist eine regelmäßige bundesweite Erhebung zu einem Stichtag notwendig, um die korrekte Einwohnerzahl ermitteln zu können. Angaben zu Bildung und Erwerbstätigkeit wiederum liegen nicht in Registern vor und müssen daher ebenfalls primärstatistisch erhoben werden.\par

Eine präzise Ermittlung der Einwohnerzahl ist wichtig, weil für die Vergabe bzw. Verteilung finanzieller Mittel seitens der EU und der Bundes- oder Landesregierungen an Gemeinden sowie für den Kommunalen Finanzausgleich die amtliche Einwohnerzahl ausschlaggebend ist. Sie wird mit dem regelmäßigen Zensus ermittelt und dann für die nächsten 10 Jahre im Rahmen der Bevölkerungsfortschreibung fortgeschrieben. Um diese und weitere Angaben über die Bevölkerung zu erhalten, die für Politik und Verwaltung sowohl auf nationaler als auch auf europäischer Ebene zentral sind, um aber andererseits die Bürger nicht über Gebühr zu belasten, werden nicht alle Haushalte sondern nur annähernd 10\% – die sogenannte Haushaltsstichprobe – befragt. Im Rahmen der jährlichen Mikrozensus-Befragung liegen zwar detaillierte Angaben auch zur Bildungs- und Erwerbsbiografie der Befragten vor, allerdings handelt es sich hierbei lediglich um eine 1\%-Stichprobe. Zeitgemäße Erhebungswege, wie die Möglichkeit, Online Angaben zu machen, und weitere Maßnahmen sollen die auskunftspflichtigen Bürgerinnen und Bürger der Haushaltsstichprobe entlasten.\par

\section[Wer ist auskunftspflichtig?]{Wer ist für die Haushaltebefragung im Zensus 2021 auskunftspflichtig?}
Die Stichprobe für die Haushaltebefragung wird auf Basis der im Steuerungsregister konsolidierten Daten gezogen. Das Steuerregister ist das zentrale Anschriften- und Personenregister des Zensus. In Hessen werden insgesamt etwa 750.000 Bürgerinnen und Bürger innerhalb von 12 Wochen befragt.\par Auskunftspflichtig sind alle Personen und Haushalte einer Anschrift, die in die Stichprobe gezogen wurde (siehe Infokasten 1). Es wird das sogenannte ``Wohnhaushaltsprinzip'' zugrunde gelegt, dementsprechend gehören diejenigen Personen zu einem Haushalt, die gemeinsam dort wohnhaft sind, unabhängig davon, ob sie auch gemeinsam wirtschaften. Wer alleine wohnt, bildet einen eigenen Haushalt, auch wenn die Person noch minderjährig ist. Jede an einer Anschrift wohnhafte Person wird erfasst, unabhängig davon, ob es sich bei der Anschrift um den Haupt- oder Nebenwohnsitz der Person handelt.\par
Um die Erhebung so belastungsarm wie möglich zu gestalten, können einzelne Personen auch Auskunft für andere im selben Haushalt lebende Personen machen, bspw. Partner für ihre Partnerinnen und Eltern für ihre Kinder. Diese sogenannten Proxy-Interviews dienen dazu, die Angaben schnell und unkompliziert für alle auskunftspflichtigen Personen eines Haushalts erheben zu können.\par

\begin{myblock}{Infokasten 1: Auskunftspflicht}
Um die Vollzähligkeit und Vollständigkeit der Erhebung zu gewährleisten, sieht der Bundesgesetzgeber eine Auskunftspflicht aller Personen bzw. Haushalte vor, die Teil der Stichprobe sind. Die Auskunftspflicht für den Zensus wird im Zensusgesetz 2021 in § 3 geregelt. Auskunftspflichtig sind demnach alle Personen und Haushalte, die zur Bevölkerung Deutschlands zählen. Dies sind Einwohner der Gemeinden, unabhängig davon, ob sie dort mit Haupt- oder Nebenwohnsitz gemeldet sind, sowie Personen, die im Rahmen ihrer beruflichen Tätigkeit für den deutschen Staat im Ausland tätig sind, bspw. Angehörige der Bundeswehr, der Polizeibehörden des Bundes und der Länder und des Auswärtigen Dienstes sowie ihre dort ansässigen Familien. Für die Ermittlung der Einwohnerzahl der Gemeinden werden dann in einem weiteren Schritt nur diejenigen Personen berücksichtigt, die dort ihren Haupt- bzw. alleinigen Wohnsitz haben. Die Auskunft kann elektronisch, schriftlich, mündlich oder telefonisch erfolgen, soweit diese Möglichkeit zur Antworterteilung von der Erhebungsstelle angeboten wird.
\end{myblock}
\section[Wer erhebt?]{Wie ist die Durchführung des Zensus organisiert?}
Die Erhebungen der Haushaltsstichprobe und auch die Primärerhebungen in den Sonderbereichen (Wohnheime und Gemeinschaftsunterkünfte) sollen, wie schon 2011, von kommunalen Erhebungsstellen organisiert und durchgeführt werden. Daher kommt den Kommunen eine wichtige Rolle in der Realisierung dieses Großprojekts der amtlichen Statistik zu. Die Kommunen in Hessen, genauer gesagt die Kreisverwaltungen, die kreisfreien Städte sowie die Sonderstatusstädte (Städte mit über 50.000 Einwohnern) richten nach jetziger Planung daher ab Sommer 2020 Erhebungsstellen ein. Diese Erhebungsstellen sind insbesondere auf der Grundlage der datenschutzrechtlichen Vorgaben (§ 16 Bundesstatistikgesetz zur Geheimhaltung und europäische Datenschutz-Grundverordnung, DSGVO) von der übrigen Verwaltung personell, organisatorisch und räumlich getrennt und gewährleisten bei der Erfüllung ihrer Aufgaben die statistische Geheimhaltung. Die statistische Geheimhaltung ist ein Grundprinzip der amtlichen Statistik und stellt sicher, dass Erkenntnisse aus der Erhebungstätigkeit nicht für andere Zwecke (bspw. andere Verwaltungsaufgaben) verwendet werden. Ebenso werden alle von den Erhebungsstellen zur Wahrnehmung ihrer Aufgaben eingesetzten Interviewerinnen und Interviewer (Erhebungsbeauftragte) zur Geheimhaltung verpflichtet. Allein in Hessen werden voraussichtlich ca. 7.500 Interviewende angeworben, verpflichtet und von den Erhebungsstellen geschult. Alle Erhebungsunterlagen werden bis zur Beendigung der Feldphase in den Erhebungsstellen auf Vollzähligkeit und auf Vollständigkeit hin überprüft und dann zur weiteren Bearbeitung an die Beleglesung zur Digitalisierung der Erhebungsunterlagen bzw. an die Statistischen Ämter weitergeleitet.

\section[Wie wird erhoben?]{Wie laufen die Haushaltebefragungen ab?}
Um die Befragten zu entlasten wird die Haushaltebefragung im Zensus 2021 drei-stufig erfolgen. Zunächst einmal werden die Stichprobenanschriften durch die Interviewerinnen und die Interviewer begangen, um zu ermitteln, wer an der Anschrift wohnt und um erste Informationen zum Zensus sowie eine Terminankündigung für ein persönliches Interview zu hinterlassen bzw. in den Briefkasten einzulegen. Im anschließenden persönlichen Kontakt werden dann die Merkmale zu den im Haushalt lebenden Personen wie bspw. Vor- und Nachname aufgenommen und entsprechend der Anzahl der Haushaltsmitglieder weitere Informationen und eine individuelle Zugangskennung für die Online-Befragung ausgegeben. So haben die auskunftspflichtigen Bürgerinnen und Bürger im dritten Schritt die Gelegenheit, ihre Angaben bspw. zu Bildungsabschlüssen und ihrer Erwerbstätigkeit zeit- und ortsunabhängig, einfach, schnell und unkompliziert online einzugeben.\par

Wenn eine (weitere) Kontaktaufnahme mit den Auskunftspflichtigen durch die Erhebungsbeauftragten nicht zustande kommt, nimmt die zuständige kommunale Erhebungsstelle schriftlich mit den entsprechenden auskunftspflichtigen Haushalten Kontakt auf, sodass diese ihre Angaben online machen. Ist dies nicht möglich oder nicht gewünscht, gibt es für die Auskunftspflichtigen die Möglichkeit ihre Angaben schriftlich auf einem Papier-Fragebogen oder ggfs. auch telefonisch zu machen.\par

Zwar gab es bereits im Zensus 2011 die Möglichkeit für die Auskunftspflichtigen Ihre Angaben online zu machen, allerdings wurde diese mit einem Anteil von 7 Prozent nur relativ selten genutzt. Dafür sind überwiegend zwei Gründe zu nennen: Zum einen wurden primär Erhebungsbeauftragte eingesetzt, die alle Angaben direkt vor Ort erhoben haben. Zum anderen waren die technischen Möglichkeiten eingeschränkter. Neu ist dieses Mal daher die sogenannte Online-First-Strategie (siehe Infokasten 2).\par

\begin{myblock}{Infokasten 2: Online-First-Strategie}
Die Online-First-Strategie des Zensus 2021 baut auf die bei der Bevölkerung vorhandene Infrastruktur zur Internetnutzung und verspricht dabei den Befragten dieselben Vorteile, die zu Anschaffung und Unterhalt von Smartphone, Tablet, Laptop oder Desktop-Computer geführt haben: Das zeitsouveräne, leichtere, schnellere und kostengünstigere Erledigen von Aufgaben des täglichen Lebens. Damit einher geht ein Zensus, der genauere Ergebnisse in kürzerer Zeit liefert und dabei Ressourcen schont. Alle Erhebungsprozesse werden zuerst auf dieses Ziel hin konzipiert. Dies schließt die Bereitstellung von Papierfragebogen bei Bedarf sowie, wo erforderlich, Interviewer-gestützte Haushaltebefragungen mit ein. Darüber hinaus ist geplant, dass die Angaben auch telefonisch übermittelt werden können.
\end{myblock}

Der öffentliche Dienst und vor allem auch die amtliche Statistik nutzt für den Zensus 2021 einen modernen und zeitgemäßen Erhebungsweg wie die Online-Selbstauskunft. Die Online-Befragung kann mit verschiedensten internetfähigen Geräten neben Desktop-PC, auch mit einem Tablet oder Smartphone durchgeführt werden. Das Layout der Fragebogen passt sich dabei entsprechend des verwendeten Endgerätes der Bildschirmgröße selbstständig an. Zusätzlich wird darauf geachtet, die Befragung nutzerfreundlich zu gestalten. Das heißt, dass sowohl der Login-Bereich zur Online-Befragung leicht erreichbar ist z. B. mittels eines QR-Codes, als auch übersichtlich gestaltet ist, um sich rasch einloggen zu können.\par

Die Online-First-Strategie bietet sowohl für die Befragten als auch für die statistischen Ämter Vorteile. Da bei Online-Meldungen die Filterführung so programmiert ist, dass die Befragten lediglich die für sie relevanten Fragen angezeigt bekommen und zudem Fehler unmittelbar während der Erfassung angezeigt werden können, ist die Beantwortung einfacher und zugleich die Datenqualität höher. Mittels elektronischem Eingang der Meldungen können die Angaben unmittelbar auf Plausibilität geprüft werden und die Daten liegen direkt elektronisch vor. Folglich müssen nicht erst noch die auf Papier ausgefüllten Bögen zu einem Beleglesezentrum transportiert, dort verarbeitet und anschließend plausibilisiert werden. Dies spart auch Zeit – den Befragten und den Statistischen Ämtern. Somit können die Ergebnisse des Zensus 2021 bereits 18 Monate nach Stichtag veröffentlicht werden.\par

\section[Was wird gefragt?]{Was wird gefragt? Welche Merkmale werden erfasst?}
Wie bereits im vorherigen Abschnitt beschrieben erfolgt die Befragung mehrstufig. Die Erhebungsbeauftragten suchen zunächst den persönlichen Kontakt zu den Haushalten, um erste für die Existenzfeststellung notwendige Merkmale zu erheben. Um die Bürgerinnen und Bürger der Haushaltsstichprobe soweit möglich zu entlasten, wird sich dabei auf ein Minimum an Fragen fokussiert (primär auf die Merkmale 1 bis 9). Die Beantwortung aller weiteren Merkmale (ab Merkmal 10) können die Auskunftspflichtigen dann im Anschluss an den Besuch der Interviewerin oder des Interviewers selbstständig vornehmen.\par

Folgende Erhebungsmerkmale werden laut § 13 Zensusgesetz 2021-Entwurf erfragt:
\setlength{\parskip}{1pt}
\begin{enumerate}[label=\arabic*.]
    \item Wohnungsstatus,
    \item Wohnungsstatus,
    \item Geschlecht,
    \item Staatsangehörigkeiten,
    \item Monat und Jahr der Geburt,
    \item Familienstand,
    \item nichteheliche Lebensgemeinschaften,
    \item Jahr der Ankunft in Deutschland (für Personen, die nach dem 31. Dezember 1955 nach Deutschland zugezogen sind),
    \item Anzahl der Personen im Haushalt,
    \item Geburtsstaat,
    \item Erwerbsstatus in der Woche des Zensusstichtags,
    \item Hauptstatus in der Woche des Zensusstichtags,
    \item Stellung im Beruf,
    \item ausgeübter Beruf,
    \item Wirtschaftszweig des Betriebs,
    \item Anschrift des Betriebs (nur Postleitzahl und Gemeinde),
    \item höchster allgemeiner Schulabschluss,
    \item höchster beruflicher Bildungsabschluss,
    \item aktueller Schulbesuch.
\end{enumerate}
\setlength{\parskip}{\baselineskip}
Darüber hinaus werden noch folgende Hilfsmerkmale benötigt:
\setlength{\parskip}{1pt}
\begin{enumerate}[label=\arabic*.]
    \item Familienname und Vornamen,
    \item Anschrift der Wohnung und Lage der Wohnung im Gebäude,
    \item Tag der Geburt ohne Monats- und Jahresangabe,
    \item Kontaktdaten der Auskunftspflichtigen oder einer anderen für Rückfragen zur Verfügung stehenden Person.
\end{enumerate}
\setlength{\parskip}{\baselineskip}
Die Datenschutzbestimmungen werden sowohl im Zensusgesetz 2021 (ZensG § 27 ff) als auch nach der europäischen Datenschutz-Grundverordnung (DSGVO) geregelt. Während im ZensG überwiegend die Verantwortlichkeit und die Verarbeitungsrechte der zu erhebenden Daten geregelt werden, befasst sich die DSGVO verstärkt mit den Rechten der betroffenen Personen, bspw. Artikel 13 ``Informationspflicht bei Erhebung von personenbezogenen Daten''.\par

\section[Wie wird berechnet?]{Wie fließen die Befragungsergebnisse in die Ermittlung der amtlichen Einwohnerzahl ein?}
Die Melderegisterdaten spielen eine große Rolle bei der Ermittlung der amtlichen Einwohnerzahl. Sie dienen auch als Ausgangsbasis für die Stichprobenziehung zur Haushaltebefragung. Zur Ermittlung der Einwohnerzahl werden vom Melderegisterbestand zum Zensusstichtag zunächst die Nebenwohnsitze und die freiwilligen Meldungen abgezogen und Zuzüge sowie Geburten hinzugezählt. Anschließend werden für eine erste Korrektur im Rahmen der sogenannten Mehrfachfallprüfung Meldungen von Personen, die bspw. an zwei Adressen einen Hauptwohnsitz angemeldet haben, subtrahiert und Personen, die bspw. bislang ausschließlich mit einem Nebenwohnsitz gemeldet waren, entsprechend addiert.

\setlength{\parskip}{1pt}
\begin{description}
    \item[]Melderegisterbestand (zum Zensusstichtag)
    \item[--]Nebenwohnsitze
    \item[--]Freiwillige Meldungen (Angehörige ausländischer Streitkräfte, Diplomaten)
    \item[+]Zuzüge und Geburten
    \item[=]\textbf{Konsolidierter Melderegisterbestand}
    \item[--]Abgänge durch Mehrfachfallprüfung
    \item[+]Zugänge durch Mehrfachfallprüfung
    \item[=]\textbf{1. korrigierter Melderegisterbestand}
    \item[--]Abgänge Sonderbereichserhebung
    \item[+]Zugänge Sonderbereichserhebung
    \item[=]\textbf{2. korrigierter Melderegisterbestand}
    \item[--]Karteileichen (hochgerechnet) aus Haushaltsstichprobe
    \item[+]Fehlbestände (hochgerechnet) aus Haushaltsstichprobe
    \item[=]\textbf{Einwohnerzahl}
\end{description}
\setlength{\parskip}{\baselineskip}

Im zweiten Korrekturschritt wird der Melderegisterbestand dann um Meldungen von Personen aktualisiert, die in Sonderbereichen wie Pflegeheimen gemeldet sind. Mittels der Erhebungen im Rahmen der Haushaltsstichprobe können die an einer Anschrift tatsächlich wohnhaften Personen – durch die kommunalen Erhebungsstellen bestätigt sowie an der Anschrift nicht (mehr) wohnhafte Personen (``Karteileichen'') und dort noch nicht gemeldete Personen (``Fehlbestände'') ermittelt werden. Die so festgestellten Karteileichen und Fehlbestände auf Basis der Haushaltsstichprobe werden entsprechend auf die Gesamtzahl der Anschriften der Gemeinde hochgerechnet und dem 2. korrigierten Melderegisterbestand entsprechend ab- bzw. aufgeschlagen. Der Prozess vom Stichtag bis zur Ermittlung der amtlichen Einwohnerzahl soll innerhalb von 18 Monaten mit der Veröffentlichung der Ergebnisse abgeschlossen sein.

\section{Fazit}
Die mithilfe des Zensus, also der Volkszählung auf Basis einer rund 10\%-Stichprobe ermittelten amtlichen Einwohnerzahl dient unter anderem der statistischen Korrektur der Bevölkerungsfortschreibung bis zum nächsten Zensus, der die Fortschreibung dann wieder bereinigt. Darüber hinaus dient die amtliche Einwohnerzahl als Bemessungsgrundlage für die Finanzausgleiche auf Ebene der Länder und der Kommunen. Des Weiteren wird sie als Richtgröße für die Einteilung der Bundestagswahlkreise, für die Berechnung der Zahl der Stimmen der Länder im Bundesrat und für die Berechnung der Zahl der Sitze in den Gemeinderäten genutzt.\par

Auch für die Statistik selbst kommt den Zensus-Daten eine zentrale Stellung zu, da sie als Auswahlgrundlage und Hochrechnungsrahmen für weitere amtliche und nichtamtliche Stichprobenerhebungen dienen. Dazu zählen bspw. der Mikrozensus, die größte amtliche Haushaltsbefragung in Deutschland, oder die Allgemeine Bevölkerungsumfrage (ALLBUS) sowie zahlreiche weitere Erhebungen in Wissenschaft und Wirtschaft. Die ebenfalls im Rahmen des Zensus 2021 zu erhebenden strukturellen Zusatzmerkmale sollen laut Bundesinnenministerium für politische Planungen und Entscheidungen genutzt werden. Hier geht es beispielsweise etwa um die Frage, wo Schulen, Studienplätze und Altersheime benötigt werden.


\chapter{Referentenentwurf Mietspiegelverordnung}
\minitoc
    \section{Allgemeinde Regelungen}
        \subsection{\S 1 - Gegenstand}
        Gegenstand der Verordnung sind der Inhalt und das Verfahren zur Erstellung und Anpassung von Mietspiegeln im Sinne des § 558c Absatz 1 des Bürgerlichen Gesetzbuchs. Die Verordnung betrifft sowohl qualifizierte Mietspiegel (§ 558d Absatz 1 des Bürgerlichen Gesetzbuchs) als auch Mietspiegel, die keine qualifizierten Mietspiegel sind (einfache Mietspiegel).
        \subsection{\S2 - Begriffsbestimmungen}
        \begin{enumerate}[label=(\arabic*)]
            \item Wohnwertrelevante gesetzliche Merkmale sind die in § 558 Absatz 2 Satz 1 des Bürgerlichen Gesetzbuchs genannten Merkmale Art, Größe, Ausstattung, Beschaffenheit und Lage einer Wohnung, soweit sie für die Mietpreisbildung relevant sind oder im Erstellungsstadium des Mietspiegels relevant sein können.
            \item Außergesetzliche Merkmale sind Merkmale in Bezug auf die Wohnung oder das Mietverhältnis, die in § 558 Absatz 2 Satz 1 des Bürgerlichen Gesetzbuchs nicht genannt sind, aber dennoch für die Mietpreisbildung relevant sind oder im Erstellungsstadium des Mietspiegels relevant sein können.
            \item Die Auswertungsgrundgesamtheit ist die Gesamtheit der mietspiegelrelevanten Wohnungen.
            \item Die Erhebungsgrundgesamtheit ist die Gesamtheit der Wohnungen, aus der die Bruttostichprobe gezogen wird, um nach Aussortierung nicht mietspiegelrelevanter Wohnungen die für den Mietspiegel relevante Stichprobe der Auswertungsgrundgesamtheit zu generieren.
        \end{enumerate}

\section{Einfache Mietspiegel}
    \subsection{\S 3 - Erstellung und Anpassung}
    Die Erstellung und Anpassung eines einfachen Mietspiegels ist vorbehaltlich der §§ 4 und 5 an kein Verfahren gebunden.
    \subsection{\S 4 - Dokumentation}
    Die Erstellung und Anpassung eines einfachen Mietspiegels und die dafür verwendeten tatsächlichen Grundlagen sollen in Grundzügen im Mietspiegel oder in einer gesonderten Dokumentation angezeigt und erläutert werden.
    \subsection{\5 - Veröffentlichung}
    Ein einfacher Mietspiegel und seine Dokumentation sollen kostenfrei im Internet veröffentlicht werden. Für ihre Ausgabe in gedruckter Form können angemessene Entgelte verlangt werden.
        
\section{Qualifizierte Mietspiegel}
    \subsection{\S 6 - Allgemeine Anforderungen}
    \begin{enumerate}[label=(\arabic*)]
        \item Das für qualifizierte Mietspiegel bestehende Erfordernis der Erstellung nach wissenschaftlichen Grundsätzen (§ 558d Absatz 1 Satz 1 des Bürgerlichen Gesetzbuchs) betrifft alle Phasen der Mietspiegelerstellung.
        \item Mietspiegel entsprechen wissenschaftlichen Grundsätzen, soweit sie unter Beachtung der in den §§ 7 bis 21 geregelten Anforderungen erstellt wurden. Soweit Mietspiegel diese Anforderungen nicht erfüllen, sind sie einfache Mietspiegel.
    \end{enumerate}
\section{Qualifizierte Mietspiegel - Erstellung}
    \subsection{\S7 - Methoden}
    \begin{enumerate}[label=(\arabic*)]
        \item Qualifizierte Mietspiegel können mittels Regressions- oder mittels Tabellenanalyse oder durch eine Kombination beider Methoden oder durch eine vergleichbar geeignete Methode erstellt werden.
        \item Auf qualifizierte Mietspiegel, die mittels einer Kombination der Regressions- und Tabellenanalyse erstellt werden, sind die §§ 11 bis 16 nur insoweit anzuwenden, als sie die jeweils angewandte Methode betreffen. Entsprechendes gilt für qualifizierte Mietspiegel, die durch eine vergleichbar geeignete Methode (Absatz 1) erstellt werden. 
    \end{enumerate}
    \subsection{\S8 - Datengrundlagen}
    \begin{enumerate}
        \item Qualifizierte Mietspiegel müssen vorbehaltlich des Absatzes 3 auf der Grundlage einer direkten Datenerhebung durch Befragung von Vermietern oder Mietern oder von beiden Gruppen erstellt werden (Primärdatenerhebung). Eine Vollerhebung ist nicht erforderlich. Qualifizierte Mietspiegel sind zumindest auf der Basis einer repräsentativen Stichprobe zu erstellen mit dem Ziel, die Auswertungsgrundgesamtheit möglichst wirklichkeitsgetreu abzubilden. Als repräsentativ gilt eine Stichprobe mit einer nach § 11 ausreichenden Datenmenge, wenn sie auf einer Zufallsauswahl beruht, bei der im Wesentlichen jede Wohnung der Auswertungsgrundgesamtheit eine positive und bekannte Wahrscheinlichkeit hat, in die Erhebung einbezogen zu werden.
        \item Nicht durch eine Primärdatenerhebung ermittelte Daten über Wohnungen (Sekundärdaten) dürfen zur Vorbereitung der Datenerhebung oder zur Plausibilitätsprüfung (§ 9 Absatz 3 Satz 1) verwendet werden.
        \item Qualifizierte Mietspiegel können auch Angaben enthalten, die auf einer Auswertung solcher Primärdaten beruhen, die mangels ausreichender Fallzahlen keine verlässlichen Angaben zur Mietpreisbildung zulassen. Sie können auch Angaben aufgrund der Auswertung von Sekundärdaten oder fachkundlichen Schätzungen enthalten. Angaben nach den Sätzen 1 und 2 sind nicht Teil des qualifizierten Mietspiegels; hierauf ist im Mietspiegel ausdrücklich hinzuweisen. Die Angaben sollen in entsprechender Anwendung des § 4 dokumentiert werden.
        \item In der Dokumentation sind die Erstellung der Erhebungsgrundgesamtheit und die dafür verwendeten Datengrundlagen darzustellen.
    \end{enumerate}
    \subsection{\S9 - Bruttostichprobe}
    \begin{enumerate}[label=(\arabic*)]
        \item Beim Ziehen einer Stichprobe von Wohnungen, hinsichtlich derer eine Primärdatenerhebung stattfinden soll (Bruttostichprobe), ist sicherzustellen, dass es sich um eine
repräsentative Stichprobe nach § 8 Absatz 1 Satz 4 handelt.
        \item Die Bruttostichprobe kann nach wohnwertrelevanten gesetzlichen Merkmalen oder außergesetzlichen Merkmalen proportional oder disproportional geschichtet werden. Eine Schichtung kann insbesondere nach Vermietertypen, Größenklassen, Ausstattungsmerkmalen, Wohnlagen und Baualtersklassen vorgenommen werden. Die Schichtung erfolgt aufgrund einer Aufteilung der Erhebungsgrundgesamtheit in homogene und überschneidungsfreie Teilgruppen. Wurde eine disproportional geschichtete Zufallsstichprobe gezogen, so ist bei der Datenauswertung eine entsprechende Rückgewichtung vorzunehmen, sofern ansonsten eine Verzerrung der Ergebnisse zu erwarten ist. 
        \item Liegen gesicherte Erkenntnisse über die statistische Ausprägung wesentlicher wohnwertrelevanter gesetzlicher oder außergesetzlicher Merkmale und über ihre Anteile an der Erhebungsgrundgesamtheit vor, so soll die Bruttostichprobe darauf überprüft werden, ob Wohnungen mit solchen statistischen Ausprägungen entsprechend ihrem Anteil an der Erhebungsgrundgesamtheit vertreten sind (Plausibilitätsprüfung). Sind Wohnungen mit solchen statistischen Ausprägungen offensichtlich nicht angemessen vertreten und sind dadurch Verzerrungen der Ergebnisse zu erwarten, soll einer Verzerrung durch geeignete Maßnahmen, beispielsweise durch eine korrigierende Gewichtung bei der Datenauswertung, begegnet werden.
        \item In der Dokumentation ist nachvollziehbar darzustellen, wie die Bruttostichprobe gezogen wurde, einschließlich etwaiger Schichtungen und dadurch notwendiger Rückgewichtungen, ob und in welcher Weise eine Plausibilitätsprüfung durchgeführt wurde, zu welchem Ergebnis eine solche Überprüfung geführt hat und welche Folgerungen daraus gezogen wurden. 
    \end{enumerate}

§ 10
Nettostichprobe
(1) Die Nettostichprobe ist der Rücklauf aus der Befragung von Vermietern oder Mietern oder beider Gruppen. Die Nettostichprobe ist um die Rückläufer zu bereinigen, die
mangels Zugehörigkeit zur Auswertungsgrundgesamtheit oder aufgrund einer Mehrfachzählung derselben Wohnung oder aufgrund grob unvollständiger oder offensichtlich unzutreffender Antworten für die Auswertung nicht verwendet werden können (bereinigte Nettostichprobe).
- 8 -
(2) Die Rücklaufquote und die Bereinigung der Nettostichprobe sind zu dokumentieren. In der Dokumentation ist zu darzustellen, ob durch einen unvollständigen oder selektiven Rücklauf oder durch die Bereinigung der Nettostichprobe Verzerrungen der Ergebnisse möglich sind.
§ 11
Stichprobenumfang
(1) Die bereinigte Nettostichprobe muss eine ausreichende Datenmenge enthalten.
(2) Bei Tabellenanalysen ist hierfür im Regelfall eine Belegung von mindestens 30
Wohnungen pro Tabellenfeld erforderlich.
(3) Bei Regressionsanalysen soll die bereinigte Nettostichprobe Wohnungen in einer
Anzahl enthalten, die wenigstens ein Prozent der Wohnungen der Gemeinde entspricht.
Unterschreitet die nach Satz 1 erforderliche Anzahl an Wohnungen 500, so bedarf es in der
Regel eines Stichprobenumfangs von mindestens 500 Wohnungen. Übersteigt die nach
Satz 1 erforderliche Anzahl an Wohnungen 3 000, so genügt ein Stichprobenumfang von
3 000 Wohnungen.
(4) Die Erfüllung der Anforderungen nach den Absätzen 1 bis 3 ist in der Dokumentation nachzuweisen.
§ 12
Datenaufbereitung
(1) Die erhobenen Mietwerte sollen so aufbereitet werden, dass eine einheitliche Ausweisung der ortsüblichen Vergleichsmiete im qualifizierten Mietspiegel als Nettokaltmiete
pro Quadratmeter ermöglicht wird.
(2) Die erhobenen Daten können um Ausreißermieten bereinigt werden. Ausreißermieten sind besonders geringe oder besonders hohe Mieten, die unter Berücksichtigung
der wohnwertrelevanten Eigenschaften der Wohnung mit der weit überwiegenden Zahl der
übrigen Mietwerte unvereinbar erscheinen. Die Ermittlung von Ausreißermieten kann durch
statistische Standardverfahren erfolgen und soll auf Plausibilität überprüft werden. Für die
Prüfung können sowohl wohnwertrelevante gesetzliche als auch außergesetzliche Merkmale herangezogen werden.
(3) In der Dokumentation ist eine Bereinigung um Ausreißermieten einschließlich des
gewählten Verfahrens zu erläutern und es ist darzustellen, welche Mietwerte aus welchen
Gründen ausgesondert wurden.
§ 13
Datenauswertung bei der Tabellenanalyse
(1) Wird die ortsübliche Vergleichsmiete mithilfe der Tabellenanalyse ermittelt, so sind
Tabellenfelder durch Kombinationen wohnwertrelevanter gesetzlicher Merkmale zu bilden
mit dem Ziel, in sich möglichst homogene Tabellenfelder zu erzeugen, die gegenüber anderen Tabellenfeldern möglichst verschieden sind.
- 9 -
(2) Lassen sich ungeachtet des Vorgehens nach Absatz 1 abweichende homogene
Teilmengen innerhalb eines Tabellenfeldes feststellen, die sich in ihren Mieten signifikant
von den restlichen Mieten des Tabellenfeldes unterscheiden, so soll überprüft werden, ob
hierfür separate Tabellenfelder gebildet oder ergänzende Hinweise für die Bewertung dieser Teilmengen gegeben werden können.
(3) In der Dokumentation ist darzustellen, nach welchen Kriterien und Verfahren die
Tabellenfelder gebildet wurden, wie viele Wohnungen für ein Tabellenfeld ausgewertet wurden und wie hoch die Mieten dieser Wohnungen waren.
§ 14
Datenauswertung bei der Regressionsanalyse
(1) Wird die ortsübliche Vergleichsmiete nach der Regressionsanalyse ermittelt, so
sind wohnwertrelevante gesetzliche Merkmale daraufhin zu untersuchen, ob sie einen statistisch signifikanten Einfluss auf den Mietpreis haben mit dem Ziel, den Zusammenhang
zwischen der Miethöhe und der gesetzlichen wohnwertrelevanten Merkmale möglichst gut
zu beschreiben. Außergesetzliche Merkmale können insbesondere zur Wahl des Regressionsmodells und bei der Bemessung von Spannen nach § 16 Absatz 3 herangezogen werden.
(2) In der Dokumentation ist darzustellen und zu erläutern,
1. welche Regressionsfunktion der Analyse zugrunde liegt,
2. welche Merkmale sich mit welchem Einfluss auf die Miethöhe auswirken, ob dieser
Einfluss statistisch signifikant ist und welches Signifikanzniveau dabei zugrunde gelegt
wird,
3. wie hoch der Erklärungsgehalt der verwendeten Regressionsfunktion ist und
4. inwieweit die tatsächlich vorgefundenen Mieten von den Ergebniswerten der Regressionsformel abweichen.
In der Dokumentation ist weiter zu erklären, ob und in welcher Weise eine Modellvalidierung
erfolgte und zu welchem Ergebnis sie führte.
§ 15
Bestimmung und Darstellung der ortsüblichen Vergleichsmiete bei der Tabellenanalyse
(1) In einem nach der Tabellenanalyse erstellten qualifizierten Mietspiegel wird die
ortsübliche Vergleichsmiete in den Tabellenfeldern durch einen Mittelwert und eine um diesen gebildete Spanne dargestellt. Die ortsübliche Vergleichsmiete soll im Einzelfall innerhalb der Spanne durch Zu- und Abschläge vom Mittelwert bestimmt werden.
(2) Der Mittelwert ist das arithmetische Mittel oder der Median und wird aus allen Mieten eines Tabellenfeldes nach einer etwaigen Bereinigung um Ausreißermieten gebildet.
Der Mittelwert entspricht der ortsüblichen Vergleichsmiete für eine Wohnung, die im Vergleich zu anderen Wohnungen des entsprechenden Tabellenfeldes unter Berücksichtigung
von Qualität und Quantität weiterer wohnwertrelevanter gesetzlicher Merkmale, die nicht
mittels der Tabellenfelder beschrieben werden, als durchschnittlich zu bewerten ist.
- 10 -
(3) Für die Bildung der Spanne sollen in der Regel je ein Sechstel bis ein Achtel der
nach Ausreißerbereinigung in einem Tabellenfeld verbliebenen Mieten am oberen und am
unteren Ende der größengeordneten Mieten unberücksichtigt bleiben. Bei der Bildung der
Spanne kann berücksichtigt werden, wie stark die Streuung der Mieten insgesamt oder im
jeweiligen Tabellenfeld ist.
(4) Aus wohnwertrelevanten gesetzlichen Merkmalen, die nicht mittels der Tabellenfelder beschrieben werden, können sich Zu- und Abschläge ausgehend vom Mittelwert des
Tabellenfeldes ergeben. Der Mietspiegel kann Bewertungshilfen für die Zu- und Abschläge
vorsehen, um die Einordnung einer Wohnung innerhalb der Spanne eines Tabellenfeldes
zu erleichtern. Machen besondere Merkmale eine Überschreitung des Oberwertes oder
eine Unterschreitung des Unterwertes der Spanne notwendig, ist dies im Mietspiegel gesondert auszuweisen.
(5) Die Bildung der Mittelwerte und der Spannen ist in der Dokumentation zu erläutern.
Sieht der Mietspiegel Bewertungshilfen für Zu- und Abschläge vor, ist in der Dokumentation
darzulegen, nach welchen Kriterien und auf welche Weise diese Bewertungshilfen erstellt
wurden.
§ 16
Bestimmung und Darstellung der ortsüblichen Vergleichsmiete bei der Regressionsanalyse
(1) Die ortsübliche Vergleichsmiete in einem mittels Regressionsanalyse erstellten
qualifizierten Mietspiegel wird im Einzelfall durch Anwendung der Regressionsfunktion ermittelt. Die Vergleichsmiete für eine bestimmte Wohnung kann insbesondere als wohnungsspezifischer Punktwert oder klassifiziert in Tabellenform gegebenenfalls mit Zu- und Abschlägen ausgewiesen werden.
(2) Im qualifizierten Mietspiegel ist darzustellen, wie die durch Regression festgestellten wohnwertrelevanten gesetzlichen Merkmale definiert werden und welchen Einfluss das
jeweilige Merkmal auf die Miethöhe hat.
(3) In dem mittels Regressionsanalyse erstellten qualifizierten Mietspiegel kann die
Schwankungsbreite der ermittelten ortsüblichen Vergleichsmiete durch Spannen berücksichtigt werden. Bei der Bildung von Spannen soll dargestellt werden, inwieweit die durch
Befragung erhobenen Mieten von den auf Basis der Regressionsanalyse errechneten Mieten nach oben oder unten abweichen. Dies kann insbesondere dadurch erfolgen, dass von
der Abweichung zwischen den vorhergesagten und den beobachteten Mieten am oberen
und unteren Ende je ein Sechstel bis ein Achtel nicht berücksichtigt wird.
(4) In der Dokumentation ist zu erläutern, wie das Ergebnis der Regressionsanalyse
im qualifizierten Mietspiegel dargestellt und die ortsübliche Vergleichsmiete einer Wohnung
konkret berechnet wird. Eine etwaige Bildung von Spannen ist darzustellen und zu erläutern.
- 11 -
Unterabschnitt 2
Inhalt des qualifizierten Mietspiegels
§ 17
Art der Wohnungen
(1) Der qualifizierte Mietspiegel soll in der Regel Wohnungen in Mehrfamilienhäusern
mit mehr als zwei Wohnungen erfassen. Andere Wohnungen sowie besondere Wohnungsund Vertragstypen in Mehrfamilienhäusern mit mehr als zwei Wohnungen können bei der
Erstellung eines qualifizierten Mietspiegels unberücksichtigt bleiben oder Gegenstand von
getrennten Erhebungen sein.
(2) Der qualifizierte Mietspiegel muss Angaben dazu enthalten, welche Wohnungsarten von ihm erfasst sind.
§ 18
Größe, Beschaffenheit und Ausstattung der Wohnungen
Im qualifizierten Mietspiegel soll dargestellt sein, welche Auswirkung die Größe sowie
die Beschaffenheit und die Ausstattung der Wohnung, einschließlich der energetischen
Ausstattung und Beschaffenheit, auf die Höhe der Miete pro Quadratmeter hat. Hierzu können Wohnungen in geeigneten Größenklassen zusammengefasst werden und es kann auf
Untermerkmale sowie auf deren Gruppierung und Klassifizierung zurückgegriffen werden,
sofern keine Mehrfachberücksichtigung erfolgt.
§ 19
Wohnlagen
(1) Unterschiedliche Wohnlagen müssen im qualifizierten Mietspiegel nur insoweit gesondert ausgewiesen werden, als eine sachgerechte Unterteilung in Wohnlagen möglich ist
und ein Einfluss der Lage auf die Mietpreisbildung festgestellt werden kann. Unterschiedlich
beschriebene Wohnlagen einer Gemeinde können im Mietspiegel nur dann zusammengefasst werden, wenn der lagebedingte Wohnwert vergleichbar ist.
(2) Zur Ermittlung von Wohnlagen soll untersucht werden, inwiefern sich durch Beschreibungen mittels vor Ort feststellbarer Faktoren wie insbesondere Bebauungs- und Verkehrsdichte, Zentralität, Infrastruktur, Begrünung oder vergleichbarer Kriterien Wohnlagen
einteilen lassen. Wird hierdurch die Einteilung von Wohnlagen nicht sachgerecht ermöglicht, können weitere Bewertungsmaßstäbe wie Bodenrichtwerte oder Kriterien der allgemeinen Beliebtheit bestimmter Wohngegenden berücksichtigt werden.
(3) Weist ein qualifizierter Mietspiegel unterschiedliche Wohnlagen aus, so sollen
diese exakt verortet werden, etwa durch ein Straßenverzeichnis oder durch eine aussagekräftige Wohnlagenkarte.
(4) Soweit wohnwertrelevante Lagemerkmale nicht bereits in eine Wohnlageneinteilung einbezogen wurden oder soweit die Lage vom Durchschnitt vergleichbarer Wohnun-
- 12 -
gen in derselben Wohnlage wesentlich abweicht, können wohnwertrelevante Lagemerkmale durch Zu- oder Abschläge zum Ergebniswert oder innerhalb der nach § 15 Absatz 1
oder § 16 Absatz 3 gebildeten Spanne berücksichtigt werden.
(5) Die Einteilung von Wohnlagen muss in der Dokumentation unter Darlegung der
Beurteilungskriterien und ihrer Zusammenhänge nachvollziehbar erläutert werden. In einem früheren Mietspiegel gebildete Wohnlageeinteilungen können fortgeschrieben werden,
wenn
1. die Dokumentation für den früheren Mietspiegel eine Dokumentation nach Satz 1 enthält und
2. eine Plausibilitätsprüfung erfolgt, die geänderte Verhältnisse vor Ort berücksichtigt.
Die Voraussetzung des Satzes 2 Nummer 1 muss nicht gegeben sein für qualifizierte Mietspiegel, deren Stichtag innerhalb von zwei Jahren nach dem … [einsetzen: Datum des Inkrafttretens dieser Verordnung nach § 25] liegt. Die Durchführung der Plausibilitätsprüfung
und ihre Ergebnisse sind in der Dokumentation zu erläutern.
Unterabschnitt 3
Dokumentation und Veröffentlichung des qualifizierten Mietspiegels
§ 20
Dokumentation
(1) Angaben, die für die Anwendung des qualifizierten Mietspiegels notwendig sind,
einschließlich des Stichtags, zu dem die Daten für den Mietspiegel erhoben wurden, sind
in den Mietspiegel aufzunehmen.
(2) Erläuterungen, die notwendig sind, um das Verfahren und die Bewertungen, die
zu den Angaben im qualifizierten Mietspiegel, auch in der fortgeschriebenen Form, geführt
haben, nachzuvollziehen, sind in einer Dokumentation darzulegen. Die Dokumentation soll
vom Text- und Ergebnisteil des Mietspiegels getrennt sein. Sie soll es ermöglichen, die im
qualifizierten Mietspiegel angegebenen Werte in ihrer Herleitung nachzuvollziehen; nicht
erforderlich ist eine Dokumentation, die eine vollständige Nachberechnung der Ergebnisse
ermöglicht.
(3) In der Dokumentation ist in allgemeiner Form darzustellen, welche der personenbezogenen Daten, die ursprünglich für andere Zwecke erhoben wurden, der Mietspiegelersteller von öffentlichen und nichtöffentlichen Stellen erhalten hat und wozu diese Daten
benötigt und verwendet wurden.
(4) Weitere Anforderungen an die Dokumentation ergeben sich aus § 8 Absatz 4, § 9
Absatz 4, § 10 Absatz 2, § 11 Absatz 4, § 12 Absatz 3, § 13 Absatz 3, § 14 Absatz 2, § 15
Absatz 5, § 16 Absatz 4, § 19 Absatz 5 und § 23 Absatz 3.
- 13 -
§ 21
Veröffentlichung
(1) Der qualifizierte Mietspiegel und seine Dokumentation sollen kostenfrei im Internet
veröffentlicht werden. Für ihre Abgabe in gedruckter Form können angemessene Entgelte
verlangt werden.
(2) Die Veröffentlichung des qualifizierten Mietspiegels soll binnen einer Frist von
neun Monaten nach dem Stichtag, auf den sich die Erhebung bezieht, erfolgen.
(3) Die Dokumentation soll zeitgleich mit der Veröffentlichung des qualifizierten Mietspiegels veröffentlicht werden.
Unterabschnitt 4
Anpassung des qualifizierten Mietspiegels
§ 22
Anpassung mittels Index
Erfolgt die Anpassung des qualifizierten Mietspiegels unter Zugrundelegung der Entwicklung des vom Statistischen Bundesamt oder vom zuständigen Statistischen Landesamt
veröffentlichten Indexes für die Nettokaltmiete, so gelten die §§ 20 und 21 entsprechend.
§ 23
Anpassung mittels Stichprobe
(1) Bei der Anpassung eines qualifizierten Mietspiegels mittels Stichprobe können vereinfachende, mit der Fortschreibung auf der Grundlage eines Indexes vergleichbare Annahmen getroffen werden.
(2) Die §§ 7 bis 21 sind auf die Anpassung mittels Stichprobe entsprechend anwendbar. Der Umfang der bereinigten Nettostichprobe kann von den in § 11 bezeichneten Werten abweichen, sofern nach Absatz 1 getroffene, vereinfachende Annahmen dies zulassen.
(3) Vereinfachende Annahmen nach Absatz 1 sowie ein von den Werten des § 11 abweichender Stichprobenumfang sind in der Dokumentation darzulegen und zu begründen.
- 14 -
A b s c h n i t t 4
S c h l u s s v o r s c h r i f t e n
§ 24
Übergangsvorschrift
Diese Verordnung ist auf Mietspiegel anzuwenden, die ein Jahr nach dem … [einsetzen: Datum der Verkündung des Gesetzes] veröffentlicht werden.
§ 25
Inkrafttreten
Diese Verordnung tritt am Tag nach der Verkündung in Kraft.
Der Bundesrat hat zugestimmt.
    
\chapter{Links}
\minitoc
    \section{Rechtsgrundlagen}
        \begin{description}
            \item[\qrcode{https://openjur.de/u/268440.html}] Volkszählungsurteil 1987
            \item[\qrcode{https://eur-lex.europa.eu/legal-content/DE/TXT/PDF/?uri=CELEX:32015R0759}] EU-Statistikverordnung
            \item[\qrcode{https://www.gesetze-im-internet.de/bstatg_1987/BStatG.pdf}] Bundesstatistikgesetz - BStatG
            \item[\qrcode{https://www.destatis.de/DE/Methoden/Rechtsgrundlagen/_inhalt.html}] Rechtsgrundlagen-Sammlung bei Destatis
            \item[\qrcode{https://eur-lex.europa.eu/legal-content/DE/TXT/PDF/?uri=CELEX:32016R0679&from=DE}] EU-DSGVO
            \item[\qrcode{https://www.gesetze-im-internet.de/bdsg_2018/BDSG.pdf}] Bundesdatenschutzgesetz - BDSG
            
            
        \end{description}
    \section{Weiterführende Links}
        \begin{description}
            \item[\qrcode{https://www.staedtestatistik.de/}] Homepage VDST/KOSIS
            \item[\qrcode{https://statistikhessen-blog.de/category/zensus/}] Blog zum Zensus 2021
            \item[\qrcode{https://www.ted.com/talks/hans_and_ola_rosling_how_not_to_be_ignorant_about_the_world}] Hans Rosling: "How not to be ignorant about the world"
        \end{description}











\end{document}
