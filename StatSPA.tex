\chapter{Statistiksatzung der Stadt Passau}{Satzung über die Kommunalstatistik der Stadt Passau}
\minitoc
  Die Stadt Passau erlässt aufgrund des Art. 23 der Gemeindeordnung für den Freistaat Bayern (Ge-meindeordnung - GO) in der Fassung der Bekanntmachung vom 22. August 1998 (GVBl S. 796, BayRS 2020-1-1-I), zuletzt geändert durch \S 10 des Gesetzes vom 27. Juli 2009 (GVBl S. 400), und der Art. 22, 23 und 24 des Bayerischen Statistikgesetzes (BayStatG) vom 10. August 1990 (GVBl S. 270, BayRS 290-1-I), zuletzt geändert durch \S 14 des Gesetzes vom 24.12.2002 (GVBl. S. 962), folgende Satzung:
  \section{\S1 Geltungsbereich}
    \begin{enumerate}[label=(\arabic*)]
      \item Diese Satzung gilt für Kommunalstatistiken der Stadt Passau. Für Auftragsstatistiken gilt sie nur, soweit dies ausdrücklich bestimmt ist. Die statistische Aufbereitung von Daten, die bei städtischen Dienststellen im Vollzug ihrer Aufgaben erhoben werden oder auf sonstige Weise anfallen und nicht die ausschließliche Durchführung von Statistiken betreffen (Geschäftsstatis-tiken), bleibt unberührt. 
      \item Die Verarbeitung von Daten, die nicht dem Datenschutz oder der statistischen Geheimhaltung unterliegen, ist von den Bestimmungen dieser Satzung ebenfalls ausgenommen. 
    \end{enumerate}

  \section{\S2 Kommunalstatistik der Stadt Passau}
    \begin{enumerate}[label=(\arabic*)]
      \item Die Stadt Passau betreibt - soweit Einzelangaben oder Ergebnisse vom Bayerischen Landesamt für Statistik und Datenverarbeitung oder von anderen öffentlichen Stellen weder zur Verfügung gestellt noch anderweitig ermittelt werden können - eine Kommunalstatistik und bestimmt eine gem. Art. 20 Abs. 2 Satz 1 BayStatG für die Leitung verantwortliche Person.
      \item Im Rahmen der Kommunalstatistik nach Maßgabe dieser Satzung dürfen bei der Stadt Passau gesetzlich geschützte Daten aus unterschiedlichen Quellen und für nicht abschließend be-stimmte statistische Auswertungszwecke erhoben und verarbeitet werden.
    \end{enumerate}
 
 \section{\S3 Aufgaben der Kommunalstatistik}
 Die Statistikstelle hat insbesondere folgende Aufgaben:
  \begin{enumerate}[label=\arabic*.]
    \item Vorbereitung und Durchführung statistischer Erhebungen aufgrund Bundes- oder Lan-desgesetze sowie freiwilliger kommunalstatistischer Erhebungen und Umfragen, Gewin-nung statistischer Daten aus Verwaltungstätigkeiten, aus Quellen der Landes- und Bun-desstatistiken und aus sonstigen Quellen, 
    \item Aufbau, Pflege und Betreuung der städtischen Datensammlungen zur statistischen Infor-mation in Form von Einzel- und Aggregatdaten aus unterschiedlichen Quellen und für nicht abschließend bestimmte statistische Auswertungszwecke, 
    \item Aufbau, Pflege und Betreuung der Instrumente zur Gewinnung und Darstellung statisti-scher Informationen,
    \item Aufbau, Pflege und Betreuung eines kleinräumig gegliederten Raumbezugssystems sowie der sich daraus ergebenden Schlüsselsysteme,
    \item Datenaufbereitung, Durchführung statistischer Analysen, Prognosen und Mo\-dell\-rech\-nun\-gen (Stadtforschung), Erstellung statistischer Gutachten, 
    \item Erhebung, Aufbereitung und Analyse der Grundlagen, 
    \item Aufgaben der örtlichen Erhebungs- und Berichtsstelle für Volkszählungen, Bundes- und Landesstatistiken, soweit durch Bundes- und Landesrecht nichts anderes bestimmt ist,
    \item Wahrnehmung der Verbindung zum statistischen Bundesamt sowie zu den statistischen Landesämtern, Mitwirkung in den einschlägigen Facharbeitskreisen und im Verband Deutscher Städtestatistiker (VDSt).
  \end{enumerate} 
  \section{\S4 Geheimhaltung}
    Einzelangaben über persönliche und sachliche Verhältnisse, die für die Kommunalstatistik der Stadt Passau gemacht oder zu diesem Zweck an die Statistikstelle übermittelt werden, sind von den Amtsträgern und für den öffentlichen Dienst besonders Verpflichteten, die mit der Durchführung einer solchen Statistik betraut sind, geheim zu halten, soweit durch besondere Rechtsvorschrift nichts anderes bestimmt ist. Die Regelungen von Art. 17 BayStatG bleiben unberührt.
    
  \section{\S5 Abschottung}
    \begin{enumerate}[label=(\arabic*)]
      \item Die Statistikstelle ist räumlich, organisatorisch und personell von anderen Verwaltungsstellen getrennt zu führen. Die Räume, in denen geschützte Einzeldaten verwahrt oder bearbeitet wer-den, sind gegen Zutritt Unbefugter bestmöglich zu sichern. Die Räume der Statistikstelle dür-fen nur von deren Mitarbeitern und den zuständigen Datenschutzbeauftragten betreten werden. Sollte der Zutritt weiterer Personen notwendig sein (z. B. IT-Firmen-Personal, Reini-gungspersonal u. ä.), so sind diese vor Betreten ausdrücklich auf ihre Geheimhaltungspflichten hinzuweisen. 
      \item Die in der Statistikstelle tätigen Personen dürfen nicht gleichzeitig bei anderen Dienststellen der Stadtverwaltung eingesetzt werden und müssen die Gewähr für Zuverlässigkeit und Ver-schwiegenheit bieten. Sie sind auf die Wahrung des Datengeheimnisses nach Art. 5 des Baye-rischen Datenschutzgesetz - BayDSG und des Statistikgeheimnisses nach \S 4 dieser Satzung schriftlich zu verpflichten. Sie sind zur Einhaltung dieser Verpflichtungen auch gegenüber den Dienstvorgesetzten verpflichtet. Die dienst- und arbeitsrechtlichen Befugnisse des Dienstvor-gesetzten bleiben unberührt. 
      \item Zur Erfüllung ihrer Aufgaben bedient sich die Statistikstelle der zentralen Datenverarbeitung. Dabei müssen die Einhaltung der Vorschriften des BayDSG, des Statistikgeheimnisses und der Vorgaben dieser Satzung gewährleistet sein. 
    \end{enumerate}
    
    \section{\S6 Inkrafttreten}
    Diese Satzung tritt eine Woche nach ihrer Bekanntmachung im Amtsblatt der Stadt Passau in Kraft.