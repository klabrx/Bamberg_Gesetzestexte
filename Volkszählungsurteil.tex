\chapter{Volkszählungsurteil 1983}
\qrcode{https://openjur.de/u/268440.html}
\newline
\url{https://openjur.de/u/268440.html}

    \section{Tenor}
        \begin{enumerate}
            \item Unter den Bedingungen der modernen Datenverarbeitung wird der Schutz des Einzelnen gegen unbegrenzte Erhebung, Speicherung, Verwendung und Weitergabe seiner persönlichen Daten von dem allgemeinen Persönlichkeitsrecht des Art. 2 Abs. 1 GG in Verbindung mit Art. 1 Abs. 1 GG umfaßt. Das Grundrecht gewährleistet insoweit die Befugnis des Einzelnen, grundsätzlich selbst über die Preisgabe und Verwendung seiner persönlichen Daten zu bestimmen.
            \item Einschränkungen dieses Rechts auf "informationelle Selbstbestimmung" sind nur im überwiegenden Allgemeininteresse zulässig. Sie bedürfen einer verfassungsgemäßen gesetzlichen Grundlage, die dem rechtsstaatlichen Gebot der Normenklarheit entsprechen muß. Bei seinen Regelungen hat der Gesetzgeber ferner den Grundsatz der Verhältnismäßigkeit zu beachten. Auch hat er organisatorische und verfahrensrechtliche Vorkehrungen zu treffen, welche der Gefahr einer Verletzung des Persönlichkeitsrechts entgegenwirken.     \item Bei den verfassungsrechtlichen Anforderungen an derartige Einschränkungen ist zu unterscheiden zwischen personenbezogenen Daten, die in individualisierter, nicht anonymer Form erhoben und verarbeitet werden, und solchen, die für statistische Zwecke bestimmt sind. Bei der Datenerhebung für statistische Zwecke kann eine enge und konkrete Zweckbindung der Daten nicht verlangt werden. Der Informationserhebung und Informationsverarbeitung müssen aber innerhalb des Informationssystems zum Ausgleich entsprechende Schranken gegenüberstehen.
            \item Das Erhebungsprogramm des Volkszählungsgesetzes 1983 (§ 2 Nr. 1 bis 7, §§ 3 bis 5) führt nicht zu einer mit der Würde des Menschen unvereinbaren Registrierung und Katalogisierung der Persönlichkeit; es entspricht auch den Geboten der Normenklarheit und der Verhältnismäßigkeit. Indessen bedarf es zur Sicherung des Rechts auf informationelle Selbstbestimmung ergänzender verfahrensrechtlicher Vorkehrungen für Durchführung und Organisation der Datenerhebung. 
            \item Die in § 9 Abs. 1 bis 3 VoZählG 1983 vorgesehenen Übermittlungsregelungen (unter anderem Melderegisterabgleich) verstoßen gegen das allgemeine Persönlichkeits\-recht. Die Weitergabe zu wissenschaftlichen Zwecken (§ 9 Abs. 4 VoZählG 1983) ist mit dem Grundgesetz vereinbar. 
        \end{enumerate}
    \section{Randnummer 99}
    \begin{enumerate}[label=\arabic*,start=99]
        \item Wieweit Informationen sensibel sind, kann hiernach nicht allein davon abhängen, ob sie intime Vorgänge betreffen. Vielmehr bedarf es zur Feststellung der per\-sön\-lich\-keits\-recht\-lichen Bedeutung eines Datums der Kenntnis seines Verwendungszusammenhangs: Erst wenn Klarheit darüber besteht, zu welchem Zweck Angaben verlangt werden und welche Verknüpfungsmöglichkeiten und Ver\-wend\-ungs\-mög\-lich\-kei\-ten bestehen, läßt sich die Frage einer zulässigen Beschränkung des Rechts auf informationelle Selbstbestimmung beantworten. Dabei ist zu unterscheiden zwischen personenbezogenen Daten, die in individualisierter, nicht anonymisierter Form erhoben und verarbeitet werden (dazu unter a), und solchen, die für statistische Zwecke bestimmt sind (dazu unter b).
    \end{enumerate}
        
    \section{Randnummer 104-112}
        \begin{enumerate}[label=\arabic*,start=104]
            \item b) Die Erhebung und Verarbeitung von Daten für statistische Zwecke weisen Besonderheiten auf, die bei der verfassungsrechtlichen Beurteilung nicht außer acht bleiben können.
            \item aa) Die Statistik hat erhebliche Bedeutung für eine staatliche Politik, die den Prinzipien und Richtlinien des Grundgesetzes verpflichtet ist. Wenn die ökonomische und soziale Entwicklung nicht als unabänderliches Schicksal hingenommen, sondern als permanente Aufgabe verstanden werden soll, bedarf es einer umfassenden, kontinuierlichen sowie laufend aktualisierten Information über die wirtschaftlichen, ökologischen und sozialen Zusammenhänge. Erst die Kenntnis der relevanten Daten und die Möglichkeit, die durch sie vermittelten Informationen mit Hilfe der Chancen, die eine automatische Datenverarbeitung bietet, für die Statistik zu nutzen, schafft die für eine am Sozialstaatsprinzip orientierte staatliche Politik unentbehrliche Handlungsgrundlage (vgl BVerfGE 27, 1 [9]).
            \item Bei der Datenerhebung für statistische Zwecke kann eine enge und konkrete Zweckbindung der Daten nicht verlangt werden. es gehört zum Wesen der Statistik, daß die Daten nach ihrer statistischen Aufbereitung für die verschiedensten, nicht von vornherein bestimmbaren Aufgaben verwendet werden sollen; demgemäß besteht auch ein Bedürfnis nach Vorratsspeicherung. Das Gebot einer konkreten Zweckumschreibung und das strikte Verbot der Sammlung personenbezogener Daten auf Vorrat kann nur für Datenerhebungen zu nichtstatistischen Zwecken gelten, nicht jedoch bei einer Volkszählung, die eine gesicherte Datenbasis für weitere statistische Untersuchungen ebenso wie für den politischen Planungsprozeß durch eine verläßliche Feststellung der Zahl und der Sozialstruktur der Bevölkerung vermitteln soll. Die Volkszählung muß Mehrzweckerhebung und Mehrzweckverarbeitung, also Datensammlung und Datenspeicherung auf Vorrat sein, wenn der Staat den Entwicklungen der industriellen Gesellschaft nicht unvorbereitet begegnen soll. Auch wären Weitergabeverbote und Verwertungsverbote für statistisch aufbereitete Daten zweckwidrig.
            \item bb) Ist die Vielfalt der Verwendungsmöglichkeiten und Verknüpfungsmöglichkeiten damit bei der Statistik von der Natur der Sache her nicht im voraus bestimmbar, müssen der Informationserhebung und Informationsverarbeitung innerhalb des Informationssystems zum Ausgleich entsprechende Schranken gegenüberstehen. Es müssen klar definierte Verarbeitungsvoraussetzungen geschaffen werden, die sicherstellen, daß der Einzelne unter den Bedingungen einer automatischen Erhebung und Verarbeitung der seine Person betreffenden Angaben nicht zum bloßen Informationsobjekt wird. Beides, die mangelnde Anbindung an einen bestimmten, jederzeit erkennbaren und nachvollziehbaren Zweck sowie die multifunktionale Verwendung der Daten, verstärkt die Tendenzen, welche durch die Datenschutzgesetze aufgefangen und eingeschränkt werden sollen, die das verfassungsrechtlich gewährleistete Recht auf informationelle Selbstbestimmung konkretisieren. Gerade weil es von vornherein an zweckorientierten Schranken fehlt, die den Datensatz eingrenzen, bringen Volkszählungen tendenziell die schon im Mikrozensus-Beschluß (BVerfGE 27, 1 [6]) hervorgehobene Gefahr einer persönlichkeitsfeindlichen Registrierung und Katalogisierung des Einzelnen mit sich. Deshalb sind an die Datenerhebung und Datenverarbeitung für statistische Zwecke besondere Anforderungen zum Schutz des Persönlichkeitsrechts der auskunftspflichtigen Bürger zu stellen.
            \item Unbeschadet des multifunktionalen Charakters der Datenerhebung und Datenverarbeitung zu statistischen Zwecken ist Voraussetzung, daß diese allein als Hilfe zur Erfüllung öffentlicher Aufgaben erfolgen. Es kann auch hier nicht jede Angabe verlangt werden. Selbst bei der Erhebung von Einzelangaben, die für statistische Zwecke gebraucht werden, muß der Gesetzgeber schon bei der Anordnung der Auskunftspflicht prüfen, ob sie insbesondere für den Betroffenen die Gefahr der sozialen Abstempelung (etwa als Drogensüchtiger, Vorbestrafter, Geisteskranker, Asozialer) hervorrufen können und ob das Ziel der Erhebung nicht auch durch eine anonymisierte Ermittlung erreicht werden kann. Dies dürfte beispielsweise bei dem in § 2 Nr 8 VZG 1983 geregelten Erhebungstatbestand der Fall sein, wonach die Volkszählung und Berufszählung im Anstaltsbereich die Eigenschaft als Insasse oder die Zugehörigkeit zum Personal oder zum Kreis der Angehörigen des Personals erfaßt. Diese Erhebung soll Anhaltspunkte über die Belegung der Anstalten liefern (BTDrucks 9/451, S. 9). Ein solches Ziel ist - abgesehen von der Gefahr sozialer Etikettierung - auch ohne Personenbezug zu erreichen. Es genügt, daß der Leiter der Anstalt verpflichtet wird, zum Stichtag der Volkszählung die zahlenmäßige Belegung nach den in § 2 Nr 8 VZG 1983 aufgeführten Merkmalen ohne jeden Bezug auf die einzelne Person mitzuteilen. Eine personenbezogene Erhebung des Tatbestandes des § 2 Nr 8 VZG 1983 wäre deshalb von vornherein ein Verstoß gegen das durch Art 2 Abs. 1 in Verbindung mit Art 1 Abs. 1 GG geschützte Persönlichkeitsrecht.
            \item Zur Sicherung des Rechts auf informationelle Selbstbestimmung bedarf es ferner besonderer Vorkehrungen für Durchführung und Organisation der Datenerhebung und Datenverarbeitung, da die Informationen während der Phase der Erhebung - und zum Teil auch während der Speicherung - noch individualisierbar sind; zugleich sind Löschungsregelungen für solche Angaben erforderlich, die als Hilfsangaben (Identifikationsmerkmale) verlangt wurden und die eine Deanonymisierung leicht ermöglichen würden, wie Name, Anschrift, Kennummer und Zählerliste (vgl auch § 11 Abs. 7 Satz 1 BStatG). Von besonderer Bedeutung für statistische Erhebungen sind wirksame Abschottungsregelungen nach außen. Für den Schutz des Rechts auf informationelle Selbstbestimmung ist - und zwar auch schon für das Erhebungsverfahren - die strikte Geheimhaltung der zu statistischen Zwecken erhobenen Einzelangaben unverzichtbar, solange ein Personenbezug noch besteht oder herstellbar ist (Statistikgeheimnis); das gleiche gilt für das Gebot einer möglichst frühzeitigen (faktischen) Anonymisierung, verbunden mit Vorkehrungen gegen eine Deanonymisierung.
            \item Erst die vom Recht auf informationelle Selbstbestimmung geforderte und gesetzlich abzusichernde Abschottung der Statistik durch Anonymisierung der Daten und deren Geheimhaltung, soweit sie zeitlich begrenzt noch einen Personenbezug aufweisen, öffnet den Zugang der staatlichen Organe zu den für die Planungsaufgaben erforderlichen Informationen. Nur unter dieser Voraussetzung kann und darf vom Bürger erwartet werden, die von ihm zwangsweise verlangten Auskünfte zu erteilen. Dürften personenbezogene Daten, die zu statistischen Zwecken erhoben wurden, gegen den Willen oder ohne Kenntnis des Betroffenen weitergeleitet werden, so würde das nicht nur das verfassungsrechtlich gesicherte Recht auf informationelle Selbstbestimmung unzulässig einschränken, sondern auch die vom Grundgesetz selbst in Art 73 Nr 11 vorgesehene und damit schutzwürdige amtliche Statistik gefährden. Für die Funktionsfähigkeit der amtlichen Statistik ist ein möglichst hoher Grad an Genauigkeit und Wahrheitsgehalt der erhobenen Daten notwendig. Dieses Ziel kann nur erreicht werden, wenn bei dem auskunftspflichtigen Bürger das notwendige Vertrauen in die Abschottung seiner für statistische Zwecke erhobenen Daten geschaffen wird, ohne welche seine Bereitschaft, wahrheitsgemäße Angaben zu machen, nicht herzustellen ist (so bereits zutreffend die Begründung der Bundesregierung zum Entwurf des Volkszählungsgesetzes 1950; vgl BTDrucks I/982, S. 20 zu § 10). Eine Staatspraxis, die sich nicht um die Bildung eines solchen Vertrauens durch Offenlegung des Datenverarbeitungsprozesses und strikte Abschottung bemühte, würde auf längere Sicht zu schwindender Kooperationsbereitschaft führen, weil Mißtrauen entstünde. Da staatlicher Zwang nur begrenzt wirksam werden kann, wird ein die Interessen der Bürger überspielendes staatliches Handeln allenfalls kurzfristig vorteilhaft erscheinen; auf Dauer gesehen wird es zu einer Verringerung des Umfangs und der Genauigkeit der Informationen führen (BTDrucks I/982, a.a.O.). Läßt sich die hochindustrialisierte Gesellschaften kennzeichnende ständige Zunahme an Komplexität der Umwelt nur mit Hilfe einer zuverlässigen Statistik aufschlüsseln und für gezielte staatliche Maßnahmen aufbereiten, so läuft die Gefährdung der amtlichen Statistik darauf hinaus, eine wichtige Voraussetzung sozialstaatlicher Politik in Frage zu stellen. Kann damit nur durch eine Abschottung der Statistik die Staatsaufgabe "Planung" gewährleistet werden, ist das Prinzip der Geheimhaltung und möglichst frühzeitigen Anonymisierung der Daten nicht nur zum Schutz des Rechts auf informationelle Selbstbestimmung des Einzelnen vom Grundgesetz gefordert, sondern auch für die Statistik selbst konstitutiv.
            \item cc) Wird den erörterten Anforderungen in wirksamer Weise Rechnung getragen, ist die Erhebung von Daten zu ausschließlich statistischen Zwecken nach dem derzeitigen Erkenntnisstand und Erfahrungsstand verfassungsrechtlich unbedenklich. Es ist nicht erkennbar, daß das Persönlichkeitsrecht der Bürger beeinträchtigt werden könnte, wenn die erhobenen Daten nach ihrer Anonymisierung oder statistischen Aufbereitung (vgl § 11 Abs. 5 und 6 BStatG) von Statistischen Ämtern anderen staatlichen Organen oder sonstigen Stellen zur Verfügung gestellt werden.
            \item Besondere Probleme wirft eine etwaige Übermittlung (Weitergabe) der weder anonymisierten noch statistisch aufbereiteten, also noch personenbezogenen Daten auf. Erhebungen zu statistischen Zwecken umfassen auch individualisierte Angaben über den einzelnen Bürger, die für die statistischen Zwecke nicht erforderlich sind und die - davon muß der befragte Bürger ausgehen können - lediglich als Hilfsmittel für das Erhebungsverfahren dienen. Alle diese Angaben dürfen zwar kraft ausdrücklicher gesetzlicher Ermächtigung weitergeleitet werden, soweit und sofern dies zur statistischen Aufbereitung durch andere Behörden geschieht und dabei die zum Schutz des Persönlichkeitsrechts gebotenen Vorkehrungen, insbesondere das Statistikgeheimnis und das Gebot der frühzeitigen Anonymisierung, ebenso durch Organisation und Verfahren zuverlässig sichergestellt sind wie bei den Statistischen Ämtern des Bundes und der Länder. Eine Weitergabe der für statistische Zwecke erhobenen, nicht anonymisierten oder statistisch aufbereiteten Daten für Zwecke des Verwaltungsvollzugs kann hingegen in unzulässiger Weise in das Recht auf informationelle Selbstbestimmung eingreifen (vgl ferner unten C IV 1).
        \end{enumerate}
    \section{Randnummer 145}
        \begin{enumerate}[label=\arabic*,start=145]
            \item Die zu statistischen Zwecken erhobenen, noch nicht anonymisierten, also noch personenbezogenen Daten dürfen - wie bereits ausgeführt (oben C II 2 cc) - kraft ausdrücklicher gesetzlicher Ermächtigung weitergeleitet werden, soweit und sofern dies zur statistischen Aufbereitung durch andere Behörden erfolgt und wenn dabei die zum Schutz des Persönlichkeitsrechts gebotenen Vorkehrungen, insbesondere das Statistikgeheimnis und das Gebot der Anonymisierung, in gleicher Weise zuverlässig sichergestellt sind wie bei den Statistischen Ämtern des Bundes und der Länder. Würden hingegen personenbezogene, nicht anonymisierte Daten, die zu statistischen Zwecken erhoben wurden und nach der gesetzlichen Regelung dafür bestimmt sind, für Zwecke des Verwaltungsvollzuges weitergegeben (Zweckentfremdung), würde in unzulässiger Weise in das Recht auf informationelle Selbstbestimmung eingegriffen. Es kann offenbleiben, ob eine direkte Weiterleitung dieser Daten generell und selbst dann als unvereinbar mit dem Grundsatz der Trennung von Statistik und Vollzug zu beanstanden wäre, wenn der Gesetzgeber diese Weiterleitung ausdrücklich vorsähe. Es bedarf auch keiner abschließenden Erörterung, ob die gleichzeitige Durchführung einer an sich statthaften Erhebung personenbezogener Daten für statistische Zwecke mit einer an sich statthaften Erhebung personenbezogener Daten für bestimmte Zwecke des Verwaltungsvollzugs auf verschiedenen Bögen (kombinierte Erhebung) zulässig wäre. Sowohl die direkte Übermittlung von zu statistischen Zwecken erhobenen Daten als auch die kombinierte Erhebung wären schon deshalb nicht bedenkenfrei, weil die Verknüpfung zweier unterschiedlicher Zwecke mit unterschiedlichen Anforderungen den Bürger angesichts der für ihn undurchsichtigen Möglichkeiten der automatischen Datenverarbeitung in hohem Maße verunsichert und dadurch die Zuverlässigkeit der Angaben und deren Eignung für statistische Zwecke gefährden kann. Ferner wären die unterschiedlichen Voraussetzungen zu beachten: So gelten für die Erhebung und Verwertung zu statistischen Zwecken das Statistikgeheimnis, das Gebot der Anonymisierung und das Nachteilsverbot; für die Erhebung zu Verwaltungsvollzugszwecken ist dies hingegen nicht oder nicht in gleicher Weise der Fall; während für die Statistik Identifikationsmerkmale (etwa Name und Anschrift) nur als Hilfsmittel dienen, sind sie in aller Regel für die Erhebung zu Verwaltungsvollzugszwecken wesentlicher Bestandteil. Zudem wird dabei die auf statistische Datensammlung zugeschnittene Ermittlungsorganisation zugleich für andere Erhebungszwecke eingesetzt, die für sich allein eine solche Organisation schwerlich rechtfertigen würden. Auch wäre zu beachten, daß das Rechtsschutzverfahren bei den beiden Erhebungsarten auseinanderlaufen kann.
        \end{enumerate}


















        
