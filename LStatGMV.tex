\chapter[LStatGMV]{Landesstatistikgesetz Mecklenburg-Vorpommern}
\minitoc
    \section{\S1 Geltungsbereich}
    Dieses Gesetz gilt
    \begin{enumerate}[label=\arabic*.]
        \item ergänzend zum Bundesstatistikgesetz (BStatG) für die Durchführung von
            \begin{enumerate}[label=\alph*)]
                \item Statistiken aufgrund von unmittelbar geltenden Rechtsakten der Europäischen Union (EU-Statistiken) und
                \item Statistiken aufgrund von Rechtsvorschriften des Bundes (Bundesstatistiken),
            \end{enumerate}
        \item für die Durchführung von
            \begin{enumerate}[label=\alph*)]
                \item Landestatistiken und
                \item Kommunalstatistiken,
            \end{enumerate}
        \item für Statistiken, bei denen Daten verwendet werden, die im Geschäftsgang der Behörden und Gerichte des Landes sowie der der Aufsicht des Landes unterstehenden juristischen Personen des öffentlichen Rechts anfallen, die bei diesen oder den übergeordneten Behörden oder Stellen geführt werden (Geschäftsstatistiken), sowie
        \item für die statistische Aufbereitung und Auswertung von Daten aus dem Verwaltungsvollzug.
    \end{enumerate}


    \section[Grundsätze]{\S2 Grundsätze der Landesstatistik und Kommunalstatistik}
        \begin{enumerate}[label=(\arabic*)]
            \item Die amtliche Statistik des Landes (Landes- und Kommunalstatistik) hat im föderativ gegliederten Gesamtsystem der amtlichen Statistik die Aufgabe, für den Informationsbedarf von Bund, Ländern, Landkreisen, Gemeinden, Gesellschaft, Wirtschaft, Wissenschaft und Forschung Daten über Massenerscheinungen zu erheben, zu sammeln, aufzubereiten, darzustellen und zu analysieren. Sie gewinnt die Daten unter Verwendung wissenschaftlicher Erkenntnisse und unter Einsatz der jeweils sachgerechten Methoden und Informationstechniken. Für die amtliche Statistik gelten die Grundsätze der Neutralität, Objektivität, wissenschaftlichen Unabhängigkeit und der statistischen Geheimhaltung.
            \item Landesstatistiken und Kommunalstatistiken sollen nur angeordnet werden, wenn die benötigten Informationen nicht auf andere Art beschafft werden können. Sie sind auf das notwendige Maß zu beschränken. Erhobene Einzelangaben dienen ausschließlich den durch dieses Gesetz oder durch eine andere eine Landes- oder Kommunalstatistik anordnende Rechtsvorschrift festgelegten Zwecken.
        \end{enumerate}

    \section{\S3 Statistisches Amt}
        \begin{enumerate}[label=(\arabic*)]
            \item Die Aufgaben der amtlichen Statistik in Mecklenburg-Vorpommern werden vom Statistischen Amt wahrgenommen. Es ist als besondere Organisationseinheit in das Landesamt für innere Verwaltung eingegliedert. Das Statistische Amt ist organisatorisch und räumlich von den anderen Verwaltungsstellen des Landesamtes für innere Verwaltung und der sonstigen Landesverwaltung abzugrenzen, gegen den Zutritt unbefugter Personen ausreichend zu sichern und mit gesondertem Personal auszustatten. Das Weisungsrecht gegenüber dem Statistischen Amt erstreckt sich nicht auf die Weitergabe von Einzeldaten, die der statistischen Geheimhaltung unterliegen.
            \item Das Statistische Amt ist zuständige Behörde für die Durchführung von Bundes- und Landesstatistiken sowie für statistische Erhebungen aufgrund unmittelbar geltender Rechtsakte der Europäischen Union. Die Aufgabe des Statistischen Amtes ist es,
            \begin{enumerate}[label=\arabic*.]
                \item EU-, Bundes- und Landesstatistiken zu erheben und aufzubereiten, soweit in diesem Gesetz oder in einer sonstigen Rechtsvorschrift nichts anderes bestimmt ist, und statistische Ergebnisse zusammenzustellen, auszuwerten, darzustellen und zu veröffentlichen,
                \item Landesstatistiken methodisch und technisch vorzubereiten und weiterzuentwickeln sowie bei der Vorbereitung und Weiterentwicklung von EU- und Bundesstatistiken mitzuwirken,
                \item Volkswirtschaftliche und Umweltökonomische Gesamtrechnungen sowie andere Gesamtsysteme statistischer Daten für Bundes- und Landeszwecke darzustellen und zu veröffentlichen,
                \item das statistische Informationssystem des Landes einzurichten, zu betreiben und inhaltlich und technisch weiterzuentwickeln sowie an der Koordinierung von speziellen Informationssystemen anderer Stellen des Landes mitzuwirken,
                \item wissenschaftliche Analysen, Prognosen und Modellrechnungen auf der Grundlage statistischer Daten vorzunehmen,
                \item auf Anforderung insbesondere der Kommission der Europäischen Union, oberster Bundesbehörden oder oberster Landesbehörden Forschungsaufträge auszuführen, Gutachten zu erstellen und sonstige Arbeiten statistischer Art durchzuführen,
                \item die Behörden und Gerichte des Landes, die Landkreise, kreisfreien Städte, Ämter und amtsfreien Gemeinden sowie die sonstigen der Aufsicht des Landes unterstehenden juristischen Personen des öffentlichen Rechts in statistischen Angelegenheiten zu beraten und zu unterstützen,
                \item an der Vorbereitung von Rechts- und Verwaltungsvorschriften mitzuwirken, die die Bundes- und Landesstatistik betreffen,
                \item bei der Durchführung von allgemeinen Wahlen und Volksabstimmungen mitzuwirken,
                \item sonstige durch Rechtsvorschrift oder durch die fachlich zuständige oberste Landesbehörde im Einvernehmen mit dem Innenminister übertragene Aufgaben wahrzunehmen.
            \end{enumerate}
            \item Die im Statistischen Amt tätigen Personen dürfen die aus oder gelegentlich ihrer Tätigkeit gewonnenen Erkenntnisse mit Personenbezug auch nach ihrem Ausscheiden aus dieser Stelle nicht in anderen Verfahren oder für andere Zwecke verwenden. Sie sind vor ihrem Einsatz auf die Wahrung der statistischen Geheimhaltung, zur Beachtung der gesetzlichen Gebote und Verbote zur Sicherung des Datenschutzes schriftlich zu verpflichten und über die Folgen ihrer Verletzung zu belehren. Sie dürfen während der Tätigkeit im Statistischen Amt nicht mit anderen Aufgaben des Verwaltungsvollzuges betraut werden.
            \item Das Ministerium für Inneres und Europa legt die zur Durchführung der Absätze 1 und 3 erforderlichen Maßnahmen in einer schriftlichen Dienstanweisung fest.
        \end{enumerate}

    \section[Zusammenarbeit]{\S4 Zusammenarbeit der statistischen Ämter}
        \begin{enumerate}[label=(\arabic*)]
            \item Das Statistische Amt darf, soweit es für die Durchführung von Landesstatistiken und für sonstige Arbeiten statistischer Art im Rahmen der Landesstatistik zuständig ist, die Ausführung einzelner Arbeiten oder hierzu erforderlicher Hilfsmaßnahmen durch Verwaltungsvereinbarung oder aufgrund einer Verwaltungsvereinbarung auf andere statistische Ämter übertragen. Davon ausgenommen sind die Heranziehung zur Auskunftserteilung und die Durchsetzung der Auskunftspflicht.
            \item Zu den statistischen Arbeiten nach Absatz 1 gehört auch die Bereitstellung von Daten für die Wissenschaft.
        \end{enumerate}
 
    \section{\S5 Landesstatistiken}
        \begin{enumerate}[label=(\arabic*)]
            \item Landesstatistiken werden, soweit in diesem Gesetz oder in einem anderen Landesgesetz nichts anderes bestimmt ist, durch Gesetz angeordnet.
            \item Die Landesregierung wird ermächtigt, Landesstatistiken mit Auskunftspflicht für die Dauer von bis zu drei Jahren durch Rechtsverordnung anzuordnen, wenn die Ergebnisse der Statistik für Zwecke der Planung oder zur Vorbereitung einer Entscheidung erforderlich sind und die Erhebung nur einen begrenzten Befragtenkreis betrifft.
            \item Landesstatistiken, die auf freiwilliger Grundlage durchgeführt werden, bedürfen keiner Anordnung durch Rechtsvorschrift. Das gleiche gilt für Landesstatistiken, bei denen ausschließlich Angaben aus allgemein zugänglichen Quellen oder aus öffentlichen Registern, zu denen dem Statistischen Amt in einer Rechtsvorschrift ein besonderes Zugangsrecht gewährt wird, verwendet werden. Landesstatistiken nach Satz 1 werden durch Verwaltungsvorschriften der Landesregierung oder der fachlich zuständigen obersten Landesbehörde im Einvernehmen mit dem Ministerium für Inneres und Europa angeordnet; die Finanzierung muß gesichert sein.
            \item Die eine Landesstatistik anordnende Rechts- oder Verwaltungsvorschrift muß die Erhebungsmerkmale, die Hilfsmerkmale, die Art und Weise der Erhebung, den Berichtszeitraum, den Berichtszeitpunkt, die Periodizität und den Kreis der zu Befragenden bestimmen. Ferner ist festzulegen, ob und in welchem Umfang die Erhebung mit oder ohne Auskunftspflicht erfolgen soll. Laufende Nummern und Ordnungsnummern sind nur dann anzuordnen und inhaltlich zu bestimmen, wenn sie Angaben über persönliche und sachliche Verhältnisse enthalten, die über die Erhebungs- und Hilfsmerkmale hinausgehen.
            \item Die Landesregierung wird ermächtigt, durch Rechtsverordnung die Durchführung einer durch Rechtsvorschrift angeordneten Landesstatistik oder die Erhebung einzelner Merkmale auszusetzen, die Periodizität zu verlängern, Erhebungstermine zu ändern sowie den Kreis der zu Befragenden einzuschränken, wenn und soweit die Ergebnisse nicht mehr benötigt werden. Die Landesregierung wird außerdem ermächtigt, durch Rechtsverordnung von der in einer Rechtsvorschrift vorgesehenen Befragung mit Auskunftspflicht zu einer Befragung ohne Auskunftspflicht überzugehen, wenn und soweit ausreichende Ergebnisse einer Landesstatistik auch durch Befragung ohne Auskunftspflicht erreicht werden können.
        \end{enumerate}

    \section{\S6 Statistiken aus dem Verwaltungsvollzug}
        \begin{enumerate}[label=(\arabic*)]
            \item Die Landesregierung kann durch Rechtsverordnung bestimmen, daß dem Statistischen Amt für statistische Zwecke solche personenbezogenen Daten aus automatisierten Registern des Verwaltungsvollzugs zur Verfügung gestellt werden, die beim Vollzug eines Landesgesetzes erhoben worden sind, soweit das Gesetz dies vorsieht. Personenbezogene Daten, die freiwillig für Zwecke des Verwaltungsvollzugs gegeben wurden und in automatisierten Registern oder Dateien verarbeitet werden, dürfen mit Einwilligung des Betroffenen dem Statistischen Amt für die Erfüllung seiner Aufgaben zur Verfügung gestellt werden.
            \item Die Rechtsverordnung muß folgende Angaben enthalten:
            \begin{enumerate}[label=\arabic*.]
                \item Bezeichnung des Registers und der Datei,
                \item speichernde Stelle,
                \item die an das Statistische Amt zu übermittelnden Daten,
                \item den statistischen Zweck, für den die Daten verwendet werden sollen,
                \item Zeitpunkt und Periodizität der Übermittlung. 
            \end{enumerate}
            \item Vor Erlaß der Rechtsverordnung ist der Landesbeauftragte für den Datenschutz zu hören.
        \end{enumerate}


 
    \section{\S7 Maßnahmen zur Vorbereitung von Landesstatistiken}
    Das Statistische Amt kann zur Vorbereitung einer eine Landesstatistik anordnenden Rechtsvorschrift
        \begin{enumerate}[label=\arabic*.]
            \item zur Bestimmung des Kreises der zu Befragenden und deren statistischer Zuordnung Angaben erheben sowie
            \item Erhebungsunterlagen und Erhebungsverfahren auf ihre Zweckmäßigkeit erproben.
        \end{enumerate}
    Für die Angaben nach Nummern 1 und 2 besteht keine Auskunftspflicht. Sie sind zum frühestmöglichen Zeitpunkt zu löschen, die Angaben nach Nummer 2 sind spätestens drei Jahre nach Durchführung der Erprobung zu löschen.

    \section{\S8 Geschäftsstatistiken}
        \begin{enumerate}[label=(\arabic*)]
            \item Geschäftsstatistiken bedürfen, auch soweit personenbezogene Daten verwendet werden, keiner Anordnung durch Rechtsvorschrift, wenn sie ausschließlich der Aufgabenbewältigung der Dienststelle, in deren Geschäftsgang die Daten anfallen, oder der Ausübung von Aufgaben oder Befugnissen der jeweils übergeordneten Dienststellen dienen.
            \item Die statistische Aufbereitung von Geschäftsstatistiken der Behörden und Gerichte des Landes sowie der der Aufsicht des Landes unterstehenden juristischen Personen des öffentlichen Rechts kann mit Zustimmung des fachlich zuständigen Ministeriums und des Ministeriums für Inneres und Europa ganz oder teilweise dem Statistischen Amt übertragen werden. Zu diesem Zweck dürfen mit Ausnahme von Name und Anschrift auch Einzelangaben über persönliche und sachliche Verhältnisse übermittelt werden. Gesetzliche Übermittlungs- und Offenbarungsverbote bleiben unberührt. Das Statistische Amt ist mit Einwilligung der auftraggebenden Stelle berechtigt, aus aufbereiteten Daten der Geschäftsstatistiken statistische Ergebnisse für allgemeine Zwecke darzustellen und zu veröffentlichen.
        \end{enumerate}

    \section{\S9 Erhebungsstellen}
        \begin{enumerate}[label=(\arabic*)]
            \item Die Landesregierung wird ermächtigt, durch Rechtsverordnung zu bestimmen, daß andere staatliche Stellen sowie Landkreise, kreisfreie Städte, Ämter und amtsfreie Gemeinden Erhebungsstellen einzurichten oder in sonstiger Weise an der amtlichen Statistik mitzuwirken haben, wenn dies wegen der Art der Erhebung, der Zahl oder der räumlichen Verteilung der zu Befragenden oder zur Sicherung der Qualität der Erhebung zweckmäßig ist. Die Landkreise, kreisfreien Städte, Ämter und amtsfreien Gemeinden nehmen die Aufgaben nach Satz 1 als Aufgaben im übertragenen Wirkungskreis wahr.
            \item Werden zur Erhebung von EU-, Bundes- oder Landesstatistiken örtliche Erhebungsstellen eingerichtet, so haben diese, soweit durch Rechtsvorschrift nichts anderes bestimmt ist, insbesondere
                \begin{enumerate}[label=\arabic*.]
                    \item die Erhebungsbeauftragten auszuwählen, zu bestellen, über ihre Rechte und Pflichten zu belehren, auf die in § 12 Abs. 2 genannten Geheimhaltungspflichten schriftlich zu verpflichten und zu beaufsichtigten,
                    \item bei der Auswahl der Berichtstellen mitzuwirken, die Erhebungsunterlagen auszuteilen und einzusammeln, die zu Befragenden über die Erhebung zu unterrichten und zur Auskunft aufzufordern, soweit Auskunftspflicht besteht,
                    \item unvollständige oder fehlerhaft ausgefüllte Erhebungsunterlagen durch Nachfrage bei den Befragten zu ergänzen oder zu berichtigen und
                    \item die Erhebungsunterlagen nach Prüfung auf Vollzähligkeit dem Statistischen Amt oder der überörtlichen Erhebungsstelle zuzuleiten.
                \end{enumerate}
            \item Werden überörtliche Erhebungsstellen eingerichtet, so haben diese, soweit durch Rechtsvorschrift nichts anderes bestimmt ist, insbesondere
                \begin{enumerate}[label=\arabic*.]
                    \item die Erhebungsunterlagen an die örtlichen Erhebungsstellen zu verteilen und von diesen wieder einzusammeln und
                    \item die empfangenen Erhebungsunterlagen auf Vollzähligkeit zu überprüfen und dem Statistischen Amt zuzuleiten.

                \end{enumerate}
            \item Die Erhebungsstellen sind für die Dauer der Bearbeitung von statistischen Einzelangaben von anderen Verwaltungsstellen zu trennen. § 11 Abs. 1 bis 3 gilt entsprechend.

            \item Sind bei Landkreisen, kreisfreien Städten, Ämtern und amtsfreien Gemeinden kommunale Statistikstellen eingerichtet, so können diese die Aufgaben der Erhebungsstellen wahrnehmen.
            \item Nehmen die Landkreise, kreisfreien Städte, Ämter und amtsfreien Gemeinden die Einrichtung der Erhebungsstellen als Aufgabe im übertragenen Wirkungskreis wahr, so unterliegen sie insoweit vorbehaltlich abweichender Regelungen der Fachaufsicht der nachfolgenden Behörden:
                \begin{enumerate}[label=\arabic*.]
                    \item Fachaufsichtsbehörde ist der Landrat, soweit örtliche Erhebungsstellen bei einer Gemeinde oder einem Amt eingerichtet sind, die der Rechtsaufsicht des Landrates unterstehen, im übrigen das Landesamt für innere Verwaltung,
                    \item obere Fachaufsichtsbehörde ist das Landesamt für innere Verwaltung,
                    \item oberste Fachaufsichtsbehörde ist das für die Erhebung jeweils fachlich zuständige Ministerium.
                \end{enumerate}
                Soweit das Landesamt für innere Verwaltung Fachaufsichtsbehörde oder obere Fachaufsichtsbehörde ist, werden diese Aufgaben vom Statistischen Amt wahrgenommen.
        \end{enumerate}

    \section{\S10 Kommunalstatistiken}
        Die Landkreise, kreisfreien Städte, Ämter und amtsfreien Gemeinden können zur Wahrnehmung ihrer öffentlichen Aufgaben statistische Erhebungen durchführen, soweit weder die benötigten statistischen Einzelangaben noch die erforderlichen Ergebnisse vom Statistischen Amt zur Verfügung gestellt werden können. Kommunalstatistiken mit Auskunftspflicht bedürfen einer Regelung durch Satzung. Kommunalstatistiken ohne Auskunftspflicht können auch durch Anordnung des Landrates, Oberbürgermeisters, Amtsvorstehers oder Bürgermeisters geregelt werden. § 5 Abs. 4 gilt jeweils entsprechend.

    \section{\S11 Kommunale Statistikstellen}
    \begin{enumerate}[label=(\arabic*)]
        \item Die Aufgaben der Kommunalstatistik dürfen nur von einer Dienststelle des Landkreises, der kreisfreien Stadt, des Amtes und der amtsfreien Gemeinde wahrgenommen werden, die organisatorisch und räumlich von den anderen Verwaltungsstellen der Körperschaft getrennt, gegen den Zutritt unbefugter Personen hinreichend gesichert und mit eigenem Personal ausgestattet ist (kommunale Statistikstelle).
        \item § 3 Abs. 3 gilt entsprechend für die in den kommunalen Statistikstellen tätigen Personen.
        \item Der Landrat, Oberbürgermeister, Amtsvorsteher oder Bürgermeister legt die zur Durchführung der Absätze 1 und 2 erforderlichen Maßnahmen in einer schriftlichen Dienstanweisung fest.
        \item Die Einrichtung sowie die Auflösung einer kommunalen Statistikstelle ist ortsüblich bekanntzugeben sowie dem Statistischen Amt, der Rechtsaufsichtsbehörde und dem Landesbeauftragten für den Datenschutz schriftlich anzuzeigen.
        \item Für ausschließlich statistische Zwecke dürfen an die kommunale Statistikstelle Daten, die im Geschäftsgang anderer Verwaltungsstellen der Landkreise, kreisfreien Städte, Ämter und amtsfreien Gemeinden anfallen, weitergegeben werden, soweit die Auswertungen zur Wahrnehmung der Aufgaben erforderlich sind und gesetzliche Weitergabeverbote nicht entgegenstehen. Regelmäßige Weitergaben sind nur aufgrund einer Satzung zulässig. § 6 Abs. 2 gilt dabei entsprechend.

    \end{enumerate}

    \section{\S12 Erhebungsbeauftragte}
        \begin{enumerate}[label=(\arabic*)]
            \item Werden zur Durchführung einer Landes- oder Kommunalstatistik Erhebungsbeauftragte eingesetzt, müssen sie die Gewähr für Zuverlässigkeit und Verschwiegenheit bieten. Erhebungsbeauftragte dürfen nicht eingesetzt werden, wenn aufgrund der beruflichen Tätigkeit oder aus anderen Gründen Anlaß zur Besorgnis besteht, daß Erkenntnisse aus der Tätigkeit als Erhebungsbeauftragte zu Lasten der Auskunftspflichtigen genutzt werden.
            \item Erhebungsbeauftragte dürfen die aus ihrer Tätigkeit gewonnenen Erkenntnisse nicht in anderen Verfahren oder für andere Zwecke verwenden. Sie sind über ihre Rechte und Pflichten zu belehren und auf die Wahrung der statistischen Geheimhaltung schriftlich zu verpflichten. Die Verpflichtung gilt auch nach Beendigung ihrer Tätigkeit fort.
            \item Erhebungsbeauftragte sind verpflichtet, die Anweisungen der mit der Durchführung der Landes- oder Kommunalstatistik betrauten Dienststellen zu befolgen. Bei der Ausübung ihrer Tätigkeit haben sie sich auszuweisen.
            \item Die Landkreise, kreisfreien Städte, Ämter und amtsfreien Gemeinden sind verpflichtet, bei der Bestellung von Erhebungsbeauftragten, insbesondere bei deren Benennung und Auswahl, mitzuwirken.
        \end{enumerate}
 
    \section{\S13 Erhebungs- und Hilfsmerkmale}
        \begin{enumerate}[label=(\arabic*)]
            \item Erhebungsmerkmale umfassen Angaben über persönliche und sachliche Verhältnisse, die zur statistischen Verwendung bestimmt sind. Hilfsmerkmale sind Angaben, die der technischen Durchführung von Landes- oder Kommunalstatistiken dienen.
            \item Für die regionale Zuordnung der Erhebungsmerkmale und für die regionale Darstellung statistischer Ergebnisse darf innerhalb einer Gemeinde als kleinste regionale Einheit die Blockseite genutzt und gespeichert werden. Besondere Regelungen in einer eine Landes- oder Kommunalstatistik anordnenden Rechts- oder Verwaltungsvorschrift bleiben unberührt.
            \item Soweit nicht eine Rechtsvorschrift etwas anderes bestimmt, sind die Hilfsmerkmale zu löschen, sobald die Überprüfung der Erhebungs- und Hilfsmerkmale auf ihre Schlüssigkeit und Vollständigkeit abgeschlossen ist. Sie sind von den Erhebungsmerkmalen zum frühestmöglichen Zeitpunkt zu trennen und gesondert aufzubewahren.
            \item Bei periodischen Erhebungen dürfen die zur Bestimmung des Kreises der zu Befragenden erforderlichen Hilfsmerkmale, soweit sie für nachfolgende Erhebungen benötigt werden, gesondert aufbewahrt werden. Nach Beendigung des Zeitraums der wiederkehrenden Erhebungen sind sie zu löschen.
            \item Absatz 3 gilt nicht für Einzelangaben, die ausschließlich einer öffentlichen Stelle zugeordnet werden können.
        \end{enumerate}

    \section{\S14 Auskunftspflicht}
    \begin{enumerate}[label=(\arabic*)]
        \item Besteht eine Auskunftspflicht, so sind alle in die Erhebung einbezogenen Personen und Stellen zur Beantwortung der gestellten Fragen gegenüber den mit der Durchführung der Statistik betrauten Stellen und Personen verpflichtet. Die Antwort ist für den Empfänger kosten- und portofrei zu erteilen.
        \item Die Antwort ist wahrheitsgemäß, vollständig und innerhalb der durch Rechtsvorschrift oder von der Erhebungsstelle gesetzten Frist zu erteilen. Eine schriftlich oder elektronisch zu übermittelnde Auskunft ist erst erteilt, wenn sie der Erhebungsstelle zugegangen ist. Elektronisch übermittelte Erhebungsvordrucke sind zugegangen, sobald die für den Empfang bestimmte Einrichtung sie in einer für die Erhebungsstelle bearbeitbaren Weise aufgezeichnet hat.
        \item Sind von den Auskunftspflichtigen Erhebungsvordrucke auszufüllen, sind die Antworten in den Vordrucken schriftlich oder elektronisch in der vorgegebenen Form zu erteilen, soweit in einer Rechtsvorschrift nichts anderes bestimmt ist. Die Richtigkeit ist unterschriftlich zu bestätigen, soweit dies in den Erhebungsvordrucken vorgesehen ist. Öffentliche Stellen des Landes haben aus dem Verwaltungsvollzug gewonnene Daten elektronisch zu übermitteln, soweit diese in geeigneter Form vorliegen.
        \item Werden Erhebungsbeauftragte eingesetzt, können die Fragen mündlich, schriftlich oder elektronisch beantwortet werden. Bei schriftlicher oder elektronischer Beantwortung sind die ausgefüllten Erhebungsvordrucke den Erhebungsbeauftragten offen oder in einem verschlossenen Umschlag zu übergeben oder bei der Erhebungsstelle abzugeben oder dorthin zu übersenden oder elektronisch zu übermitteln.
        \item Widerspruch und Anfechtungsklage gegen die Aufforderung zur Auskunftserteilung bei der Durchführung von Landes- oder Kommunalstatistiken haben keine aufschiebende Wirkung.
    \end{enumerate}

    \section{\S15 Informationspflicht}
    Die zu Befragenden sind über die Informationspflichten gemäß Artikel 13 und Artikel 14 der Verordnung (EU) 2016/679 des Europäischen Parlaments und des Rates vom 27. April 2016 zum Schutz natürlicher Personen bei der Verarbeitung personenbezogener Daten zum freien Datenverkehr und zur Aufhebung der Richtlinie 95/46/EG (ABl. L 119 vom 04.05.2016, S. 1; L 314 vom 22.11.2016, S. 72) hinaus schriftlich oder elektronisch zu unterrichten über
        \begin{enumerate}[label=\arabic*.]
            \item Art und Umfang der Erhebung,
            \item die statistische Geheimhaltung,
            \item die Auskunftspflicht oder die Freiwilligkeit der Auskunftserteilung,
            \item die bei der Durchführung der Erhebung verwendeten Hilfsmerkmale,
            \item die Trennung der Erhebungsmerkmale von den Hilfsmerkmalen und die Löschung der Hilfsmerkmale,
            \item die Hilfs- und Erhebungsmerkmale zur Führung von Adreßdateien,
            \item die Rechte und Pflichten der Erhebungsbeauftragten,
            \item die verschiedenen Möglichkeiten, Auskunft zu erteilen,
            \item die Möglichkeit der Übermittlung von Einzelangaben,
            \item die Bedeutung und den Inhalt von laufenden Nummern und Ordnungsnummern,
            \item den Ausschluß der aufschiebenden Wirkung von Widerspruch und Anfechtungsklage gegen die Aufforderung zur Auskunftserteilung.
        \end{enumerate}

    \section{\S15a Beschränkung von Rechten der betroffenen Personen}
    Die in den Artikeln 15, 16, 18 und 21 der Verordnung (EU) 2016/679 vorgesehenen Rechte der betroffenen Person sind insoweit beschränkt, als diese Rechte voraussichtlich die Verwirklichung der statistischen Zwecke unmöglich machen oder ernsthaft beeinträchtigen und solche Ausnahmen für die Erfüllung der Statistikzwecke notwendig sind.

    \section{\S16 Adreßdateien}
    Adreßdateien, die nach den jeweils geltenden bundesrechtlichen Vorschriften geführt werden, führt und nutzt das Statistische Amt in entsprechender Anwendung dieser Bestimmungen.
    
    \section{\S17 Statistische Geheimhaltung}
        \begin{enumerate}[label=(\arabic*)]
            \item Einzelangaben, die für eine Landes- oder Kommunalstatistik gemacht werden und die dem Befragten oder Betroffenen zugeordnet werden können, sind von den mit der Durchführung der Statistik betrauten Personen geheimzuhalten, soweit in diesem Gesetz oder in einer eine Landes- oder Kommunalstatistik anordnenden Rechtsvorschrift nichts anderes bestimmt ist. Die Pflicht zur statistischen Geheimhaltung gilt nicht für
                \begin{enumerate}[label=\arabic*.]
                    \item Einzelangaben, in deren Übermittlung oder Veröffentlichung der Befragte oder Betroffene schriftlich eingewilligt hat,
                    \item Einzelangaben, die aus allgemein zugänglichen Quellen entnommen werden können, auch soweit sie aufgrund einer Auskunftspflicht erlangt wurden,
                    \item Einzelangaben, die ausschließlich einer öffentlichen Stelle, die nicht am wirtschaftlichen Wettbewerb teilnimmt, zugeordnet werden können.
                \end{enumerate}
            \item Die Pflicht zur Geheimhaltung besteht auch für Personen, die Empfänger von Einzelangaben nach § 18 und § 19 Abs. 3 oder einer anderen Rechtsvorschrift sind.
        \end{enumerate}



    \section[Übermittlung von Einzelangaben]{\S18 Übermittlung von Einzelangaben aus Landes- und Kommunalstatistiken}
        \begin{enumerate}
            \item Die Übermittlung von Einzelangaben zwischen den mit der Durchführung der Statistik betrauten Personen und Stellen ist zulässig, soweit dies zur Erstellung der Statistik erforderlich ist. Darüber hinaus ist die Übermittlung von Einzelangaben zwischen den an einer Zusammenarbeit nach § 4 beteiligten statistischen Ämtern und die zentrale Verarbeitung dieser Einzelangaben in einem oder mehreren statistischen Ämtern zulässig.
            \item Für ausschließlich statistische Zwecke darf das Statistische Amt den kommunalen Statistikstellen Einzelangaben für ihren Zuständigkeitsbereich übermitteln, wenn die Voraussetzungen des § 11 Abs. 1 bis 4 erfüllt sind und die Übermittlung in einer eine Landesstatistik anordnenden Rechtsvorschrift vorgesehen ist sowie Art und Umfang der zu übermittelnden Einzelangaben bestimmt sind. Vor der erstmaligen Übermittlung von Einzelangaben ist dem Statistischen Amt die Dienstanweisung nach § 11 Abs. 3 vorzulegen
            \item Das Statistische Amt darf dem Statistischen Bundesamt und den Statistischen Ämtern der anderen Länder zur Erstellung koordinierter Länderstatistiken oder für methodische Untersuchungen Einzelangaben übermitteln.
            \item Für die Verwendung gegenüber dem Landtag und für Zwecke der Planung, jedoch nicht für die Regelung von Einzelfällen, dürfen den obersten Landesbehörden Tabellen mit statistischen Ergebnissen übermittelt werden, auch soweit Tabellenfeldern nur ein einziger Fall zugrunde liegt. Entsprechendes gilt für die Übermittlung von Daten an oberste Bundesbehörden und an oberste Behörden anderer Länder. Die Übermittlung nach Satz 1 und 2 ist nur zulässig, soweit in der eine Landesstatistik anordnenden Rechtsvorschrift die Übermittlung von Einzelangaben an oberste Landesbehörden oder oberste Bundesbehörden zugelassen ist.
            \item Für die Durchführung wissenschaftlicher Vorhaben darf das Statistische Amt Einzelangaben an Hochschulen oder sonstige Einrichtungen mit der Aufgabe unabhängiger wissenschaftlicher Forschung übermitteln, wenn die Einzelangaben nur mit einem unverhältnismäßig großen Aufwand an Zeit, Kosten und Arbeitskraft den Betroffenen zugeordnet werden können. Sofern es sich bei den Empfängern nicht um Amtsträger oder für den öffentlichen Dienst besonders Verpflichtete handelt, sind sie vor der Übermittlung besonders zur Geheimhaltung zu verpflichten. § 1 Abs. 2 und 3 und § 1 Abs. 4 Nr. 2 des Verpflichtungsgesetzes vom 2. März 1974 (BGBl. I S. 469), geändert durch Gesetz vom 15. August 1974 (BGBl. I S. 1942), gilt entsprechend. Personen, die nach Satz 2 besonders verpflichtet worden sind, stehen für die Anwendung der Vorschriften des Strafgesetzbuches über die Verletzung von Privatgeheimnissen ( § 203 Abs. 2, 4, 5 StGB und §§ 204 , 205 StGB ) und des Dienstgeheimnisses ( § 353 b Abs. 1 StGB ) den für den öffentlichen Dienst besonders Verpflichteten gleich. Empfänger haben durch technische und organisatorische Maßnahmen sicherzustellen, daß sonstige Personen keine Kenntnis von Einzelangaben erhalten. Die Einzelangaben sind zu löschen, sobald das wissenschaftliche Vorhaben abgeschlossen ist, zu dessen Durchführung sie übermittelt wurden. Die Löschung ist dem Statistischen Amt anzuzeigen.
            \item Die übermittelten Einzelangaben dürfen nur für Zwecke verwendet werden, für die sie übermittelt wurden. Die Übermittlung ist vom Statistischen Amt unter Angabe von Inhalt, empfangender Stelle, Datum und Zweck aufzuzeichnen. Die Aufzeichnungen sind von der empfangenden Stelle gegenzuzeichnen und beim Statistischen Amt fünf Jahre lang aufzubewahren.
            \item Die Vorschriften der Absätze 5 und 6 gelten für kommunale Statistikstellen entsprechend.
        \end{enumerate}
 
    \section{\S19 Vergabe statistischer Arbeiten}
    Behörden des Landes dürfen privaten oder öffentlichen Stellen Forschungs-, Planungs- und Untersuchungsaufträge, deren Erledigung statistische Erhebungen oder die Auswertung von Angaben aus Statistiken nach § 1 Nr. 1 a und Nr. 4 erfordern, nur im Einvernehmen mit dem Statistischen Amt erteilen. Vor der Auftragserteilung sind Art und Umfang der statistischen Erhebungen oder Auswertungen mit dem Statistischen Amt abzustimmen. Können die benötigten Angaben vom Statistischen Amt zur Verfügung gestellt werden, darf der Auftrag insoweit nicht erteilt werden.
 
    \section{\S20 Verbot der Reidentifizierung}
    Eine Zusammenführung von Einzeldaten aus Landes- oder Kommunalstatistiken oder von Einzelangaben mit anderen Angaben zum Zwecke der Herstellung eines Personen-, Unternehmens-, Betriebs- oder Arbeitsstättenbezuges außerhalb der Aufgabenstellung dieses Gesetzes oder einer eine Statistik anordnenden Rechtsvorschrift ist verboten.

    \section{\S21 Strafvorschrift}
    Wer entgegen § 20 Einzeldaten aus Landesstatistiken oder Kommunalstatistiken oder solche Einzelangaben mit anderen Angaben zusammenführt, wird mit Freiheitsstrafe bis zu einem Jahr oder mit Geldstrafe bestraft.
 
    \section{\S22 Bußgeldvorschrift}
        \begin{enumerate}
            \item Ordnungswidrig handelt, wer vorsätzlich oder fahrlässig entgegen § 14 Abs. 1 Satz 1, Absatz 2 oder 3 eine Auskunft nicht, nicht wahrheitsgemäß, nicht vollständig, nicht rechtzeitig oder nicht auf den Erhebungsvordrucken in der dort vorgegebenen Form erteilt.
            \item Ordnungswidrig handelt auch, wer vorsätzlich oder fahrlässig einer Auskunftspflicht zuwiderhandelt, die in einer nach § 10 Satz 2 erlassenen Satzung festgelegt ist, soweit die Satzung für einen bestimmten Tatbestand auf diese Bußgeldvorschrift verweist.
            \item Die Ordnungswidrigkeit kann mit einer Geldbuße bis zu 5 000 Euro geahndet werden.
            \item Verwaltungsbehörde im Sinne von § 36 Abs. 1 Nr. 1 des Gesetzes über Ordnungswidrigkeiten ist bei EU-, Bundes- und Landesstatistiken das Landesamt für innere Verwaltung, soweit durch Rechtsvorschrift nichts anderes bestimmt wird. Diese Aufgaben werden vom Statistischen Amt wahrgenommen. Verwaltungsbehörde im Sinne von § 36 Abs. 1 Nr. 1 des Gesetzes über Ordnungswidrigkeiten ist bei Statistiken nach § 10 Satz 2 der Landrat, der Oberbürgermeister, der Amtsvorsteher oder der Bürgermeister der amtsfreien Gemeinden.
        \end{enumerate}

    \section{\S23 (aufgehoben)}
    
    \section{\S24 Inkrafttreten}
    Dieses Gesetz tritt am Tage nach seiner Verkündung in Kraft.
