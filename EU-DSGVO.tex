\chapter{EU-DSGVO}
\qrcode{https://eur-lex.europa.eu/legal-content/DE/TXT/PDF/?uri=CELEX:32016R0679}
\newline
\url{https://eur-lex.europa.eu/legal-content/DE/TXT/PDF/?uri=CELEX:32016R0679}
    \section{Art. 1: Gegenstand und Ziele}
        \begin{enumerate}[label=(\arabic*)]
            \item Diese Verordnung enthält Vorschriften zum Schutz natürlicher Personen bei der Verarbeitung personenbezogener Daten und zum freien Verkehr solcher Daten.
            \item Diese Verordnung schützt die Grundrechte und Grundfreiheiten natürlicher Personen und insbesondere deren Recht auf Schutz personenbezogener Daten.
            \item Der freie Verkehr personenbezogener Daten in der Union darf aus Gründen des Schutzes natürlicher Personen bei der Verarbeitung personenbezogener Daten weder eingeschränkt noch verboten werden. 
        \end{enumerate}


    \section{Art. 4: Begriffsbestimmungen}
    Im Sinne dieser Verordnung bezeichnet der Ausdruck:
        \begin{enumerate}[label=\arabic*.]
            \item ``personenbezogene Daten'' alle Informationen, die sich auf eine identifizierte oder identifizierbare natürliche Person (im Folgenden ``betroffene Person'') beziehen; als identifizierbar wird eine natürliche Person angesehen, die direkt oder indirekt, insbesondere mittels Zuordnung zu einer Kennung wie einem Namen, zu einer Kennnummer, zu Standortdaten, zu einer Online-Kennung oder zu einem oder mehreren besonderen Merkmalen, die Ausdruck der physischen, physiologischen, genetischen, psychischen, wirtschaftlichen, kulturellen oder sozialen Identität dieser natürlichen Person sind, identifiziert werden kann; 
            \item ``Verarbeitung'' jeden mit oder ohne Hilfe automatisierter Verfahren ausgeführten Vorgang oder jede solche Vorgangsreihe im Zusammenhang mit personenbezogenen Daten wie das Erheben, das Erfassen, die Organisation, das Ordnen, die Speicherung, die Anpassung oder Veränderung, das Auslesen, das Abfragen, die Verwendung, die Offenlegung durch Übermittlung, Verbreitung oder eine andere Form der Bereitstellung, den Abgleich oder die Verknüpfung, die Einschränkung, das Löschen oder die Vernichtung;
            \item ``Einschränkung der Verarbeitung'' die Markierung gespeicherter personenbezogener Daten mit dem Ziel, ihre künftige Verarbeitung einzuschränken;
            \item ``Profiling'' jede Art der automatisierten Verarbeitung personenbezogener Daten, die darin besteht, dass diese personenbezogenen Daten verwendet werden, um bestimmte persönliche Aspekte, die sich auf eine natürliche Person beziehen, zu bewerten, insbesondere um Aspekte bezüglich Arbeitsleistung, wirtschaftliche Lage, Gesundheit, persönliche Vorlieben, Interessen, Zuverlässigkeit, Verhalten, Aufenthaltsort oder Ortswechsel dieser natürlichen Person zu analysieren oder vorherzusagen;
            \item ``Pseudonymisierung'' die Verarbeitung personenbezogener Daten in einer Weise, dass die personenbezogenen Daten ohne Hinzuziehung zusätzlicher Informationen nicht mehr einer spezifischen betroffenen Person zugeordnet werden können, sofern diese zusätzlichen Informationen gesondert aufbewahrt werden und technischen und organisatorischen Maßnahmen unterliegen, die gewährleisten, dass die personenbezogenen Daten nicht einer identifizierten oder identifizierbaren natürlichen Person zugewiesen werden; 
            \item ``Dateisystem'' jede strukturierte Sammlung personenbezogener Daten, die nach bestimmten Kriterien zugänglich sind, unabhängig davon, ob diese Sammlung zentral, dezentral oder nach funktionalen oder geografischen Gesichtspunkten geordnet geführt wird;
            \item ``Verantwortlicher'' die natürliche oder juristische Person, Behörde, Einrichtung oder andere Stelle, die allein oder gemeinsam mit anderen über die Zwecke und Mittel der Verarbeitung von personenbezogenen Daten entscheidet; sind die Zwecke und Mittel dieser Verarbeitung durch das Unionsrecht oder das Recht der Mitgliedstaaten vorgegeben, so kann der Verantwortliche beziehungsweise können die bestimmten Kriterien seiner Benennung nach dem Unionsrecht oder dem Recht der Mitgliedstaaten vorgesehen werden;
            \item ``Auftragsverarbeiter'' eine natürliche oder juristische Person, Behörde, Einrichtung oder andere Stelle, die personenbezogene Daten im Auftrag des Verantwortlichen verarbeitet;
            \item ``Empfänger'' eine natürliche oder juristische Person, Behörde, Einrichtung oder andere Stelle, der personenbezogene Daten offengelegt werden, unabhängig davon, ob es sich bei ihr um einen Dritten handelt oder nicht. Behörden, die im Rahmen eines bestimmten Untersuchungsauftrags nach dem Unionsrecht oder dem Recht der Mitgliedstaaten möglicherweise personenbezogene Daten erhalten, gelten jedoch nicht als Empfänger; die Verarbeitung dieser Daten durch die genannten Behörden erfolgt im Einklang mit den geltenden Datenschutzvorschriften gemäß den Zwecken der Verarbeitung;
            \item  ``Dritter'' eine natürliche oder juristische Person, Behörde, Einrichtung oder andere Stelle, außer der betroffenen Person, dem Verantwortlichen, dem Auftragsverarbeiter und den Personen, die unter der unmittelbaren Verantwortung des Verantwortlichen oder des Auftragsverarbeiters befugt sind, die personenbezogenen Daten zu verarbeiten;
            \item ``Einwilligung'' der betroffenen Person jede freiwillig für den bestimmten Fall, in informierter Weise und unmissverständlich abgegebene Willensbekundung in Form einer Erklärung oder einer sonstigen eindeutigen bestätigenden Handlung, mit der die betroffene Person zu verstehen gibt, dass sie mit der Verarbeitung der sie betreffenden personenbezogenen Daten einverstanden ist;
            \item ``Verletzung des Schutzes personenbezogener Daten'' eine Verletzung der Sicherheit, die, ob unbeabsichtigt oder unrechtmäßig, zur Vernichtung, zum Verlust, zur Veränderung, oder zur unbefugten Offenlegung von beziehungsweise zum unbefugten Zugang zu personenbezogenen Daten führt, die übermittelt, gespeichert oder auf sonstige Weise verarbeitet wurden;
            \item ``genetische Daten'' personenbezogene Daten zu den ererbten oder erworbenen genetischen Eigenschaften einer natürlichen Person, die eindeutige Informationen über die Physiologie oder die Gesundheit dieser natürlichen Person liefern und insbesondere aus der Analyse einer biologischen Probe der betreffenden natürlichen Person gewonnen wurden;
            \item ``biometrische Daten'' mit speziellen technischen Verfahren gewonnene personenbezogene Daten zu den physischen, physiologischen oder verhaltenstypischen Merkmalen einer natürlichen Person, die die eindeutige Identifizierung dieser natürlichen Person ermöglichen oder bestätigen, wie Gesichtsbilder oder daktyloskopische Daten;
            \item ``Gesundheitsdaten'' personenbezogene Daten, die sich auf die körperliche oder geistige Gesundheit einer natürlichen Person, einschließlich der Erbringung von Gesundheitsdienstleistungen, beziehen und aus denen Informationen über deren Gesundheitszustand hervorgehen;
            \item ``Hauptniederlassung''
                \begin{enumerate}[label=\alph*)]
                    \item im Falle eines Verantwortlichen mit Niederlassungen in mehr als einem Mitgliedstaat den Ort seiner Hauptverwaltung in der Union, es sei denn, die Entscheidungen hinsichtlich der Zwecke und Mittel der Verarbeitung personenbezogener Daten werden in einer anderen Niederlassung des Verantwortlichen in der Union getroffen und diese Niederlassung ist befugt, diese Entscheidungen umsetzen zu lassen; in diesem Fall gilt die Niederlassung, die derartige Entscheidungen trifft, als Hauptniederlassung;
                    \item im Falle eines Auftragsverarbeiters mit Niederlassungen in mehr als einem Mitgliedstaat den Ort seiner Hauptverwaltung in der Union oder, sofern der Auftragsverarbeiter keine Hauptverwaltung in der Union hat, die Niederlassung des Auftragsverarbeiters in der Union, in der die Ver\-ar\-bei\-tungs\-tä\-tig\-kei\-ten im Rahmen der Tätigkeiten einer Niederlassung eines Auftragsverarbeiters hauptsächlich stattfinden, soweit der Auftragsverarbeiter spezifischen Pflichten aus dieser Verordnung unterliegt;
                \end{enumerate} 
            \item ``Vertreter'' eine in der Union niedergelassene natürliche oder juristische Person, die von dem Verantwortlichen oder Auftragsverarbeiter schriftlich gemäß Artikel 27 bestellt wurde und den Verantwortlichen oder Auftragsverarbeiter in Bezug auf die ihnen jeweils nach dieser Verordnung obliegenden Pflichten vertritt;
            \item ``Unternehmen'' eine natürliche und juristische Person, die eine wirtschaftliche Tätigkeit ausübt, unabhängig von ihrer Rechtsform, einschließlich Personengesellschaften oder Vereinigungen, die regelmäßig einer wirtschaftlichen Tätigkeit nachgehen;
            \item ``Unternehmensgruppe'' eine Gruppe, die aus einem herrschenden Unternehmen und den von diesem abhängigen Unternehmen besteht; 
            \item ``verbindliche interne Datenschutzvorschriften'' Maßnahmen zum Schutz personenbezogener Daten, zu deren Einhaltung sich ein im Hoheitsgebiet eines Mitgliedstaats niedergelassener Verantwortlicher oder Auftragsverarbeiter verpflichtet im Hinblick auf Datenübermittlungen oder eine Kategorie von Datenübermittlungen personenbezogener Daten an einen Verantwortlichen oder Auftragsverarbeiter derselben Unternehmensgruppe oder derselben Gruppe von Unternehmen, die eine gemeinsame Wirtschaftstätigkeit ausüben, in einem oder mehreren Drittländern;
            \item ``Aufsichtsbehörde'' eine von einem Mitgliedstaat gemäß Artikel 51 eingerichtete unabhängige staatliche Stelle; 
            \item ``betroffene Aufsichtsbehörde'' eine Aufsichtsbehörde, die von der Verarbeitung personenbezogener Daten betroffen ist, weil
                \begin{enumerate}[label=\alph*)]
                    \item der Verantwortliche oder der Auftragsverarbeiter im Hoheitsgebiet des Mitgliedstaats dieser Aufsichtsbehörde niedergelassen ist,
                    \item diese Verarbeitung erhebliche Auswirkungen auf betroffene Personen mit Wohnsitz im Mitgliedstaat dieser Aufsichtsbehörde hat oder haben kann oder
                    \item eine Beschwerde bei dieser Aufsichtsbehörde eingereicht wurde;
                \end{enumerate}
            \item ``grenzüberschreitende Verarbeitung'' entweder
                \begin{enumerate}[label=\alph*)]
                    \item eine Verarbeitung personenbezogener Daten, die im Rahmen der Tätigkeiten von Niederlassungen eines Verantwortlichen oder eines Auftragsverarbeiters in der Union in mehr als einem Mitgliedstaat erfolgt, wenn der Verantwortliche oder Auftragsverarbeiter in mehr als einem Mitgliedstaat niedergelassen ist, oder
                    \item eine Verarbeitung personenbezogener Daten, die im Rahmen der Tätigkeiten einer einzelnen Niederlassung eines Verantwortlichen oder eines Auftragsverarbeiters in der Union erfolgt, die jedoch erhebliche Auswirkungen auf betroffene Personen in mehr als einem Mitgliedstaat hat oder haben kann;
                \end{enumerate}              
            \item ``maßgeblicher und begründeter Einspruch'' einen Einspruch gegen einen Beschlussentwurf im Hinblick darauf, ob ein Verstoß gegen diese Verordnung vorliegt oder ob beabsichtigte Maßnahmen gegen den Verantwortlichen oder den Auftragsverarbeiter im Einklang mit dieser Verordnung steht, wobei aus diesem Einspruch die Tragweite der Risiken klar hervorgeht, die von dem Beschlussentwurf in Bezug auf die Grundrechte und Grundfreiheiten der betroffenen Personen und gegebenenfalls den freien Verkehr personenbezogener Daten in der Union ausgehen;
            \item ``Dienst der Informationsgesellschaft'' eine Dienstleistung im Sinne des Artikels 1 Nummer 1 Buchstabe b der Richtlinie (EU) 2015/1535 des Europäischen Parlaments und des Rates 
            \item ``internationale Organisation'' eine völkerrechtliche Organisation und ihre nachgeordneten Stellen oder jede sonstige Einrichtung, die durch eine zwischen zwei oder mehr Ländern geschlossene Übereinkunft oder auf der Grundlage einer solchen Übereinkunft geschaffen wurde. 
        \end{enumerate}
    \section{Art. 5: Grundsätze für die Verarbeitung personenbezogener Daten}
        \begin{enumerate}[label=(\arabic*)]
            \item Personenbezogene Daten müssen
                \begin{enumerate}[label=\alph*)]
                    \item auf rechtmäßige Weise, nach Treu und Glauben und in einer für die betroffene Person nachvollziehbaren Weise verarbeitet werden (``Rechtmäßigkeit, Verarbeitung nach Treu und Glauben, Transparenz'');
                    \item für festgelegte, eindeutige und legitime Zwecke erhoben werden und dürfen nicht in einer mit diesen Zwecken nicht zu vereinbarenden Weise weiterverarbeitet werden; eine Weiterverarbeitung für im öffentlichen Interesse liegende Archivzwecke, für wissenschaftliche oder historische Forschungszwecke oder für statistische Zwecke gilt gemäß Artikel 89 Absatz 1 nicht als unvereinbar mit den ursprünglichen Zwecken (``Zweckbindung'');
                    \item dem Zweck angemessen und erheblich sowie auf das für die Zwecke der Verarbeitung notwendige Maß beschränkt sein (``Datenminimierung''); 
                    \item sachlich richtig und erforderlichenfalls auf dem neuesten Stand sein; es sind alle angemessenen Maßnahmen zu treffen, damit personenbezogene Daten, die im Hinblick auf die Zwecke ihrer Verarbeitung unrichtig sind, unverzüglich gelöscht oder berichtigt werden (``Richtigkeit'');
                    \item in einer Form gespeichert werden, die die Identifizierung der betroffenen Personen nur so lange ermöglicht, wie es für die Zwecke, für die sie verarbeitet werden, erforderlich ist; personenbezogene Daten dürfen länger gespeichert werden, soweit die personenbezogenen Daten vorbehaltlich der Durchführung geeigneter technischer und organisatorischer Maßnahmen, die von dieser Verordnung zum Schutz der Rechte und Freiheiten der betroffenen Person gefordert werden, ausschließlich für im öffentlichen Interesse liegende Archivzwecke oder für wissenschaftliche und historische Forschungszwecke oder für statistische Zwecke gemäß Artikel 89 Absatz 1 verarbeitet werden (``Speicherbegrenzung'');
                    \item in einer Weise verarbeitet werden, die eine angemessene Sicherheit der personenbezogenen Daten gewährleistet, einschließlich Schutz vor unbefugter oder unrechtmäßiger Verarbeitung und vor unbeabsichtigtem Verlust, unbeabsichtigter Zerstörung oder unbeabsichtigter Schädigung durch geeignete technische und organisatorische Maßnahmen (``Integrität und Vertraulichkeit'');
                \end{enumerate}
            \item Der Verantwortliche ist für die Einhaltung des Absatzes 1 verantwortlich und muss dessen Einhaltung nachweisen können (``Rechenschaftspflicht'')
        \end{enumerate}
    \section[Art 9: Verarbeitung besonderer Daten]{Art. 9: Verarbeitung besonderer Kategorien personenbezogener Daten}
        \begin{enumerate}
            \item Die Verarbeitung personenbezogener Daten, aus denen die rassische und ethnische Herkunft, politische Meinungen, religiöse oder weltanschauliche Überzeugungen oder die Gewerkschaftszugehörigkeit hervorgehen, sowie die Verarbeitung von genetischen Daten, biometrischen Daten zur eindeutigen Identifizierung einer na\-tür\-lichen Person, Gesundheitsdaten oder Daten zum Sexualleben oder der sexuellen Orientierung einer natürlichen Person ist untersagt.
            \item Absatz 1 gilt nicht in folgenden Fällen:
                \begin{enumerate}[label=\alph*)]
                    \item Die betroffene Person hat in die Verarbeitung der genannten personenbezogenen Daten für einen oder mehrere festgelegte Zwecke ausdrücklich eingewilligt, es sei denn, nach Unionsrecht oder dem Recht der Mitgliedstaaten kann das Verbot nach Absatz 1 durch die Einwilligung der betroffenen Person nicht aufgehoben werden,
                    \item die Verarbeitung ist erforderlich, damit der Verantwortliche oder die betroffene Person die ihm bzw. ihr aus dem Arbeitsrecht und dem Recht der sozialen Sicherheit und des Sozialschutzes erwachsenden Rechte ausüben und seinen bzw. ihren diesbezüglichen Pflichten nachkommen kann, soweit dies nach Unionsrecht oder dem Recht der Mitgliedstaaten oder einer Kollektivvereinbarung nach dem Recht der Mitgliedstaaten, das geeignete Garantien für die Grundrechte und die Interessen der betroffenen Person vorsieht, zulässig ist,
                    \item die Verarbeitung ist zum Schutz lebenswichtiger Interessen der betroffenen Person oder einer anderen natürlichen Person erforderlich und die betroffene Person ist aus körperlichen oder rechtlichen Gründen außerstande, ihre Einwilligung zu geben, 
                    \item die Verarbeitung erfolgt auf der Grundlage geeigneter Garantien durch eine politisch, weltanschaulich, religiös oder gewerkschaftlich ausgerichtete Stiftung, Vereinigung oder sonstige Organisation ohne Gewinnerzielungsabsicht im Rahmen ihrer rechtmäßigen Tätigkeiten und unter der Voraussetzung, dass sich die Verarbeitung ausschließlich auf die Mitglieder oder ehemalige Mitglieder der Organisation oder auf Personen, die im Zusammenhang mit deren Tätigkeitszweck regelmäßige Kontakte mit ihr unterhalten, bezieht und die personenbezogenen Daten nicht ohne Einwilligung der betroffenen Personen nach außen offengelegt werden,
                    \item die Verarbeitung bezieht sich auf personenbezogene Daten, die die betroffene Person offensichtlich öffentlich gemacht hat,
                    \item die Verarbeitung ist zur Geltendmachung, Ausübung oder Verteidigung von Rechtsansprüchen oder bei Handlungen der Gerichte im Rahmen ihrer justiziellen Tätigkeit erforderlich, 
                    \item die Verarbeitung ist auf der Grundlage des Unionsrechts oder des Rechts eines Mitgliedstaats, das in angemessenem Verhältnis zu dem verfolgten Ziel steht, den Wesensgehalt des Rechts auf Datenschutz wahrt und angemessene und spezifische Maßnahmen zur Wahrung der Grundrechte und Interessen der betroffenen Person vorsieht, aus Gründen eines erheblichen öffentlichen Interesses erforderlich,
                    \item die Verarbeitung ist für Zwecke der Gesundheitsvorsorge oder der Arbeitsmedizin, für die Beurteilung der Arbeitsfähigkeit des Beschäftigten, für die medizinische Diagnostik, die Versorgung oder Behandlung im Gesundheitsoder Sozialbereich oder für die Verwaltung von Systemen und Diensten im Gesundheits- oder Sozialbereich auf der Grundlage des Unionsrechts oder des Rechts eines Mitgliedstaats oder aufgrund eines Vertrags mit einem Angehörigen eines Gesundheitsberufs und vorbehaltlich der in Absatz 3 genannten Bedingungen und Garantien erforderlich,
                    \item die Verarbeitung ist aus Gründen des öffentlichen Interesses im Bereich der öffentlichen Gesundheit, wie dem Schutz vor schwerwiegenden grenz\-über\-schrei\-ten\-den Gesundheitsgefahren oder zur Gewährleistung hoher Qualitäts- und Sicherheitsstandards bei der Gesundheitsversorgung und bei Arzneimitteln und Medizinprodukten, auf der Grundlage des Unionsrechts oder des Rechts eines Mitgliedstaats, das angemessene und spezifische Maßnahmen zur Wahrung der Rechte und Freiheiten der betroffenen Person, insbesondere des Berufsgeheimnisses, vorsieht, erforderlich, oder
                    \item die Verarbeitung ist auf der Grundlage des Unionsrechts oder des Rechts eines Mitgliedstaats, das in angemessenem Verhältnis zu dem verfolgten Ziel steht, den Wesensgehalt des Rechts auf Datenschutz wahrt und angemessene und spezifische Maßnahmen zur Wahrung der Grundrechte und Interessen der betroffenen Person vorsieht, für im öffentlichen Interesse liegende Archivzwecke, für wissenschaftliche oder historische Forschungszwecke oder für statistische Zwecke gemäß Artikel 89 Absatz 1 erforderlich.
                \end{enumerate}
            \item Die in Absatz 1 genannten personenbezogenen Daten dürfen zu den in Absatz 2 Buchstabe h genannten Zwecken verarbeitet werden, wenn diese Daten von Fachpersonal oder unter dessen Verantwortung verarbeitet werden und dieses Fachpersonal nach dem Unionsrecht oder dem Recht eines Mitgliedstaats oder den Vorschriften nationaler zuständiger Stellen dem Berufsgeheimnis unterliegt, oder wenn die Verarbeitung durch eine andere Person erfolgt, die ebenfalls nach dem Unionsrecht oder dem Recht eines Mitgliedstaats oder den Vorschriften nationaler zuständiger Stellen einer Geheimhaltungspflicht unterliegt.
            \item Die Mitgliedstaaten können zusätzliche Bedingungen, einschließlich Be\-schrän\-kun\-gen, einführen oder aufrechterhalten, soweit die Verarbeitung von genetischen, biometrischen oder Gesundheitsdaten betroffen ist. 
        \end{enumerate}
    \section[Art. 14: Informationspflicht]{Art. 14: Informationspflicht, wenn die personenbezogenen Daten nicht bei der betroffenen Person
erhoben wurden}
    \begin{enumerate}[label=(\arabic*)]
        \item Werden personenbezogene Daten nicht bei der betroffenen Person erhoben, so teilt der Verantwortliche der betroffenen Person Folgendes mit:
            \begin{enumerate}[label=\alph*)]
                \item den Namen und die Kontaktdaten des Verantwortlichen sowie gegebenenfalls seines Vertreters;
                \item zusätzlich die Kontaktdaten des Datenschutzbeauftragten;
                \item die Zwecke, für die die personenbezogenen Daten verarbeitet werden sollen, sowie die Rechtsgrundlage für die
Verarbeitung;
                \item die Kategorien personenbezogener Daten, die verarbeitet werden;
                \item gegebenenfalls die Empfänger oder Kategorien von Empfängern der personenbezogenen Daten;
                \item gegebenenfalls die Absicht des Verantwortlichen, die personenbezogenen Daten an einen Empfänger in einem Drittland oder einer internationalen Organisation zu übermitteln, sowie das Vorhandensein oder das Fehlen eines Angemessenheitsbeschlusses der Kommission oder im Falle von Übermittlungen gemäß Artikel 46 oder Artikel 47 oder Artikel 49 Absatz 1 Unterabsatz 2 einen Verweis auf die geeigneten oder angemessenen Garantien und die Möglichkeit, eine Kopie von ihnen zu erhalten, oder wo sie verfügbar sind. 
            \end{enumerate}
        \item Zusätzlich zu den Informationen gemäß Absatz 1 stellt der Verantwortliche der betroffenen Person die folgenden Informationen zur Verfügung, die erforderlich sind, um der betroffenen Person gegenüber eine faire und transparente Verarbeitung zu gewährleisten:
            \begin{enumerate}[label=\alph*)]
                \item die Dauer, für die die personenbezogenen Daten gespeichert werden oder, falls dies nicht möglich ist, die Kriterien für die Festlegung dieser Dauer;
                \item wenn die Verarbeitung auf Artikel 6 Absatz 1 Buchstabe f beruht, die berechtigten Interessen, die von dem Verantwortlichen oder einem Dritten verfolgt werden;
                \item das Bestehen eines Rechts auf Auskunft seitens des Verantwortlichen über die betreffenden personenbezogenen Daten sowie auf Berichtigung oder Löschung oder auf Einschränkung der Verarbeitung und eines Widerspruchsrechts gegen die Verarbeitung sowie des Rechts auf Datenübertragbarkeit; 
                \item wenn die Verarbeitung auf Artikel 6 Absatz 1 Buchstabe a oder Artikel 9 Absatz 2 Buchstabe a beruht, das Bestehen eines Rechts, die Einwilligung jederzeit zu widerrufen, ohne dass die Rechtmäßigkeit der aufgrund der Einwilligung bis zum Widerruf erfolgten Verarbeitung berührt wird;
                \item das Bestehen eines Beschwerderechts bei einer Aufsichtsbehörde;
                \item aus welcher Quelle die personenbezogenen Daten stammen und gegebenenfalls ob sie aus öffentlich zugänglichen Quellen stammen;
                \item das Bestehen einer automatisierten Entscheidungsfindung einschließlich Profiling gemäß Artikel 22 Absätze 1 und 4 und — zumindest in diesen Fällen — aussagekräftige Informationen über die involvierte Logik sowie die Tragweite und die angestrebten Auswirkungen einer derartigen Verarbeitung für die betroffene Person.
            \end{enumerate}
        \item Der Verantwortliche erteilt die Informationen gemäß den Absätzen 1 und 2
            \begin{enumerate}
                \item unter Berücksichtigung der spezifischen Umstände der Verarbeitung der personenbezogenen Daten innerhalb einer angemessenen Frist nach Erlangung der personenbezogenen Daten, längstens jedoch innerhalb eines Monats, 
                \item falls die personenbezogenen Daten zur Kommunikation mit der betroffenen Person verwendet werden sollen, spätestens zum Zeitpunkt der ersten Mitteilung an sie, oder, 
                \item falls die Offenlegung an einen anderen Empfänger beabsichtigt ist, spätestens zum Zeitpunkt der ersten Offenlegung.
            \end{enumerate}
        \item Beabsichtigt der Verantwortliche, die personenbezogenen Daten für einen anderen Zweck weiterzuverarbeiten als den, für den die personenbezogenen Daten erlangt wurden, so stellt er der betroffenen Person vor dieser Weiterverarbeitung Informationen über diesen anderen Zweck und alle anderen maßgeblichen Informationen gemäß Absatz 2 zur Verfügung.
        \item Die Absätze 1 bis 4 finden keine Anwendung, wenn und soweit
        \begin{enumerate}[label=\alph*)]
            \item die betroffene Person bereits über die Informationen verfügt,
            \item die Erteilung dieser Informationen sich als unmöglich erweist oder einen unverhältnismäßigen Aufwand erfordern würde; dies gilt insbesondere für die Verarbeitung für im öffentlichen Interesse liegende Archivzwecke, für wissenschaftliche oder historische Forschungszwecke oder für statistische Zwecke vorbehaltlich der in Artikel 89 Absatz 1 genannten Bedingungen und Garantien oder soweit die in Absatz 1 des vorliegenden Artikels genannte Pflicht voraussichtlich die Verwirklichung der Ziele dieser Verarbeitung unmöglich macht oder ernsthaft beeinträchtigt In diesen Fällen ergreift der Verantwortliche geeignete Maßnahmen zum Schutz der Rechte und Freiheiten sowie der berechtigten Interessen der betroffenen Person, einschließlich der Bereitstellung dieser Informationen für die Öffentlichkeit,
            \item die Erlangung oder Offenlegung durch Rechtsvorschriften der Union oder der Mitgliedstaaten, denen der Verantwortliche unterliegt und die geeignete Maßnahmen zum Schutz der berechtigten Interessen der betroffenen Person vorsehen, ausdrücklich geregelt ist oder
            \item die personenbezogenen Daten gemäß dem Unionsrecht oder dem Recht der Mitgliedstaaten dem Berufsgeheimnis, einschließlich einer satzungsmäßigen Geheimhaltungspflicht, unterliegen und daher vertraulich behandelt werden müssen. 
        \end{enumerate}
    \end{enumerate}

    \section[Art. 15: Auskunftsrecht]{Art. 15: Auskunftsrecht der betroffenen Person}
        \begin{enumerate}[label=(\arabic*)]
            \item Die betroffene Person hat das Recht, von dem Verantwortlichen eine Bestätigung darüber zu verlangen, ob sie betreffende personenbezogene Daten verarbeitet werden; ist dies der Fall, so hat sie ein Recht auf Auskunft über diese personenbezogenen Daten und auf folgende Informationen:
                \begin{enumerate}[label=\alph*)]
                    \item die Verarbeitungszwecke;
                    \item die Kategorien personenbezogener Daten, die verarbeitet werden;
                    \item die Empfänger oder Kategorien von Empfängern, gegenüber denen die personenbezogenen Daten offengelegt worden sind oder noch offengelegt werden, insbesondere bei Empfängern in Drittländern oder bei internationalen Organisationen;
                    \item falls möglich die geplante Dauer, für die die personenbezogenen Daten gespeichert werden, oder, falls dies nicht möglich ist, die Kriterien für die Festlegung dieser Dauer;
                    \item das Bestehen eines Rechts auf Berichtigung oder Löschung der sie betreffenden personenbezogenen Daten oder auf Einschränkung der Verarbeitung durch den Verantwortlichen oder eines Widerspruchsrechts gegen diese Verarbeitung;
                    \item das Bestehen eines Beschwerderechts bei einer Aufsichtsbehörde;
                    \item wenn die personenbezogenen Daten nicht bei der betroffenen Person erhoben werden, alle verfügbaren Informationen über die Herkunft der Daten;
                    \item das Bestehen einer automatisierten Entscheidungsfindung einschließlich Profiling gemäß Artikel 22 Absätze 1 und 4 und — zumindest in diesen Fällen — aussagekräftige Informationen über die involvierte Logik sowie die Tragweite und die angestrebten Auswirkungen einer derartigen Verarbeitung für die betroffene Person.
                \end{enumerate}
            \item Werden personenbezogene Daten an ein Drittland oder an eine internationale Organisation übermittelt, so hat die betroffene Person das Recht, über die geeigneten Garantien gemäß Artikel 46 im Zusammenhang mit der Übermittlung unterrichtet zu werden.
            \item Der Verantwortliche stellt eine Kopie der personenbezogenen Daten, die Gegenstand der Verarbeitung sind, zur Verfügung. Für alle weiteren Kopien, die die betroffene Person beantragt, kann der Verantwortliche ein angemessenes Entgelt auf der Grundlage der Verwaltungskosten verlangen. Stellt die betroffene Person den Antrag elektronisch, so sind die Informationen in einem gängigen elektronischen Format zur Verfügung zu stellen, sofern sie nichts anderes angibt.
            \item Das Recht auf Erhalt einer Kopie gemäß Absatz 1b darf die Rechte und Freiheiten anderer Personen nicht beeinträchtigen. 
        \end{enumerate}

    \section{Art. 16: Recht auf Berichtigung}
        Die betroffene Person hat das Recht, von dem Verantwortlichen unverzüglich die Berichtigung sie betreffender unrichtiger personenbezogener Daten zu verlangen. Unter Berücksichtigung der Zwecke der Verarbeitung hat die betroffene Person das Recht, die Vervollständigung unvollständiger personenbezogener Daten — auch mittels einer ergänzenden Erklärung — zu verlangen.



    \section[Art. 17 : Recht auf Löschung]{Art. 17: Recht auf Löschung (``Recht auf Vergessenwerden'')}
        \begin{enumerate}[label=(\arabic*)]
            \item Die betroffene Person hat das Recht, von dem Verantwortlichen zu verlangen, dass sie betreffende personenbezogene Daten unverzüglich gelöscht werden, und der Verantwortliche ist verpflichtet, personenbezogene Daten unverzüglich zu löschen, sofern einer der folgenden Gründe zutrifft:
                \begin{enumerate}[label=\alph*)]
                    \item Die personenbezogenen Daten sind für die Zwecke, für die sie erhoben oder auf sonstige Weise verarbeitet wurden, nicht mehr notwendig.
                    \item Die betroffene Person widerruft ihre Einwilligung, auf die sich die Verarbeitung gemäß Artikel 6 Absatz 1 Buchstabe a oder Artikel 9 Absatz 2 Buchstabe a stützte, und es fehlt an einer anderweitigen Rechtsgrundlage für die Verarbeitung.
                    \item Die betroffene Person legt gemäß Artikel 21 Absatz 1 Widerspruch gegen die Verarbeitung ein und es liegen keine vorrangigen berechtigten Gründe für die Verarbeitung vor, oder die betroffene Person legt gemäß Artikel 21 Absatz 2 Widerspruch gegen die Verarbeitung ein.
                    \item Die personenbezogenen Daten wurden unrechtmäßig verarbeitet. 
                    \item Die Löschung der personenbezogenen Daten ist zur Erfüllung einer rechtlichen Verpflichtung nach dem Unionsrecht oder dem Recht der Mitgliedstaaten erforderlich, dem der Verantwortliche unterliegt. 
                    \item Die personenbezogenen Daten wurden in Bezug auf angebotene Dienste der Informationsgesellschaft gemäß Artikel 8 Absatz 1 erhoben.
                \end{enumerate}
            \item Hat der Verantwortliche die personenbezogenen Daten öffentlich gemacht und ist er gemäß Absatz 1 zu deren Löschung verpflichtet, so trifft er unter Berücksichtigung der verfügbaren Technologie und der Implementierungskosten angemessene Maßnahmen, auch technischer Art, um für die Datenverarbeitung Verantwortliche, die die personenbezogenen Daten verarbeiten, darüber zu informieren, dass eine betroffene Person von ihnen die Löschung aller Links zu diesen personenbezogenen Daten oder von Kopien oder Replikationen dieser personenbezogenen Daten verlangt hat.
            \item Die Absätze 1 und 2 gelten nicht, soweit die Verarbeitung erforderlich ist
                \begin{enumerate}[label=\alph*)]
                    \item zur Ausübung des Rechts auf freie Meinungsäußerung und Information;
                    \item zur Erfüllung einer rechtlichen Verpflichtung, die die Verarbeitung nach dem Recht der Union oder der Mitgliedstaaten, dem der Verantwortliche unterliegt, erfordert, oder zur Wahrnehmung einer Aufgabe, die im öffentlichen Interesse liegt oder in Ausübung öffentlicher Gewalt erfolgt, die dem Verantwortlichen übertragen wurde;
                    \item aus Gründen des öffentlichen Interesses im Bereich der öffentlichen Gesundheit gemäß Artikel 9 Absatz 2 Buchstaben h und i sowie Artikel 9 Absatz 3; 
                    \item für im öffentlichen Interesse liegende Archivzwecke, wissenschaftliche oder historische Forschungszwecke oder für statistische Zwecke gemäß Artikel 89 Absatz 1, soweit das in Absatz 1 genannte Recht voraussichtlich die Verwirklichung der Ziele dieser Verarbeitung unmöglich macht oder ernsthaft beeinträchtigt, oder 
                    \item zur Geltendmachung, Ausübung oder Verteidigung von Rechtsansprüchen. 
                \end{enumerate}
        \end{enumerate}

    \section{Art. 18: Recht auf Einschränkung der Verarbeitung}
        \begin{enumerate}[label=(\arabic*)]
            \item Die betroffene Person hat das Recht, von dem Verantwortlichen die Einschränkung der Verarbeitung zu verlangen, wenn eine der folgenden Voraussetzungen gegeben ist:
                \begin{enumerate}[label=\alph*)]
                    \item die Richtigkeit der personenbezogenen Daten von der betroffenen Person bestritten wird, und zwar für eine Dauer, die es dem Verantwortlichen ermöglicht, die Richtigkeit der personenbezogenen Daten zu überprüfen,
                    \item die Verarbeitung unrechtmäßig ist und die betroffene Person die Löschung der personenbezogenen Daten ablehnt und stattdessen die Einschränkung der Nutzung der personenbezogenen Daten verlangt; 
                    \item der Verantwortliche die personenbezogenen Daten für die Zwecke der Verarbeitung nicht länger benötigt, die betroffene Person sie jedoch zur Geltendmachung, Ausübung oder Verteidigung von Rechtsansprüchen benötigt, oder
                    \item die betroffene Person Widerspruch gegen die Verarbeitung gemäß Artikel 21 Absatz 1 eingelegt hat, solange noch nicht feststeht, ob die berechtigten Gründe des Verantwortlichen gegenüber denen der betroffenen Person überwiegen.
                \end{enumerate}
            \item Wurde die Verarbeitung gemäß Absatz 1 eingeschränkt, so dürfen diese personenbezogenen Daten — von ihrer Speicherung abgesehen — nur mit Einwilligung der betroffenen Person oder zur Geltendmachung, Ausübung oder Verteidigung von Rechtsansprüchen oder zum Schutz der Rechte einer anderen natürlichen oder juristischen Person oder aus Gründen eines wichtigen öffentlichen Interesses der Union oder eines Mitgliedstaats verarbeitet werden.
            \item Eine betroffene Person, die eine Einschränkung der Verarbeitung gemäß Absatz 1 erwirkt hat, wird von dem Verantwortlichen unterrichtet, bevor die Einschränkung aufgehoben wird. 
        \end{enumerate}

    \section{Art. 21: Widerspruchsrecht}
        \begin{enumerate}[label=(\arabic*]
            \item Die betroffene Person hat das Recht, aus Gründen, die sich aus ihrer besonderen Situation ergeben, jederzeit gegen die Verarbeitung sie betreffender personenbezogener Daten, die aufgrund von Artikel 6 Absatz 1 Buchstaben e oder f erfolgt, Widerspruch einzulegen; dies gilt auch für ein auf diese Bestimmungen gestütztes Profiling. Der Verantwortliche verarbeitet die personenbezogenen Daten nicht mehr, es sei denn, er kann zwingende schutzwürdige Gründe für die Verarbeitung nachweisen, die die Interessen, Rechte und Freiheiten der betroffenen Person überwiegen, oder die Verarbeitung dient der Geltendmachung, Ausübung oder Verteidigung von Rechtsansprüchen.
            \item Werden personenbezogene Daten verarbeitet, um Direktwerbung zu betreiben, so hat die betroffene Person das Recht, jederzeit Widerspruch gegen die Verarbeitung sie betreffender personenbezogener Daten zum Zwecke derartiger Werbung einzulegen; dies gilt auch für das Profiling, soweit es mit solcher Direktwerbung in Verbindung steht. 
            \item Widerspricht die betroffene Person der Verarbeitung für Zwecke der Direktwerbung, so werden die personenbezogenen Daten nicht mehr für diese Zwecke verarbeitet.
            \item Die betroffene Person muss spätestens zum Zeitpunkt der ersten Kommunikation mit ihr ausdrücklich auf das in den Absätzen 1 und 2 genannte Recht hingewiesen werden; dieser Hinweis hat in einer verständlichen und von anderen Informationen getrennten Form zu erfolgen.
            \item Im Zusammenhang mit der Nutzung von Diensten der Informationsgesellschaft kann die betroffene Person ungeachtet der Richtlinie 2002/58/EG ihr Widerspruchsrecht mittels automatisierter Verfahren ausüben, bei denen technische Spezifikationen verwendet werden. 
            \item Die betroffene Person hat das Recht, aus Gründen, die sich aus ihrer besonderen Situation ergeben, gegen die sie betreffende Verarbeitung sie betreffender personenbezogener Daten, die zu wissenschaftlichen oder historischen Forschungszwecken oder zu statistischen Zwecken gemäß Artikel 89 Absatz 1 erfolgt, Widerspruch einzulegen, es sei denn, die Verarbeitung ist zur Erfüllung einer im öffentlichen Interesse liegenden Aufgabe erforderlich. 
        \end{enumerate}


    \section[Art. 89: Garantien]{Art. 89: Garantien und Ausnahmen in Bezug auf die Verarbeitung zu im öffentlichen Interesse liegenden Archivzwecken, zu wissenschaftlichen oder historischen Forschungszwecken und zu statistischen Zwecken}
        \begin{enumerate}[label=(\arabic*)]
            \item Die Verarbeitung zu im öffentlichen Interesse liegenden Archivzwecken, zu wissenschaftlichen oder historischen Forschungszwecken oder zu statistischen Zwecken unterliegt geeigneten Garantien für die Rechte und Freiheiten der betroffenen Person gemäß dieser Verordnung. Mit diesen Garantien wird sichergestellt, dass technische und organisatorische Maßnahmen bestehen, mit denen insbesondere die Achtung des Grundsatzes der Datenminimierung gewährleistet wird. Zu diesen Maßnahmen kann die Pseudonymisierung gehören, sofern es möglich ist, diese Zwecke auf diese Weise zu erfüllen. In allen Fällen, in denen diese Zwecke durch die Weiterverarbeitung, bei der die Identifizierung von betroffenen Personen nicht oder nicht mehr möglich ist, erfüllt werden können, werden diese Zwecke auf diese Weise erfüllt.
            \item Werden personenbezogene Daten zu wissenschaftlichen oder historischen Forschungszwecken oder zu statistischen Zwecken verarbeitet, können vorbehaltlich der Bedingungen und Garantien gemäß Absatz 1 des vorliegenden Artikels im Unionsrecht oder im Recht der Mitgliedstaaten insoweit Ausnahmen von den Rechten gemäß der Artikel 15, 16, 18 und 21 vorgesehen werden, als diese Rechte voraussichtlich die Verwirklichung der spezifischen Zwecke unmöglich machen oder ernsthaft beeinträchtigen und solche Ausnahmen für die Erfüllung dieser Zwecke notwendig sind.
            \item Werden personenbezogene Daten für im öffentlichen Interesse liegende Archivzwecke verarbeitet, können vorbehaltlich der Bedingungen und Garantien gemäß Absatz 1 des vorliegenden Artikels im Unionsrecht oder im Recht der Mitgliedstaaten insoweit Ausnahmen von den Rechten gemäß der Artikel 15, 16, 18, 19, 20 und 21 vorgesehen werden, als diese Rechte voraussichtlich die Verwirklichung der spezifischen Zwecke unmöglich machen oder ernsthaft beeinträchtigen und solche Ausnahmen für die Erfüllung dieser Zwecke notwendig sind. 
            \item Dient die in den Absätzen 2 und 3 genannte Verarbeitung gleichzeitig einem anderen Zweck, gelten die Ausnahmen nur für die Verarbeitung zu den in diesen Absätzen genannten Zwecken. 
        \end{enumerate}