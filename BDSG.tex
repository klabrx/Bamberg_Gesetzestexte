\chapter{Bundesdatenschutzgesetz}
\qrcode{https://www.gesetze-im-internet.de/bdsg_2018/BDSG.pdf}
\newline
\url{https://www.gesetze-im-internet.de/bdsg_2018/BDSG.pdf}
    \section[\S 27: Datenverarbeitung zu \dots statistischen Zwecken]{\S 27: Datenverarbeitung zu wissenschaftlichen oder historischen Forschungszwecken und zu statistischen Zwecken}
        \begin{enumerate}[label=(\arabic*)]
            \item Abweichend von Artikel 9 Absatz 1 der Verordnung (EU) 2016/679\footnote{EU-DSGVO} ist die Verarbeitung besonderer Kategorien personenbezogener Daten im Sinne des Artikels 9 Absatz 1 der Verordnung (EU) 2016/679 auch ohne Einwilligung für wissenschaftliche oder historische Forschungszwecke oder für statistische Zwecke zulässig, wenn die Verarbeitung zu diesen Zwecken erforderlich ist und die Interessen des Verantwortlichen an der Verarbeitung die Interessen der betroffenen Person an einem Ausschluss der Verarbeitung erheblich überwiegen. Der Verantwortliche sieht angemessene und spezifische Maßnahmen zur Wahrung der Interessen der betroffenen Person gemäß \S 22 Absatz 2 Satz 2 vor.
            \item Die in den Artikeln 15, 16, 18 und 21 der Verordnung (EU) 2016/679 vorgesehenen Rechte der betroffenen Person sind insoweit beschränkt, als diese Rechte voraussichtlich die Verwirklichung der Forschungs- oder Statistikzwecke unmöglich machen oder ernsthaft beinträchtigen und die Beschränkung für die Erfüllung der Forschungs- oder Statistikzwecke notwendig ist. Das Recht auf Auskunft gemäß Artikel 15 der Verordnung (EU) 2016/679 besteht darüber hinaus nicht, wenn die Daten für Zwecke der wissenschaftlichen Forschung erforderlich sind und die Auskunftserteilung einen unverhältnismäßigen Aufwand erfordern würde.
            \item Ergänzend zu den in \S 22 Absatz 2 genannten Maßnahmen sind zu wissenschaftlichen oder historischen Forschungszwecken oder zu statistischen Zwecken verarbeitete besondere Kategorien personenbezogener Daten im Sinne des Artikels 9 Absatz 1 der Verordnung (EU) 2016/679 zu anonymisieren, sobald dies nach dem Forschungs- oder Statistikzweck möglich ist, es sei denn, berechtigte Interessen der betroffenen Person stehen dem entgegen. Bis dahin sind die Merkmale gesondert zu speichern, mit denen Einzelangaben über persönliche oder sachliche Verhältnisse einer bestimmten oder bestimmbaren Person zugeordnet werden können. Sie dürfen mit den Einzelangaben nur zusammengeführt werden, soweit der Forschungs- oder Statistikzweck dies erfordert.
            \item Der Verantwortliche darf personenbezogene Daten nur veröffentlichen, wenn die betroffene Person eingewilligt hat oder dies für die Darstellung von Forschungsergebnissen über Ereignisse der Zeitgeschichte unerlässlich ist.
        \end{enumerate}

    \section[\S 50 Verarbeitung zu \dots statistischen Zwecken]{\S 50 Verarbeitung zu archivarischen, wissenschaftlichen  und statistischen Zwecken}
    Personenbezogene Daten dürfen im Rahmen der in \S 45 genannten Zwecke in archivarischer, wissenschaftlicher oder statistischer Form verarbeitet werden, wenn hieran ein öffentliches Interesse besteht und geeignete Garantien für die Rechtsgüter der betroffenen Personen vorgesehen werden. Solche Garantien können in einer so zeitnah wie möglich erfolgenden Anonymisierung der personenbezogenen Daten, in Vorkehrungen gegen ihre unbefugte Kenntnisnahme durch Dritte oder in ihrer räumlich und organisatorisch von den sonstigen Fachaufgaben getrennten Verarbeitung bestehen.