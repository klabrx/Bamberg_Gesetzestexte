\chapter{Referentenentwurf Mietspiegelverordnung}
\minitoc
    \section{Allgemeinde Regelungen}
        \subsection{\S 1 - Gegenstand}
        Gegenstand der Verordnung sind der Inhalt und das Verfahren zur Erstellung und Anpassung von Mietspiegeln im Sinne des § 558c Absatz 1 des Bürgerlichen Gesetzbuchs. Die Verordnung betrifft sowohl qualifizierte Mietspiegel (§ 558d Absatz 1 des Bürgerlichen Gesetzbuchs) als auch Mietspiegel, die keine qualifizierten Mietspiegel sind (einfache Mietspiegel).
        \subsection{\S2 - Begriffsbestimmungen}
        \begin{enumerate}[label=(\arabic*)]
            \item Wohnwertrelevante gesetzliche Merkmale sind die in § 558 Absatz 2 Satz 1 des Bürgerlichen Gesetzbuchs genannten Merkmale Art, Größe, Ausstattung, Beschaffenheit und Lage einer Wohnung, soweit sie für die Mietpreisbildung relevant sind oder im Erstellungsstadium des Mietspiegels relevant sein können.
            \item Außergesetzliche Merkmale sind Merkmale in Bezug auf die Wohnung oder das Mietverhältnis, die in § 558 Absatz 2 Satz 1 des Bürgerlichen Gesetzbuchs nicht genannt sind, aber dennoch für die Mietpreisbildung relevant sind oder im Erstellungsstadium des Mietspiegels relevant sein können.
            \item Die Auswertungsgrundgesamtheit ist die Gesamtheit der mietspiegelrelevanten Wohnungen.
            \item Die Erhebungsgrundgesamtheit ist die Gesamtheit der Wohnungen, aus der die Bruttostichprobe gezogen wird, um nach Aussortierung nicht mietspiegelrelevanter Wohnungen die für den Mietspiegel relevante Stichprobe der Auswertungsgrundgesamtheit zu generieren.
        \end{enumerate}

\section{Einfache Mietspiegel}
    \subsection{\S 3 - Erstellung und Anpassung}
    Die Erstellung und Anpassung eines einfachen Mietspiegels ist vorbehaltlich der §§ 4 und 5 an kein Verfahren gebunden.
    \subsection{\S 4 - Dokumentation}
    Die Erstellung und Anpassung eines einfachen Mietspiegels und die dafür verwendeten tatsächlichen Grundlagen sollen in Grundzügen im Mietspiegel oder in einer gesonderten Dokumentation angezeigt und erläutert werden.
    \subsection{\5 - Veröffentlichung}
    Ein einfacher Mietspiegel und seine Dokumentation sollen kostenfrei im Internet veröffentlicht werden. Für ihre Ausgabe in gedruckter Form können angemessene Entgelte verlangt werden.
        
\section{Qualifizierte Mietspiegel}
    \subsection{\S 6 - Allgemeine Anforderungen}
    \begin{enumerate}[label=(\arabic*)]
        \item Das für qualifizierte Mietspiegel bestehende Erfordernis der Erstellung nach wissenschaftlichen Grundsätzen (§ 558d Absatz 1 Satz 1 des Bürgerlichen Gesetzbuchs) betrifft alle Phasen der Mietspiegelerstellung.
        \item Mietspiegel entsprechen wissenschaftlichen Grundsätzen, soweit sie unter Beachtung der in den §§ 7 bis 21 geregelten Anforderungen erstellt wurden. Soweit Mietspiegel diese Anforderungen nicht erfüllen, sind sie einfache Mietspiegel.
    \end{enumerate}
\section{Qualifizierte Mietspiegel - Erstellung}
    \subsection{\S7 - Methoden}
    \begin{enumerate}[label=(\arabic*)]
        \item Qualifizierte Mietspiegel können mittels Regressions- oder mittels Tabellenanalyse oder durch eine Kombination beider Methoden oder durch eine vergleichbar geeignete Methode erstellt werden.
        \item Auf qualifizierte Mietspiegel, die mittels einer Kombination der Regressions- und Tabellenanalyse erstellt werden, sind die §§ 11 bis 16 nur insoweit anzuwenden, als sie die jeweils angewandte Methode betreffen. Entsprechendes gilt für qualifizierte Mietspiegel, die durch eine vergleichbar geeignete Methode (Absatz 1) erstellt werden. 
    \end{enumerate}
    \subsection{\S8 - Datengrundlagen}
    \begin{enumerate}
        \item Qualifizierte Mietspiegel müssen vorbehaltlich des Absatzes 3 auf der Grundlage einer direkten Datenerhebung durch Befragung von Vermietern oder Mietern oder von beiden Gruppen erstellt werden (Primärdatenerhebung). Eine Vollerhebung ist nicht erforderlich. Qualifizierte Mietspiegel sind zumindest auf der Basis einer repräsentativen Stichprobe zu erstellen mit dem Ziel, die Auswertungsgrundgesamtheit möglichst wirklichkeitsgetreu abzubilden. Als repräsentativ gilt eine Stichprobe mit einer nach § 11 ausreichenden Datenmenge, wenn sie auf einer Zufallsauswahl beruht, bei der im Wesentlichen jede Wohnung der Auswertungsgrundgesamtheit eine positive und bekannte Wahrscheinlichkeit hat, in die Erhebung einbezogen zu werden.
        \item Nicht durch eine Primärdatenerhebung ermittelte Daten über Wohnungen (Sekundärdaten) dürfen zur Vorbereitung der Datenerhebung oder zur Plausibilitätsprüfung (§ 9 Absatz 3 Satz 1) verwendet werden.
        \item Qualifizierte Mietspiegel können auch Angaben enthalten, die auf einer Auswertung solcher Primärdaten beruhen, die mangels ausreichender Fallzahlen keine verlässlichen Angaben zur Mietpreisbildung zulassen. Sie können auch Angaben aufgrund der Auswertung von Sekundärdaten oder fachkundlichen Schätzungen enthalten. Angaben nach den Sätzen 1 und 2 sind nicht Teil des qualifizierten Mietspiegels; hierauf ist im Mietspiegel ausdrücklich hinzuweisen. Die Angaben sollen in entsprechender Anwendung des § 4 dokumentiert werden.
        \item In der Dokumentation sind die Erstellung der Erhebungsgrundgesamtheit und die dafür verwendeten Datengrundlagen darzustellen.
    \end{enumerate}
    \subsection{\S9 - Bruttostichprobe}
    \begin{enumerate}[label=(\arabic*)]
        \item Beim Ziehen einer Stichprobe von Wohnungen, hinsichtlich derer eine Primärdatenerhebung stattfinden soll (Bruttostichprobe), ist sicherzustellen, dass es sich um eine
repräsentative Stichprobe nach § 8 Absatz 1 Satz 4 handelt.
        \item Die Bruttostichprobe kann nach wohnwertrelevanten gesetzlichen Merkmalen oder außergesetzlichen Merkmalen proportional oder disproportional geschichtet werden. Eine Schichtung kann insbesondere nach Vermietertypen, Größenklassen, Ausstattungsmerkmalen, Wohnlagen und Baualtersklassen vorgenommen werden. Die Schichtung erfolgt aufgrund einer Aufteilung der Erhebungsgrundgesamtheit in homogene und überschneidungsfreie Teilgruppen. Wurde eine disproportional geschichtete Zufallsstichprobe gezogen, so ist bei der Datenauswertung eine entsprechende Rückgewichtung vorzunehmen, sofern ansonsten eine Verzerrung der Ergebnisse zu erwarten ist. 
        \item Liegen gesicherte Erkenntnisse über die statistische Ausprägung wesentlicher wohnwertrelevanter gesetzlicher oder außergesetzlicher Merkmale und über ihre Anteile an der Erhebungsgrundgesamtheit vor, so soll die Bruttostichprobe darauf überprüft werden, ob Wohnungen mit solchen statistischen Ausprägungen entsprechend ihrem Anteil an der Erhebungsgrundgesamtheit vertreten sind (Plausibilitätsprüfung). Sind Wohnungen mit solchen statistischen Ausprägungen offensichtlich nicht angemessen vertreten und sind dadurch Verzerrungen der Ergebnisse zu erwarten, soll einer Verzerrung durch geeignete Maßnahmen, beispielsweise durch eine korrigierende Gewichtung bei der Datenauswertung, begegnet werden.
        \item In der Dokumentation ist nachvollziehbar darzustellen, wie die Bruttostichprobe gezogen wurde, einschließlich etwaiger Schichtungen und dadurch notwendiger Rückgewichtungen, ob und in welcher Weise eine Plausibilitätsprüfung durchgeführt wurde, zu welchem Ergebnis eine solche Überprüfung geführt hat und welche Folgerungen daraus gezogen wurden. 
    \end{enumerate}

§ 10
Nettostichprobe
(1) Die Nettostichprobe ist der Rücklauf aus der Befragung von Vermietern oder Mietern oder beider Gruppen. Die Nettostichprobe ist um die Rückläufer zu bereinigen, die
mangels Zugehörigkeit zur Auswertungsgrundgesamtheit oder aufgrund einer Mehrfachzählung derselben Wohnung oder aufgrund grob unvollständiger oder offensichtlich unzutreffender Antworten für die Auswertung nicht verwendet werden können (bereinigte Nettostichprobe).
- 8 -
(2) Die Rücklaufquote und die Bereinigung der Nettostichprobe sind zu dokumentieren. In der Dokumentation ist zu darzustellen, ob durch einen unvollständigen oder selektiven Rücklauf oder durch die Bereinigung der Nettostichprobe Verzerrungen der Ergebnisse möglich sind.
§ 11
Stichprobenumfang
(1) Die bereinigte Nettostichprobe muss eine ausreichende Datenmenge enthalten.
(2) Bei Tabellenanalysen ist hierfür im Regelfall eine Belegung von mindestens 30
Wohnungen pro Tabellenfeld erforderlich.
(3) Bei Regressionsanalysen soll die bereinigte Nettostichprobe Wohnungen in einer
Anzahl enthalten, die wenigstens ein Prozent der Wohnungen der Gemeinde entspricht.
Unterschreitet die nach Satz 1 erforderliche Anzahl an Wohnungen 500, so bedarf es in der
Regel eines Stichprobenumfangs von mindestens 500 Wohnungen. Übersteigt die nach
Satz 1 erforderliche Anzahl an Wohnungen 3 000, so genügt ein Stichprobenumfang von
3 000 Wohnungen.
(4) Die Erfüllung der Anforderungen nach den Absätzen 1 bis 3 ist in der Dokumentation nachzuweisen.
§ 12
Datenaufbereitung
(1) Die erhobenen Mietwerte sollen so aufbereitet werden, dass eine einheitliche Ausweisung der ortsüblichen Vergleichsmiete im qualifizierten Mietspiegel als Nettokaltmiete
pro Quadratmeter ermöglicht wird.
(2) Die erhobenen Daten können um Ausreißermieten bereinigt werden. Ausreißermieten sind besonders geringe oder besonders hohe Mieten, die unter Berücksichtigung
der wohnwertrelevanten Eigenschaften der Wohnung mit der weit überwiegenden Zahl der
übrigen Mietwerte unvereinbar erscheinen. Die Ermittlung von Ausreißermieten kann durch
statistische Standardverfahren erfolgen und soll auf Plausibilität überprüft werden. Für die
Prüfung können sowohl wohnwertrelevante gesetzliche als auch außergesetzliche Merkmale herangezogen werden.
(3) In der Dokumentation ist eine Bereinigung um Ausreißermieten einschließlich des
gewählten Verfahrens zu erläutern und es ist darzustellen, welche Mietwerte aus welchen
Gründen ausgesondert wurden.
§ 13
Datenauswertung bei der Tabellenanalyse
(1) Wird die ortsübliche Vergleichsmiete mithilfe der Tabellenanalyse ermittelt, so sind
Tabellenfelder durch Kombinationen wohnwertrelevanter gesetzlicher Merkmale zu bilden
mit dem Ziel, in sich möglichst homogene Tabellenfelder zu erzeugen, die gegenüber anderen Tabellenfeldern möglichst verschieden sind.
- 9 -
(2) Lassen sich ungeachtet des Vorgehens nach Absatz 1 abweichende homogene
Teilmengen innerhalb eines Tabellenfeldes feststellen, die sich in ihren Mieten signifikant
von den restlichen Mieten des Tabellenfeldes unterscheiden, so soll überprüft werden, ob
hierfür separate Tabellenfelder gebildet oder ergänzende Hinweise für die Bewertung dieser Teilmengen gegeben werden können.
(3) In der Dokumentation ist darzustellen, nach welchen Kriterien und Verfahren die
Tabellenfelder gebildet wurden, wie viele Wohnungen für ein Tabellenfeld ausgewertet wurden und wie hoch die Mieten dieser Wohnungen waren.
§ 14
Datenauswertung bei der Regressionsanalyse
(1) Wird die ortsübliche Vergleichsmiete nach der Regressionsanalyse ermittelt, so
sind wohnwertrelevante gesetzliche Merkmale daraufhin zu untersuchen, ob sie einen statistisch signifikanten Einfluss auf den Mietpreis haben mit dem Ziel, den Zusammenhang
zwischen der Miethöhe und der gesetzlichen wohnwertrelevanten Merkmale möglichst gut
zu beschreiben. Außergesetzliche Merkmale können insbesondere zur Wahl des Regressionsmodells und bei der Bemessung von Spannen nach § 16 Absatz 3 herangezogen werden.
(2) In der Dokumentation ist darzustellen und zu erläutern,
1. welche Regressionsfunktion der Analyse zugrunde liegt,
2. welche Merkmale sich mit welchem Einfluss auf die Miethöhe auswirken, ob dieser
Einfluss statistisch signifikant ist und welches Signifikanzniveau dabei zugrunde gelegt
wird,
3. wie hoch der Erklärungsgehalt der verwendeten Regressionsfunktion ist und
4. inwieweit die tatsächlich vorgefundenen Mieten von den Ergebniswerten der Regressionsformel abweichen.
In der Dokumentation ist weiter zu erklären, ob und in welcher Weise eine Modellvalidierung
erfolgte und zu welchem Ergebnis sie führte.
§ 15
Bestimmung und Darstellung der ortsüblichen Vergleichsmiete bei der Tabellenanalyse
(1) In einem nach der Tabellenanalyse erstellten qualifizierten Mietspiegel wird die
ortsübliche Vergleichsmiete in den Tabellenfeldern durch einen Mittelwert und eine um diesen gebildete Spanne dargestellt. Die ortsübliche Vergleichsmiete soll im Einzelfall innerhalb der Spanne durch Zu- und Abschläge vom Mittelwert bestimmt werden.
(2) Der Mittelwert ist das arithmetische Mittel oder der Median und wird aus allen Mieten eines Tabellenfeldes nach einer etwaigen Bereinigung um Ausreißermieten gebildet.
Der Mittelwert entspricht der ortsüblichen Vergleichsmiete für eine Wohnung, die im Vergleich zu anderen Wohnungen des entsprechenden Tabellenfeldes unter Berücksichtigung
von Qualität und Quantität weiterer wohnwertrelevanter gesetzlicher Merkmale, die nicht
mittels der Tabellenfelder beschrieben werden, als durchschnittlich zu bewerten ist.
- 10 -
(3) Für die Bildung der Spanne sollen in der Regel je ein Sechstel bis ein Achtel der
nach Ausreißerbereinigung in einem Tabellenfeld verbliebenen Mieten am oberen und am
unteren Ende der größengeordneten Mieten unberücksichtigt bleiben. Bei der Bildung der
Spanne kann berücksichtigt werden, wie stark die Streuung der Mieten insgesamt oder im
jeweiligen Tabellenfeld ist.
(4) Aus wohnwertrelevanten gesetzlichen Merkmalen, die nicht mittels der Tabellenfelder beschrieben werden, können sich Zu- und Abschläge ausgehend vom Mittelwert des
Tabellenfeldes ergeben. Der Mietspiegel kann Bewertungshilfen für die Zu- und Abschläge
vorsehen, um die Einordnung einer Wohnung innerhalb der Spanne eines Tabellenfeldes
zu erleichtern. Machen besondere Merkmale eine Überschreitung des Oberwertes oder
eine Unterschreitung des Unterwertes der Spanne notwendig, ist dies im Mietspiegel gesondert auszuweisen.
(5) Die Bildung der Mittelwerte und der Spannen ist in der Dokumentation zu erläutern.
Sieht der Mietspiegel Bewertungshilfen für Zu- und Abschläge vor, ist in der Dokumentation
darzulegen, nach welchen Kriterien und auf welche Weise diese Bewertungshilfen erstellt
wurden.
§ 16
Bestimmung und Darstellung der ortsüblichen Vergleichsmiete bei der Regressionsanalyse
(1) Die ortsübliche Vergleichsmiete in einem mittels Regressionsanalyse erstellten
qualifizierten Mietspiegel wird im Einzelfall durch Anwendung der Regressionsfunktion ermittelt. Die Vergleichsmiete für eine bestimmte Wohnung kann insbesondere als wohnungsspezifischer Punktwert oder klassifiziert in Tabellenform gegebenenfalls mit Zu- und Abschlägen ausgewiesen werden.
(2) Im qualifizierten Mietspiegel ist darzustellen, wie die durch Regression festgestellten wohnwertrelevanten gesetzlichen Merkmale definiert werden und welchen Einfluss das
jeweilige Merkmal auf die Miethöhe hat.
(3) In dem mittels Regressionsanalyse erstellten qualifizierten Mietspiegel kann die
Schwankungsbreite der ermittelten ortsüblichen Vergleichsmiete durch Spannen berücksichtigt werden. Bei der Bildung von Spannen soll dargestellt werden, inwieweit die durch
Befragung erhobenen Mieten von den auf Basis der Regressionsanalyse errechneten Mieten nach oben oder unten abweichen. Dies kann insbesondere dadurch erfolgen, dass von
der Abweichung zwischen den vorhergesagten und den beobachteten Mieten am oberen
und unteren Ende je ein Sechstel bis ein Achtel nicht berücksichtigt wird.
(4) In der Dokumentation ist zu erläutern, wie das Ergebnis der Regressionsanalyse
im qualifizierten Mietspiegel dargestellt und die ortsübliche Vergleichsmiete einer Wohnung
konkret berechnet wird. Eine etwaige Bildung von Spannen ist darzustellen und zu erläutern.
- 11 -
Unterabschnitt 2
Inhalt des qualifizierten Mietspiegels
§ 17
Art der Wohnungen
(1) Der qualifizierte Mietspiegel soll in der Regel Wohnungen in Mehrfamilienhäusern
mit mehr als zwei Wohnungen erfassen. Andere Wohnungen sowie besondere Wohnungsund Vertragstypen in Mehrfamilienhäusern mit mehr als zwei Wohnungen können bei der
Erstellung eines qualifizierten Mietspiegels unberücksichtigt bleiben oder Gegenstand von
getrennten Erhebungen sein.
(2) Der qualifizierte Mietspiegel muss Angaben dazu enthalten, welche Wohnungsarten von ihm erfasst sind.
§ 18
Größe, Beschaffenheit und Ausstattung der Wohnungen
Im qualifizierten Mietspiegel soll dargestellt sein, welche Auswirkung die Größe sowie
die Beschaffenheit und die Ausstattung der Wohnung, einschließlich der energetischen
Ausstattung und Beschaffenheit, auf die Höhe der Miete pro Quadratmeter hat. Hierzu können Wohnungen in geeigneten Größenklassen zusammengefasst werden und es kann auf
Untermerkmale sowie auf deren Gruppierung und Klassifizierung zurückgegriffen werden,
sofern keine Mehrfachberücksichtigung erfolgt.
§ 19
Wohnlagen
(1) Unterschiedliche Wohnlagen müssen im qualifizierten Mietspiegel nur insoweit gesondert ausgewiesen werden, als eine sachgerechte Unterteilung in Wohnlagen möglich ist
und ein Einfluss der Lage auf die Mietpreisbildung festgestellt werden kann. Unterschiedlich
beschriebene Wohnlagen einer Gemeinde können im Mietspiegel nur dann zusammengefasst werden, wenn der lagebedingte Wohnwert vergleichbar ist.
(2) Zur Ermittlung von Wohnlagen soll untersucht werden, inwiefern sich durch Beschreibungen mittels vor Ort feststellbarer Faktoren wie insbesondere Bebauungs- und Verkehrsdichte, Zentralität, Infrastruktur, Begrünung oder vergleichbarer Kriterien Wohnlagen
einteilen lassen. Wird hierdurch die Einteilung von Wohnlagen nicht sachgerecht ermöglicht, können weitere Bewertungsmaßstäbe wie Bodenrichtwerte oder Kriterien der allgemeinen Beliebtheit bestimmter Wohngegenden berücksichtigt werden.
(3) Weist ein qualifizierter Mietspiegel unterschiedliche Wohnlagen aus, so sollen
diese exakt verortet werden, etwa durch ein Straßenverzeichnis oder durch eine aussagekräftige Wohnlagenkarte.
(4) Soweit wohnwertrelevante Lagemerkmale nicht bereits in eine Wohnlageneinteilung einbezogen wurden oder soweit die Lage vom Durchschnitt vergleichbarer Wohnun-
- 12 -
gen in derselben Wohnlage wesentlich abweicht, können wohnwertrelevante Lagemerkmale durch Zu- oder Abschläge zum Ergebniswert oder innerhalb der nach § 15 Absatz 1
oder § 16 Absatz 3 gebildeten Spanne berücksichtigt werden.
(5) Die Einteilung von Wohnlagen muss in der Dokumentation unter Darlegung der
Beurteilungskriterien und ihrer Zusammenhänge nachvollziehbar erläutert werden. In einem früheren Mietspiegel gebildete Wohnlageeinteilungen können fortgeschrieben werden,
wenn
1. die Dokumentation für den früheren Mietspiegel eine Dokumentation nach Satz 1 enthält und
2. eine Plausibilitätsprüfung erfolgt, die geänderte Verhältnisse vor Ort berücksichtigt.
Die Voraussetzung des Satzes 2 Nummer 1 muss nicht gegeben sein für qualifizierte Mietspiegel, deren Stichtag innerhalb von zwei Jahren nach dem … [einsetzen: Datum des Inkrafttretens dieser Verordnung nach § 25] liegt. Die Durchführung der Plausibilitätsprüfung
und ihre Ergebnisse sind in der Dokumentation zu erläutern.
Unterabschnitt 3
Dokumentation und Veröffentlichung des qualifizierten Mietspiegels
§ 20
Dokumentation
(1) Angaben, die für die Anwendung des qualifizierten Mietspiegels notwendig sind,
einschließlich des Stichtags, zu dem die Daten für den Mietspiegel erhoben wurden, sind
in den Mietspiegel aufzunehmen.
(2) Erläuterungen, die notwendig sind, um das Verfahren und die Bewertungen, die
zu den Angaben im qualifizierten Mietspiegel, auch in der fortgeschriebenen Form, geführt
haben, nachzuvollziehen, sind in einer Dokumentation darzulegen. Die Dokumentation soll
vom Text- und Ergebnisteil des Mietspiegels getrennt sein. Sie soll es ermöglichen, die im
qualifizierten Mietspiegel angegebenen Werte in ihrer Herleitung nachzuvollziehen; nicht
erforderlich ist eine Dokumentation, die eine vollständige Nachberechnung der Ergebnisse
ermöglicht.
(3) In der Dokumentation ist in allgemeiner Form darzustellen, welche der personenbezogenen Daten, die ursprünglich für andere Zwecke erhoben wurden, der Mietspiegelersteller von öffentlichen und nichtöffentlichen Stellen erhalten hat und wozu diese Daten
benötigt und verwendet wurden.
(4) Weitere Anforderungen an die Dokumentation ergeben sich aus § 8 Absatz 4, § 9
Absatz 4, § 10 Absatz 2, § 11 Absatz 4, § 12 Absatz 3, § 13 Absatz 3, § 14 Absatz 2, § 15
Absatz 5, § 16 Absatz 4, § 19 Absatz 5 und § 23 Absatz 3.
- 13 -
§ 21
Veröffentlichung
(1) Der qualifizierte Mietspiegel und seine Dokumentation sollen kostenfrei im Internet
veröffentlicht werden. Für ihre Abgabe in gedruckter Form können angemessene Entgelte
verlangt werden.
(2) Die Veröffentlichung des qualifizierten Mietspiegels soll binnen einer Frist von
neun Monaten nach dem Stichtag, auf den sich die Erhebung bezieht, erfolgen.
(3) Die Dokumentation soll zeitgleich mit der Veröffentlichung des qualifizierten Mietspiegels veröffentlicht werden.
Unterabschnitt 4
Anpassung des qualifizierten Mietspiegels
§ 22
Anpassung mittels Index
Erfolgt die Anpassung des qualifizierten Mietspiegels unter Zugrundelegung der Entwicklung des vom Statistischen Bundesamt oder vom zuständigen Statistischen Landesamt
veröffentlichten Indexes für die Nettokaltmiete, so gelten die §§ 20 und 21 entsprechend.
§ 23
Anpassung mittels Stichprobe
(1) Bei der Anpassung eines qualifizierten Mietspiegels mittels Stichprobe können vereinfachende, mit der Fortschreibung auf der Grundlage eines Indexes vergleichbare Annahmen getroffen werden.
(2) Die §§ 7 bis 21 sind auf die Anpassung mittels Stichprobe entsprechend anwendbar. Der Umfang der bereinigten Nettostichprobe kann von den in § 11 bezeichneten Werten abweichen, sofern nach Absatz 1 getroffene, vereinfachende Annahmen dies zulassen.
(3) Vereinfachende Annahmen nach Absatz 1 sowie ein von den Werten des § 11 abweichender Stichprobenumfang sind in der Dokumentation darzulegen und zu begründen.
- 14 -
A b s c h n i t t 4
S c h l u s s v o r s c h r i f t e n
§ 24
Übergangsvorschrift
Diese Verordnung ist auf Mietspiegel anzuwenden, die ein Jahr nach dem … [einsetzen: Datum der Verkündung des Gesetzes] veröffentlicht werden.
§ 25
Inkrafttreten
Diese Verordnung tritt am Tag nach der Verkündung in Kraft.
Der Bundesrat hat zugestimmt.