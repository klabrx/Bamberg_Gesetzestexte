\chapter{Leitfaden zur Einrichtung abgeschotteter kommunaler Statistikstellen in Bayern}
Jede Kommune hat die Möglichkeit, eine abgeschottete Statistikstelle entsprechend Art. 20 BayStatG einzurichten, um in bestimmten, gesetzlich geregelten Fällen Einzeldaten übermittelt zu bekommen (vgl. \S 16 Abs. 5 BStatG), die Auswertungen in tiefer regionaler Gliederung erlauben (z.B. Zensus, Unternehmensregister). Ferner können die abgeschotteten Statistikstellen bei künftigen Großzählungen die Aufgaben einer örtlichen Erhebungsstelle wahrnehmen (vgl. Art. 27 BayStatG; die relevanten Regelungen im BayStatG sind im Anhang aufgeführt). Durch die Abschottung der kommunalen Statistikstellen soll eine eindeutige und für jedermann nachvollziehbare Trennung der Statistik von der übrigen Verwaltung sichergestellt werden. Abschottung bedeutet dabei organisatorische, personelle und räumliche Trennung der Statistikstelle von anderen Verwaltungsstellen (Art. 21 Abs. 3 i.V.m. Art. 20 Abs. 2 und 3 BayStatG). Insbesondere ist auch die IT-Infrastruktur der abgeschotteten Statistikstelle hinreichend vom übrigen Verwaltungsnetz zu trennen. Das strenge Abschottungsgebot resultiert aus dem besonderen Schutzbedarf der statistischen Einzeldaten. Die Geheimhaltung der statistischen Einzelangaben (\S 16 BStatG, Art. 17 BayStatG) als spezielle Form des Datenschutzes ist seit jeher das Fundament der Statistik. Ihre Gewährleistung dient folgenden Zielen:
\begin{itemize}
    \item Schutz des Einzelnen vor der Offenlegung seiner persönlichen und sachlichen Verhältnisse,
    \item Erhaltung des Vertrauensverhältnisses zwischen den Befragten und den statistischen Ämtern,
    \item Gewährleistung der Zuverlässigkeit der Angaben und der Berichtswilligkeit der Befragten.
\end{itemize}
Das Bundesverfassungsgericht hat im Volkszählungsurteil (BVerfGE 65,1) die herausragende Bedeutung des Statistikgeheimnisses hervorgehoben. Es betrachtet den Grundsatz, die zu statistischen Zwecken erhobenen Einzelangaben strikt geheim zu halten, nicht nur als konstitutiv für die Funktionsfähigkeit der Statistik, sondern auch im Hinblick auf den Schutz des Rechts auf informationelle Selbstbestimmung als unverzichtbar.
Da es in Bayern für die zu treffenden Abschottungsmaßnahmen  -- insbesondere im IT-Bereich -- keine weiterführende rechtliche Regelung gibt, wird hiermit ein mit dem Bayerischen Landesbeauftragten für den Datenschutz abgestimmter Leitfaden zur Verfügung gestellt. Eine individuelle, davon abweichende kommunale Regelung sollte trotzdem immer noch mit dem behördlichen Datenschutzbeauftragten abgestimmt werden. 
Der Leitfaden richtet sich insbesondere an Kommunen, die erstmalig eine abgeschottete Statistikstelle einrichten möchten; es ist nicht erforderlich, dass eine Kommune mit bereits bestehender Statistikstelle Änderungen an ihrer bewährten individuellen Lösung vornimmt, sofern diese mit dem behördlichen Datenschutzbeauftragten abgestimmt ist.
Die im Folgenden aufgeführten Maßnahmen sind zu unterscheiden nach Maßnahmen, die konkret (z.B. im BayStatG) gefordert sind und Maßnahmen, die mangels konkreter gesetzlicher Vorgaben aus dem allgemeinen Abschottungsgebot abgeleitet wurden. Erstere sind direkt durch den Verweis auf die entsprechende Rechtsgrundlage gekennzeichnet, letztere (vorrangig die Maßnahmen zur IT-Abschottung) sind grundsätzlich nicht als Vorschrift, sondern als Empfehlung zu verstehen. D.h., eine Kommune, die sich bei der Einrichtung einer abgeschotteten Statistikstelle an die aufgeführten Maßnahmen hält, kann mit einer problemlosen Abstimmung mit dem behördlichen Datenschutzbeauftragten rechnen; weicht die Kommune von den Empfehlungen ab, entsteht evtl. weiterer Abstimmungsbedarf. Für die statistischen Einzeldaten wird das Schutzbedarfsniveau "normal" vorausgesetzt (Schutzstufe C gemäß Datensicherungskatalog des Koordinierungsausschusses Datenverarbeitung; Nä-heres hierzu in Abschnitt 4).