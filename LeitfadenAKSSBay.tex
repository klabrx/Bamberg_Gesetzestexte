\chapter[Leitfaden Abschottung]{Leitfaden zur Einrichtung abgeschotteter kommunaler Statistikstellen in Bayern}
Jede Kommune hat die Möglichkeit, eine abgeschottete Statistikstelle entsprechend Art. 20 BayStatG einzurichten, um in bestimmten, gesetzlich geregelten Fällen Einzeldaten übermittelt zu bekommen (vgl. \S 16 Abs. 5 BStatG), die Auswertungen in tiefer regionaler Gliederung erlauben (z.B. Zensus, Unternehmensregister). Ferner können die abgeschotteten Statistikstellen bei künftigen Großzählungen die Aufgaben einer örtlichen Erhebungsstelle wahrnehmen (vgl. Art. 27 BayStatG; die relevanten Regelungen im BayStatG sind im Anhang aufgeführt). Durch die Abschottung der kommunalen Statistikstellen soll eine eindeutige und für jedermann nachvollziehbare Trennung der Statistik von der übrigen Verwaltung sichergestellt werden. Abschottung bedeutet dabei organisatorische, personelle und räumliche Trennung der Statistikstelle von anderen Verwaltungsstellen (Art. 21 Abs. 3 i.V.m. Art. 20 Abs. 2 und 3 BayStatG). Insbesondere ist auch die IT-Infrastruktur der abgeschotteten Statistikstelle hinreichend vom übrigen Verwaltungsnetz zu trennen. Das strenge Abschottungsgebot resultiert aus dem besonderen Schutzbedarf der statistischen Einzeldaten. Die Geheimhaltung der statistischen Einzelangaben (\S 16 BStatG, Art. 17 BayStatG) als spezielle Form des Datenschutzes ist seit jeher das Fundament der Statistik. Ihre Gewährleistung dient folgenden Zielen:
\begin{itemize}
    \item Schutz des Einzelnen vor der Offenlegung seiner persönlichen und sachlichen Verhältnisse,
    \item Erhaltung des Vertrauensverhältnisses zwischen den Befragten und den statistischen Ämtern,
    \item Gewährleistung der Zuverlässigkeit der Angaben und der Berichtswilligkeit der Befragten.
\end{itemize}
Das Bundesverfassungsgericht hat im Volkszählungsurteil (BVerfGE 65,1) die herausragende Bedeutung des Statistikgeheimnisses hervorgehoben. Es betrachtet den Grundsatz, die zu statistischen Zwecken erhobenen Einzelangaben strikt geheim zu halten, nicht nur als konstitutiv für die Funktionsfähigkeit der Statistik, sondern auch im Hinblick auf den Schutz des Rechts auf informationelle Selbstbestimmung als unverzichtbar.
Da es in Bayern für die zu treffenden Abschottungsmaßnahmen  - insbesondere im IT-Bereich - keine weiterführende rechtliche Regelung gibt, wird hiermit ein mit dem Bayerischen Landesbeauftragten für den Datenschutz abgestimmter Leitfaden zur Verfügung gestellt. Eine individuelle, davon abweichende kommunale Regelung sollte trotzdem immer noch mit dem behördlichen Datenschutzbeauftragten abgestimmt werden. 
Der Leitfaden richtet sich insbesondere an Kommunen, die erstmalig eine abgeschottete Statistikstelle einrichten möchten; es ist nicht erforderlich, dass eine Kommune mit bereits bestehender Statistikstelle Änderungen an ihrer bewährten individuellen Lösung vornimmt, sofern diese mit dem behördlichen Datenschutzbeauftragten abgestimmt ist.
Die im Folgenden aufgeführten Maßnahmen sind zu unterscheiden nach Maßnahmen, die konkret (z.B. im BayStatG) gefordert sind und Maßnahmen, die mangels konkreter gesetzlicher Vorgaben aus dem allgemeinen Abschottungsgebot abgeleitet wurden. Erstere sind direkt durch den Verweis auf die entsprechende Rechtsgrundlage gekennzeichnet, letztere (vorrangig die Maßnahmen zur IT-Abschottung) sind grundsätzlich nicht als Vorschrift, sondern als Empfehlung zu verstehen. D.h., eine Kommune, die sich bei der Einrichtung einer abgeschotteten Statistikstelle an die aufgeführten Maßnahmen hält, kann mit einer problemlosen Abstimmung mit dem behördlichen Datenschutzbeauftragten rechnen; weicht die Kommune von den Empfehlungen ab, entsteht evtl. weiterer Abstimmungsbedarf. Für die statistischen Einzeldaten wird das Schutzbedarfsniveau``normal'' vorausgesetzt (Schutzstufe C gemäß Datensicherungskatalog des Koordinierungsausschusses Datenverarbeitung; Näheres hierzu in Abschnitt ``IT-Abschottung''.
\section{Organisatorische Maßnahmen}
    \begin{itemize}
        \item Zur Einrichtung einer abgeschotteten Statistikstelle ist eine Satzung erforderlich, die u.a. die wesentlichen organisatorischen Bestimmungen zur Wahrung des Statistikgeheimnisses zum Inhalt haben muss (Art. 24 Abs. 2 Satz 1 BayStatG).
        \item Bei der abgeschotteten Statistikstelle muss es sich um eine eigenständige organisatorische Einheit handeln (Art. 20 Abs. 2 Satz 2 BayStatG).
        \item Die Statistikstelle muss über eine eigene Postanschrift verfügen; Post an die Statistikstelle darf nur dort geöffnet werden.
        \item Alle ergriffenen Maßnahmen zur Abschottung müssen in geeigneter Weise dokumentiert werden. Grundlage für die Dokumentation sollte ein Betriebskonzept\footnote{Hilfreich wäre ein Muster-Betriebskonzept, das von einer Kommune erstellt werden sollte.} sein, in dem konkrete Vorgaben für die Umsetzung der im vorliegenden Leitfaden aufgeführten organisatorischen, personellen und technischen Maßnahmen aufgeführt sind. Darin muss beispielsweise ein Berechtigungskonzept für die Zugriffsrechte auf Datenverarbeitungsanlagen sowie ein Konzept für die Außerbetriebnahme von Datenverarbeitungsanlagen enthalten sein. Auf dieses Betriebskonzept ist in der Satzung zu verweisen.
        \item Vom Personal ist stets auf einen „aufgeräumten Arbeitsplatz“ zu achten, d.h. bei Verlassen des Arbeitsplatzes sind sämtliche sensiblen Unterlagen in einem abgeschlossenen Schrank unterzubringen und der Arbeitsplatzrechner gegen unbefugte Zugriffe zu sichern.\footnote{Weiteres zum Thema „aufgeräumter Arbeitsplatz“ kann dem Abschnitt M 2.37 im Maßnahmenkatalog des Bundesamtes für Sicherheit in der Informationstechnik entnommen werden.}
    \end{itemize}
 
 \section{Räumliche Abschottung}
    \begin{itemize}
        \item Die Statistikstelle muss in eigenen Räumlichkeiten eingerichtet werden (Art. 20 Abs. 2 Satz 2 BayStatG).
        \item Die Räumlichkeiten der Statistikstelle müssen gegen den Zutritt unbefugter Personen hinreichend gesichert sein (Art. 20 Abs. 2 Satz 2 BayStatG).
        \item Die räumliche Abschottung ist durch entsprechende zusätzliche Sicherungen, wie festgelegte Arbeitszeit- und Schlüsselregime für das in der Statistikstelle eingesetzte Personal, zu gewährleisten.
        \item Es müssen abschließbare Schränke zur Aufbewahrung der Unterlagen mit Personenbezug vorhanden sein.
        \item Dritte dürfen nur dann Zugang zur Statistikstelle haben, wenn zumindest ein/e Mitarbeiter/in der Statistikstelle präsent ist. Ausgenommen hiervon ist der Zugang durch Reinigungspersonal, Wachpersonal sowie die Hausverwaltung sofern alle dem Datenschutz unterliegenden Unterlagen in abgeschlossenen Schränken verwahrt sind.
    \end{itemize}

\section{Personelle Anforderungen}
Die personellen Anforderungen sind größtenteils in Art. 20 Abs. 2 und 3 BayStatG festgelegt:
\begin{itemize}
    \item Für jede Statistikstelle ist ein Statistikstellenleiter (Art. 20 Abs. 2 Satz 1 BayStatG) und sein Vertreter zu bestimmen.
    \item Die in Statistikstellen tätigen Personen müssen ausschließlich in der Statistikstelle tätig sein; soweit und solange sie Einzelangaben bearbeiten, dürfen sie keine anderen Aufgaben des Verwaltungsvollzugs wahrnehmen (Art. 20 Abs. 3 Satz 3 BayStatG; Ausnahme: Durchführung von Wahlen).
    \item Die in Statistikstellen tätigen Personen müssen die Gewähr auf Zuverlässigkeit und Verschwiegenheit bieten.
    \item Sie sind vor ihrem Einsatz in der Statistikstelle auf die Wahrung des Statistikgeheimnisses und über die Folgen seiner Verletzung zu belehren und schriftlich zu verpflichten (Art. 20 Abs. 3 Satz 2 BayStatG). Die Verpflichtung gilt auch nach Beendigung des Dienstverhältnisses.
    \item Die in Statistikstellen tätigen Personen dürfen statistische Einzelangaben und gelegentlich ihrer Tätigkeit gewonnene Erkenntnisse auch nach Beendigung ihrer Tätigkeit nicht in anderen Verfahren oder für andere Zwecke verarbeiten oder nutzen, soweit nicht durch Rechtsvorschrift etwas anderes zugelassen ist (Art. 20 Abs. 3 Satz 1 BayStatG).
    \item Im Anschluss an eine Tätigkeit in der Statistikstelle sollen sie nicht für Aufgaben eingesetzt werden, bei denen eine Nutzung der in den Statistikstellen gewonnenen Erkenntnisse möglich ist, soweit das die organisatorischen und personellen Verhältnisse zulassen (Art. 20 Abs. 3 Satz 4 BayStatG).
\end{itemize}
Einen Sonderfall stellt, sofern vorhanden, externes IT-Wartungspersonal dar. Hierauf wird im anschließenden Abschnitt „IT-Abschottung“ näher eingegangen.
\section{IT-Abschottung} \label{ITA}
Wenn Einzelangaben in Datenverarbeitungsanlagen verarbeitet werden, ist die Abschottung dieser Daten gegenüber anderen Verwaltungsdaten und ihre Zweckbindung zuverlässig zu gewährleisten. Bevor eine Bewertung einer IT-Infrastruktur in datenschutzrechtlicher Hinsicht erfolgen kann, muss der Schutzbedarf der verarbeiteten Daten ermittelt werden (z.B. gemäß BSI-Grundschutz). Für die im Folgenden beschriebenen Maßnahmen wird davon ausgegangen, dass die statistischen Einzeldaten, die bei den abgeschotteten Statistikstellen verarbeitetet werden, Schutzbedarfsniveau "normal" haben, vergleichbar den entsprechenden Einzeldaten im LfStaD (entspricht Schutzstufe C gemäß Datensicherungskatalog des Koordinierungsausschusses Datenverarbeitung). Sollte eine Kommune im Einzelfall besonders sensible Daten verarbeiten, die erhöhten Schutzbedarf haben, müssten die Maßnahmen entsprechend verschärft werden (z.B. Schutz einzelner Bereiche durch besonders starke Schlüssel).
Die IT-Infrastruktur einer Statistikstelle besteht üblicherweise aus verschiedenen Komponenten, für die auch verschiedene Abschottungsmaßnahmen erforderlich sind. Im Einzelnen sind das
    \begin{itemize}
        \item der Server, auf dem die Einzeldaten liegen und ggf. verarbeitet werden,
        \item das Datensichtgerät, mit dem auf die Statistikdaten zugegriffen wird und auf dem die Daten ggf. verarbeitet werden (üblicherweise ein PC, der auch E-Mail und Internetdienste bereit stellt sowie ggf. ins kommunale Netz eingebunden ist)\footnote{Bei kleineren kommunalen Statistikstellen kann es sich bei diesem Datensichtgerät und dem vorgenannten Server, auf dem die Einzeldaten liegen, auch um ein und denselben PC handeln.}
        \item Backup-Server und externe Speichermedien
        \item die Netzwerkkomponenten, über die eine elektronische Übermittlung statistischer Einzeldaten erfolgt.
    \end{itemize}
Ein weiterer zu betrachtender Aspekt ist die Wartung bzw. Administration der Datenverarbeitungsanlagen, die – sofern sie nicht durch das eigene Personal der Statistikstelle erfolgt – ein Sicherheitsrisiko darstellt. In Einzelfällen kann es auch vorkommen, dass Auswertungsarbeiten an externe Dienstleister ausgelagert werden. Auch dies muss gesondert betrachtet werden.

Im Folgenden werden zwei Grundszenarien für den Aufbau einer IT-Infrastruktur unterschieden und bewertet, die Gefährdungspotenziale kurz dargelegt und abschließend die erforderlichen Abschottungsmaßnahmen aufgeführt.\footnote{Hierbei wird davon ausgegangen, dass eine Auslagerung der Datenspeicherung oder –verarbeitung an externe Dienstleister (also außerhalb der kommunalen Rechenzentren) nicht in Frage kommt. Sollte dies in Einzelfällen dennoch in Erwägung gezogen werden, so ist eine gesonderte Klärung mit dem Datenschutzbeauftragten erforderlich.}
    \begin{description}
        \item[Szenario 1:] Sämtliche Komponenten (Server oder dedizierte Arbeitsplätze mit Datenbestand, Systeme für Datensicherungen) befinden sich (auch räumlich) innerhalb der abgeschotteten Statistikstelle und werden ausschließlich vom eigenen Personal der Statistikstelle betrieben. Die Wartung der Komponenten findet entweder durch eigenes Personal (Szenario 1a) oder durch Personal des kommunalen Rechenzentrums (Szenario 1b) statt. 
        \item[Szenario 2:] Ein oder mehrere Server, auf dem/denen statistische Einzeldaten oder deren Datensicherungen gespeichert sind, befindet/befinden sich außerhalb der abgeschotteten Statistikstelle in einem kommunalen Rechenzentrum und werden vom dortigen Personal betreut.
    \end{description}
Aus der Perspektive des Datenschutzes ist grundsätzlich Szenario 1 zu bevorzugen, da hier sämtliche Zuständigkeiten in der Hand der Statistikstelle bleiben und somit die Abschottung am zuverlässigsten gewährleistet werden kann. Hierfür ist es Voraussetzung, dass die Statistikstelle die Grundschutzempfehlungen des BSI einhält. Um die Realisierung dieses Szenarios auch kleineren Gemeinden zu ermöglichen, können mehrere Gemeinden die Aufgaben einer abgeschotteten Statistikstelle im Wege der interkommunalen Zusammenarbeit (nach KommZG) einer gemeinsamen Statistikstelle übertragen und so diese bestmögliche Abschottung erfüllbar machen.

Da die Satzungen der meisten bereits bestehenden kommunalen Statistikstellen ausdrücklich vorsehen, dass sich die Statistikstelle „der zentralen Datenverarbeitung bedient“, ist davon auszugehen, dass auch Szenario 2 für die IT-Infrastruktur einer abgeschotteten Statistikstelle zumindest nicht ausgeschlossen werden sollte. Diese Lösung ist auch in anderen Bundesländern in entsprechenden Verordnungen berücksichtigt.

Bei Szenario 2 besteht durch die räumliche Nähe der Statistikserver zu den anderen kommunalen Servern die Gefahr einer technischen oder funktionalen Kopplung. Die Sicherheit der Statistikserver darf hierbei nicht von der Sicherheit der anderen Server abhängen bzw. durch diese nicht beeinträchtigt werden. Es ist deshalb eine wichtige Voraussetzung, dass das kommunale Rechenzentrum die Grundschutzempfehlungen des BSI einhält. Bei der Entscheidung, welches Szenario für den Aufbau der IT-Infrastruktur gewählt werden soll, sollte in jedem Fall Sicherheit vor Wirtschaftlichkeit gehen. Im Folgenden sind die zur IT-Abschottung empfohlenen Maßnahmen aufgeführt; in Klammern ist bei jeder Maßnahme angegeben, welche Szenarien sie betrifft.
    \subsection{Maßnahmen zur IT-Abschottung in kommunalen Statistikstellen:}
        \begin{enumerate}[label=arabic*.]
            \item Datenverarbeitungsanlagen, auf denen statistische Einzeldaten gespeichert oder verarbeitet werden, müssen durch geeignete technische Maßnahmen (abgeschottetes Netzsegment, z.B. durch Firewall, VPN-Gateway) gegenüber dem übrigen kommunalen Verwaltungsnetz und gegenüber dem Internet sowie ggf. innerhalb des kommunalen Rechenzentrums abgesichert sein (1, 2).
            \item Sind Datenserver, auf denen statistische Einzeldaten gespeichert sind (kurz: Statistikdatenserver) räumlich nicht innerhalb der abgeschotteten Statistikstelle untergebracht, so sollten sie innerhalb des kommunalen Rechenzentrums separat stehen (z.B. in einem RZ-Schrank). Die Daten müssen in diesem Fall verschlüsselt gespeichert werden. Die Verschlüsselung muss auch gegenüber den Systemadministratoren, soweit sie nicht Mitarbeiter/-innen der Statistikstelle sind, wirken (d.h. die Verschlüsselung muss bereits vor der Übertragung der Einzeldaten auf den Statistikdatenserver erfolgen) (2).
            \item Der Zugriff auf statistische Einzeldaten sowie auf Programme zu deren Verarbeitung darf nur von einer sich räumlich innerhalb der abgeschotteten Statistikstelle befindlichen Datenverarbeitungsanlage möglich sein. Ausnahmen sind zulässig (z.B. Telearbeit, Einsatz mobiler Geräte), die technischen Lösungen hierfür müssen jedoch gesondert mit dem Datenschutzbeauftragten abgestimmt werden (1, 2).
            \item Der Zugriff auf Dateien und Programme ist durch Identifikation (Benutzerkennung) und Authentifizierung (Bestätigung der Identität) zu sichern. Hierfür kommen Passwort-, Chipkarten- und biometrische sowie kombinierte Verfahren in Betracht. Die Zugriffsrechte sind in einem Berechtigungskonzept festzuhalten (Bestandteil des Betriebskonzepts, beschrieben werden müssen u.a. die Befugnisse verschiedener Personengruppen, die Prozesse zur Vergabe und zum Entzug von Rechten sowie Protokollierungspflichten). Sie sind vom Leiter der kommunalen Statistikstelle schriftlich zu vergeben (1, 2).
            \item Es ist zu protokollieren, zu welcher Zeit auf welche statistischen Einzeldaten zugegriffen wurde, mit welchen Programmen sie verarbeitet wurden und an wen solche Daten übermittelt worden sind. Die Protokollierung erfolgt automatisch, soweit technisch möglich, ansonsten manuell. Unberechtigte Zugriffsversuche müssen auf jeden Fall automatisch protokolliert werden. Für die jeweiligen Protokollierungen sind geeignete Auswertungsverfahren festzulegen (Bestandteil des Betriebskonzepts) (1, 2).
            \item Jedes Endgerät, das eine Ein- oder Ausgabe von Daten ermöglicht, darf vom Bearbeiter nur so hinterlassen werden, dass in seiner Abwesenheit durch Nichtberechtigte ein Zugriff auf Einzeldaten nicht möglich ist (1, 2).
            \item Alle externen, mobilen Datenträger, auf denen statistische Einzeldaten gesichert werden, sind eindeutig zu kennzeichnen, zu katalogisieren und auf Vollzähligkeit zu prüfen; sie sind unter Verschluss aufzubewahren. Dasselbe gilt für die Aufbewahrung von Programmakten und -dokumentationen (1, 2).
            \item Die Aufbewahrung von maschinell verwendbaren Datenträgern ist außerhalb der kommunalen Statistikstelle nur in einem zugangsgeschützten Raum erlaubt. In diesem Fall sind die Datenträger so aufzubewahren, dass sie gegen den Zugriff von denjenigen Personen geschützt sind, die nicht schriftlich durch den Leiter der kommunalen Statistikstelle ermächtigt worden sind und ein Entfernen bzw. Manipulieren der Datenträger zuverlässig verhindert wird. Jede Entnahme ist unter Angabe des Zeitpunktes und des Namens des Entnehmenden aufzuzeichnen (2).
            \item Sind Einzelangaben, die auf maschinell verwendbaren Datenträgern gespeichert sind, zu löschen, so hat dies physisch zu geschehen (dies gilt auch für Festplatten, interne Speicher von Druckern etc.). Datenträger mit temporären Daten oder Daten, die für statistische Auswertungen nicht mehr benötigt werden, sind unverzüglich physisch zu löschen. Die Einzelheiten sind in einem Konzept für die Außerbetriebnahme einer Datenverarbeitungsanlage (Bestandteil des Betriebskonzepts) festzulegen (1, 2).
            \item Wenn für die Datensicherung Ressourcen außerhalb der abgeschotteten Statistikstelle genutzt werden (externe Backup-Server), muss die Datensicherung in verschlüsselter Form stattfinden (analog zu Maßnahme 2) (1,2).
            \item Werden die Daten über elektronische Netze übermittelt, müssen sie verschlüsselt übertragen werden (dies gilt auch für die Übermittlung statistischer Einzeldaten vom Bayerischen Landesamt für Statistik und Datenverarbeitung an die kommunalen Statistikstel-len). Es ist sicherzustellen, dass die Entschlüsselung der Daten nur durch den berechtig-ten Empfänger vorgenommen werden kann (1, 2).
            \item Die Tätigkeiten der Systemadministration für die Datenverarbeitungsanlagen und Verfah-ren dürfen nur von den für die Datenverarbeitungsanlagen der kommunalen Statistikstelle zuständigen Systemadministratoren vorgenommen werden und müssen protokolliert werden. Die Protokollierungen sind durch geeignete und angemessene Maßnahmen vor Manipulationen auch durch die Administratoren zu schützen. Der Einsatz von externem Personal muss gesondert mit dem Datenschutzbeauftragten abgestimmt werden (1, 2).
            \item Die Tätigkeiten der Systemadministration für die Datenverarbeitungsanlagen und Verfah-ren innerhalb der abgeschotteten Statistikstelle sollten nach Möglichkeit von fachkundi-gen Bediensteten der kommunalen Statistikstelle vorgenommen werden (1a). Ist dies aus personellen Gründen nicht möglich, so kann die Systemadministration auch durch Mitar-beiter des kommunalen Rechenzentrums erfolgen. Es sollte sich dabei um einen einge-schränkten Kreis von bewährten und zuverlässigen Personen handeln, die vor ihrem Ein-satz auf die Wahrung des Statistikgeheimnisses und über die Folgen seiner Verletzung zu belehren und schriftlich zu verpflichten sind. (1b).
            \item Die Tätigkeiten der Systemadministration für Statistikdatenserver und Backup-Server, die sich außerhalb der abgeschotteten Statistikstelle befinden, müssen ebenfalls durch fach-kundige und zuverlässige Personen durchgeführt werden. Eine schriftliche Verpflichtung auf die Wahrung des Statistikgeheimnisses und über die Folgen seiner Verletzung ist nicht erforderlich, sofern, wie vorgeschrieben, die Speicherung der Daten auf diesen Ser-vern in verschlüsselter Form erfolgt und diese Verschlüsselung auch gegenüber den Administratoren wirkt (vgl. Maßnahme 2) (2).
            \item Eine Fernwartung der Datenverarbeitungsanlagen und Verfahren der kommunalen Sta-tistikstelle ist nur in begründeten Ausnahmefällen, nur nach vorheriger Gestattung und bei einer Überwachung der Fernwartungstätigkeiten erlaubt. Die Verbindung sollte stets vom Kunden (also von der kommunalen Statistikstelle) aus aufgebaut werden. Die Maß-nahmen 12 bis 14 sind explizit anzuwenden. Technische Einzelheiten sind gesondert in einem Konzept zu regeln und mit dem Datenschutzbeauftragten abzustimmen (1b, 2).
            \item Sollte es in Einzelfällen erforderlich sein, Auswertungsarbeiten an einen externen Dienst-leister auszulagern, ist sicherzustellen, dass die vorstehenden Maßnahmen entspre-chend angewendet werden. Hier ist eine gesonderte Abstimmung mit dem Datenschutz-beauftragten erforderlich (1, 2).
        \end{enumerate}
